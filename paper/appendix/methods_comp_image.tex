
\subsubsection{Computational methods for image processing}

\paragraph{Confocal Image Processing}
Image data were acquired at 8 or 16 bits, with voxel sizes of 1 µm x 0.62 µm x 0.62 µm (MitoHealth) or 1 µm x 0.31 µm x 0.31 µm (PLIN2, LipidSpot, Aβ, β-secretase). Image files extracted from the confocal Zeiss microscope (.czi format) were loaded into Python using the `aicsimageio` package and converted to floating-point format in the range $[0,1]$. Confocal image acquisition settings were kept consistent within each imaging batch.

A pre-trained model ("cyto2" from Cellpose\cite{Stringer2021-ek}) was applied to segment NeuN+ cell bodies per image. For images sampled at 0.62 µm along the xy-plane, segmentation on the NeuN channel in 3D produced the best results. For images sampled at 0.31 µm along the xy-plane, segmentation on the NeuN and Hoechst channels in 2D (xy), with subsequent stitching along the z-axis, produced the best results. Specific segmentation settings were determined for each imaging experiment. Segmentation quality was assessed manually, blinded to condition, and images with low-quality segmentations were discarded.

The model outputs per-voxel probabilities representing the Bernoulli probability that a given voxel lies within a cell (any cell) and per-voxel masks—recovered from flow vectors and from the pixel probabilities output by the model—representing regions of interest (cells). We leveraged these per-voxel probabilities to compute the expected fluorescence intensity $E(I_t)$ for our target channel $t$ in each cell $c$. This is calculated as $E(I_t) = a \cdot b$, where $a$ is a 1-dimensional vector containing the measured intensities for channel $t$ across all $n$ voxels annotated as part of the region of interest for cell $c$, and $b$ is a 1-dimensional vector of the same length $n$. Each element $b_i$ in $b$ represents the normalized probability that the corresponding voxel $i$ belongs to cell $c$, calculated as:

$b_i = \frac{\text{Pr}(v_i \in c)}{\sum_{j=0}^{n-1} \text{Pr}(v_j \in c)}$

This normalization ensures that the probabilities sum to 1, providing a weighted contribution of each voxel to the total expected fluorescence intensity for the cell.

A linear mixed-effects model was fit (using `mixedlm()` from the `statsmodels` package) to cell-level average fluorescence intensities, with treatment or genotype as a fixed effect and well-of-origin as a random effect; formalized as follows:

$ Y_{ij} = \beta_0 + \beta_1 X_{ij} + u_j + \epsilon_{ij} $

where:
- $Y_{ij}$ is the observed fluorescence intensity for cell $i$ in well $j$,
- $\beta_0$ is the intercept,
- $\beta_1$ is the coefficient for the fixed effect (treatment or genotype),
- $X_{ij}$ is the fixed effect predictor (treatment or genotype) for cell $i$ in well $j$,
- $u_j$ is the random effect for well $j$, assumed to be normally distributed with mean 0 and variance $\sigma_u^2$,
- $\epsilon_{ij}$ is the residual error for cell $i$ in well $j$, assumed to be normally distributed with mean 0 and variance $\sigma^2$.

Where indicated, measurements were combined over multiple differentiation batches (independent staining and imaging experiments). To this end, an equal number of cells from each experimental condition were sampled uniformly per batch, fluorescent values were z-scaled within that batch, and then combined. Indicator vectors for well-of-origin and batch-of-origin were included in the model. Before applying this linear transformation, per-cell per-image clipping was determined to be low (< 0.1\%) and the response function of the confocal microscope was assumed to be linear.

For each condition, representative images were chosen from a single batch as the images closest to the mean fluorescence intensity for each condition. Voxels not belonging to a cell (i.e., not used in quantification) were masked prior to mean-projection for visualization.

For the confocal image analysis code, see GitHub Repository.
