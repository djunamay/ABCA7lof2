Loss-of-function (LoF) variants in the lipid transporter ABCA7 significantly increase the risk of Alzheimer’s disease (odds ratio ~2), yet the pathogenic mechanisms and the neural cell types affected by these variants remain largely unknown. Here, we performed single-nuclear RNA sequencing of 36 human post-mortem samples from the prefrontal cortex of 12 ABCA7 LoF carriers and 24 matched non-carrier control individuals. ABCA7 LoF was associated with gene expression changes in all major cell types. Excitatory neurons, which expressed the highest levels of ABCA7, showed transcriptional changes related to lipid metabolism, mitochondrial function, cell cycle-related pathways, and synaptic signaling. ABCA7 LoF-associated transcriptional changes in neurons were similarly perturbed in carriers of the common AD missense variant ABCA7 p.Ala1527Gly (n = 240 controls, 135 carriers), indicating that findings from our study may extend to large portions of the at-risk population. Consistent with ABCA7’s function as a lipid exporter, lipidomic analysis of isogenic iPSC-derived neurons (iNs) revealed profound intracellular triglyceride accumulation in ABCA7 LoF, which was accompanied by a relative decrease in phosphatidylcholine abundance. Metabolomic and biochemical analyses of iNs further indicated that ABCA7 LoF was associated with disrupted mitochondrial bioenergetics that suggested impaired lipid breakdown by uncoupled respiration. Treatment of ABCA7 LoF iNs with CDP-choline (a rate-limiting precursor of phosphatidylcholine synthesis) reduced triglyceride accumulation and restored mitochondrial function, indicating that ABCA7 LoF-induced phosphatidylcholine dyshomeostasis may directly disrupt mitochondrial metabolism of lipids. Treatment with CDP-choline also rescued intracellular amyloid 𝛽-42 levels in ABCA7 LoF iNs, further suggesting a link between ABCA7 LoF metabolic disruptions in neurons and AD pathology. This study provides a detailed transcriptomic atlas of ABCA7 LoF in the human brain and mechanistically links ABCA7 LoF-induced lipid perturbations to neuronal energy dyshomeostasis. In line with a growing body of evidence, our study highlights the central role of lipid metabolism in the etiology of Alzheimer’s disease.

