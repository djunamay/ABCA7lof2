\subsubsection{\underline{LC-MS Experiments on iNs}} 

\paragraph{Biphasic Extraction.} 
iPSC-derived neurons were washed once with cold PBS (Fisher; Cat\#MT21040CM) and lifted off plate with a cell scraper in 1 mL cold PBS. Cells were centrifuged at 2000 xg for 5 min. PBS was removed, and cells were resuspended in 2 mL cold methanol for biphasic extraction. Chloroform (Sigma 1.02444) (4 mL; cold) was added to each vial, and mixed by vortexing for 1 min. Water (Sigma WX0001) (2 mL; cold) was added to each vial, and mixed by vortexing for 1 min. Vials were placed in 50 mL conical tubes and centrifuged for 10 min at 3000 rcf for phase separation. The lower, chloroform phase was collected (3 mL from each sample) and transferred to new vials. 
In instances where samples were prepared by the Harvard Center for Mass Spectrometry, cell pellets were provided in 500 µL of methanol, vortexed, and transferred to 8 mL glass vials. Each sample received an additional 1.5 mL of methanol and 4 mL of chloroform, followed by vortexing and incubation for 10 min in an ultrasound bath. Next, 2 mL of water was added, and samples were again vortexed. Phase separation was achieved by centrifugation at 800 rcf for 10 min at 4°C. The resulting upper aqueous phases were transferred into new glass vials designated for metabolomics analysis, while the lower chloroform phases were transferred separately for lipidomics analysis.
At least one blank sample (containing no cells) was prepared alongside each biphasic extraction experiment and processed identically through the LC-MS analysis pipeline.

\paragraph{Cell pellet sample preparation for LC-MS Lipidomics.} 
Subsequent sample preparation for lipidomics was performed by the Harvard Center for Mass Spectrometry. Samples were dried under nitrogen flow until approximately 1 mL remained, transferred into microcentrifuge tubes, and completely evaporated to dryness. Dried samples were resuspended in chloroform, with volumes scaled according to biomass (cell count) using a minimum of 60 µL, then split into two equal aliquots for positive and negative ionization mode analyses. For experiments using only positive ionization mode, samples were resuspended in a smaller biomass-scaled volume (minimum 20–25 µL) without splitting. Following resuspension, samples were centrifuged (maximum speed for 10 min or 18,000 rcf for 20 min at 4°C), and supernatants were transferred into microinserts for LC-MS analysis.

\paragraph{Cell pellet sample preparation for LC-MS Metabolomics.}
Subsequent sample preparation for metabolomics was performed by the Harvard Center for Mass Spectrometry. Samples were dried under nitrogen flow until approximately 1 mL remained, transferred into microcentrifuge tubes, and evaporated completely to dryness. Dried samples were resuspended in 50\% acetonitrile in water, using volumes scaled according to provided biomass (minimum ~20 µL). Following centrifugation at maximum speed for 10 min, a consistent volume (either 12 µL or 15 µL, depending on the batch) of supernatant from each sample was transferred into microinserts. The remaining supernatants from each batch were pooled separately to create batch-specific quality control (QC) samples.

\paragraph{Media preparation for LC-MS Metabolomics.}
Media samples (100 µL each) were transferred into microcentrifuge tubes containing 1 mL of methanol and incubated at -20°C for 2 hours. Following incubation, samples were centrifuged at 18,000 rcf for 20 min at -9°C, and supernatants were transferred into new tubes and evaporated to dryness under nitrogen flow. The dried samples were resuspended in 50 µL of 30\% acetonitrile in water containing 2 mM medronic acid, centrifuged again at 18,000 rcf for 20 min at 4°C, and the resulting supernatants were transferred into glass microinserts for LC-MS analysis.

\paragraph{LC-MS Lipidomics.}
LC-MS lipidomics was performed by the Harvard Center for Mass Spectrometry. LC-MS analyses were modified from \cite{Miraldi2013-ng} and were performed on an Orbitrap Exactive plus MS (Thermo Scientific) in line with an Ultimate 3000 LC (Thermo Scientific). Each sample was analyzed in positive and negative modes, in top 5 automatic data-dependent MS/MS mode. Column hardware consisted of a Biobond C4 column (4.6 x 50 mm, 5 μm, Dikma Technologies). Flow rate was set to 100 µL min-1 for 5 min with 0\% mobile phase B (MB), then switched to 400 µL min-1 for 50 min, with a linear gradient of MB from 20\% to 100\%. The column was then washed at 500 µL min-1 for 8 min at 100\% MB before being re-equilibrated for 7 min at 0\% MB and 500 µL min-1. For positive mode runs, buffers consisted for mobile phase A (MA) of 5mM ammonium formate, 0.1 \% formic acid and 5\% methanol in water, and for MB of 5 mM ammonium formate, 0.1\% formic acid, 5\% water, 35\% methanol in isopropanol. For negative runs, buffers consisted for MA of 0.03\% ammonium hydroxide, 5\% methanol in water, and for MB of 0.03\% ammonium hydroxide, 5\% water, 35\% methanol in isopropanol. Lipids were identified and their signal integrated using the Lipidsearch © software (version 4.2.27, Mitsui Knowledge Industry, University of Tokyo). Integrations and peak quality were curated manually before exporting. 

\paragraph{LC-MS Metabolomics.}
LC-MS metabolomics was performed by the Harvard Center for Mass Spectrometry. Samples were analyzed by LC-MS on a Vanquish LC coupled to an ID-X MS (Thermofisher Scientific). Five µL of sample was injected on a ZIC-pHILIC peek-coated column (150 mm x 2.1 mm, 5 micron particles, maintained at 40 °C, SigmaAldrich). Buffer A was 20 mM Ammonium Carbonate, 0.1\% Ammonium hydroxide in water and Buffer B was Acetonitrile 97\% in water. The LC program was as follows: starting at 93\% B, to 40\% B in 19 min, then to 0\% B in 9 min, maintained at 0\% B for 5 min, then back to 93\% B in 3 min and re-equilibrated at 93\% B for 9 min. The flow rate was maintained at 0.15 mL min-1, except for the first 30 seconds where the flow rate was uniformly ramped from 0.05 to 0.15 mL min-1. Data was acquired on the ID-X in switching polarities at 120000 resolution, with an AGC target of 1e5, and a m/z range of 65 to 1000. MS1 data is acquired in switching polarities for all samples. MS2 and MS3 data was acquired on the pool sample using the AquirX DeepScan function, with 5 reinjections, separately in positive and negative ion mode. A mixture for standards of interest was prepared and analyzed immediately following the sample runs for targeted metabolite analysis.

