\textbf{Discussion}\newline\newline

Here, we generated a transcriptional atlas to identify potential cell-type-specific impacts of ABCA7 LoF variants in the human prefrontal cortex. Excitatory neurons expressed the highest levels of ABCA7 and showed transcriptional alterations in pathways related to lipid biosynthesis, mitochondrial respiration, and cellular stress, including upregulation of DNA damage response genes, as well as changes in inflammatory and synaptic genes. Following experimental validation of predictions from this atlas, we observed that ABCA7 LoF variants impaired mitochondrial uncoupling, elevated mitochondrial membrane potential, and increased ROS levels in human induced neurons. Consistent with ABCA7's function as a phospholipid transporter, ABCA7 LoF altered PC composition in these neurons, characterized by increased saturated PCs and reduced highly polyunsaturated PCs. CDP-choline treatment increased PC synthesis, elevated expression of PC remodeling enzymes, and corrected mitochondrial uncoupling deficits, mitochondrial membrane potential, and oxidative stress. Furthermore, CDP-choline supplementation reduced neuronal hyperexcitability and amyloid-β secretion. 

Our data indicate that ABCA7 LoF neurons accumulate saturated PCs, consistent with phospholipid saturation imbalances recently reported in human ALS/FTD neurons\supercite{Giblin2025-ri}. Treatment with CDP-choline, which boosts \textit{de novo} PC synthesis, effectively mitigated downstream effects of ABCA7 LoF, although it did not fully normalize the lipid profile. Consistent with previous findings \supercite{Boumann2003-im}, increased \textit{de novo} PC synthesis through CDP-choline supplementation may expand the diversity of PC species—potentially by enhancing substrate availability for Lands cycle remodeling—and thereby counteract the observed enrichment of saturated species. Although further research is needed to fully characterize the nature and functional impact of these lipid alterations, our data suggest that neuronal dysfunction linked to ABCA7 LoF variants may partly stem from disrupted phosphatidylcholine metabolism.

Phosphatidylcholine species are abundant components of biological membranes, including in mitochondria, and changes to their fatty acyl chain composition impact mitochondrial bioenergetics, dynamics, and membrane potential \supercite{Decker2024-ae,Wang2019-om,Van_der_Veen2017-ei,Adachi2016-tg}. Mitochondrial dysfunction, including altered mitochondrial uncoupling, is increasingly linked to aging and neurodegeneration, though its specific role in AD remains unclear \supercite{Bano2023-qz,Zong2024-tn,Demine2019-qj,Picca2023-gt}. Neurons depend heavily on mitochondrial OXPHOS to meet their energy demands \supercite{Morant-Ferrando2023-va,Trigo2022-ym}, and regulated mitochondrial uncoupling supports neuronal health by modulating mitochondrial membrane potential, reducing reactive oxygen species, and promoting mitochondrial biogenesis \supercite{Park2023-fa,Demine2019-qj,Shadel2015-kt,Korshunov1997-aj,Wisloff2005-ho,Andrews2005-yy}. Elevated oxidative stress from impaired uncoupling may further contribute to the neuronal DNA damage and inflammation observed in AD brains, and both transcriptionally evident in ABCA7 LoF carriers \supercite{Robert2020-sc,Volanti2002-mc,Canty1999-oj,Schreck1992-zr}. 

Further studies are needed to clarify how ABCA7 regulates PC composition. Previous work suggests ABCA7 transports PCs and their Lands cycle derivatives, LPCs. We speculate that impaired ABCA7 floppase activity—responsible for flipping phospholipids between membrane leaflets—could lead to initial PC accumulation, followed by compensatory reductions in PC synthesis and remodeling towards more saturated species. Yeast studies indicate that reduced de novo PC synthesis triggers remodeling towards increased saturation of the existing PC pool. Alternatively, disrupted floppase activity affecting PCs, LPCs, or other phospholipids transported by ABCA7, such as PE, could indirectly alter PC composition by influencing membrane fluidity and curvature \supercite{Takada2018-ce,Renne2018-fc}. These membrane changes may impact cellular functions \supercite{McMahon2015-gy,Yang2024-tz} and alter transcriptional regulation of lipid biosynthetic and remodeling enzymes (e.g., LPCATs), which are responsive to membrane properties \supercite{Ballweg2020-rv,Covino2018-hz}.

Consistent with our findings linking phosphatidylcholine imbalances to mitochondrial impairment in ABCA7 LoF neurons, a recent independent study identified mitochondrial dysfunction linked to phosphatidylglycerol deficiency in ABCA7-deficient neurospheroids \supercite{Kawatani2023-vf}, further highlighting lipid metabolism as a therapeutic target. Here, we demonstrate that CDP-choline, a safe and widely available dietary supplement \supercite{Gavrilova2018-oi,Zeisel2009-xv,Blusztajn2017-nv}, can reverse ABCA7 LoF-induced neuronal dysfunction, including AD pathology and neuronal hyperexcitability. Recent findings from our lab have similarly implicated phosphatidylcholine and fatty acid saturation imbalances in APOE4-associated dysfunction \supercite{Sienski2021-zt} and cognitive resilience to AD pathology \supercite{Mathys2024-ex}, suggesting broad relevance of phosphatidylcholine disruptions in AD risk. Indeed, our results suggest that the common missense variant p.Ala1527Gly likely produces effects convergent with ABCA7 LoF. Genetic interactions with other risk factors, such as APOE4, could exacerbate mild ABCA7 dysfunction, contributing significantly to AD risk \supercite{Wang2021-oa,Hemani2013-zr,Haig2011-vs,Zuk2012-uz}. Our findings thus support a growing literature, including recent studies on APOE4 \supercite{Haney2024-fx,Victor2022-tl}, underscoring lipid metabolic disruptions as central to AD pathogenesis and identifying additional genotypes that may benefit from phosphatidylcholine-targeted interventions.

