\begin{figure}[H]
    \begin{subfigure}[t]{.6\textwidth}
        \caption{}
        \includegraphics[width=\textwidth]{./extended_plots/plot_rama4.png}        
    \end{subfigure}
    \begin{subfigure}[t]{.4\textwidth}
        \caption{}
        \includegraphics[width=\textwidth]{./extended_plots/plot_rama-joint.png}        
    \end{subfigure}
    \begin{subfigure}[t]{.5\textwidth}
        \caption{}
        \includegraphics[width=\textwidth]{./extended_plots/dssp.png}        
    \end{subfigure}
    \begin{subfigure}[t]{.45\textwidth}
        \caption{}
        \includegraphics[width=\textwidth]{./extended_plots/alpha_helix_distribution_1517_1537.png}        
    \end{subfigure}
    \caption{
        \textbf{Analysis of local conformational fluctuations and secondary structure variations induced by the p.Ala1527Gly substitution in ABCA7 open and closed conformations.}\\
        }
        \label{fig:md_simulations_2}
    \end{figure}
    \begin{itemize}
        \item[\textbf{(A)}] Phi vs. Psi dihedral angle distribution of residue 1527 as a function of simulation time for open and closed conformations.
        \item[\textbf{(B)}] Overall Phi vs. Psi dihedral angle distributions for residue 1527, comparing open and closed conformations throughout the entire simulation period.
        \item[\textbf{(C)}] Time-resolved secondary structure assignments for residues 1517–1537. Alpha-helical structures are highlighted in red; other colors represent distinct secondary structures.
        \item[\textbf{(D)}] Fraction of alpha-helical content observed for residues 1517–1537 during the simulations. A value of 1 indicates uninterrupted preservation of the alpha-helical structure throughout the simulation duration.
    \end{itemize}

