\subsubsection{\underline{Experimental Methods on human \textit{postmortem} brain tissue}}

\paragraph{Isolation of nuclei from frozen post-mortem brain tissue.}
For batch \#1: The protocol for the isolation of nuclei from frozen post-mortem brain tissue (region BA10) was adapted for smaller sample volumes from a previous study \cite{Mathys2019-wb}. All procedures were carried out on ice or at 4°C. In brief, post-mortem brain tissue was homogenized in 700 µl Homogenization Buffer (320 mM sucrose, 5 mM CaCl2, 3 mM Mg(CH3COO)2, 10 mM Tris HCl pH 7.8, 0.1 mM EDTA pH 8.0, 0.1\% IGEPAL CA-630, 1 mM β-mercaptoethanol, and 0.4 U µl-1 recombinant RNase inhibitor (Clontech)) using a Wheaton Dounce tissue grinder (15 strokes with the loose pestle). Homogenized tissue was filtered through a 40-µm cell strainer, mixed with an equal volume of Working Solution, which is prepared by mixing Diluent (30mM CaCl2, 18mM Mg(CH3COO)2, 60mM Tris pH 7.8, 0.6mM EDTA, 6mM β-mercaptoethanol) with Optiprep density gradient solution (Sigma-Aldrich D1556-250ML) in a 1:5 ratio. The sample mix was then loaded on top of an Optiprep density gradient consisting of 750 µl 30\% OptiPrep solution (1.5:1 ratio of Working Solution:Homogenization Buffer) on top of 300 µl 40\% OptiPrep solution (4:1 ratio of Working Solution:Homogenization Buffer). The nuclei were separated by centrifugation (5 min, 10,000 g, 4 °C). Approximately 100µl of nuclei were collected from the 30\%/40\% interface and washed twice with 1 ml of PBS containing 0.04\% BSA, centrifuging 300g for 3 min (4 °C) in between, then resuspended in 100µl PBS containing 0.04\% BSA. The nuclei were counted on C-Chip disposable hemocytometer and diluted to 1000 nuclei per µl in PBS containing 0.04\% BSA. 

For batch \#2: These samples (fresh post-mortem brain; PFC BA10) were prepared as part of and according to a previous study \cite{Mathys2019-wb}.

Informed consent and Anatomical Gift Act consent were obtained from each participant. The Religious Orders Study and Rush Memory and Aging Project were approved by the Institutional Review Board (IRB) of Rush University Medical Center. All participants signed a repository consent, allowing their data and biospecimens to be shared.

\paragraph{Droplet-based snRNA-seq.}\\ 
\newcommand{\quoteJ}{\textcolor{blue}{For batch \#1: cDNA libraries were generated using the Chromium Single Cell 3′ Reagent Kits v3 following the manufacturer's protocol (10x Genomics). Libraries were sequenced on the NovaSeq 6000 S2 platform (paired-end, 28 + 91 bp, with an 8-nucleotide index). Samples were distributed across two lanes and sequenced twice on separate flow cells to enhance sequencing depth.\label{quoteJ-label}}}
\quoteJ\\\\
\newcommand{\quoteK}{\textcolor{blue}{For batch \#2: Libraries were prepared using Chromium Single Cell 3′ Reagent Kits v2 and sequenced with the NextSeq 500/550 High Output v2 kits (150 cycles), as described in our previously published study \cite{Mathys2019-wb}.}}
\quoteK\\\\
\newcommand{\quoteZ}{\textcolor{blue}{Raw sequencing reads from all samples were processed jointly for alignment and gene counting.}}
\quoteZ\\

% \paragraph{RNAscope in post-mortem human brain tissue.}
% Fresh frozen PFC tissue (region BA10) was sectioned on a cryostat microtome (Leica CM3050 S) at 10µm. RNAscope was performed using RNAscope Multiplex Fluorescent Reagent Kit v2 with TSA Vivid Dyes (ACD Bio 323270) according to manufacturer’s instructions with probes targeting human ACLY (Cat. No. 460391-C2), SCP2 (Cat. No. 875961), COX7A2 (Cat. No. 1288461-C2), or TLK2 (Cat. No. 1288451-C2). All hybridizations were performed with an additional probe to label vGlut1 positive cells, human SLC17A7 (Cat. No. 415611-C3). Images were acquired using a Zeiss LSM 880 confocal microscope at 40X for quantification. Images were analyzed blinded to genotype with a custom ImageJ macros. In brief, the macros first identified ROIs based on DAPI that were positive for SLC17A7 signal, and within those ROIs, performed a particle analysis to count RNAscope probe dots per cell. Results were reported as dot/ROI and as H-score (defined by ACD Bio analysis guidelines). 3-4 images were acquired and scored per individual, n = 4 individuals per genotype.

% \paragraph{Lipidomics of post-mortem human brain tissue.}
% A biphasic extraction protocol was used to isolate a lipid fraction from frozen post-mortem prefrontal cortex (~100mg). Briefly, weighed tissue was homogenized using Bio-vortexer Homogenizer (Daigger) in 1 mL cold methanol (Sigma MX0486) and transferred to glass vials (VWR 66011-550) with an additional 1mL of cold methanol. Chloroform (Sigma 1.02444) (4 mL; cold) was added to each vial, and mixed by vortexing for 1 min. Water (Sigma WX0001) (2 mL; cold) was added to each vial, and mixed by vortexing for 1 min. Vials were placed in 50 mL conical tubes and centrifuged for 10 min at 3000 rcf for phase separation. The lower, chloroform phase was collected (3 mL from each sample) and transferred to new vials. A blank treated the same as samples was included throughout the biphasic extraction for lipidomic analysis. Lipidomics and analysis was performed in collaboration with the Harvard Center for Mass Spectrometry (HCMS). 

% For lipidomics, samples were dried under nitrogen flow and resuspended in 60 µL of chloroform. Each sample was split into two equal aliquots, one for each polarity analysis. LC–MS analyses were modified from \cite{Miraldi2013-ng} and were performed on an Orbitrap Exactive plus MS (Thermo Scientific) in line with an Ultimate 3000 LC (Thermo Scientific). Each sample was analyzed in positive and negative modes, in top 5 automatic data-dependent MS/MS mode. Column hardware consisted of a Biobond C4 column (4.6 × 50 mm, 5 μm, Dikma Technologies). Flow rate was set to 100 µL min−1 for 5 min with 0\% mobile phase B (MB), then switched to 400 µL min−1 for 50 min, with a linear gradient of MB from 20\% to 100\%. The column was then washed at 500 µL min−1 for 8 min at 100\% MB before being re-equilibrated for 7 min at 0\% MB and 500 µL min−1. For positive mode runs, buffers consisted for mobile phase A (MA) of 5mM ammonium formate, 0.1 \% formic acid and 5\% methanol in water, and for MB of 5 mM ammonium formate, 0.1\% formic acid, 5\% water, 35\% methanol in isopropanol. For negative runs, buffers consisted for MA of 0.03\% ammonium hydroxide, 5\% methanol in water, and for MB of 0.03\% ammonium hydroxide, 5\% water, 35\% methanol in isopropanol. Lipids were identified and their signal integrated using the Lipidsearch © software (version 4.2.27, Mitsui Knowledge Industry, University of Tokyo). Integrations and peak quality were curated manually before exporting. 
