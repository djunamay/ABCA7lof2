\begin{figure}[ht]
    \begin{subfigure}[t]{.3\textwidth}
        \caption{}
        \includegraphics[width=\textwidth]{./extended_plots/rna_correlation_both_lof_lines.png}        
    \end{subfigure}
    \begin{subfigure}[t]{.3\textwidth}
        \caption{}
        \includegraphics[width=\textwidth]{./extended_plots/pm_e3_y622_jaccard.png}        
    \end{subfigure}
    \begin{subfigure}[t]{.3\textwidth}
        \caption{}
        \includegraphics[width=\textwidth]{./extended_plots/rna_correlation_miocarta_both_lines.png}        
    \end{subfigure}
    \caption{
         \textbf{Lipid and Mitochondrial Effects of Treatment with CDP-choline.}\\[1ex]
         (A) Per-cell correlation of average PLIN2 and LipidSpot fluorescent intensities shown as a density plot. 
         (B) Per-batch LipidSpot fluorescence intensities (related to Figure~\ref{fig:main_choline}A) in ABCA7 p.Tyr622* iNs treated with CDP-choline or H20 vehicle control. X-axis indicates z-scaled log-fluorescence intensity. 
         (C) Example oxygen consumption rate (OCR) curves used for analysis in Figure~\ref{fig:main_choline}. The line plot indicates the per-condition mean estimator, and the error bars indicate the 95\% confidence interval. 
         (D) Representative per-well traces from (C). 
         (E) Quantification of SRC from curves in (D). P-values computed by independent sample t-test. $N$ wells = 6 (p.Tyr622* + H20), 8 (p.Tyr622* + CDP-choline). Boxes indicate per-condition dataset quartiles, and whiskers extend to the most extreme data points not considered outliers (i.e., within 1.5 times the interquartile range from the first or third quartile). 
     }
     \label{fig:rnaseq_both_lof_lines}
 \end{figure}