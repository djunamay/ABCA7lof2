\renewcommand{\thefigure}{S\arabic{figure}}
\setcounter{figure}{0}

\clearpage

% Cohort metadata
\begin{figure}[H]
    \begin{subfigure}[t]{.5\textwidth}
        \caption{}
        \includegraphics[width=\textwidth]{./extended_plots/sanger_seq.png}        
    \end{subfigure}  
    \begin{subfigure}[t]{.5\textwidth}
        \begin{subfigure}[t]{\textwidth}
            \caption{}
            \includegraphics[width=\textwidth]{./extended_plots/sample_swap.png}        
        \end{subfigure}  
        \begin{subfigure}[t]{\textwidth}
            \caption{}
            \includegraphics[width=\textwidth]{./extended_plots/protein_levels_extended.pdf}        
        \end{subfigure}  
    \end{subfigure}  
    \begin{subfigure}[t]{\textwidth}
        \caption{}
        \includegraphics[width=\textwidth]{./extended_plots/batch_cont_var.png}        
    \end{subfigure}  
    \begin{subfigure}[t]{\textwidth}
        \caption{}
        \includegraphics[width=\textwidth]{./extended_plots/batch_categorical_vars.png}        
    \end{subfigure}  
    \caption{
        \textbf{Overview of Human snRNA-Sequencing Cohort.}\\
    }
    \label{fig:snRNA_cohort}
\end{figure}
\begin{itemize}
    \item[\textbf{(A)}] Sanger sequencing of ABCA7 LoF variants in prefrontal cortex genomic DNA samples from 3 ABCA7 LoF carriers and 3 controls from the snRNA-seq cohort. Sequencing confirmed heterozygosity of the indicated variant in LoF samples, with variant location marked by a black box. 
    \item[\textbf{(B)}] Example plots validating matches between whole genome sequencing (WGS) and snRNA-seq libraries. Each plot shows the concordance of homo- and heterozygous SNP calls between WGS and snRNA-seq data for a single individual. Matches between WGS SNP calls and snRNA-seq BAM inferred SNP calls are indicated by extreme outliers. Expected (i.e., correct) matches are indicated in blue/purple. 
    \item[\textbf{(C)}] Protein levels from post-mortem human prefrontal cortex (see Table~\ref{tab:external_datasets} for external dataset used) showing ABCA7 protein levels (left) and NeuN (RBFOX3) levels (middle) for a subset of individuals in the snRNA-seq cohort (N=6 control and N=4 ABCA7 LoF carriers). The right panel shows NeuN (RBFOX3) protein levels by genotype in all available control samples (N=180) vs. ABCA7 LoF carriers (N=5). 
    \item[\textbf{(D)}] Distributions of continuous metadata variables (see Supplementary Text for descriptions) for control individuals (N=24) vs. ABCA7 LoF carriers (N=12). For panels C and D, boxes indicate dataset quartiles per condition, and whiskers extend to the most extreme data points not considered outliers (i.e., within 1.5 times the interquartile range from the first or third quartile). 
    \item[\textbf{(E)}] Distributions of discrete metadata variables for control individuals (N=24) vs. ABCA7 LoF carriers (N=12). Con=control, LoF=ABCA7 loss-of-function. P-values in panels C and D were computed by two-sided Wilcoxon rank sum test. P-values in panel E were computed by two-sided Fisher’s exact test.
\end{itemize}
\clearpage
% Batch quality
\begin{figure}[H]
    \begin{subfigure}[t]{\textwidth}
        \caption{}
        \includegraphics[width=\textwidth]{../paper/extended_plots/seq_batch_cont.png}        
    \end{subfigure}
    \begin{subfigure}[t]{\textwidth}
        \caption{}
        \includegraphics[width=\textwidth]{../paper/extended_plots/seq_batch_cat.png}        
    \end{subfigure}  
    \begin{subfigure}[t]{\textwidth}
        \caption{}
        \includegraphics[width=\textwidth]{../paper/extended_plots/additional_projections.png}        
    \end{subfigure}   
    \begin{subfigure}[t]{\textwidth}
        \caption{}
        \includegraphics[width=\textwidth]{../paper/extended_plots/score_correlations_by_batch.png}        
    \end{subfigure}   
\end{figure}
\textbf{Fig. S2: Overview of snRNA-sequencing Batch Correction and Data Quality.}
\textbf{a,} Distribution of continuous metadata variables by sequencing batch. P-values in panels were computed by two-sided Wilcoxon rank sum test.
\textbf{b,} Distribution of discrete metadata variables by sequencing batch. P-values were computed by two-sided Fisher's exact test.
\textbf{c,} 2D UMAP projection of snRNA-seq cells after quality control, colored by selected metadata variables.
\textbf{d,} Correlation of gene perturbation scores ($S = -\log_{10}(p)\times\text{sign}(\log_2(\text{fold-change}))$) computed using all samples versus excluding batch 2 (v2 chemistry), demonstrating that results are robust and not driven by batch-specific effects.

\clearpage
% Annotation
\begin{figure}[H]
    \begin{subfigure}[t]{.5\textwidth}
        \begin{subfigure}[t]{.45\textwidth}
            \caption{}
            \includegraphics[width=\textwidth]{./extended_plots/cell_proj_with_leiden.png}        
        \end{subfigure}
        \hspace{0.5cm}  
        \begin{subfigure}[t]{.45\textwidth}
            \caption{}
            \includegraphics[width=\textwidth]{./extended_plots/leiden_heatmap.png}        
        \end{subfigure}   
        \begin{subfigure}[t]{.3\textwidth}
            \caption{}
            \includegraphics[width=\textwidth]{./extended_plots/hierarchical_tree.png}        
        \end{subfigure}  
        \hspace{1cm}
        \begin{subfigure}[t]{.45\textwidth}
            \caption{}
            \includegraphics[width=\textwidth]{./extended_plots/marker_hmap.png}        
        \end{subfigure}  
    \end{subfigure}
    \hspace{2cm}  
    \begin{subfigure}[t]{.25\textwidth}
        \caption{}
        \includegraphics[width=\textwidth]{./extended_plots/marker_boxplot.png}        
    \end{subfigure}    
    \par
    \begin{subfigure}[t]{.15\textwidth}
        \caption{}
        \includegraphics[width=\textwidth]{./extended_plots/median_cells.png}        
    \end{subfigure} 
    \hspace{0.5cm}   
    \begin{subfigure}[t]{.25\textwidth}
        \caption{}
        \includegraphics[width=\textwidth]{./extended_plots/individual_fractions.png}        
    \end{subfigure}  
    \hspace{0.5cm}   
    \begin{subfigure}[t]{.25\textwidth}
        \caption{}
        \includegraphics[width=\textwidth]{./extended_plots/celltype_heatmap.png}        
    \end{subfigure}  
    \hspace{0.5cm}
    \begin{subfigure}[t]{.15\textwidth}
        \caption{}
        \includegraphics[width=\textwidth]{./extended_plots/individual_correlations.png}        
    \end{subfigure}       
    \caption{
        \textbf{Overview of snRNA-sequencing Cell Type Annotations.}\\
    }
    \label{fig:snRNA_quality_annotation}
\end{figure}
\begin{itemize}
    \item[\textbf{(A)}] Two-dimensional UMAP projections of individual cells from gene expression space, colored by Leiden clusters. 
    \item[\textbf{(B)}] Average marker gene expression (per-cluster mean log(fold-change)) for all marker genes for the cell type indicated along the x-axis. Log(fold-changes) are computed for the cluster of interest vs. all remaining clusters. Reference 1 (Table~\ref{tab:external_datasets}) marker genes were used. 
    \item[\textbf{(C)}] Cladogram visualizing subcluster relationships based on pairwise distances between per-cluster gene expression profiles. 
    \item[\textbf{(D)}] Average marker gene expression profiles (x-axis) per major cell type annotation (y-axis) for two marker gene references (Table~\ref{tab:external_datasets}). 
    \item[\textbf{(E)}] Per-cell distribution of select marker gene expression by cell type. Y-axis indicates log-counts. 
    \item[\textbf{(F)}] Median number of cells per cell type per individual. 
    \item[\textbf{(G)}] Cell type fraction by individual. 
    \item[\textbf{(H)}] Heatmap of individual-level gene expression correlations by cell type. 
    \item[\textbf{(I)}] Boxplot of individual-level gene expression correlations by cell type. 
\end{itemize}
\clearpage
% Gene scores
\begin{figure}[H]
    \begin{subfigure}[t]{1\textwidth}
        \caption{}
        \centering
        \includegraphics[width=0.7\textwidth]{../paper/extended_plots/umap_projection_more_genes.png}        
    \end{subfigure}    
\end{figure}
\textbf{Extended Data Fig. 2}
\clearpage
% ABCA7 expression
\begin{figure}[H]
    \begin{subfigure}[t]{.25\textwidth}
        \caption{}
        \includegraphics[width=\textwidth]{../paper/extended_plots/abca7_detection_rate.png}        
    \end{subfigure}
    \par
    \begin{subfigure}[t]{\textwidth}
        \caption{}
        \includegraphics[width=\textwidth]{../paper/extended_plots/scRNAseq_bulk_rna.pdf}        
    \end{subfigure}
    \par
    \begin{subfigure}[t]{\textwidth}
        \caption{}
        \includegraphics[width=\textwidth]{../paper/extended_plots/welch_et_al_bulk_rna.pdf}        
    \end{subfigure}
\end{figure}
\textbf{Extended Data Fig. 4}
\clearpage
% Benchmarking clustering
\begin{figure}[ht]
    \begin{subfigure}[t]{0.5\textwidth}
        \caption{}
        \includegraphics[width=\textwidth]{./extended_plots/jaccard_mat_sub.png}        
    \end{subfigure}
    \begin{subfigure}[t]{0.5\textwidth}
        \caption{}
        \includegraphics[width=\textwidth]{./extended_plots/partitioning_losses.pdf}        
    \end{subfigure}
    \begin{subfigure}[t]{1\textwidth}
        \caption{}
        \includegraphics[width=\textwidth]{./extended_plots/adjacency_matrices_ordered_by_cluster_labels.pdf}        
    \end{subfigure}
    \begin{subfigure}[t]{0.33\textwidth}
        \caption{}
        \includegraphics[width=\textwidth]{./extended_plots/argmins_jaccard.png}        
    \end{subfigure}
    \begin{subfigure}[t]{0.33\textwidth}
        \caption{}
        \includegraphics[width=\textwidth]{./extended_plots/random_jaccard.png}        
    \end{subfigure}
    \begin{subfigure}[t]{0.33\textwidth}
        \caption{}
        \includegraphics[width=\textwidth]{./extended_plots/rand_indices.pdf}        
    \end{subfigure}
    \caption{
        \textbf{Benchmarking Partitioning and Clustering Algorithms for Gene-Pathway Grouping.}\\[1ex]
        (A) Jaccard indices quantifying overlap of genes for all 111 pathways in Figure~\ref{fig:main_excitatory}B (see Methods; Supplementary Text). 
        (B) Average loss (total cut size; see Methods) associated with applying each algorithm (spectral clustering (SC), METIS, Kernighan-Lin (K/L), spectral bisection (SB), or random permutation) to G (with 379 vertices; see Methods) over 1000 initiations (SC, random permutation) or $5 \times 10^5$ initiations (METIS, K/L). The SB implementation is deterministic and was run only once. Error bars indicate the standard deviation. 
        (C) Unweighted adjacency matrix for G sorted by labels assigned by the indicated algorithm. Red indicates the presence of an edge between two vertices. For each algorithm, labels corresponding to the best initiation (lowest loss) over 1000 initiations (SC, random permutation) or $5 \times 10^5$ initiations (METIS, K/L) are shown. 
        (D) Pairwise labeling consistency for the best K/L initiation and the best METIS initiation. Cluster labels corresponding to each are shown on the X- and Y-axes, respectively. Each color entry indicates the fraction of shared vertices per cluster across two initiations. Consistency is quantified using the Jaccard Index (JI). $\text{JI} = \frac{|A \cap B|}{|A \cup B|}$, where A and B are two sets (i.e., cluster A from initiation \#1 and cluster B from initiation \#2). 
        (E) Same as (D), but comparing the best K/L initiation against the best random permutation initiation. 
        (F) Average Rand index (RI) for all pairwise initiations from (B). “METIS,” “Kernighan-Lin,” and “Permuted” labels on the Y-axis indicate the average RI (consistency across two sets of labels) for all combinations of initiations within the specified algorithm. “METIS-K/L” indicates the average RI for all combinations of initiations across the METIS and Kernighan-Lin algorithms. Error bars indicate standard deviations. ($\text{RI} = \frac{\text{number of agreeing vertex pairs}}{\text{number of vertex pairs}}$).
    }
    \label{fig:benchmarking_clustering}
\end{figure}
\clearpage
% MD simulations
\begin{figure}[ht]
    \begin{subfigure}[t]{.6\textwidth}
        \caption{}
        \includegraphics[width=\textwidth]{./extended_plots/abca7_structure_with_inset.png}        
    \end{subfigure}
    \hspace{1cm}
    \begin{subfigure}[t]{.3\textwidth}
        \caption{}
        \includegraphics[width=\textwidth]{./extended_plots/abca7_structure_inset_only.png}        
    \end{subfigure}
    \par
    \begin{subfigure}[t]{.6\textwidth}
        \caption{}
        \includegraphics[width=\textwidth]{./extended_plots/rmsd_time.png}        
    \end{subfigure}
    \hspace{1cm}
    \begin{subfigure}[t]{.3\textwidth}
        \caption{}
        \includegraphics[width=\textwidth]{./extended_plots/rmsd_projection_open.png}        
    \end{subfigure}
    \par
    \begin{subfigure}[t]{.6\textwidth}
        \caption{}
        \includegraphics[width=\textwidth]{./extended_plots/rmsd_volcano.png}        
    \end{subfigure}
    \caption{
        \textbf{Molecular Dynamics Simulations of ABCA7 open conformations with p.Ala1527Gly substitution.}\\[1ex]
        (A) Open conformation ABCA7 protein structure. ABCA7 domain between residues 1517 and 1756 used for simulations is shown in yellow. Expanded yellow domain (inset from left), with A1527 variant (light grey) and G1527 variant (purple). 
        (B) Expanded inset from A with residues of interest rendered. 
        (C) Root mean squared deviations of open conformation domains from B with A1527 (light grey) or G1527 (purple) under simulation. Structural deviations over time were computed with respect to reference open structures from B. 
        (D) Projection of $C_\alpha$ atom positional fluctuations under simulation onto the first two principal components, for open conformation domain from B with A1527 (top, light grey) or G1527 (bottom, purple). 
        (E) Violin plot indicating average $C_\alpha$ atom positional fluctuations over time. 
    }
    \label{fig:md_simulations}
\end{figure}
\clearpage
% Differentiating iPSC neurons
\begin{figure}[H]
    \begin{subfigure}[t]{0.4\textwidth}
        \caption{}
        \hspace{1.5cm}
        \includegraphics[width=\textwidth]{../paper/extended_plots/tyr622_cartoon.png}        
    \end{subfigure}  
    \hspace{1.5cm}
    \begin{subfigure}[t]{0.4\textwidth}
        \caption{}
        \includegraphics[width=\textwidth]{../paper/extended_plots/glu50fs3_cartoon.png}        
    \end{subfigure}  
    \par
    \begin{subfigure}[t]{0.9\textwidth}
        \caption{}
        \hspace{1.5cm}
        \includegraphics[width=\textwidth]{../paper/extended_plots/karyotypes.png}        
    \end{subfigure}  
\end{figure}
\textbf{Fig. S4: Generation of iPSC-Derived Cells Harboring ABCA7 PTC Variants.}
\textbf{a,} Sanger sequencing chromatogram confirming single nucleotide insertion in ABCA7 exon 3 of the ABCA7 p.Glu50fs*3 isogenic iPSC line.
\textbf{b,} Sanger sequencing chromatogram confirming patient single nucleotide polymorphism in ABCA7 exon 15 of the ABCA7 p.Tyr622* isogenic iPSC line. 
\textbf{c,} Normal karyotypes were observed for WT, ABCA7 p.Glu50fs*3, and ABCA7 p.Tyr622* isogenic iPSC lines. 

\clearpage
% bulk RNAseq supplement
\begin{figure}[H]
    \begin{subfigure}[t]{.3\textwidth}
        \caption{}
        \includegraphics[width=\textwidth]{./main_plots/g2_kl_clusters_network.pdf}        
    \end{subfigure}
    \begin{subfigure}[t]{.3\textwidth}
        \caption{}
        \includegraphics[width=\textwidth]{./extended_plots/jaccard_pT622_pG50fs3_vs_wt.png}        
    \end{subfigure}
    \begin{subfigure}[t]{.3\textwidth}
        \caption{}
        \includegraphics[width=\textwidth]{./extended_plots/g2_pm_jaccard.png}        
    \end{subfigure}
    \begin{subfigure}[t]{.6\textwidth}
        \caption{}
        \includegraphics[width=\textwidth]{./main_plots/kl_densities_g.png}        
    \end{subfigure}
    \hspace{0.5cm}
    \begin{subfigure}[t]{.3\textwidth}
        \caption{}
        \includegraphics[width=\textwidth]{./extended_plots/rna_correlation_miocarta_both_lines.png}        
    \end{subfigure}
    \caption{
        \textbf{mRNA-seq analysis of p.Glu50fs* vs. WT iNs.}\\
    }
    \label{fig:bulk_RNAseq_supplement}
\end{figure}
\begin{itemize}
    \item[\textbf{(A)}] Kernighan-Lin (K/L) clustering of leading-edge genes from significantly perturbed pathways (Benjamini–Hochberg (BH) FDR-adjusted $p<0.05$) in p.Glu50fs*3 vs. WT iNs. Colors represent distinct K/L clusters.
    \item[\textbf{(B)}] Heatmap of Jaccard index overlap between K/L gene clusters from p.Glu50fs*3 neurons and clusters identified in p.Tyr622* vs WT iNs. Red text denotes clusters with average score $S$ upregulated in ABCA7 LoF; blue text denotes clusters with average $S$ downregulated in ABCA7 LoF.
    \item[\textbf{(B)}] Heatmap showing Jaccard index overlap between K/L clusters identified in p.Glu50fs* vs. WT iNs and p.Tyr622* vs. WT iNs.
    \item[\textbf{(C)}] Heatmap of Jaccard index overlap between K/L gene clusters from p.Glu50fs*3 neurons and clusters identified in human postmortem excitatory neurons.
    \item[\textbf{(D)}] Gaussian kernel density plots of gene perturbation scores ($S$) within each cluster. Positive $S$ indicates upregulation in p.Glu50fs*3. Solid lines denote cluster means. Top enriched pathways with highest intra-cluster connectivity indicated.
    \item[\textbf{(E)}] Correlation of per-gene perturbation scores ($S = -\log_{10}(\text{p-value}) \times \text{sign}(\log_2(\text{fold change}))$) between p.Glu50fs*3 vs. WT and p.Tyr622* vs. WT iNs for MitoCarta genes.
\end{itemize}
\clearpage
% LCMS supplement
\begin{figure}[ht]
    \begin{subfigure}[t]{0.5\textwidth}
        \caption{}
        \includegraphics[width=\textwidth]{./extended_plots/lcms_corr_heatmap.png}        
    \end{subfigure}  
    \begin{subfigure}[t]{0.5\textwidth}
        \caption{}
        \includegraphics[width=\textwidth]{./extended_plots/lcms_corr_scatterplot.png}        
    \end{subfigure}  
    \begin{subfigure}[t]{0.5\textwidth}
        \caption{}
        \includegraphics[width=\textwidth]{./extended_plots/lcms_corr_scatterplot_metab.png}        
    \end{subfigure}  
    \begin{subfigure}[t]{0.5\textwidth}
        \caption{}
        \includegraphics[width=\textwidth]{./extended_plots/beta_ox_genes_pm.png}        
    \end{subfigure}  
    \caption{
        \textbf{LCMS supplement in iN.}\\[1ex]
        (A) Lipid synthesis and storage pathways perturbed in ABCA7 LoF excitatory neurons vs. control as measured by snRNA-seq on the post-mortem human PFC. Enrichments of biological processes were computed using FGSEA. Red = enrichment > 0, Blue = enrichment < 0. * = $p<0.05$. 
        (B) Schematic model showing anabolic processes feeding from the TCA cycle towards fatty acid (FA) and triglyceride (TG) synthesis. DG = diacylglyceride, PA = phosphatidic acid, PC = phosphatidylcholine, PE = phosphatidylethanolamine, PS = phosphatidylserine. * = differentially expressed in ABCA7 LoF vs. control excitatory neurons from post-mortem human brain at $p<0.05$ and $\log\text{FC}<0$. 
        (C) 𝛽-oxidation and TCA pathways perturbed in ABCA7 LoF excitatory neurons vs. control as measured by snRNA-seq on the post-mortem human PFC. Enrichments of biological processes were computed using FGSEA. Red = enrichment > 0, Blue = enrichment < 0. * = $p<0.05$. 
        (D) Schematic model showing catabolic processes feeding into the TCA cycle and oxidative phosphorylation with key genes from (C) highlighted in red or blue. * = $p<0.05$. For (A, C,) boxes indicate per-condition dataset quartiles, and whiskers extend to the most extreme data points not considered outliers (i.e., within 1.5 times the interquartile range from the first or third quartile). 
        (E, F) Transcript levels of ACLY (E) and SCP2 (F) assessed in post-mortem human PFC by RNAscope. Transcript counts per SLC17A7+ cell are reported in each bar chart. $N = 8$ individuals per genotype. Per-cell Wilcoxon rank-sum p-values are reported.
    }
    \label{fig:lipid_mitochondrial_perturbations}
\end{figure}

\clearpage
% mitochondrial o2 consumption
\begin{figure}[H]
    \begin{subfigure}[t]{0.33\textwidth}
        \caption{}
        \includegraphics[width=\textwidth]{./extended_plots/rep_seahorse_curves_by_line.png}        
    \end{subfigure}
    \begin{subfigure}[t]{0.33\textwidth}
        \caption{}
        \includegraphics[width=\textwidth]{./extended_plots/rep_seahorse_curves_all.png}        
    \end{subfigure}   
    \begin{subfigure}[t]{0.33\textwidth}
        \caption{}
        \includegraphics[width=\textwidth]{./main_plots/uncoupling_cartoon.png}        
    \end{subfigure}  
    \begin{subfigure}[t]{0.33\textwidth}
        \caption{}
        \includegraphics[width=\textwidth]{./extended_plots/src_cartoon.png}        
    \end{subfigure} 
    \begin{subfigure}[t]{0.33\textwidth}
        \caption{}
        \includegraphics[width=\textwidth]{./main_plots/seahorse_cartoon.png}        
    \end{subfigure}  
    \begin{subfigure}[t]{0.25\textwidth}
        \caption{}
        \includegraphics[width=\textwidth]{./extended_plots/SRC.png}        
    \end{subfigure} 
    \begin{subfigure}[t]{0.25\textwidth}
        \caption{}
        \includegraphics[width=\textwidth]{./extended_plots/uncoupling_quantification_batch1.png}        
    \end{subfigure} 
    \hspace{0.1\textwidth}
    \begin{subfigure}[t]{0.25\textwidth}
        \caption{}
        \includegraphics[width=\textwidth]{./extended_plots/uncoupling_quantification_batch2.png}        
    \end{subfigure} 
    \begin{subfigure}[t]{0.25\textwidth}
        \caption{}
        \includegraphics[width=\textwidth]{./extended_plots/UCP_levels.png}        
    \end{subfigure} 
    \caption{
         \textbf{Analysis of Oxygen Consumption Rates in ABCA7 LoF vs. Control iNs.}\\
     }
     \label{fig:oxygen_consumption_rates_iPSC_neurons}
 \end{figure}
 \begin{itemize}
    \item[\textbf{(A)}] Example oxygen consumption rate (OCR) curves from Batch 1 of the two differentiation batches used for analysis in Figure~\ref{fig:main_mitochondrial}G. The line plot indicates the per-condition mean estimator, and the error bars indicate the 95\% confidence interval. 
    \item[\textbf{(B)}] Representative per-well traces from (A). 
    \item[\textbf{(C)}] Schematic indicating the relationship between oxygen consumption as a measure of proton current (I), which sustains the proton motive force (voltage, V). Regulation of ATP synthase and uncoupling protein (UCP) activity modifies resistance (R) and depletes the proton motive force.
    \item[\textbf{(D)}] Schematic indicating measurement of maximal and basal oxygen consumption to compute SRC.
    \item[\textbf{(E)}] Schematic indicating measurement of uncoupled oxygen consumption.
    \item[\textbf{(F)}] SRC computed for WT, ABCA7 p.Glu50fs*3, and ABCA7 p.Tyr622* iNs. P-values computed by independent sample t-test. $N$ wells = 18 (WT), 17 (p.Tyr622*), 13 (p.Glu50fs*3) across two independent differentiation batches and Seahorse experiments. 
    \item[\textbf{(G, H)}] Relative uncoupling measured for two independent iN differentiation batches and separate Seahorse experiments shown combined in Figure~\ref{fig:main_mitochondrial}G. P-values computed by independent sample t-test. Batch 1 (left); $N$ wells = 10 (WT), 7 (p.Tyr622*), 7 (p.Glu50fs*3). Batch 2 (right); $N$ wells = 8 (WT), 10 (p.Tyr622*), 6 (p.Glu50fs*3) shown per differentiation batch.
    \item[\textbf{(I)}] UCP2 mRNA levels. 
 \end{itemize}
\clearpage
% mitochondrial mmp
% \begin{figure}[ht]
    \begin{subfigure}[t]{0.3\textwidth}
        \caption{}
        \includegraphics[width=\textwidth]{./extended_plots/tmrm_quantification_with_FCCP.png}        
    \end{subfigure}
    \begin{subfigure}[t]{0.7\textwidth}
        \caption{}
        \includegraphics[width=\textwidth]{./extended_plots/tmrm_images_with_FCCP.png}        
    \end{subfigure}
    \begin{subfigure}[t]{0.7\textwidth}
        \caption{}
        \includegraphics[width=\textwidth]{./extended_plots/mitohealth_dye.png}        
    \end{subfigure}
    \par
    \begin{subfigure}[t]{\textwidth}
        \caption{}
        \includegraphics[width=\textwidth]{./extended_plots/mitohealth_per_cell.png}        
    \end{subfigure}
    \caption{
         \textbf{Mitochondrial MMP in iN.}\\[1ex]
         (A) Example oxygen consumption rate (OCR) curves from Batch 1 of the two differentiation batches used for analysis in Figure~\ref{fig:main_mitochondrial}G. The line plot indicates the per-condition mean estimator, and the error bars indicate the 95\% confidence interval. 
         (B) Representative per-well traces from (A). 
         (C) Schematic indicating measurement of maximal and basal oxygen consumption to compute SRC, as shown in 
         (D) for WT, ABCA7 p.Glu50fs*3, and ABCA7 p.Tyr622* iNs. P-values computed by independent sample t-test. $N$ wells = 18 (WT), 17 (p.Tyr622*), 13 (p.Glu50fs*3) across two independent differentiation batches and Seahorse experiments. 
         (E) Relative uncoupling measured for two independent iN differentiation batches and separate Seahorse experiments shown combined in Figure~\ref{fig:main_mitochondrial}G. P-values computed by independent sample t-test. Batch 1 (left); $N$ wells = 10 (WT), 7 (p.Tyr622*), 7 (p.Glu50fs*3). Batch 2 (right); $N$ wells = 8 (WT), 10 (p.Tyr622*), 6 (p.Glu50fs*3) shown per differentiation batch. For (D, E) boxes indicate per-condition dataset quartiles, and whiskers extend to the most extreme data points not considered outliers (i.e., within 1.5 times the interquartile range from the first or third quartile). 
         (F) Per-batch cell-level MioHealth fluorescence intensities (related to Figure~\ref{fig:main_mitochondrial}H).
     }
     \label{fig:oxygen_consumption_rates_iPSC_neurons}
 \end{figure}
% \clearpage
% choline treatment
\begin{figure}[H]
    %ROW 1
    \begin{subfigure}[t]{.7\textwidth}
        \caption{}
        \includegraphics[width=\textwidth]{./extended_plots/choline_media_lcms.png}        
    \end{subfigure}
    \begin{subfigure}[t]{.3\textwidth}
        \caption{}
        \includegraphics[width=\textwidth]{./extended_plots/choline_in_cells_lcms.png}        
    \end{subfigure}
    \begin{subfigure}[t]{0.25\textwidth}
        \caption{}
        \includegraphics[width=\textwidth]{./extended_plots/choline_synth_genes.png}        
    \end{subfigure}
    \begin{subfigure}[t]{0.25\textwidth}
        \caption{}
        \includegraphics[width=\textwidth]{./extended_plots/choline_lpcat.png}        
    \end{subfigure}
    \begin{subfigure}[t]{.35\textwidth}
        \caption{}
        \includegraphics[width=\textwidth]{./main_plots/pc_unsat_with_choline_batch1.png}        
    \end{subfigure} 
    \begin{subfigure}[t]{0.2\textwidth}
        \caption{}
        \includegraphics[width=\textwidth]{./main_plots/pca_rna_batch1.png}        
    \end{subfigure}  
    \hspace{.25cm}
    \begin{subfigure}[t]{0.2\textwidth}
        \caption{}
        \includegraphics[width=\textwidth]{./extended_plots/rna_correlation_miocarta_choline.png}        
    \end{subfigure}
    \hspace{.25cm}
    \begin{subfigure}[t]{.25\textwidth}
        \caption{}
        \includegraphics[width=\textwidth]{./extended_plots/ocr_choline_curves_by_treatment.png}        
    \end{subfigure}
    \begin{subfigure}[t]{.25\textwidth}
        \caption{}
        \includegraphics[width=\textwidth]{./extended_plots/ocr_choline_rep_curves.png}        
    \end{subfigure}
    \begin{subfigure}[t]{.22\textwidth}
        \caption{}
        \includegraphics[width=\textwidth]{./extended_plots/src_choline_quantification.png}        
    \end{subfigure}
    \hspace{.25cm}
    \begin{subfigure}[t]{.45\textwidth}
        \caption{}
        \includegraphics[width=\textwidth]{./extended_plots/mitohealth_treatment.png}        
    \end{subfigure}
    \hspace{.25cm}
    \begin{subfigure}[t]{.25\textwidth}
        \caption{}
        \vspace{.3cm}
        \includegraphics[width=\textwidth]{./main_plots/pca_plot_y622_choline_metab.png}        
    \end{subfigure} 
    \caption{
         \textbf{Effects of CDP-choline treatment in p.Tyr622* iNs.}\\
     }
     \label{fig:choline_treatment}
\end{figure}
\begin{itemize}
    \item[\textbf{(A)}] Choline metabolites detected in the media by targeted LCMS.
    \item[\textbf{(B)}] Choline metabolite detected in the cells by targeted LCMS.
    \item[\textbf{(C)}] Select choline synthesis and transport differentially expressed in p.Tyr622* +/- CDP-choline iNs.
    \item[\textbf{(D)}] LPCAT gene changes in in p.Tyr622* +/- CDP-choline iNs.
    \item[\textbf{(E)}] Fold-changes distribution of phosphatidylcholine species in p.Tyr622* +/- CDP-choline iNs by chain length and saturation.
    \item[\textbf{(F)}] Correlation of differentially expressed genes in p.Tyr622* +/- CDP-choline iNs and p.Tyr622* vs WT iNs.
    \item[\textbf{(G)}] Correlation of differentially expressed genes in p.Tyr622* +/- CDP-choline iNs and p.Glu50fs*3 vs WT iNs - mitochondrial genes only.
    \item[\textbf{(H)}] Example oxygen consumption rate (OCR) curves used for analysis in Fig. 5. The line plot indicates the per-condition mean estimator, and the error bars indicate the 95\% confidence interval. 
    \item[\textbf{(I)}] Representative per-well traces from (H).
    \item[\textbf{(J)}] Quantification of SRC from curves in (I). P-values computed by independent sample t-test.  wells = 6 (p.Tyr622* + H20), 8 (p.Tyr622* + CDP-choline). Boxes indicate per-condition dataset quartiles, and whiskers extend to the most extreme data points not considered outliers (i.e., within 1.5 times the interquartile range from the first or third quartile). 
    \item[\textbf{(K)}] Per-batch cell-level MioHealth fluorescence intensities in ABCA7 p.Tyr622* iNs treated with CDP-choline or H20 vehicle control. X-axis indicates z-scaled fluorescence intensity.
\end{itemize}
\clearpage
% neurospheroid figure
\begin{figure}[H]
    \begin{subfigure}[t]{0.7\textwidth}
        \caption{}
        \includegraphics[width=\textwidth]{./extended_plots/neurospheroid_markers.png}        
    \end{subfigure}
    \par
    \begin{subfigure}[t]{0.5\textwidth}
        \caption{}
        \includegraphics[width=\textwidth]{./main_plots/abeta_elisa_iN.png}        
    \end{subfigure}     
    \par
    \begin{subfigure}[t]{0.7\textwidth}
        \caption{}
        \includegraphics[width=\textwidth]{./main_plots/abeta_elisa_3weeks.png}        
    \end{subfigure}
    % \begin{subfigure}[t]{0.4\textwidth}
    %     \caption{}
    %     \includegraphics[width=\textwidth]{./extended_plots/calcium_traces.png}        
    % \end{subfigure}
    \caption{
        \textbf{CDP-choline treatment in cortical organoids}\\
    }
    \label{fig:neurospheroid_figure}
\end{figure}
\begin{itemize}
    \item[\textbf{(A)}] Representative images of sliced cortical organoids per genotype. 
    \item[\textbf{(B)}] Amyloid-$\beta$ levels measured by ELISA from media of cortical organoids (176 days old) treated with 500 µM or 1 mM CDP-choline for 3 weeks, grouped by genotype. These samples correspond to the organoids shown in Figure \ref{fig:main_choline}K, but were analyzed one week prior to the assays presented there.\end{itemize}
\clearpage
% updated model
\input{/Users/djuna/Documents/ABCA7lof2/paper/appendix/extended_figures/model.tex}
\clearpage




% \begin{figure}[ht]
%    % \centerline{\includegraphics[width=\textwidth]{./extended_plots/quantification_Abeta42_iPSC_neurons.pdf}}
%     \caption{
%         \textbf{Quantification of Aβ42 in iPSC-Derived Neurons Harboring ABCA7 PTC Variants.}\\[1ex]
%         (A) Quantification of neuronal Aβ42 fluorescence intensity. P-values were computed by a linear mixed-effects model on per-NeuN+ volume averages, including well-of-origin as a random effect. $N = 16$ (WT; 2261 cells), $N=8$ (p.Tyr622*; 1466 cells), $N=6$ wells (p.Glu50fs*3; 999 cells) from 4-week-old iNs. Boxes indicate per-condition dataset quartiles, and whiskers extend to the most extreme data points not considered outliers (i.e., within 1.5 times the interquartile range from the first or third quartile). Individual data points represent per-well averages of cell-level intensities. 
%         (B) Representative images per condition showing mean-intensity projections of the entire image (NeuN+) and projections within NeuN+ volumes considered for quantification (Aβ42). Representative images for the Aβ42 channel were processed with condition-wide percentile-based background subtraction and thresholding. Representative images of cell soma underwent per-image percentile-based background subtraction and thresholding, reflecting the segmentation methodology.
%     }
%     \label{fig:quantification_Abeta42_iPSC_neurons}
% \end{figure}


