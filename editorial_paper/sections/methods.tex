\subsubsection{Methods}

\subsubsection{Isolation of nuclei from postmortem brain tissue.}
Batch \#1 nuclei (BA10 region, frozen tissue) were isolated following an adapted protocol from Mathys et al.\supercite{Mathys2019-wb}, performed entirely at 4°C or on ice. Briefly, tissue was homogenized (700 µl homogenization buffer: 320 mM sucrose, 5 mM CaCl2, 3 mM Mg(CH3COO)2, 10 mM Tris-HCl pH 7.8, 0.1 mM EDTA pH 8.0, 0.1\% IGEPAL CA-630, 1 mM β-mercaptoethanol, 0.4 U/µl recombinant RNase inhibitor (Clontech)) using a Wheaton Dounce tissue grinder (15 strokes, loose pestle), filtered (40 µm cell strainer), then mixed 1:1 with working solution (diluent (30mM CaCl2, 18mM Mg(CH3COO)2, 60mM Tris pH 7.8, 0.6mM EDTA, 6mM β-mercaptoethanol) and OptiPrep density gradient solution (Sigma-Aldrich D1556-250ML), 1:5). The sample was layered onto an OptiPrep density gradient consisting of 750 µl of 30\% OptiPrep (1.5:1 ratio of Working Solution:Homogenization Buffer) above 300 µl of 40\% OptiPrep (4:1 ratio of Working Solution:Homogenization Buffer), centrifuged (10,000 g, 5 min, 4°C), and nuclei collected from the 30/40\% interface (100µl). Nuclei were washed twice (1 ml PBS, 0.04\% BSA, 300 g, 3 min), resuspended (100µl PBS, 0.04\% BSA ), counted (C-Chip hemocytometer), and diluted to 1,000 nuclei/µl (PBS 0.04\% BSA).

Batch \#2 nuclei (fresh postmortem PFC BA10 tissue) were prepared as part of a previous study\supercite{Mathys2019-wb}.

Informed consent and Anatomical Gift Act consent were obtained, including repository consent to allow sharing of data and biospecimens. Rush University Medical Center IRB approved protocols (Religious Orders Study, Rush Memory and Aging Project).

\subsubsection{Droplet-based snRNA-sequencing}
Batch \#1 libraries were prepared using Chromium Single Cell 3′ Reagent Kits v3 (10x Genomics) and sequenced on a NovaSeq 6000 S2 (paired-end, 28+91 bp, 8-nt index). Each sample was sequenced twice across two lanes to increase depth. Batch \#2 libraries were prepared using Chromium Single Cell 3′ Reagent Kits v2 and sequenced with NextSeq 500/550 High Output v2 kits (150 cycles), as previously described\supercite{Mathys2019-wb}. All raw reads were processed together for alignment and gene counting.

\subsubsection{Variant calling and ROSMAP subject selection.}
We selected 36 individuals from the ROSMAP cohort, a longitudinal study of aging and dementia\supercite{Bennett2018-tn}. Whole-genome sequencing (WGS) variant calls (N=1249 available ROSMAP samples) were downloaded from Synapse (accession ID: syn11724057) for genes with rare damaging variants linked to AD: SORL1, TREM2, ABCA7, ATP8B4, ABCA1, and ADAM10\supercite{Holstege2022-vp}. For subjects with multiple WGS samples, the highest-quality sample was chosen (`Genomic Quality Score`). Samples with sex mismatches or genotype inconsistencies were excluded (see accession ID: syn12178037). Only variants passing quality control (`FILTER\_PASS`) were used.

PTC variants flagged as splice, frameshift, nonsense, missense, or premature stop variants annotated as loss-of-function (`LOF`) were identified. For ABCA7, known LoF variants from literature were captured, except c.5570+5G>C, which was manually added. Additional WGS details (library prep, QC, annotations, impact predictions) can be viewed on Synapse (accession ID: syn10901595). 

We selected 12 subjects (`LoF` samples) who carried ABCA7 PTC variants, had no PTC variants in the other candidate genes listed above, and had fresh-frozen tissue available from Rush University. Additionally, we chose 24 matched controls without any PTC variants in ABCA7 or the other listed genes. Controls were matched by age, sex, and pathology.

\subsubsection{Read counting & alignment.}
Libraries were demultiplexed using the MIT BioMicroCenter BMC/BCC 1.8 pipeline (updated 12/09/2020, \url{https://openwetware.org/wiki/BioMicroCenter:Software\#BMC-BCC_Pipeline}). Fastq reads were aligned to the human genome (GRCh38) and counted using Cell Ranger (v6.1.2; 10x Genomics) with intron counting enabled and an expected cell count of 5000 per sample. Default parameters were otherwise used. Counts from all samples were aggregated via a custom script, yielding 150,456 total cells.

\subsubsection{Sample swap analysis.}
Sample swap analysis was performed using an established pipeline (MVV; QTLtools v1.1)\supercite{Fort2017-jq}, comparing allelic concordance between genomic (VCF) and transcriptomic (BAM, generated by Cell Ranger) data. We specifically analyzed chromosome 19 variants (location of ABCA7). Each single-cell sample matched the expected WGS sample clearly, showing higher concordance (fewer mismatches) than all other ROSMAP WGS samples (examples shown in Extended Data Fig. 1e).

\subsubsection{Cell filtering metrics.}
Aggregated counts underwent stringent quality control prior to cell annotation. Cells with fewer than 500 or more than 10,000 detected genes (count > 0) were removed. Next, we filtered cells by mitochondrial fraction (total mitochondrial counts divided by total gene counts), a measure of nuclear integrity. We log-transformed mitochondrial fractions and fitted a Gaussian Mixture Model (GMM, sklearn GaussianMixture) to identify and remove cells assigned to the GMM component with the highest mean mitochondrial fraction. This step removed approximately 20,000 low-quality cells.

We next considered cells in marker-gene expression space defined by known major cell-type markers for human PFC: astrocytes (159 markers), excitatory neurons (113), inhibitory neurons (83), microglia (97), oligodendrocytes (179), OPCs (143), and vascular cells (124) (Reference 1; Supplementary Table 3). Marker counts were normalized to total library size, mean-centered, and scaled to unit variance. Incremental PCA (sklearn IncrementalPCA) reduced dimensionality (top 50 PCs). Visually, cells projected onto the first two principal components formed distinct Gaussian-like clusters. Assuming each Gaussian cluster corresponded to a distinct brain cell type, we fitted another GMM to the projected data. The resulting ten clusters aligned clearly with known brain cell types.

Cells poorly modeled by this GMM (log-probability < -100) and two clearly outlying clusters were removed. These excluded cells had lower total counts and higher mitochondrial fractions, suggesting low quality. This step removed approximately 12,000 cells, leaving a final dataset of 118,668 cells.

\subsubsection{Gene filtering metrics.}
Downstream analyses included only nuclear-encoded, protein-coding genes (total 19,384) based on Ensembl GRCh38p12 annotations.

\subsubsection{Cell type annotations.}
We first corrected variance due to sequencing batch and individual-of-origin by applying Harmony\supercite{Korsunsky2019-qz} to the top 50 PCs from the quality-controlled data. Using Harmony-corrected PCs, we computed a neighborhood graph (default Scanpy parameters)\supercite{Wolf2018-jx} and clustered cells with the Leiden algorithm (Scanpy implementation)\supercite{Traag2019-xu}.

Major cell types (Ex, In, Ast, Mic, Oli, Opc, Vascular) were assigned to Leiden clusters by computing cell-type-specific marker gene enrichment. Specifically, we calculated enrichment scores as the average log-ratio of expression for marker genes inside versus outside each cluster and assigned labels based on the highest enrichment.

We then sub-clustered each major cell type with the Leiden algorithm and removed subclusters with excessively high mitochondrial fraction or extreme total counts. Thresholds were set at two standard deviations above the mean for these metrics within each major cell type. Removed clusters were small, poorly represented across individuals, and weakly connected upon manual inspection.

\subsubsection{Individual-level filtering.}
Following all previous quality-control steps, six individuals with fewer than 500 cells were excluded from further analyses, leaving 24 control and 12 ABCA7 LoF individuals. None of these excluded individuals carried ABCA7 PTC variants, and their removal did not substantially affect clinical variable distributions across genotypes.

\subsubsection{Differential gene expression.}
Pseudo-bulk gene expression values were generated by summing cell-level counts per gene per individual (matrix multiplication). For each major cell type, we considered genes detected in >10\% of cells. Counts were normalized by TMM (edgeR), and residual mean-variance trends were removed using Limma-Voom. Unknown variance was modeled via surrogate variable analysis (SVA). Differential expression analysis (Limma: lmFit, eBayes, topTable) was performed separately for each major cell type using the following linear model for each gene ($G_i$):

$ G_i = \beta_0 \times \text{ABCA7LoF} + \beta_1 \times \text{msex} + \beta_2 \times \text{nft} + \beta_3 \times \text{amyloid} + \beta_4 \times \text{age\_death} + \beta_5 \times \text{PMI} + \beta_6 \times \text{batch} + \beta_7 \times \text{APOE4} + \sum_{j=1}^{n} \beta_{\text{SV}_j} \times \text{SV}_j$

where $n$ is the number of surrogate variables determined by num.sv() per cell type and  ABCA7LoF indicates individuals carrying ABCA7 loss-of-function variants. Additional covariates (defined in Supplementary Note 1) include sex, NFT, amyloid burden, age at death, PMI, sequencing batch, and APOE4 status.

\subsubsection{Gene perturbation projections across cell types.}
We computed cell type-specific gene perturbation scores summarizing differential expression significance and direction associated with ABCA7 LoF as $S = \operatorname{sign}(\log_2 \text{FC}) \times -\log_{10}(\text{p-value})$, where positive log2FC indicates upregulation in ABCA7 LoF. Scores for genes not detected in >10\% of cells per cell type were set to zero. Genes with $|S|>1.3$ in at least one of six major cell types (Ex, In, Ast, Mic, Oli, OPC) were projected from 6D perturbation-score space into 2D using UMAP (Python umap).

Genes were clustered in the resulting 2D embedding using Gaussian mixture modeling (Python sklearn). Clusters were annotated by hypergeometric enrichment (Python gseapy) for Gene Ontology Biological Process pathways (Supplementary Table 3), using all genes in the embedding as background. Pathways with enrichment p-values <0.01 were selected for naming each cluster. Per-cell-type perturbation scores for each cluster were calculated as the mean gene score within clusters. Statistical significance was assessed by permuting cluster assignments (100,000 permutations).

\subsubsection{Gene-set enrichment and K/L pathway clustering.}
Genes were ranked by perturbation scores S (see Gene perturbation projections). Fast gene set enrichment analysis (fGSEA; R implementation\supercite{Subramanian2005-gt}) with 10,000 permutations tested enrichment of WikiPathways gene sets (Supplementary Table 3) among differentially expressed genes. Only gene sets with 5-1000 genes were considered.

To simplify gene-pathway associations, we constructed a bipartite graph using genes from the fGSEA Leading Edge (LE) subset (268 genes, enriched at p<0.05 in ABCA7 LoF excitatory neurons) and WikiPathways associated with ≥4 LE genes. We treated gene-pathway grouping as a graph partitioning problem. Among three graph-partitioning algorithms tested (Supplementary Note 2), METIS and Kernighan-Lin (K/L) showed the lowest loss and highly comparable performance (within 1.8\% loss; Rand index = 0.98 after $5.0 \times 10^4$ \text{ random initiations}). We selected K/L because it consistently outperformed METIS across a wider range of graph sizes. The K/L algorithm was implemented in Python based on Kernighan and Lin\supercite{Kernighan1970-zl} with parameters C=0, KL\_modified=True, random\_labels=True, unweighted=True, and K=50 to partition the graph into 8 groups. We performed $5.0 \times 10^4$ random initiations and selected the lowest-loss solution.

Graph layouts were computed using the spring layout algorithm (networkx, 10,000 iterations) and visualized using matplotlib. Representative pathways for each cluster were identified by averaging ABCA7 LoF perturbation scores (S) of genes in the cluster connected directly to each pathway. Pathways with ≥5 intra-cluster gene connections were highlighted in figures.

\subsubsection{Gene-set enrichment and K/L pathway clustering.}
Genes were ranked by perturbation scores S (see Gene perturbation projections). Fast gene set enrichment analysis (fGSEA; R implementation\supercite{Subramanian2005-gt}) with 10,000 permutations tested enrichment of WikiPathways gene sets (Supplementary Table 3) among differentially expressed genes. Only gene sets sized between 5 and 1000 genes were considered.

To simplify gene-pathway associations, we constructed a bipartite graph from genes in the fGSEA Leading Edge (LE) subset (268 genes, enriched at p<0.05 in ABCA7 LoF excitatory neurons) and WikiPathways associated with ≥4 LE genes. We treated gene-pathway grouping as a graph partitioning problem. Of three graph partitioning algorithms tested (Supplementary Note 2), METIS and Kernighan-Lin (K/L) showed lowest loss and near-identical performance (within 1.8\% loss, Rand index=0.98, after $5.0 \times 10^4$ random initiations). We selected K/L as it consistently outperformed METIS across varied graph sizes. The K/L algorithm was implemented in Python (see GitHub Repository) based on its original paper \supercite{Kernighan1970-zl} and run with parameters set as C=0, KL\_modified=True, random\_labels=True, unweighted=True, and K=50 to partition the graph into 8 groups. We performed $5.0 \times 10^4$ random initiations and report the partitioning with the lowest loss among all initiations.

Graph layouts were computed using the spring layout algorithm (networkx, 10,000 iterations) and visualized with Python's matplotlib. Representative pathways for each cluster were identified by averaging ABCA7 LoF perturbation scores (SSS) of genes in the cluster connected directly to each pathway. Only pathways with at least 5 intra-cluster gene connections and manually selected subsets of genes with high perturbation scores (∣S∣>1|S| > 1∣S∣>1) were highlighted in the figures.

\subsubsection{Excitatory neuronal layer annotation.}
Excitatory neurons were annotated by cortical layer using published marker gene sets\supercite{He2017-dq} (Supplementary Table 3), following the procedures described in “Cell type annotations.” Briefly, the normalized expression matrix was filtered to include only layer-specific marker genes and cells expressing ≥15\% of these genes. Dimensionality was reduced using iterative PCA, followed by batch-effect correction using Harmony. A neighborhood graph was constructed, and cells were clustered using the Leiden algorithm. Clusters enriched for layer-specific markers (average log-fold-change >0.1) were labeled accordingly, while ambiguous clusters were excluded. Layers 5 and 6 were combined into a single ‘L5/6’ category. Annotations were validated using independent marker genes\supercite{Maynard2021-mz} (Supplementary Table 3). Layer-specific differential expression analysis was performed as described (“Differential gene expression”), followed by gene-set enrichment analysis (fGSEA, as described in “Gene-set enrichment and K/L pathway clustering”) testing enrichment of ABCA7 LoF-associated gene clusters identified by K/L clustering (as described in “Gene-set enrichment and K/L pathway clustering”).

\subsubsection{ABCA7 p.Ala1527Gly variant calling and gene-pathway clustering comparisons.}
Subjects carrying the ABCA7 p.Ala1527Gly variant with available PFC snRNA-seq data from a previous study (Supplementary Table 3) were identified using methods described in “Variant calling and ROSMAP subject selection.” Differential expression was computed as described (“Differential gene expression”), followed by gene-set enrichment analysis (fGSEA) to test enrichment of ABCA7 LoF-associated gene clusters identified by K/L clustering (“Gene-set enrichment and K/L pathway clustering”).

\subsubsection{Culture and generation of human isogenic iPSCs.}
A control parental iPSC line (AG09173; 75-year-old female, APOE3/3 genotype) was generated previously by the Picower Institute iPSC Facility\supercite{Lin2018-ma}. Two ABCA7 LoF isogenic lines were derived from AG09173: ABCA7 p.Glu50fs*3, containing a novel premature stop codon in exon 3 (generated by Synthego), and ABCA7 p.Tyr622*, containing a patient-derived mutation (Y622*)\supercite{De_Roeck2019-bs} generated in-house by CRISPR-Cas9 editing.

For the ABCA7 p.Tyr622* line, an sgRNA targeting ABCA7 (oligos: forward, 5’- CACCGCCCCTACAGCCACCCGGGCG -3’; reverse, 5’- AAACCGCCCGGGTGGCTGTAGGGGC -3’; designed at \url{http://crispr.mit.edu}) was cloned into pSpCas9-2A-GFP (PX458, Addgene \#48138) as previously described \supercite{Ran2013-sf}. The plasmid was confirmed by Sanger sequencing, then nucleofected (Amaxa, Lonza Human Stem Cell Nucleofector Kit I, program A-23) along with 15 μg of a single-strand oligodeoxynucleotide (ssODN) template \seqsplit{(5'-GGTGCGCGCCCCCAGGCCAATCCAGGAGCTGCACCCTAAGCTCCCGTTGCCTCTCACAGCTGGGAGACATCCTCCCCTAGAGCCACCCGGGCGTCGTCTTCCTGTTCTTGGCAGCCTTCGCGGTGGCCACGGTGACCCAGAGCTTCCTGCTCAGCGCCTTCTTCTCCCGCGCCAACCTGG-3')} into dissociated AG09173 iPS cells (Accutase, Thermo Fisher; 10 μM ROCK inhibitor, Tocris). Cells ($\sim 5 \times 10^6$) were sorted (BD FACS Aria IIU, Whitehead Institute), plated at single-cell density in media supplemented with Penicillin-Streptomycin (Gemini Bio-products) and ROCK inhibitor. Colonies were expanded, screened by genomic DNA extraction (DNeasy Blood & Tissue Kit, Qiagen, \#69504) and Sanger sequencing to confirm the Y622* mutation.

All iPSC lines were regularly tested for karyotypic normality (Cell Line Genetics) and cultured at 37°C, 5\% CO2, in feeder-free conditions using mTeSR-1 medium (STEMCELL Technologies, \#85850) on Matrigel-coated plates (Corning; hESC-qualified, \#354277). Cells were passaged at 60-80\% confluence using ReLeSR (STEMCELL Technologies, \#05872) onto Matrigel-coated plates at a 1:6 to 1:24 split ratio.

\subsubsection{rTTA and NGN2 virus production.}
HEK293T cells were seeded at $5 \times 10^6$ cells per 10 cm plate and transfected using a 3rd-generation lentiviral system. Per plate, transfection mixtures contained 10 µg plasmid DNA (EF1a-rtTA-Hygro, Addgene \#66810, or pLV-TetO-hNGN2-eGFP-Puro, Addgene \#79823), 5 µg pMDL g/pRRE, 2.5 µg pRSV-Rev, 2.5 µg MD2.G, and 48 µL polyethyleneimine (1 mg/mL) diluted in 600 µL OptiMEM (Fisher, \#51-985-034). Mixtures were incubated 20 min at room temperature, added dropwise to cells, and replaced with fresh media after 16 hours. Virus-containing supernatant collected 72 hours post-transfection was clarified ($3000 \times $g, 5 min, 4°C) and supernatant ultracentrifuged (Beckman Optima L-90K Ultracentrifuge, SW32Ti rotor, 25,000 rpm, 2 hours), resuspended in 1 mL PBS per 10 cm plate, and stored at -80°C.

\subsubsection{Lentivirus-mediated NGN2 induction in iPSCs and drug treatments.}
iPSCs were dissociated into single-cell suspensions (Cell Dissociation Buffer, Life Technologies, \#13151-014), resuspended in mTeSR1 media with ROCK inhibitor (Rockout; Abcam, ab285418), and plated onto Matrigel-coated 6-well plates at 50-60\% confluence after 24 hours. After one day, cells were co-transduced overnight with 80 µL each of pLV-TetO-hNGN2-eGFP-Puro and EF1a-rtTA-Hygro lentivirus per well. NGN2 expression was induced 24 hours later with doxycycline (DOX, 1 µg/mL) and ROCK inhibitor . puromycin selection was performed 24 hours post viral transduction. Immature neurons were replated on PDL/laminin-coated plates ($1 \times 10^6$ cells per well in 6-well plates, or $5 \times 10^4$ cells per well in 96-well plates), and maintained in BrainPhys Neuronal Media (STEMCELL Technologies, \#05793) with Neurocult SM1 Neuronal Supplement (STEMCELL Technologies, \#05711), (N2-supplement-A STEMCELL Technologies, \#07152), laminin (1 µg/mL), and DOX (1 µg/mL). Half-media changes were performed every 3-4 days, and cultures matured for 28 days before experiments.

Neurons were treated with cytidine 5'-diphosphocholine (CDP-choline, Millipore Sigma, \#30290) at a final concentration of 100 µM starting at day 14, continuing with each media change until day 28. Choice of treatment concentration and duration was based on a previous study by our lab \supercite{Sienski2021-zt}.

\subsubsection{Cortical organoid generation.}
Dorsal cortical organoids were generated as previously described\supercite{Sloan2018-ja}. Briefly, iPSCs at 80-90\% confluence were dissociated into single-cell suspensions ($1 \times 10^5$ cells/mL) in mTeSR with ROCK inhibitor, seeded at 100 µL/well in PrimeSurface® 96 Slit-well plates (S-Bio, \#MS9096SZ), and induced to differentiate using Neural Induction Media (DMEM/F12, Life Technologies \#11330-032, KnockOut serum replacement (Life Technologies, \#10828-028), GlutaMAX, 2-Mercaptoethanol, Penicillin-Streptomycin, and SMAD inhibitors SB-431542 and dorsomorphin) with daily media changes (days 0-5). Media was then switched (days 6-16) to Neural Differentiation Media (Neurobasal A, B27 supplement, GlutaMAX, Penicillin-Streptomycin, human recombinant EGF and FGF2, 20 ng/mL each), with daily changes until day 16, then every other day until day 25. From day 25 onwards, EGF and FGF2 were replaced with 20 ng/mL each of BDNF and NT3, with media changes twice weekly after day 45.

Dorsal cortical organoids were generated as previously described\supercite{Sloan2018-ja}. Briefly, iPSCs at 80-90\% confluence were dissociated into single-cell suspensions ($1\times10^5$ cells/mL) in mTeSR with 10µM ROCK inhibitor, seeded at 100 µL/well in PrimeSurface® 96 Slit-well plates (S-Bio, MS9096SZ), and induced to differentiate using Neural Induction Media consisting of DMEM/F12 (Life Technologies, 11330-032), 100mM GlutaMAX (Life Technologies, 35050-061), 0.1mM 2-Mercaptoethanol (Sigma-Aldrich, M3148), 1\% Penicillin-Streptomycin (Life Technologies, 15070-063), and 10µM SB-431542 (R&D Systems, 1614) and 2.5µM Dorsomorphin (Sigma-Aldrich, P5499-CONF) with daily media changes (days 0-5). Media was then switched (days 6-16) to Neural Differentiation Media (Neurobasal A, B27 supplement, GlutaMAX, Penicillin-Streptomycin, human recombinant EGF and FGF2, 20 ng/mL each), with daily changes until day 16, then every other day until day 25. From day 25 onwards, EGF and FGF2 were replaced with 20 ng/mL each of BDNF and NT3, with media changes twice weekly after day 45.

\subsubsection{Confocal imaging experiments.}
All confocal images were acquired on a Zeiss LSM900 microscope using ZEN software.

For mitochondrial health staining, live cells were incubated with MitoHealth dye (ThermoFisher, \# H10295) according to manufacturer's protocols for 30 min at 37°C, fixed (4\% paraformaldehyde/4\% sucrose, 15 min, room temperature), permeabilized (0.1\% Triton-X, 5 min), blocked (2\% BSA, Fisher Bioreagents, BP9703), and incubated overnight at 4°C with NeuN antibody (1:500), followed by incubation with secondary antibodies (1:1000) for 2 hours and Hoechst (1:2000, Invitrogen, H3570) for 10 minutes. Images were captured as z-stacks (1 µm intervals).

Live imaging of mitochondrial membrane potential used TMRM (0.1 µM, 30 min at 37°C; ThermoFisher, \# I34361), followed by imaging before and immediately after adding FCCP (1 µM; Cayman Chemical, \#15218). Reactive oxygen species were assessed by live staining with CellROX Orange (5 µM, 30 min at 37°C; ThermoFisher, \# C10443). TMRM and CellROX images were acquired as single optical sections.

For immunostaining, iNs cultured on coverslips and cortical organoid cryosections (20 µm) were fixed (4\% formaldehyde, 10 min), permeabilized (0.2\% Triton-X) and blocked (10\% BSA, 1 hour), and incubated overnight at 4°C with primary antibodies (MAP2 and NeuN, both 1:1000). Alexa Fluor-conjugated secondary antibodies (1:500) and Hoechst (1:1000) were used for visualization. Coverslips were mounted with Fluoromount-G, and images were captured as single optical sections.

\subsubsection{Confocal image quantification.}
Confocal images (.czi format; 8 or 16 bits; voxel size: $1 \times 0.62 \times 0.62$ µm) were loaded into Python (aicsimageio) and normalized to floating-point format [0,1]. Acquisition settings were consistent within each imaging batch.

For fixed z-stack images, NeuN-positive cell bodies were segmented in 3D using the pre-trained "cyto2" model (Cellpose\supercite{Stringer2021-yn}). Segmentation quality was manually verified (blinded), and low-quality images were excluded. Cell-level fluorescence intensities were computed as probability-weighted sums of voxel intensities, using segmentation-derived voxel probabilities. Measurements from multiple differentiation batches (independent staining and imaging experiments) were combined by uniformly sampling cells per condition per batch, batch-wise z-scaling fluorescence values, and including batch and well-of-origin indicator variables in downstream analyses. Clipping was minimal (<0.1\%), and the confocal microscope response was assumed linear. A linear mixed-effects model (mixedlm() from statsmodels) tested cell-level fluorescence intensities, modeling genotype or treatment as a fixed effect and well-of-origin as a random effect.

For single-plane live imaging (TMRM, CellROX), images were binarized at the 75th percentile intensity threshold per channel to identify regions occupied by neuronal soma or processes, following established methodology\supercite{Esteras2020-md}. Mean fluorescence intensities were quantified within these masked areas. For FCCP time-course imaging, images were spatially aligned by Fourier-based registration (phase cross-correlation), with alignment accuracy confirmed manually. A mask from the baseline (pre-FCCP) TMRM image (75th percentile threshold) was consistently applied across time points. For all live-imaging experiments, masked regions (wells) were treated as individual observations in statistical tests. Batch-wise z-scaling was not required here, as data were not combined across batches for these experiments.

\subsubsection{Aβ enzyme-linked immunosorbent assays (ELISAs).}
Culture media were collected and analyzed for Aβ40 and Aβ42 levels using ELISA kits (ThermoFisher Scientific, KHB3481 and KHB3441, respectively) per the manufacturer's protocols. For 4-week-old iNs, media were flash-frozen before analysis. For cortical organoids (age: 5-6 months; days 176-182), media were analyzed immediately after collection following 3-4 weeks of treatment with 500 µM or 1 mM CDP-choline. 

\subsubsection{Electrophysiological recordings.}
Electrophysiological recordings were performed using an Axon Multiclamp 700B amplifier and Clampex 11.2 software (Molecular Devices). Cells were visualized using infrared differential interference contrast (IR-DIC) imaging (Olympus BX-50WI microscope), placed in a recording chamber, and perfused continuously at 2 mL/min (~32°C) with oxygenated artificial cerebrospinal fluid (ACSF, containing 125 mM NaCl, 2.5 mM KCl, 1.2 mM NaH2PO4•H2O, 2.4 mM CaCl2•2H2O, 1.2 mM MgCl2•6H2O, 26 mM NaHCO3, and 11 mM D-Glucose).

Action potentials were elicited by injecting current steps in current-clamp mode. Whole-cell currents were recorded from a holding potential of -80 mV by stepping to various voltages in voltage-clamp mode. Spontaneous firing was recorded in cell-attached configuration. Recordings were filtered at 1 kHz (four-pole Bessel filter), digitized at 10 kHz with a Digidata 1550B interface (Molecular Devices). Pipette solution contained 120 mM K-gluconate, 5 mM KCl, 2 mM MgCl2•6H2O, 10 mM HEPES, 4 mM ATP, and 0.2 mM GTP. Data were analyzed using pClamp 11.2 and GraphPad Prism 10.

For electrophysiological recordings from cortical organoids, day 150 organoids were dissociated using Accutase (Stem Cell Technologies, \#07920, 40 min, 37°C), plated onto \#1 glass coverslips (Fisher Scientific, \#50-194-4702) coated with PDL, laminin, and Matrigel, and maintained in 2D culture with or without 100 µM CDP-choline for two weeks prior to recordings. 

Spontaneous action potential outliers were identified using the interquartile range (IQR) method (values outside \( Q1 - 2 \times \mathrm{IQR} \) or \( Q3 + 2 \times \mathrm{IQR} \)) and removed, resulting in the exclusion of two data points (values: 9.38 in p.Tyr622*; 6.15 in p.Tyr622*+Choline). Cells recording zero spontaneous potentials (likely glial) were also excluded.

\subsubsection{Seahorse metabolic assays and OCR analysis.}
iPSC-derived neurons were differentiated directly in Seahorse XFe96/XF Pro microplates for 28 days before metabolic assays on a Seahorse XFe96 Analyzer. Seahorse XF Cell Mito Stress and Oxidation Stress Tests were conducted according to manufacturer protocols, using final drug concentrations of 2.5 µM oligomycin, 1 µM FCCP, and 0.5 µM rotenone/antimycin.

Oxygen consumption rates (OCR) were monitored over time, with curves visually inspected (blinded) to exclude those not responsive to drug injections. The following OCR metrics were computed from integrals of OCR curves between specific experimental intervals: (1) Basal respiration (prior to oligomycin injection), (2) Proton leak (post-oligomycin, pre-FCCP), (3) Maximal respiration (post-FCCP, pre-rotenone/antimycin), (4) Relative uncoupling (proton leak divided by basal respiration), and (5) Spare respiratory capacity (maximal respiration divided by basal respiration).


\subsubsection{mRNA sequencing and analysis of iNs.}
Total RNA was extracted from iNs using the RNeasy Mini Kit (Qiagen). RNA quality was assessed (Fragment Analyzer, Agilent), and only samples with RNA Quality Number (RQN) > 9.5 were selected. Full-length cDNA libraries were generated (SMART-seq v4 kit, Takara Bio), and sequencing libraries prepared (Nextera XT DNA Library Preparation Kit, Illumina) for sequencing on an Element AVITI platform (Element Biosciences; 75 bp paired-end reads with dual 8-nucleotide indexes) at the MIT BioMicro Center.

Sequencing data were processed through the MIT BioMicro Center BMC/BCC pipeline v1.8 (updated 06/06/2023; details: \url{https://openwetware.org/wiki/BioMicroCenter:Software\#BMC-BCC_Pipeline}). Reads were adapter-trimmed (Trim Galore, Nextera-specific settings, minimum overlap 3 bases), aligned to the human reference genome (GRCh38.p14, GENCODE release 47; STAR aligner), and counted (featureCounts, paired-end settings). Read counts were summarized at the exon level and aggregated by gene identifier.

Differential expression analysis (edgeR, limma-voom) retained protein-coding genes expressed at ≥1 CPM in ≥1 sample, normalized counts, and employed linear modeling with empirical Bayes moderation with contrasts based on experimental conditions (treatment/genotype). Gene set enrichment analysis (FGSEA, 10,000 permutations) of WikiPathways gene sets (Supplementary Table 3) was performed using ranked differentially expressed genes (score: sign(log-fold-change) × −log10(p-value)), as described above (‘Gene-set enrichment and K/L pathway clustering’). Significant pathways (adjusted p-value <0.05) were identified, and leading-edge genes underwent gene-pathway clustering (Kernighan-Lin heuristic, described above).
Gene-pathway cluster similarity was assessed by computing Jaccard indices based on pathways and genes assigned to each K/L cluster. Significance of observed overlaps was determined empirically through comparison to 1,000 random permutations, with p-values adjusted using the Benjamini-Hochberg method to control the false discovery rate.

\subsubsection{LC-MS lipidomics on iNs.}
iPSC-derived neurons were washed in cold PBS, scraped, centrifuged ($2000 \times $g, 5 min), counted, pelleted to equal number, and resuspended in cold methanol (2 mL). Biphasic extraction was performed by sequentially adding cold chloroform (4 mL) and cold water (2 mL), vortexing after each addition, then centrifuging (3000 rcf, 10 min) for phase separation. Samples prepared at the Harvard Center for Mass Spectrometry were similarly processed from provided pellets (in 500 µL methanol), supplemented with additional methanol (1.5 mL) and chloroform (4 mL), sonicated (10 min), mixed with water (2 mL), and centrifuged (800 rcf, 10 min, 4°C). Upper aqueous phases were collected for metabolomics, while chloroform phases were reserved for lipidomics. At least one blank control (no cells) was included in each extraction run. All LC-MS analyses were performed by the Harvard Center for Mass Spectrometry.

Extracted samples were dried under nitrogen, fully evaporated, resuspended in chloroform (scaled by biomass (cell count); ≥60 µL), and split equally for positive and negative ionization analyses (or unsplit if only positive mode). Following centrifugation (18,000 rcf, 20 min, 4°C), supernatants were transferred into microinserts for LC-MS.

LC-MS analyses were performed on an Orbitrap Exactive plus MS (Thermo Scientific) in line with an Ultimate 3000 LC (Thermo Scientific) in both positive and negative ionization modes, in top 5 automatic data-dependent MS/MS mode. Chromatographic separation was performed on a Biobond C4 column (4.6 x 50 mm, 5 µm particle size; Dikma Technologies). The flow rate began at 100 µL/min with 0\% mobile phase B (MB) for the initial 5 minutes, followed by an increase to 400 µL/min over the next 50 minutes with a linear gradient of MB from 20\% to 100\%. The column was subsequently washed at 500 µL/min for 8 minutes with 100\% MB, then re-equilibrated for 7 minutes at 500 µL/min using 0\% MB. For positive ion mode, mobile phases consisted of buffer A (MA: 5 mM ammonium formate, 0.1\% formic acid, and 5\% methanol in water) and buffer B (MB: 5 mM ammonium formate, 0.1\% formic acid, 5\% water, and 35\% methanol in isopropanol). For negative ion mode, buffer A (MA) contained 0.03\% ammonium hydroxide and 5\% methanol in water, and buffer B (MB) contained 0.03\% ammonium hydroxide, 5\% water, and 35\% methanol in isopropanol.

Lipids were identified, and their signals integrated using the Lipidsearch © software (version 4.2.27, Mitsui Knowledge Industry, University of Tokyo). Integrations and peak quality were curated manually. Peak areas were background-corrected (subtracting 3× median blank peak areas; negative values set to zero). Statistical analyses used Welch's t-test (unequal variance) to compare different cell lines, and Student's t-test (equal variance) for treatment comparisons within identical genetic backgrounds.

\subsubsection{LC-MS metabolomics on iNs.}
Samples were dried under nitrogen, evaporated completely, and resuspended in biomass-scaled volumes (≥20 µL) of 50\% acetonitrile in water. Following centrifugation (max speed, 10 min), consistent volumes (12 or 15 µL, depending on batch) of supernatants were transferred to microinserts. The remaining of the sample volumes were combined to create a pool sample used for MS2/MS3 data acquisition.

LC-MS metabolomics analyses were performed at the Harvard Center for Mass Spectrometry using a Vanquish LC system coupled with an ID-X mass spectrometer (Thermo Fisher Scientific). Samples (5 µL injection) were analyzed on a ZIC-pHILIC peek-coated column (150 mm x 2.1 mm, 5 µm particle size; Sigma Aldrich) held at 40°C. Mobile phases comprised buffer A (20 mM ammonium carbonate and 0.1\% ammonium hydroxide in water) and buffer B (97\% acetonitrile in water). The gradient initiated at 93\% B, decreasing linearly to 40\% B over 19 minutes, further decreasing to 0\% B over the subsequent 9 minutes, held at 0\% B for 5 minutes, returned to 93\% B within 3 minutes, and finally re-equilibrated at 93\% B for 9 minutes. The flow rate was held constant at 0.15 mL/min, except for an initial 30-second ramp from 0.05 to 0.15 mL/min. Mass spectrometry data were acquired in polarity-switching mode at 120,000 resolution, with an AGC target of $1 \times 10^5$, covering an m/z range from 65 to 1000. MS1 acquisition employed polarity switching for all samples. MS2 and MS3 analyses were performed on pooled samples using the AcquireX DeepScan method, with five reinjections each in positive and negative ion modes separately. A mixture containing standards of targeted metabolites was prepared and analyzed immediately following the sample runs for targeted metabolite analysis.

Data were analyzed using Compound Discoverer 3.2 (Thermo Fisher Scientific). Metabolite identification was based either on MS2/MS3 spectral matching against a local mzVault library and corresponding retention times from pure standards (Level 1), or spectral matching using mzCloud (Level 2). Each metabolite identification was manually inspected. Peak areas were first background-corrected by subtracting three times the median peak area measured in blank samples; negative values resulting from this correction were set to zero. Median-centered peak areas were scaled to zero-mean and unit variance prior to principal component analysis. The Harvard Center for Mass Spectrometry identified three samples with notably low overall metabolite intensities, which were subsequently excluded from downstream analyses.

\subsubsection{LC-MS metabolomics on media.}
Media samples (100 µL each) were transferred into microcentrifuge tubes containing 1 mL of methanol and incubated at -20°C for 2 hours. Following incubation, samples were centrifuged at 18,000 rcf for 20 min at -9°C, and supernatants were transferred into new tubes and evaporated to dryness under nitrogen flow. The dried samples were resuspended in 50 µL of 30\% acetonitrile in water containing 2 mM medronic acid, centrifuged again at 18,000 rcf for 20 min at 4°C, and the resulting supernatants were transferred into glass microinserts for LC-MS analysis.

Peak areas from targeted metabolite analysis of media samples were compared for CDP, CDP-choline, and choline. To ensure accurate detection, solvent blanks were analyzed: CDP and CDP-choline were not detected in these blanks, while choline was detected at levels several orders of magnitude lower than in media samples.

\subsubsection{Molecular Dynamics Simulations.}
ABCA7 structures (unbound-open and bound-closed conformations; PDB IDs: 8EE6, 8EOP) containing the G1527 variant were retrieved from the Protein Data Bank. The A1527 variant was generated via mutation (Gly→Ala) using PyMOL. ABCA7 residues 1517-1756 were embedded in a DPPC membrane (CHARMM-GUI) and oriented according to the OPM database. Four simulations were performed (GROMACS 2022.3; CHARMM36M force field; Supplementary Table 15).

The protein-membrane system was solvated in a cubic box with a minimum distance of 1.0 nm between the protein and the box edge, using the TIP3P water model. Energy minimization was performed using the steepest descent algorithm with a maximum force threshold of 1000 kJ/mol/nm to relieve any steric clashes or bad contacts. The system was equilibrated in six phases, each 125 ps long, to equilibrate volume (NVT) and pressure (NPT). The production run, 300 ns long, was performed in the NPT ensemble at 323 K using a v-rescale thermostat and 1 bar using the Parrinello-Rahman barostat. A 2 fs time step with h-bonds constraints was used with periodic boundary conditions applied in all directions. Long-range electrostatics were handled using the Particle Mesh Ewald (PME) method with a cutoff of 1.0 nm for non-bonded interactions. 
 
RMSD was calculated to monitor the conformational stability of a given structure over the course of the simulation by comparing the position of $C_\alpha$ at time $t$ under simulation to its reference position (in 8EOP or 8EE6).The $\phi$ and $\psi$ dihedral angles were calculated using the \textit{gmx rama} tool, followed by post-processing. Secondary structure analysis was performed using \textit{gmx dssp -hmode dssp}, with subsequent post-processing using custom Python scripts. Visualization of the trajectories was carried out using VMD software. Principal Component Analysis was conducted on $C_\alpha$ atom positional fluctuations  to identify the major conformational changes during the simulation. 

