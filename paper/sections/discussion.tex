
Loss-of-function (LoF) mutations in the lipid transporter ABCA7 are among the strongest genetic risk factors for late-onset AD. Here, we generated a transcriptional atlas of ABCA7 LoF effects across all major brain cell types in the human prefrontal cortex. Our dataset showed the highest levels of ABCA7 expression in excitatory neurons and strong evidence that ABCA7 LoF led to transcriptional perturbation in pathways related to lipid biosynthesis, mitochondrial respiration, and cellular stress, including up-regulation of DNA repair pathways, and changes to inflammatory and synaptic genes. Using iPSC-derived isogenic neuronal lines with and without ABCA7 LoF variants, we show that ABCA7 LoF leads to intracellular triglyceride accumulation, decreased phosphatidylcholine levels, and a reduction in mitochondrial uncoupling.

Multiple lines of evidence highlight a causal relationship between lipid metabolism and  mitochondrial uncoupling\cite{Prola2021-uz,Tseng2010-ly,Rossmeisl2005-yi,Goedeke2021-gt,Amorim2022-zf}. While neurons rely on astrocytes to help metabolize excess  lipids\cite{Ralhan2021-kq,Islimye2022-av}\cite{Ioannou2019-hl,Moulton2021-ft}, neurons can also perform endogenous lipid breakdown through β-oxidation\cite{McFadden2014-ri,noauthor_2004-nu}. Rates of β-oxidation are tightly linked to mitochondrial uncoupling; Cells that specialize in fatty acid oxidation have among the highest uncoupling ratios (e.g. Brown adipose tissue)\cite{Jastroch2017-ve,Giralt_undated-jw,Fedorenko2012-oq} and pharmacologically stimulating mitochondrial uncoupling increases lipid catabolism\cite{Prola2021-uz,Tseng2010-ly,Rossmeisl2005-yi,Goedeke2021-gt}. We find that treatment with CDP-choline, a phosphatidylcholine precursor that bypasses the rate limiting step in phosphatidylcholine synthesis through the Kennedy pathway\cite{noauthor_1997-th}, reverts uncoupling to normal levels and reverses triglyceride accumulation,  suggesting that in addition to increasing the availability of triglyceride precursors, phosphatidylcholine deficiency may exacerbate triglyceride accumulation by impairing mitochondrial lipid catabolism in these cells (Figure~\ref{fig:main_choline}G). Further studies will be needed to distinguish if increased phosphatidylcholine synthesis restores mitochondrial uncoupling directly by modifying the mitochondrial membrane, in line with a recent study \cite{Prola2021-uz}, or indirectly by increasing the release of free fatty acids from lipid droplets\cite{Krahmer2011-xy,Guo2008-xt,Fei2011-bl,Schott2019-zq}.

While mitochondrial dysfunction is associated with aging, AD, and other neurodegenerative diseases - mitochondrial uncoupling was recently linked to frontotemporal dementia variants\cite{noauthor_2022-os} -, little is known about the role of mitochondrial uncoupling in AD etiology \cite{Bano2023-qz,Zong2024-tn,Demine2019-qj,noauthor_2013-rt,Picca2023-gt}. Neurons maintain very high levels of mitochondrial OXPHOS to meet their energy demands\cite{Morant-Ferrando2023-va,Trigo2022-ym} . Mitochondrial uncoupling, which is actively regulated by mitochondrial proteins\cite{Park2023-fa,noauthor_2016-fg}, may help sustain this high aerobic capacity by actively controlling the mitochondrial membrane potential, managing associated levels of reactive oxygen species\cite{Demine2019-qj,Shadel2015-kt}, and promoting mitochondrial biogenesis\cite{Korshunov1997-aj,Wisloff2005-ho,Andrews2005-yy,noauthor_2022-vx}. Impaired uncoupling can be neurotoxic\cite{Korshunov1997-aj,Wisloff2005-ho,Andrews2005-yy,noauthor_2022-vx}, for example, by impairing synaptic and calcium signaling, or increasing  oxidative stress, a phenotype observed recently in ABCA7 LoF neurospheroids\cite{Kawatani2023-vf}. Oxidative stress has multiple toxic downstream consequences, including the induction of DNA damage and inflammation, as suggested by ABCA7 LoF transcriptomic signatures\cite{Robert2020-sc,Volanti2002-mc,Canty1999-oj,Schreck1992-zr}. Neurons afflicted by these phenotypes also impact surrounding glial cells \cite{Byrns2024-id,Welch2022-ef}. 

In line with our findings linking phosphatidylcholine-triglyceride imbalances to mitochondrial impairments in ABCA7 LoF neurons, a recent study in ABCA7 LoF neurospheroids independently revealed a link between phosphatidylglycerol deficiency, observed simultaneously with reduced phosphatidylcholine, and mitochondrial function\cite{Kawatani2023-vf}, further highlighting the importance of lipid-centric therapeutic interventions for ABCA7 LoF. Here, we offer a therapeutic strategy to reverse these dysfunctions through CDP-choline treatment, a readily available and safe dietary supplement\cite{Gavrilova2018-oi,Zeisel2009-xv,Blusztajn2017-nv}. CDP-choline treatment also reduces toxic Aβ42 levels in ABCA7 LoF neurons, indicating that lipid metabolic defects may contribute to Alzheimer’s disease pathology in ABCA7 LoF neurons. This is consistent with previous studies linking membrane lipids and lipid modifications to APP processing \cite{Bhattacharyya2016-rs,Walter2013-qu}. 

Recent work from our lab implicates choline in APOE4 dysfunction\cite{Sienski2021-zt}, and in cognitive resilience to AD pathology (Boix et al, 2024, Nature, in press), suggesting that phosphatidylcholine disruptions may be central to AD risk in large fractions of the population. Indeed, our work suggests that the common missense variant p.Ala1527Gly likely has convergent effects with ABCA7 LoF. Genetic interactions with other risk factors, including APOE4, may exacerbate otherwise subtle ABCA7 dysfunction, and contribute to risk in a significant subset of AD cases\cite{Wang2021-oa,Hemani2013-zr,Haig2011-vs,Zuk2012-uz}. As such, our study supports a growing body of literature, including recent studies on APOE4 \cite{Haney2024-fx,Victor2022-tl}, implicating lipid disruptions in the etiology of AD and pinpoints additional genotypes that may benefit from phosphatidylcholine and triglyceride-targeting interventions. 
