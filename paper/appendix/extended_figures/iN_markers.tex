\begin{figure}[H]
    \begin{subfigure}[t]{\textwidth}
        \begin{subfigure}[t]{0.45\textwidth}
            \caption{}
            \includegraphics[width=\textwidth]{./extended_plots/iN_induction_cartoon.png}        
        \end{subfigure}
        \begin{subfigure}[t]{0.45\textwidth}
            \caption{}
            \includegraphics[width=\textwidth]{./extended_plots/iN_markers_prev.png}        
        \end{subfigure}
    \end{subfigure}
    \begin{subfigure}[t]{0.8\textwidth}
        \caption{}
        \hspace{2cm}
        \includegraphics[width=\textwidth]{./extended_plots/iN_markers.png}        
    \end{subfigure}
    \caption{
        \textbf{Differentiating iPSC-Derived Neurons Harboring ABCA7 PTC Variants.}\\
    }
    \label{fig:iN_markers}
\end{figure}
\begin{itemize}
    \item[\textbf{(A)}] iPSCs were plated at low density for NGN2 viral transduction. Expression of NGN2 was driven by doxycycline (DOX) induction with puromycin (PURO) selection, then re-plated to match neuronal densities. Neurons were maintained for 4 weeks (DIV 28) before experimentation (Created with BioRender.com). 
    \item[\textbf{(B)}] Neuronal marker gene expression in 2 and 4-week matured iNs. 
    \item[\textbf{(C)}] Neuronal marker gene expression in iNs matured for 4 weeks for indicated genotypes. CDP-choline treatment was applied for 2 weeks at 100 $\mu$M.
\end{itemize}