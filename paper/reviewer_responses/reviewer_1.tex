\subsubsection*{Referee \#1 (Remarks to the Author)}
In this manuscript, von Maydell et al. analyzed the single nucleus transcriptomes of AD patients with or without ABCA7 LoF mutations and found that ABCA7 LoF affects pathways including lipid metabolism, mitochondrial function, and others. Through cross validation with other datasets and using iPSC-derived neurons, the authors showed that ABCA7 LoF led to lipid accumulation and a decrease in mitochondrial uncoupling in neurons. Modulating ABCA7 LoF through CDP-choline treatment may reverse these effects.
While the study provides novel evidence of the important roles that phosphatidylcholine disruption plays in AD pathogenesis, some key information is missing. Detailed comments are listed below.

\textcolor{blue}{We thank the reviewer for their encouraging comments and constructive suggestions.}

Fig. 1F,\\
\dots what are the genes overlapping in the different cell types?\\ 
\textcolor{blue}{ We have now added a supplementary heat map showing overlapping genes across cell types (Figure~\ref{fig:snRNAseq_gene_scores_2}A). For visualization purposes, we included only genes significantly perturbed (p<0.05) in ABCA7 loss-of-function (LoF) brains in at least three cell types. Figure~\ref{fig:snRNAseq_gene_scores_2}B provides functional annotations for these genes, grouping them into key categories. We also provide these plots in subpanels A and B directly below.}

\begin{figure}[H] 
    \begin{subfigure}[t]{0.2\textwidth}
        \caption{DEGs across cell types}
        \includegraphics[width=\textwidth]{./extended_plots/heatmap_overlap_with_gene_summaries.png}        
    \end{subfigure}   
	\hspace{2cm}
    \begin{subfigure}[t]{0.8\textwidth}
        \caption{Functional annotations of genes in the same order as in the heatmap in A}
        \vspace{.5cm}
        \includegraphics[width=\textwidth]{./extended_plots/overlap_with_gene_summaries_legend.png}        
    \end{subfigure}   
\end{figure}

\dots are these changes relevant for the risk of ABCA7 LOF on AD?\\ 
\textcolor{blue}{These categories highlight pathways consistently identified in our study, such as lipid metabolism, mitochondrial function, and DNA damage response, suggesting these pathways may contribute to Alzheimer’s disease risk linked to ABCA7 LoF in other cell types, in addition to neurons.}

\dots The thresholds of the volcano plots should be indicated.\\ 
\textcolor{blue}{These plots represent projections of differentially expressed gene scores across major cell types for simplified visualization and comparison in 2D space, rather than standard volcano plots. To clearly distinguish them from volcano plots, we have now added axis labels indicating the UMAP dimensions in Figure~\ref{fig:main_atlas}F.}

Fig. 1G, it is not clear how the clusters were defined and genes under each pathway/cluster were selected. PCA the and thresholds of volcano plots should be provided.\\
\textcolor{blue}{We have now carefully revised our description of this analysis in the text to clarify our approach [page~\pageref{quoteF-label}]:}
\begin{quote}
	\textcolor{blue}{["\dots]} \quoteF\\\\
	\quoteG\textcolor{blue}{"}
\end{quote}

Fig. 3E-F, the data of the 3 iNs should be provided.\\
\textcolor{blue}{We have now included the data for all three lines in Figure~\ref{fig:differentiating_iPSC_neurons}A,B}.

Fig. S9. The increase of Aβ42 is not evident, Amyloid accumulation should be confirmed with another method like immunoblotting or ELISA.\\
\textcolor{blue}{Because we found secreted Aβ40 and Aβ42 levels by ELISA in media from ABCA7 LoF iNs at this developmental time point to be low (Figure~\ref{fig:neurospheroid_figure}A), we also differentiated WT and p.Tyr622* lines into dorsal cortical organoids and repeated ELISA measurements on the media at a more mature timepoint (six months). These experiments indicated elevated soluble Aβ40 and Aβ42 in media from p.Tyr622* cortical organoids compared to WT cortical organoids (Figure~\ref{fig:main_choline}J and Figure~\ref{fig:neurospheroid_figure}C) and shown directly below in panel A}.
\begin{figure}[H] 
	\centering
	\begin{subfigure}[t]{.6\textwidth}
		\caption{Abeta ELISA on the media from cortical organoids (aged to six months, with 4 weeks of CDP-choline treatment (1mM)).}
		\includegraphics[width=\textwidth]{./main_plots/abeta_elisa.png}        
	\end{subfigure}  
\end{figure}

\dots Images with more cells within the same ROI should be provided, for example like in Fig.4H. Same applies to Fig 5A, 5D and 5E.\\
\textcolor{blue}{We have now updated all images to include more cells within the same ROI. See Figure~\ref{fig:main_mitochondrial}H-J; Figure~\ref{fig:main_choline}H,I; Figure~\ref{fig:choline_treatment}L.}

Is the number of up- and down-regulated PCs highlighted in Fig. 3K the sameas in Fig.3I.\\
\textcolor{blue}{We thank the reviewer for highlighting this inconsistency. We have now revised the relevant panels, consistently applying the criteria of adjusted p-value < 0.05 and absolute log2 fold change > 1. These updated data are presented in Figure~\ref{fig:main_lipids}A,B,D,E}.

Line 365-380, the lipidomics analysis is unclear, oversimplified, and seemingly biased which I am afraid may affect accurate data interpretation. Here are my comments: 1) in Fig. 3I, the PC lipid species display 13 up- and 8 down-regluated species, but in Fig. 3Q, the relative PC abundance in total is even decreased, the two results are contradictory. 2) which PC lipid species were selected to calculate the ratio of PC:TG. TG species are uniformly increased, which can predominantly drive the decrease in this ratio, likely not specifically related to PC. 3) For MG and DG, when nearly no lipids in each class are significantly changed, interpreting the relative abundance of the entire lipid class is difficult and highly speculative. Based on the lipdidomics results, TG seems to accumulate and the complex changes in PC and other lipids are suggestive. But the conclusion that diglycerides are less often converted into phosphatidylcholine and more often into triglycerides in the presence of ABCA LoF is highly speculative. It is important here to measure lipid flux using istope tracers to confirm the conclusion.\\
\textcolor{blue}{We agree with the reviewer that our initial analysis was overly complex and that emphasizing lipid ratios could lead to misinterpretation. To address this, we have substantially revised our lipidomics analysis comparing p.Glu50fs*3 vs. WT iNs and added a new lipidomics experiment using the second ABCA7 LoF line (p.Tyr622* vs. WT iNs). We removed relative abundance analyses, instead highlighting specific changes at the individual lipid species level. These revisions allowed for a more detailed evaluation of phosphatidylcholine changes, which we discuss below.}\\
\textcolor{blue}{\underline{Regarding comment 1):} Instead of aggregating phosphatidylcholine species into a single ratio, we now categorize individual species by their fatty acid saturation profiles (Figure~\ref{fig:main_lipids}D,E). This analysis revealed that phosphatidylcholine species up-regulated in p.Glu50fs*3 vs. WT iNs  were significantly enriched for saturated fatty acids, while several down-regulated species contained highly polyunsaturated fatty acids. We confirmed a similar increase in saturated phosphatidylcholine species in p.Tyr622* vs. WT iNs  (Figure~\ref{fig:main_lipids}G,H and Figure~\ref{fig:main_mitochondrial}K). Additionally, mRNA sequencing of p.Tyr622*, p.Glu50fs*3, and WT iNs indicated decreased expression of key phosphatidylcholine remodeling enzymes, including LPCAT3, which incorporates polyunsaturated fatty acids into phosphatidylcholine (Figure~\ref{fig:main_lipids}K,L). Together, these findings reveal complex alterations in phosphatidylcholine abundance and saturation in ABCA7 LoF neurons. We have updated the subsection "ABCA7 LoF induces phosphatidylcholine imbalance in neurons" to reflect these new data [page~\pageref{quoteA-label}]:}
\begin{quote}
	\textcolor{blue}{"}\quoteA\\\\
	\quoteB"\textcolor{blue}{"}
\end{quote}  
\textcolor{blue}{\underline{Regarding comments 2 and 3):} We have removed the relative abundance panels (Fig. 3 L–R) and simply provide the unbiased visualizations of individual lipid species grouped by major lipid categories in Figure~\ref{fig:main_lipids}A,B,G,H. These updated panels highlight changes at the individual lipid-species level rather than at the lipid- class level.}

Line 380: it is Fig. S10C (instead of Fig. S9C). Furthermore, the observed trends in TG and PC levels in human samples is minimal and not significant to suggest increased TG and decreased PC levels in ABCA7 LoF carriers. These data should be removed.\\
 \textcolor{blue}{We have now removed these data from the paper.}

The increase of TG in ABCA7 LoF should be confirmed by lipid staining.\\
\textcolor{blue}{Our LC-MS data for p.Glu50fs*3 revealed increased levels of long-chain triglyceride species with a total carbon count of $\geq$32 (Figure~\ref{fig:main_lipids}C), consistent with triglyceride species recently observed to accumulate in iNs in the context of APOE4 \cite{Haney2024-bp}. However, these longer-chain triglycerides were not detected in a subsequent LC-MS experiment on p.Tyr622* iNs, where we instead observed fewer and only shorter-chain triglycerides ($\leq$32 carbons; Figure~\ref{fig:main_lipids}C vs. Figure~\ref{fig:main_lipids}I). This discrepancy may reflect technical variability in the lipidomics process or biological variability due to, eg, instability of certain triglyceride species. Therefore, we have adjusted our interpretation to emphasize the robust and consistent increase in saturated phosphatidylcholine species detected in both ABCA7 LoF lines, while presenting triglyceride findings more cautiously [page~\pageref{quoteH-label}]:}
\begin{quote}
	\textcolor{blue}{"\quoteH"}
\end{quote}

Fig. 4A: In the PCA analysis, PC1 and PC2 explain only 0.18\% and 0.15\% of the variations, which are extremely low. Was it a mistake? If not, this indicates that the differences between the two groups are not actually captured by PC1 and PC2. The authors should review their analysis and consider testing other dimension reduction methods.\\
\textcolor{blue}{We thank the reviewer for pointing out this error. PC1 and PC2 explain 18\% and 15\% of the variance, respectively, not 0.18\% and 0.15\%.}

Fig. 4B, what were the upregulated metabolites?\\
\textcolor{blue}{Only a small fraction of detected metabolites could be confidently annotated, and none of the up-regulated metabolites could be annotated with high confidence. While the metabolomics data are intriguing, we acknowledge that they have caused confusion among reviewers. Thus, we now limit our discussion of ABCA7 LoF metabolomics data to supplementary materials in the context of our CDP-choline rescue (Figure~\ref{fig:choline_treatment}F). Although our data indicate notable differences in overall metabolite composition between ABCA7 LoF and WT iNs, further studies are required to fully characterize and interpret these metabolic changes.} 

Fig. 4D, what is the significance of the correlation. If the dots in each group in the correlation plots represent technical replicates, the aggregated data (mean or median) should be used here.\\
\textcolor{blue}{Another reviewer expressed similar concerns regarding the interpretation and clarity of these plots, and asked us to remove these plots, which we did.}

Fig. 4E, how OCR data were normalized should be explained.

\textcolor{blue}{We have now updated the text to more clearly explain this point [page~\pageref{quoteC-label}]:}
\begin{quote}
	\textcolor{blue}{"\quoteC"}
\end{quote}

Line 457-459, CDP-choline cannot be directly taken up by cells. Although cells may uptake its hydrolyzed products, it remains important to confirm the cellular increase of CDP-choline.\\
\textcolor{blue}{We thank the reviewer for raising this important point. To determine whether CDP-choline treatment increases cellular choline levels, we performed targeted LC-MS analyses of both media and cell extracts with and without CDP-choline treatment. Our experiments revealed that: (1) CDP-choline treatment led to accumulation of CDP and choline in media conditioned by p.Tyr622* iNs but not in media without cells, indicating—as suggested by the reviewer—that iNs hydrolyze extracellular CDP-choline into CDP and choline (Figure~\ref{fig:choline_treatment}A, and Panel A below); and (2) CDP-choline treatment led to a  significant increase of choline in cell extracts (Figure~\ref{fig:choline_treatment}B, and Panel B below). While CDP-choline was only detected at low levels in cellular extracts, the number of  p.Tyr622* samples with detectable CDP-choline increased from 1 of 8 (13\%) to 4 of 8 (50\%) after CDP-choline treatment (Extended Data~\ref{data:targeted_metabolomics}).}\\\\
\textcolor{blue}{In agreement with this, mRNA sequencing after CDP-choline treatment showed increased expression of choline transporters (Figure~\ref{fig:choline_treatment}C). Taken together, these data suggest that CDP-choline treatment leads to cellular uptake of choline.}\\\\
\textcolor{blue}{We have now updated the text with a detailed discussion of these results [page~\pageref{quoteD-label}]:}
\begin{quote}
	\textcolor{blue}{"\quoteD"}
\end{quote}

\begin{figure}[H] 
	\begin{subfigure}[t]{\textwidth}
		\caption{Targeted LC-MS of media with and without p.Tyr622* iN, in the presence and absence of CDP-choline treatment.}
		\includegraphics[width=\textwidth]{./extended_plots/choline_media_LCMS.png}        
	\end{subfigure}
	\centering
	\begin{subfigure}[t]{.4\textwidth}
		\caption{Targeted LC-MS of p.Tyr622* iN cells, in the presence and absence of CDP-choline treatment.}
		\includegraphics[width=\textwidth]{./extended_plots/choline_in_cells_LCMS.png}        
	\end{subfigure}
\end{figure}

Line 453-455, from the data provided, it is unclear how PC and mitochondrial function are linked in ABCA7 LoF. Few questions remain to be addressed,
\begin{itemize}
	\item whether the change of PC affects mitochondrial membrane structure in ABCA7 LOF?

	\item whether CDP-choline increases cellular PC levels and whether this increase affects mitochondrial membrane structure in ABCA7 LoF?
	
	\item Whether CDP-choline specifically decreases the TG species that accumulate in ABCA7 LoF or affects different groups of TGs also remains to be determined.
\end{itemize}

\textcolor{blue}{\underline{Regarding item \# 1:} We attempted to visualize mitochondrial membrane structure using both TEM and expansion microscopy. For expansion microscopy, we first successfully tested a single-shot 20x expansion protocol (https://www.nature.com/articles/s41592-024-02454-9) in HEK cells fixed with glutaraldehyde, achieving good preservation and visualization of mitochondrial architecture. However, induced neurons could not be reliably grown on cover glass, preventing their expansion due to poor cell adhesion. We also attempted the Magnify expansion microscopy protocol (https://www.nature.com/articles/s41587-022-01546-1) on paraformaldehyde-fixed spheroids. While we could visualize mitochondrial networks, the individual mitochondrial structures remained difficult to interpret clearly, likely due to limited expansion factor and insufficient organelle preservation with paraformaldehyde. Thus, both approaches would require substantial additional optimization beyond the scope of this manuscript, and we believe this is better addressed in future follow-up studies.}

\textcolor{blue}{\underline{Regarding item \#2:} We newly performed mRNA analysis on p.Tyr622* iNs with and without CDP-choline treatment, which revealed increased expression of PCYT1B (Figure~\ref{fig:choline_treatment}C), the rate-limiting enzyme for converting choline into CDP-choline \cite{Lykidis1998-rj}, after treatment. Consistent with this, LC-MS analysis of CDP-choline-treated p.Tyr622* cells demonstrated significant (p<0.05, |logFC|>1) increases in multiple phosphatidylcholine species, as well as in lysophosphatidylcholine—a phosphatidylcholine remodeling intermediate of the Land’s cycle (Figure~\ref{fig:main_choline}A; and Panel A directly below). Expression of all known LPCATs, which catalyze phosphatidylcholine-remodeling, was also increased  after treatment (Figure~\ref{fig:choline_treatment}D; and Panel B directly below). These findings indicate that CDP-choline enhances phosphatidylcholine synthesis and remodeling, which is likely to impact mitochondrial membrane structure. We have moved previous discussions regarding the relationship between phosphatidylcholine and mitochondrial function (previously Line 453-455) to the discussion on page~\pageref{quoteI-label}, which now reads as follows:}

\begin{quote}
	\textcolor{blue}{"\quoteI"}
\end{quote}

\begin{figure}[H] 
	% \begin{subfigure}[t]{0.4\textwidth}
	% 	\caption{Up-regulated PCYT1B mRNA after CDP-choline treatment.}
	% 	\includegraphics[width=\textwidth]{./extended_plots/choline_synth_genes.png}        
	% \end{subfigure}
	\begin{subfigure}[t]{0.6\textwidth}
		\caption{LC-MS lipidomic analysis of CDP-choline-treated p.Tyr622* neurons.}
		\includegraphics[width=\textwidth]{./main_plots/all_lipids_choline_batch1.png}        
	\end{subfigure}
	\begin{subfigure}[t]{0.4\textwidth}
		\caption{Up-regulated LPCAT mRNA after CDP-choline treatment.}
		\includegraphics[width=\textwidth]{./extended_plots/choline_lpcat.png}        
	\end{subfigure}
\end{figure}

\textcolor{blue}{\underline{Regarding item \#3:} We did  observe a modest decrease in triglyceride species following CDP-choline treatment (Figure~\ref{fig:main_choline}A; and Panel A directly above). However, since we did not detect  long-chain triglycerides in p.Tyr622* iNs – the lipid species previously shown to accumulate in p.Glu50fs*3 iNs (Figure~\ref{fig:main_lipids}I vs Figure~\ref{fig:main_lipids}C), we are cautious about drawing definitive conclusions regarding triglyceride accumulation in response to CDP-choline treatment, and we have updated the manuscript text accordingly to reflect this nuanced interpretation.}

Are the dyes used in Fig. 4H and Fig.5C the same? If different, both should be examined in Fig. 4H and 5C.

\textcolor{blue}{These are the same dyes.}

Line 479-487: The clearance of amyloid-β by CDP-choline is interesting yet perplexing. In the secretase model, it is unclear whether the clearance is related to the mitochondrial membrane or the membranes of other organelles via TAG or PC. More importantly, it is unclear whether the mechanisms involving TG and PC are actually causal for ABCA7 LOF-associated AD phenotypes or risk, and whether CDP-choline has therapeutic potential for AD. A mouse model of ABCA7 LOF-induced AD is necessary to address these questions.

\textcolor{blue}{\underline{Regarding the secretase data:} We agree with the reviewer that these findings are intriguing. However, investigating the precise mechanism through which CDP-choline reduces Aβ levels—now confirmed by ELISA in media from iNs and aged cortical organoids—is outside the main focus of the current study. Therefore, we have removed the secretase data from the manuscript.}

\textcolor{blue}{\underline{Regarding the causality of lipid dysregulation in ABCA7 LoF-associated AD:} We believe that our updated analyses provide strong support for an upstream role of phosphatidylcholine imbalance in ABCA7 LoF-associated risk. Specifically: (1) CDP-choline treatment significantly reversed ABCA7 LoF-related transcriptional, metabolic, and mitochondrial phenotypes, restoring them toward WT levels (updated Figure~\ref{fig:main_choline}); (2) targeted and untargeted metabolomics and lipidomics experiments indicate that CDP-choline treatment enhances synthesis and remodeling of phosphatidylcholine and its derivatives (Figure~\ref{fig:main_choline}A; Figure~\ref{fig:choline_treatment}A-D); and (3) phosphatidylcholine species were consistently perturbed across both p.Glu50fs*3 and p.Tyr622* induced neurons (Figure~\ref{fig:main_mitochondrial}K; Figure~\ref{fig:main_lipids}). Together, these findings highlight phosphatidylcholine imbalance in ABCA7 LoF neurons and demonstrate that intervening in phosphatidylcholine metabolism ameliorates ABCA7 LoF-associated defects. Given ABCA7’s established role as a phospholipid transporter, including phosphatidylcholine transport, our data strongly support phosphatidylcholine imbalance as an upstream mechanism driving dysfunction in ABCA7 LoF neurons.}

\textcolor{blue}{\underline{Regarding the therapeutic potential of CDP-choline:} We demonstrate that CDP-choline treatment effectively reduces several key Alzheimer’s disease-related pathologies, including mitochondrial dysfunction, reactive oxygen species accumulation (Figure~\ref{fig:main_choline}I), amyloid pathology (Figure~\ref{fig:main_choline}J), and neuronal hyperexcitability (Figure~\ref{fig:main_choline}K). Importantly, our snRNAseq data suggest these benefits may generalize to carriers of the common ABCA7 variant (p.Ala1527Gly) (Figure~\ref{fig:main_neurons}), suggesting that CDP-choline—a readily available dietary supplement—may represent a promising therapeutic strategy to reduce AD risk in broader populations. This conclusion is further supported by recent findings from our lab showing CDP-choline’s effectiveness in reversing APOE4-associated cellular phenotypes \cite{Sienski2021-zt} and linking choline metabolism to cognitive resilience in AD \cite{Mathys2024-ex}. Choline supplementation is currently under clinical investigation as an AD treatment strategy \cite{Cummings2024-cu}. While testing CDP-choline efficacy in a mouse model would be valuable, the work presented here is focused on more mechanistic studies in human cells.}

Line 161-164, line 215-217, line 256-260, across many datasets, DNA damage/repair pathways were detected; this was not explored further. Yet DNA damage/repair is closely related to lipid dysregulation and protein aggregation, it is therefore interesting to explore whether DNA damage/repair was involved in the phenotypic changes upon ABCA7 LoF.\\
\textcolor{blue}{We thank the reviewer for this insightful point. Given that DNA damage pathways consistently emerged in our RNAseq datasets, we agree that exploring this connection further is valuable. DNA damage frequently occurs downstream of increased reactive oxygen species (ROS) \cite{Welch2022-bp}, and decreased mitochondrial uncoupling—as observed in our ABCA7 LoF lines—is well-known to cause oxidative damage in cells \cite{Jain2024-br, Crivelli2024-pf}. We directly tested for increased ROS generation by using CellROX staining and found significantly increased ROS levels in ABCA7 LoF neurons (Figure~\ref{fig:main_mitochondrial}J; and Panel A below). We also showed that this increase in ROS was effectively reversed by CDP-choline treatment (Figure~\ref{fig:main_choline}I), notably in parallel to normalization of mitochondrial uncoupling (Figure~\ref{fig:main_choline}G). We have updated the Results and Discussion sections accordingly, highlighting this clearer connection between mitochondrial dysfunction, ROS accumulation, and potential genotoxic effects in ABCA7 LoF neurons.}

\begin{figure}[H] 
	\centering
	\begin{subfigure}[t]{.7\textwidth}
		\caption{CellROX staining of ABCA7 LoF iNs}
		\includegraphics[width=\textwidth]{./main_plots/cellrox_images.png}        
	\end{subfigure}  
\end{figure}