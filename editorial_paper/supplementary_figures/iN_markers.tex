\begin{figure}[H]
    \begin{subfigure}[t]{\textwidth}
        \begin{subfigure}[t]{0.45\textwidth}
            \caption{}
            \includegraphics[width=\textwidth]{../paper/extended_plots/iN_induction_cartoon.png}        
        \end{subfigure}
        \begin{subfigure}[t]{0.45\textwidth}
            \caption{}
            \includegraphics[width=\textwidth]{../paper/extended_plots/iN_markers_prev.png}        
        \end{subfigure}
    \end{subfigure}
    \begin{subfigure}[t]{0.8\textwidth}
        \caption{}
        \hspace{2cm}
        \includegraphics[width=\textwidth]{../paper/extended_plots/iN_markers.png}        
    \end{subfigure}
\end{figure}
\textbf{Fig. S5: Differentiating iPSC-Derived Neurons Harboring ABCA7 PTC Variants.}
\textbf{a,} iPSCs were plated at low density for NGN2 viral transduction. Expression of NGN2 was driven by doxycycline (DOX) induction with puromycin (PURO) selection, then cells were replated to match neuronal densities. Neurons were maintained for 4 weeks (DIV 28) before experimentation (Created with BioRender.com). 
\textbf{b,} Neuronal marker expression in 2 and 4-week matured iNs. 
\textbf{c,} Neuronal marker expression in iNs matured for 4 weeks for indicated genotypes. CDP-choline treatment was applied for 2 weeks at 100 $\mu$M.