\textbf{Extended Data Figure Legends}\newline\newline
\textbf{Extended Data Fig 1. Overview of Human snRNA-Sequencing Cohort.}\\
\textbf{a,} Protein levels from post-mortem human prefrontal cortex (see Table~\ref{tab:external_datasets} for external dataset used) showing ABCA7 protein levels (left) and NeuN (RBFOX3) levels (middle) for a subset of individuals found to overlap with the snRNA-seq cohort (N=6 control and N=4 ABCA7 LoF carriers). The right panel shows NeuN (RBFOX3) protein levels by genotype in all available control samples (N=180) vs. all available ABCA7 LoF carriers with proteomic data (N=5).
\textbf{b,} Distributions of continuous metadata variables (see Supplementary Text for descriptions) for control individuals (N=24) vs. ABCA7 LoF carriers (N=12). For panels C and D, boxes indicate dataset quartiles per condition, and whiskers extend to the most extreme data points not considered outliers (i.e., within 1.5 times the interquartile range from the first or third quartile). 
\textbf{c,} Distributions of discrete metadata variables for control individuals (N=24) vs. ABCA7 LoF carriers (N=12). Con=control, LoF=ABCA7 loss-of-function. P-values in panels A and B were computed by two-sided Wilcoxon rank sum test. P-values in panel C were computed by two-sided Fisher’s exact test.
\textbf{d,} Sanger sequencing of ABCA7 LoF variants in prefrontal cortex genomic DNA samples from 3 ABCA7 LoF carriers and 3 controls from the snRNA-seq cohort. Sequencing confirmed heterozygosity of the indicated variant in LoF samples, with variant location marked by a black box. 
\textbf{e,} Example plots validating matches between whole genome sequencing (WGS) and snRNA-seq libraries. Each plot shows the concordance of homo- and heterozygous SNP calls between WGS and snRNA-seq data for a single individual. Matches between WGS SNP calls and snRNA-seq BAM inferred SNP calls are indicated by extreme outliers. Expected (i.e., correct) matches are indicated in blue/purple. 

\textbf{Extended Data Fig 2. Annotated Projections of Gene Scores.}\\
\textbf{a,} Enlarged view of the UMAP projection of Figure~\ref{fig:main_atlas}E, showing the top 20 genes by absolute score ($|S|$) for each cell type. 

\textbf{Extended Data Fig 3. Shared Differentially Expressed Genes by Cell Type.}\\
\textbf{a,} Heatmap indicating the overlap of differentially expressed genes between cell types (genes, where p-value < 0.05 in at least three cell types). 
\textbf{b,} Functional annotations of genes in the same order as in the heatmap in A.

\textbf{Extended Data Fig 4. Neuronal Expression of ABCA7 in the Postmortem Human Brain.}\\
\textbf{a,} Per cell type ABCA7 detection rate of major cell types in the \textit{postmortem} PFC as quantified by snRNA-seq. 
\textbf{b,} Normalized expression of indicated gene in glial cells (per-individual mean expression profiles across Oli, Opc, Ast, Mic) vs. neuronal cells (per-individual mean expression profiles across Ex and In) from \textit{postmortem} snRNA-seq data. 
\textbf{c,} Normalized expression of indicated genes in NeuN- vs. NeuN+ cells (N=6 individuals, from \supercite{Welch2022-ef}; Table~\ref{tab:external_datasets}). All p-values are computed by paired two-sided t-test. Boxes indicate per-condition dataset quartiles, and whiskers extend to the most extreme data points not considered outliers (i.e., within 1.5 times the interquartile range from the first or third quartile).

\textbf{Extended Data Fig 5. Annotation of Excitatory Neurons from \textit{postmortem} snRNAseq Dataset by Cortical Layer.}\\
\textbf{a,} UMAP visualization of excitatory neurons annotated by cortical layers based on Leiden clustering. 
\textbf{b,} Heatmap showing enrichment of cortical layer-specific marker genes from \supercite{He2017-dq} across annotated layers. Color indicates average marker gene expression (log2 fold change) for each layer marker gene set with respect to all other clusters.    
\textbf{c,} Heatmap displaying validation of layer annotations using an independent set of cortical layer marker genes from \supercite{Maynard2021-mz}. Color represents average marker gene expression (log2 fold change) for each layer marker gene set with respect to all other clusters.
\textbf{d,} ABCA7 LoF-associated perturbations of excitatory neuronal PM gene clusters by layer (computed by FGSEA, see Methods).

\textbf{Extended Data Fig 6. Molecular Dynamics Simulations of ABCA7 Open Conformations with p.Ala1527Gly Substitution.}\\
\textbf{a,} Open conformation ABCA7 protein structure. ABCA7 domain between residues 1517 and 1756 used for simulations is shown in yellow. Expanded yellow domain (inset from left), with A1527 variant (light grey) and G1527 variant (purple). 
\textbf{b,} Expanded inset from A. 
\textbf{c,} Root mean squared deviations of open conformation domains from B with A1527 (light grey) or G1527 (purple) under simulation. Structural deviations over time were computed with respect to reference open structures from B. 
\textbf{d,} Projection of $C_\alpha$ atom positional fluctuations under simulation onto the first two principal components, for open conformation domain from B with A1527 (top, light grey) or G1527 (bottom, purple). 
\textbf{e,} Violin plot indicating average $C_\alpha$ atom positional fluctuations over time. 

\textbf{Extended Data Fig 7. Analysis of Local Conformational Fluctuations and Secondary Structure Variations Induced by the p.Ala1527Gly Substitution in ABCA7 Open and Closed Conformations.}\\
\textbf{a,} Phi vs. Psi dihedral angle distribution of residue 1527 as a function of simulation time for open and closed conformations.
\textbf{b,} Overall Phi vs. Psi dihedral angle distributions for residue 1527, comparing open and closed conformations throughout the entire simulation period.
\textbf{c,} Time-resolved secondary structure assignments for residues 1517–1537. Alpha-helical structures are highlighted in red; other colors represent distinct secondary structures.
\textbf{d,} Fraction of alpha-helical content observed for residues 1517–1537 during the simulations. A value of 1 indicates uninterrupted preservation of the alpha-helical structure throughout the simulation duration.
\textbf{e,} ABCA1 (cyan) closed structure; PDB ID: 7TBW. ABCA7 (purple) closed structure; PDB ID: 8EOP. Positions of Gly1527 in ABCA7 and V1646 in ABCA1 are indicated as spheres.
\textbf{f,} ABCA4 (green) closed structure; PDB ID: 7LKZ. ABCA7 (purple) closed structure; PDB ID: 8EOP. Positions of Gly1527 in ABCA7 and I1671 in ABCA4 are indicated as spheres.

\textbf{Extended Data Fig 8. Measuring activity of iPSC-derived neurons harboring ABCA7 PTC variants.}\\
\textbf{a,} Representative sweeps show action potentials elisupercited by 800 ms of current injections in patched 4-week-old iNs.
\textbf{b,} Summary of action potential frequency (means ± SEM) elisupercited with different amounts of injected current in 4-week-old iNs. 
\textbf{c,} Representative sweeps of whole-cell current flow of inward (upper panel) and outward (lower panel) current recordings from WT 4-week-old neurons. 
\textbf{d,} Quantification of (C). 
\textbf{e,} Resting membrane potential (mV) of 4-week-old WT, ABCA7 p.Tyr622*, and ABCA7 p.Glufs*3 neurons. 
\textbf{f,} Rheobase (pA) of 4-week-old WT, ABCA7 p.Tyr622*, and ABCA7 p.Glufs*3 neurons. 
\textbf{g,} Action potential frequency of 4-week-old WT, ABCA7 p.Tyr622*, and ABCA7 p.Glufs*3 neurons with indicated current injections. For panels E-G: WT: $n=24$; Y622: $n=13$; G2: $n=23$. For all panels: $P*<0.05$, $P***<0.001$. Graphs are mean ± SEM.

\textbf{Extended Data Fig 9. mRNA-seq analysis of p.Glu50fs* vs. WT iNs.}\\
\textbf{a,} Kernighan-Lin (K/L) clustering of leading-edge genes from significantly perturbed pathways (Benjamini–Hochberg (BH) FDR-adjusted $p<0.05$) in p.Glu50fs*3 vs. WT iNs. Colors represent distinct K/L clusters.
\textbf{b,} Heatmap of Jaccard index overlap between K/L gene clusters from p.Glu50fs*3 neurons and clusters identified in p.Tyr622* vs WT iNs. Red text denotes clusters with average score $S$ upregulated in ABCA7 LoF; blue text denotes clusters with average $S$ downregulated in ABCA7 LoF.
\textbf{c,} Heatmap of Jaccard index overlap between K/L gene clusters from p.Glu50fs*3 neurons and clusters identified in human postmortem excitatory neurons.
\textbf{d,} Gaussian kernel density plots of gene perturbation scores ($S$) within each cluster. Positive $S$ indicates upregulation in p.Glu50fs*3. Solid lines denote cluster means. Top enriched pathways with highest intra-cluster connectivity indicated.
\textbf{e,} Correlation of per-gene perturbation scores ($S = -\log_{10}(\text{p-value}) \times \text{sign}(\log_2(\text{fold change}))$) between p.Glu50fs*3 vs. WT and p.Tyr622* vs. WT iNs for MitoCarta genes.

\textbf{Extended Data Fig 10. Analysis of Oxygen Consumption Rates in ABCA7 LoF vs. Control iNs.}\\
\textbf{a,} Example oxygen consumption rate (OCR) curves from Batch 1 of the two differentiation batches used for analysis in Figure~\ref{fig:main_mitochondrial}. The line plot indicates the per-condition mean estimator, and the error bars indicate the 95\% confidence interval. 
\textbf{b,} Representative per-well traces from (A). 
\textbf{c,} Schematic indicating the relationship between oxygen consumption as a measure of proton current (I), which sustains the proton motive force (voltage, V). Regulation of ATP synthase and uncoupling protein (UCP) activity modifies resistance (R) and depletes the proton motive force.
\textbf{d,} Schematic indicating measurement of maximal and basal oxygen consumption to compute SRC.
\textbf{e,} Schematic indicating measurement of uncoupled oxygen consumption.
\textbf{f,} SRC computed for WT, ABCA7 p.Glu50fs*3, and ABCA7 p.Tyr622* iNs. P-values computed by independent sample t-test. $N$ wells = 18 (WT), 17 (p.Tyr622*), 13 (p.Glu50fs*3) across two independent differentiation batches and Seahorse experiments. 
\textbf{g,} UCP2 mRNA levels by genotype. 
\textbf{h,} Mitochondrial membrane potential quantified by average TMRM fluorescence intensity per masked region (thresholded at 75th percentile) under baseline conditions and after addition of FCCP in ABCA7 LoF and WT iNs cultured for 4 weeks. Each datapoint represents average intensity per well. $N=4$ (WT), $5$ (p.Tyr622*) wells. Statistical comparison by independent-sample t-test. Same plot and images as shown under baseline conditions in Figure~\ref{fig:main_mitochondrial}.

\textbf{Extended Data Fig 11. LC-MS Lipidomics in ABCA7 LoF iNs.}\\
\textbf{a,} Volcano plot of significantly perturbed lipid species identified by LC-MS in p.Glu50fs*3 vs. WT iNs (BH FDR-adjusted $p<0.05$, $|\log_2(\text{FC})|>1$), colored by lipid class. $N=6$ wells per genotype.
\textbf{b,} Significantly perturbed lipid species from (A), summarized by lipid subclass.
\textbf{c,} Distribution of triglyceride fold changes grouped by fatty acid chain length and saturation for p.Glu50fs*3 vs. WT iNs.
\textbf{d,} Volcano plot highlighting significantly perturbed phosphatidylcholine species containing saturated or monounsaturated fatty acids (SFA/MUFA; BH FDR-adjusted $p<0.05$, $|\log_2(\text{FC})|>1$) for p.Glu50fs*3 vs. WT iNs.
\textbf{e,} Volcano plot highlighting significantly perturbed phosphatidylcholine species containing polyunsaturated fatty acids (PUFA; BH FDR-adjusted $p<0.05$, $|\log_2(\text{FC})|>1$) for p.Glu50fs*3 vs. WT iNs.
\textbf{f,} Distribution of phosphatidylcholine fold changes grouped by fatty acid chain length and saturation for p.Glu50fs*3 vs. WT iNs.
\textbf{g,} Table summarizing significantly perturbed lipid species by lipid subclass in p.Tyr622* vs. WT iNs ($N=10$ WT, $8$ p.Tyr622* wells; BH FDR-adjusted $p<0.05$, $|\log_2(\text{FC})|>1$).
\textbf{h,} Volcano plot showing lipid species from (G), colored by lipid class; phosphatidylcholines highlighted in blue, for p.Tyr622* vs. WT iNs.
\textbf{i,} Same analysis as (C,F), but comparing p.Tyr622* vs. WT iNs.
\textbf{j,} Expression changes (mRNA) of LPCAT genes comparing p.Tyr622* vs. WT and p.Glu50fs*3 vs. WT iNs.

\textbf{Extended Data Fig 12. Choline Metabolism in ABCA7 LoF iNs.}\\
\textbf{a,} Choline metabolites detected in media by targeted LC-MS. $N=2$ for media without cells, $N=4$ for cell-conditioned media; N/F indicates not detected.
\textbf{b,} Choline metabolites detected intracellularly by targeted LC-MS. $N=8$ per genotype, $N=4$ blanks.
\textbf{c,} Selected choline synthesis and transport genes differentially expressed in p.Tyr622* $\pm$ CDP-choline iNs (mRNA-seq).
\textbf{d,} LPCAT gene expression changes in p.Tyr622* $\pm$ CDP-choline iNs (mRNA-seq).
\textbf{e,} Distribution of phosphatidylcholine species fold-changes by fatty acid chain length and saturation in p.Tyr622* $\pm$ CDP-choline iNs.
\textbf{f,} PCA plot of untargeted LC-MS metabolite profiles from p.Tyr622* $\pm$ CDP-choline and WT iNs.
\textbf{g,} PCA plot of mRNA-seq data from p.Tyr622* $\pm$ CDP-choline iNs.
\textbf{h,} Correlation of gene perturbation scores ($S$) for mitochondrial-localized genes comparing p.Tyr622* $\pm$ CDP-choline iNs versus p.Glu50fs*3 vs. WT iNs.
\textbf{i,} Example Seahorse oxygen consumption rate (OCR) curves. Lines represent per-condition means; error bars indicate 95\% confidence intervals.
\textbf{j,} Representative per-well OCR traces from (I).
\textbf{k,} Quantification of spare respiratory capacity (SRC) from OCR curves in (I). Statistical comparisons by independent-sample $t$-tests; $N=6$ wells (p.Tyr622* + H$_2$O), $N=8$ wells (p.Tyr622* + CDP-choline). Boxes indicate quartiles; whiskers extend to data within 1.5$\times$IQR from quartiles.
\textbf{l,} Quantification of mitochondrial membrane potential via neuronal HCS MitoHealth dye fluorescence intensity. Statistical comparisons by linear mixed-effects model on per-NeuN+ volume averages, with well-of-origin as random effect. $N=11$ wells (p.Tyr622*), $N=9$ wells (p.Glu50fs*3), $N=8$ wells (WT); $\sim3000$ cells per condition from three differentiation batches. Individual data points represent per-well averages. Right: Representative images shown as mean-intensity projections of NeuN+ staining and corresponding MitoHealth and Hoechst signals within quantified NeuN+ volumes. Each NeuN/GFP image intensity was scaled relative to its maximum value, followed by gamma correction ($\gamma = 0.5$) for visualization.

\textbf{Extended Data Fig 13. CDP-choline Treatment in Cortical Organoids.}\\
\textbf{a,} Amyloid-$\beta$ levels quantified by ELISA from media of 4-week-old iNs.
\textbf{b,} Representative images of cortical organoid slices from indicated genotypes.
\textbf{c,} Amyloid-$\beta$ levels quantified by ELISA from media of cortical organoids (176 days old), grouped by genotype and treated with 500~$\mu$M or 1~mM CDP-choline for 3 weeks. Samples correspond to organoids in Figure~\ref{fig:main_choline}K, analyzed one week prior to assays presented there.
