\begin{figure}[ht]
    %\centerline{\includegraphics[width=\textwidth]{./extended_plots/lipid_mitochondrial_perturbations.pdf}}
    \caption{
        \textbf{Model of ABCA7 LoF dysfunction in neurons.}\\[1ex]
        (A) Lipid synthesis and storage pathways perturbed in ABCA7 LoF excitatory neurons vs. control as measured by snRNA-seq on the post-mortem human PFC. Enrichments of biological processes were computed using FGSEA. Red = enrichment > 0, Blue = enrichment < 0. * = $p<0.05$. 
        (B) Schematic model showing anabolic processes feeding from the TCA cycle towards fatty acid (FA) and triglyceride (TG) synthesis. DG = diacylglyceride, PA = phosphatidic acid, PC = phosphatidylcholine, PE = phosphatidylethanolamine, PS = phosphatidylserine. * = differentially expressed in ABCA7 LoF vs. control excitatory neurons from post-mortem human brain at $p<0.05$ and $\log\text{FC}<0$. 
        (C) 𝛽-oxidation and TCA pathways perturbed in ABCA7 LoF excitatory neurons vs. control as measured by snRNA-seq on the post-mortem human PFC. Enrichments of biological processes were computed using FGSEA. Red = enrichment > 0, Blue = enrichment < 0. * = $p<0.05$. 
        (D) Schematic model showing catabolic processes feeding into the TCA cycle and oxidative phosphorylation with key genes from (C) highlighted in red or blue. * = $p<0.05$. For (A, C,) boxes indicate per-condition dataset quartiles, and whiskers extend to the most extreme data points not considered outliers (i.e., within 1.5 times the interquartile range from the first or third quartile). 
        (E, F) Transcript levels of ACLY (E) and SCP2 (F) assessed in post-mortem human PFC by RNAscope. Transcript counts per SLC17A7+ cell are reported in each bar chart. $N = 8$ individuals per genotype. Per-cell Wilcoxon rank-sum p-values are reported.
    }
    \label{fig:lipid_mitochondrial_perturbations}
\end{figure}