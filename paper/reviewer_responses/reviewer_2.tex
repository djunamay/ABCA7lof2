\subsubsection*{Referee #2 (Remarks to the Author)}
LOF of ABCA7 increases risk of AD. the studies of its disease mechanisms are mostly done in animal models or in vitro cell models, which greatly limited their relavance to human disease. This study performed single-nucleus RNA sequencing (snRNA-seq) on rare ABCA7 loss-of-functon(LOF) carriers as well as an independent group of ABCA7 missense variants from the ROSMAP cohort. The study uncovered convergent neuronal pathways perturbed by ABCA7 variants, including those related to lipid metabolism and mitochondrial dysfunction. Further, robust and comprehensive omics analyses (lipidomics and metabolomics) were conducted to validate these observations using functional cellular assays in human induced pluripotent stem cell (iPSC)-derived neurons. Significantly, the study demonstrated that CDP-choline rescued LOF-induced dysregulation in mitochondrial function, lipid droplets, and amyloid biosynthesis. The mechanistic link between lipid disturbance, mitochondrial dysfunction, and amyloid pathology is novel. The robustness of the conclusions could be strengthened with additional controls.

\textbf{Major comments:}

Figure 1D shows the combined demographics. It’s important to show the diagnosis, Cerad score and Braak scores, and gender distribution, stratified by the ABCA7 genotypes (control and LOF).

\textcolor{blue}{We ****thank the reviewer for pointing this out. We purposefully show the overall demographics to emphasize the context in which our single-cell observations are made. While we do show ABCA7 genotype-stratified demographic variables in Figure S1 D-E, to emphasize that we have carefully matched the two subsets of our cohort to limit confounding, we agree that this is useful information for the reader and therefore we include some of this information in Figure 1D.}

Figure 1B, the numbers of samples for normal (n=180) and LOF (n=5) are not balanced. Increasing the number of LOF cases would strengthen the conclusion.

\textcolor{blue}{Given how rare the LoF cases are, these are the maximum number of cases for which we had available genotype data. We wanted to report all of the samples from this cohort. However, we provide an additional analysis below to support that the difference in ABCA7 protein levels remains highly significant when performing multiple random samplings of equal numbers of individuals.}

Figure 5 needs to be strengthened. The statement of ABCA7 LoF iNs recapitulate AD amyloid pathology is weak in Figure. S9A,B and Figure 5D, ELISA is needed to support the effect of pGlu50fs on Abeta42. If BACE1 is reduced by the mutation (Fig. 5E), the levels of APP-CTF in the cell, and sAPPbeta in the medium should be assessed and quantified biochemically, in addition to IF, to confirm the reduction of enzyme activity.

\textcolor{blue}{In accordance with another reviewer’s comment, we have now replaced our IHC data on iNs with ELISA on the media from neurospheroids.}

Figure 5A: a control line should be included to show the effect of pTyr622 on Lipid droplet. Also lipid staining should also be performed in pGlu50fs neurons, since the major lipidomics data were generated from pGlu50fs, not from pTyr622.

\textbf{Other comments:}

Figure. S7E, F: a nuclei staining and a merged image should be provided for identification of signals in GLUT neurons.

It appears that only one clone is generated for human iPSC carrying ABCA7 variants (related to Fig. 3), it will be more convincing to have one more independent clone or line to replicate the findings for each variant.

\textcolor{blue}{This is true, however we didn’t have the resources to perform these studies with the same number of technical replicates per line & multiple experimental lines (which we believe is high compared to previous papers) if we had included multiple clones and more than the 2 lines we already are including. However, the neurospheroids were derived from a different clone (?).}

Figure 4D: The correlation is not convincing, especially due to the combination of two genotypes and the clustering of data points in the LOF line.

\textcolor{blue}{In line with another reviewer’s comments we have now removed this panel.}

