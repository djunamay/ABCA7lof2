
Loss-of-function (LoF) mutations in the lipid transporter ABCA7 are among the strongest genetic risk factors for late-onset AD. Here, we generated a transcriptional atlas of ABCA7 LoF effects across all major brain cell types in the human prefrontal cortex. Our dataset showed the highest levels of ABCA7 expression in excitatory neurons and strong evidence that ABCA7 LoF led to transcriptional perturbation in pathways related to lipid biosynthesis, mitochondrial respiration, and cellular stress, including up-regulation of DNA repair pathways, and changes to inflammatory and synaptic genes. Using iPSC-derived isogenic neuronal lines with and without ABCA7 LoF variants, we show that ABCA7 LoF lead to mitochondrial dysfunction and lipid imbalances.  Previous studies support a causal relationship between decreased mitochondrial uncoupling, elevated mitochondrial membrane potential, and the subsequent accumulation of reactive oxygen species (ROS). Consistent with ABCA7's role as a phospholipid transporter, our data, coupled with our findings from CDP-choline treatment, indicate that phosphatidylcholine imbalances in ABCA7 LoF may precede mitochondrial dysfunction. As phosphatidylcholine is ubiquitous in biological membranes and highly abundant in mitochondrial membranes, changes in its saturation levels significantly impact membrane fluidity. Increased saturation can decrease fluidity, and lead to reduced mitochondrial uncoupling [citations].

While the precise mechanism by which ABCA7 LoF causes phosphatidylcholine imbalances remains unclear, previous studies suggest phosphatidylcholine is directly transported by ABCA7 across lipid bilayers to maintain membrane asymmetry. Disruption of this transport may alter membrane fluidity and subsequently affect lipid homeostasis, potentially by downregulating lipid biosynthesis pathways such as LPCAT, as previous studies show. Alternatively, increased oxidative stress in ABCA7 LoF neurons could contribute to phospholipid oxidation, elevating levels of saturated phosphatidylcholines. Supplementation with CDP-choline supports de novo phosphatidylcholine synthesis, consistent with our findings and previous research. CDP-choline supplementation may restore phosphatidylcholine balance by providing substrates for lipid remodeling pathways, such as the Land’s cycle. Additionally, we observed evidence of elevated neutral lipid species in ABCA7 LoF neurons. As phosphatidylcholine balance regulates the metabolism and transport of neutral lipids, supplementation with CDP-choline might help reduce neutral lipid accumulation by facilitating their mobilization and mitochondrial oxidation and transport. 

Mitochondrial dysfunction, including impaired mitochondrial uncoupling, is increasingly recognized as critical in aging, AD, and other neurodegenerative diseases. Although mitochondrial uncoupling was recently linked to frontotemporal dementia, its specific role in AD remains poorly investigated \cite{noauthor_2022-os,Bano2023-qz,Zong2024-tn,Demine2019-qj,noauthor_2013-rt,Picca2023-gt}. Neurons maintain high mitochondrial oxidative phosphorylation (OXPHOS) to meet their significant energy demands \cite{Morant-Ferrando2023-va,Trigo2022-ym}. Mitochondrial uncoupling, actively regulated by mitochondrial proteins\cite{Park2023-fa,noauthor_2016-fg}, supports this energy demand by modulating mitochondrial membrane potential, reducing reactive oxygen species (ROS) \cite{Demine2019-qj,Shadel2015-kt}, and promoting mitochondrial biogenesis\cite{Korshunov1997-aj,Wisloff2005-ho,Andrews2005-yy,noauthor_2022-vx}. Impaired mitochondrial uncoupling, as observed in ABCA7 LoF neurons, can elevate oxidative stress, impair synaptic and calcium signaling, and contribute to neurodegeneration \cite{Korshunov1997-aj,Wisloff2005-ho,Andrews2005-yy,noauthor_2022-vx}. Increased oxidative stress also triggers DNA damage and inflammatory responses, as indicated by ABCA7 LoF transcriptomic signatures\cite{Robert2020-sc,Volanti2002-mc,Canty1999-oj,Schreck1992-zr}. Furthermore, neuronal dysfunction can affect surrounding glial cells, exacerbating pathological outcomes\cite{Byrns2024-id,Welch2022-ef}.

In line with our findings linking phosphatidylcholine imbalances to mitochondrial impairments in ABCA7 LoF neurons, a recent study in ABCA7 LoF neurospheroids independently revealed a link between phosphatidylglycerol deficiency and mitochondrial function\cite{Kawatani2023-vf}, further highlighting the importance of lipid-centric therapeutic interventions for ABCA7 LoF. Here, we offer a therapeutic strategy to reverse these dysfunctions through CDP-choline treatment, a readily available and safe dietary supplement\cite{Gavrilova2018-oi,Zeisel2009-xv,Blusztajn2017-nv}.  Recent work from our lab implicates phosphatidylcholine and fatty acyl saturation imbalances in APOE4 dysfunction\cite{Sienski2021-zt}, and in cognitive resilience to AD pathology (Boix et al, 2024, Nature, in press), suggesting that phosphatidylcholine disruptions may be central to AD risk in large fractions of the population. Indeed, our work suggests that the common missense variant p.Ala1527Gly likely has convergent effects with ABCA7 LoF. Genetic interactions with other risk factors, including APOE4, may exacerbate otherwise subtle ABCA7 dysfunction, and contribute to risk in a significant subset of AD cases\cite{Wang2021-oa,Hemani2013-zr,Haig2011-vs,Zuk2012-uz}. As such, our study supports a growing body of literature, including recent studies on APOE4 \cite{Haney2024-fx,Victor2022-tl}, implicating lipid disruptions in the etiology of AD and pinpoints additional genotypes that may benefit from phosphatidylcholine and triglyceride-targeting interventions.


\textcolor{red}{How to mention the abeta and hyperactivity results? What about cholines role in acetylcholine? Previously had written something like:}

\textcolor{red}{CDP-choline treatment also reduces toxic Aβ42 levels in ABCA7 LoF neurons, indicating that lipid metabolic defects may contribute to Alzheimer’s disease pathology in ABCA7 LoF neurons. This is consistent with previous studies linking membrane lipids and lipid modifications to APP processing \cite{Bhattacharyya2016-rs,Walter2013-qu}.}