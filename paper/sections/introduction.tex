Over 50 million people worldwide have dementia, with a large fraction of cases caused by Alzheimer’s disease \cite{WHO2023Dementia}. Late-onset Alzheimer’s Disease (AD) affects individuals over the age of 65 and accounts for more than 95\% of all AD cases \cite{Alzheimers_Association2016-jx}. Though AD is a multifactorial disorder, twin studies suggest a strong genetic component (~70\% heritability) \cite{Karlsson2022-vv} contributing to AD disease risk and progression. Large scale genome-wide association studies implicate multiple genes in AD etiology \cite{Lambert2013-km,Marioni2019-os,Jansen2019-ww,Kunkle2019-yo,Wightman2021-km,Bellenguez2022-ao,Belloy2023-kj}. After APOE4, rare loss-of-function (LoF) mutations caused by premature termination codons (PTCs) in ATP-binding cassette transporter A7 (ABCA7), are among the strongest genetic factors for AD (odds ratio $\approx$ 2) \cite{Steinberg2015-mu,De_Roeck2019-bs,Reitz2013-eo,Bellenguez2022-ao,Holstege2022-vp,Lyssenko2021-gw}. In addition to LoF variants, several common single nucleotide polymorphisms in ABCA7 - depending on the population - moderately \cite{Steinberg2015-mu,De_Roeck2019-bs,Reitz2013-eo,Bellenguez2022-ao,Le_Guennec2016-mr,Hollingworth2011-tr,Naj2011-bs} to strongly \cite{Reitz2013-eo} increase AD risk, suggesting that ABCA7 dysfunction may play a role in a significant proportion of AD cases. Despite the prevalence and potential impact of ABCA7 variants, the mechanism by which ABCA7 dysfunction increases AD risk remains poorly characterized. 

ABCA7 is a member of the A subfamily of ABC transmembrane proteins \cite{Kim2008-zi} with high sequence homology to ABCA1, the primary lipid transporter responsible for cholesterol homeostasis and high-density lipoprotein genesis in the brain \cite{Koldamova2014-kd}. ABCA7 effluxes both cholesterol and phospholipids to APOA-I and APOE in \textit{in vitro} studies \cite{Abe-Dohmae2004-wb,Wang2003-wh,Tomioka2017-nv,Picataggi2022-yp,Quazi2013-pe,Fang2025} and has been shown to be a critical regulator of energy homeostasis, immune cell functions, and amyloid processing \cite{Aikawa2018-ek,Tanaka2011-zo,Duchateau2023-ji,Kawatani2023-vf,Tayran2024-bo,Wang2025-de}. To date, study of ABCA7 LoF has been predominantly pursued in rodent knock-out models or in non-neural mammalian cell lines. These studies show that ABCA7 knock-out or missense variants  cause increased amyloid processing and deposition \cite{Satoh2015-yu,Sakae2016-uy,Chan2008-qu,Bamji-Mirza2018-xt}, reduced plaque clearance by astrocytes and microglia \cite{Kim2013-sv,Fu2016-qe}, and glial-mediated inflammatory responses \cite{Aikawa2019-hv,Aikawa2021-vz}.  While these studies shed light on potential mechanisms of ABCA7 risk in AD, studies investigating the effects of ABCA7 LoF in human cells and tissue are severely lacking, with only a small number published to date \cite{Kawatani2023-vf,Allen2017-vw,Liu2021-zh,Bamji-Mirza2018-xt}.  These human studies highlight a number of potential LoF-induced defects in human cells, including impacts on lipid metabolism and mitochondrial function \cite{Kawatani2023-vf}. However, comprehensive and unbiased profiling of multiple human neural cell types is needed to elucidate the mechanism by which ABCA7 LoF increases AD risk.

Single-nucleus RNA sequencing (snRNA-seq) of human neural tissue has identified cell type-specific transcriptional changes associated with AD risk variants in genes such as \textit{APOE} and \textit{TREM2} \cite{Brase2023-xk,Blanchard2022-cf,Sayed2021-qn,Wamsley2024-zm,Kamath2022-if}, providing insights into disease mechanisms and potential therapies. Here, we generated a cell type-specific transcriptomic atlas of ABCA7 LoF in the human prefrontal cortex (PFC). SnRNA-seq of \textit{postmortem} brain tissue from ABCA7 LoF variant carriers and matched controls revealed widespread transcriptional alterations, particularly in excitatory neurons, which expressed the highest ABCA7 levels. Expression changes in these neurons indicated disruptions in lipid metabolism, mitochondrial respiration, DNA damage response, and synaptic function. Similar transcriptional changes were observed in neurons carrying the common missense variant p.Ala1527Gly, which was predicted to impair ABCA7 function based on structural simulations. This overlap indicates that p.Ala1527Gly may exert effects comparable to ABCA7 LoF, extending the relevance of our findings to a broader group of the at-risk population.

To complement our transcriptomic findings, we examined induced pluripotent stem cell (iPSC)-derived neurons harboring ABCA7 LoF variants. These neurons exhibited significant transcriptional overlap with human PFC neurons affected by ABCA7 LoF. Additionally, they demonstrated impaired uncoupled mitochondrial respiration, hyperpolarized mitochondrial membrane potential, elevated reactive oxygen species (ROS) levels, increased secretion of amyloid-$\beta$ (Aβ), and hyperexcitability. Consistent with ABCA7's known role in phospholipid transport, we also observed alterations in lipid composition, notably an increase in saturated phosphatidylcholine. Enhancing \textit{de novo} phosphatidylcholine synthesis through CDP-choline supplementation effectively reversed these ABCA7 LoF-induced transcriptional changes and phenotypes. These findings link metabolic disruptions to AD pathology and suggest that neuronal ABCA7 may impact mitochondrial function through phosphatidylcholine imbalance, highlighting a potential mechanism by which ABCA7 variants increase AD risk.
