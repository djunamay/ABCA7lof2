\subsubsection{Transcriptomic atlas of ABCA7 LoF variants}
To investigate the cell type-specific impacts of ABCA7 LoF variants in AD, we selected 12 carriers of rare ABCA7 PTC variants, including splice (c.4416+2T>G, c.5570+5G>C), frameshift (p.Leu1403fs, p.Glu709fs), and nonsense variants (p.Trp1245*, p.Trp1085*), from the ROSMAP cohort (Fig. 1a-c; Supplementary Tables 1, 2). ABCA7 PTC variants are presumed to cause ABCA7 haploinsufficiency and were associated with lower ABCA7 protein levels in the PFC compared to matched non-carriers, for a previously available dataset in which a subset overlapped with our sequencing patients (Fig. 1d, Extended Data Fig. 1a, Supplementary Table 3). Twenty-four matched controls without ABCA7 PTC variants were selected based on AD pathology, age, sex, APOE genotype, and cognitive status (Fig. 1c, Extended Data Fig. 1b,c, Supplementary Table 4, Supplementary Note 1). Absence of rare damaging variants in other AD-associated genes\supercite{Holstege2022-vp} were confirmed and a subset of genotypes were verified by Sanger sequencing (Extended Data Fig. 1d, Methods).

Raw snRNAseq data from the BA10 region of the PFC were available for 10 non-carrier samples from a previous study \supercite{Mathys2019-wb}. Fresh-frozen PFC BA10 tissues from remaining subjects underwent snRNAseq using the 10x Genomics Chromium platform. After extensive quality control— including genotype-transcriptome matching to confirm sample identities and rule out potential sample swaps and correcting for batch effects (Extended Data Fig. 1e, Supplementary Figure 1, Supplementary Table 5)—our final dataset consisted of 102,710 high-quality cells from an initial total of 150,456 cells, representing inhibitory neurons, excitatory neurons, astrocytes, microglia, oligodendrocytes, and oligodendrocyte precursor cells (Fig. 1e, Supplementary Figure 2). A small putative vascular cell cluster did not meet our quality thresholds and was excluded from further analysis.

Next, we analyzed cell type-specific transcriptional changes associated with ABCA7 LoF variants. We identified 2,389 genes with nominal evidence of perturbation (p<0.05), suggesting possible transcriptional changes across six major neural cell types after controlling for covariates and focusing on genes detected in >10\% of cells per type (Supplementary Table 6). We visualized these perturbations in two dimensions, revealing clear transcriptional patterns across cell types (Fig. 1f, Extended Data Fig. 2a). To nominate biological pathways that may be affected by ABCA7 LoF, we clustered genes based on their proximity in this visualization, as closer genes exhibited similar perturbation patterns. Each gene cluster was then analyzed for enrichment of biological pathways using the Gene Ontology Biological Process database, highlighting candidate functional themes disrupted by ABCA7 LoF, including cellular stress and apoptosis, synaptic function, DNA repair, and metabolism (Fig. 1g, Supplementary Table 7).

Specifically, microglia exhibited notable downregulation of stress-response genes (e.g., \textit{HSPH1}, cluster 11), a trend also observed, though less prominently, in neurons and oligodendrocyte precursor cells (OPCs) (Fig. 1g). Microglia and astrocytes showed increased expression of transcriptional regulatory genes (clusters 9 and 10, respectively) (Fig. 1g). OPCs and oligodendrocytes displayed changes in inflammatory signaling pathways (e.g., \textit{IL10RB}, cluster 0; \textit{STAT2}, cluster 8) (Fig. 1g). Neurons demonstrated increased expression of DNA repair genes (e.g., \textit{FANCC}, cluster 12) and reduced expression of synaptic transmission genes (e.g., \textit{NLGN1}, \textit{SHISA6}, cluster 1) (Fig. 1g). Excitatory neurons uniquely exhibited enhanced expression of cellular respiration genes (e.g., \textit{NDUFV2}, cluster 7) and decreased expression of triglyceride biosynthesis genes (e.g., \textit{PPARD}, cluster 5) (Fig. 1g). Overlaps in gene perturbations across cell types are summarized in Extended Data Fig. 3.

Together, these results highlight extensive, cell type-specific transcriptional disruptions associated with ABCA7 LoF in the human PFC. This single-cell atlas serves as a valuable resource to nominate pathways and genes for future investigation, and is accessible through the UCSC Single Cell Browser and Synapse (ID: syn53461705).

\subsubsection{ABCA7 LoF profiles in excitatory neurons}
Our snRNAseq data revealed that excitatory neurons express the highest levels of ABCA7 among major neural cell types in the brain (Fig. 2a; Extended Data Fig. 4a,b). We validated these expression patterns using an independent dataset \supercite{Welch2022-ef} (Supplementary Table 3), confirming significantly higher ABCA7 expression in neuronal versus glial populations from human temporal cortex. Expression profiles of control genes known to be neuron- or glia-specific matched expectations (Extended Data Fig. 4c).

Given this expression profile, we hypothesized that excitatory neurons may be particularly impacted by ABCA7 LoF variants. To identify transcriptional correlates, we performed gene set enrichment analysis using WikiPathways (472 pathways), which identified 34 candidate ABCA7 LoF-perturbed pathways (p<0.05) involving 268 unique genes \supercite{Subramanian2005-pu} (Supplementary Table 8). To minimize redundancy and clearly identify biological themes, we grouped these genes into non-overlapping clusters via graph partitioning (Fig. 2a, Supplementary Figure 3, Supplementary Note 2). This analysis revealed eight biologically meaningful clusters highlighting three major themes: (1) energy metabolism and homeostasis (clusters PM.0, PM.1), (2) DNA damage and cellular stress responses (clusters PM.2, PM.3, PM.4, PM.5), and (3) synaptic dysfunction (cluster PM.7) (Fig. 2b). Layer-specific analysis indicated consistent transcriptional perturbation patterns across cortical layers (Extended Data Fig. 5, Supplementary Table 9).

Clusters PM.0 and PM.1 were enriched for genes involved in lipid metabolism, mitochondrial function, and oxidative phosphorylation. Specifically, cluster PM.0, associated with lipid homeostasis (e.g., \textit{NR1H3}, \textit{ACLY}, \textit{PPARD}), was downregulated, whereas cluster PM.1, comprising mitochondrial complex genes (e.g., \textit{COX7A2}, \textit{NDUFV2}), was upregulated. Clusters PM.2, PM.3, and PM.6 contained upregulated DNA damage response and replication genes (e.g., \textit{RECQL}, \textit{TLK2}, \textit{BARD1}). Clusters PM.4 and PM.5 encompassed genes associated with proteasomal degradation, ciliogenesis, apoptosis, and inflammation, exhibiting mixed directional regulation. Similarly, cluster PM.7, linked to synaptic and developmental pathways, contained both up- and downregulated genes (Fig. 2b).

\subsubsection{Shared ABCA7 LoF and p.Ala1527Gly effects}
ABCA7 LoF variants significantly increase AD risk but are rare, thus contributing to a small fraction of AD cases \supercite{Steinberg2015-mu,Duchateau2024-rf}. To assess if transcriptional patterns associated with ABCA7 LoF extend to more common variants, we analyzed carriers of the prevalent ABCA7 missense variant p.Ala1527Gly (rs3752246; minor allele frequency ≈0.18) within the ROSMAP cohort (Fig. 2c, Supplementary Table 3). Although annotated as the reference allele, Gly1527 represents the less frequent allele associated with moderately increased AD risk (Odds Ratio = 1.15 [1.11-1.18]) \supercite{Kunkle2019-yo,Holstege2022-vp,Naj2011-bs}. We analyzed existing snRNAseq data \supercite{Mathys2023-rs} from human PFC samples of 133 p.Ala1527Gly carriers and 227 non-carriers, ensuring no overlap with the previous ABCA7 LoF cohort.

We observed directional transcriptional perturbations in excitatory neurons from p.Ala1527Gly carriers that were consistent with those previously associated with ABCA7 LoF variants across all identified clusters (PM.0-7) (Fig. 2b,e,f). Notably, clusters related to DNA damage (PM.3) and proteasomal function (PM.4) demonstrated evidence of upregulation, closely mirroring the cellular stress and genomic instability signatures associated with ABCA7 LoF (Fig. 2e,f). Additionally, we found evidence for downregulation in lipid metabolism (PM.0) and modest upregulation in mitochondrial function (PM.1), both aligning closely with changes seen in ABCA7 LoF neurons (Fig. 2e,f).

To explore structural explanations for these shared transcriptional patterns, we conducted molecular dynamics simulations comparing Ala1527 and Gly1527 variants in ATP-bound closed (Fig. 2g,h) and ATP-unbound open ABCA7 conformations (Extended Data Fig. 6a,b), embedded within a lipid bilayer over a 300-ns timescale. The Gly1527 variant exhibited increased structural flexibility specifically in the ATP-bound closed state, characterized by pronounced conformational fluctuations compared to the Ala1527 variant (Fig. 2i,j). Given that the ATP-bound closed conformation is proposed to facilitate lipid presentation to apolipoproteins \supercite{LeThiMy2022-dp,Fang2025}, the increased flexibility of the Gly1527 variant may reduce lipid extrusion efficiency, consistent with recent experimental findings \supercite{Fang2025}. Both variants remained structurally stable in the ATP-unbound open state (Extended Data Fig. 6c-e). These results are further supported by analyses of $\phi$/$\psi$ dihedral angle distributions and secondary structure persistence, as detailed in the Supplementary Text (Supplementary Note 3, Extended Data Fig. 7). These structural insights, together with our transcriptomics data, suggest that both rare, high-effect ABCA7 LoF variants and common, moderate-risk variants may influence AD risk through similar ABCA7-dependent mechanisms, indicating broader relevance to Alzheimer's disease risk.

\subsubsection{Conserved signatures in ABCA7 LoF iNs}
To experimentally validate the effects of ABCA7 LoF predicted by our single-cell dataset, we generated two isogenic iPSC lines homozygous for distinct ABCA7 LoF variants using CRISPR-Cas9 editing (Fig. 3a, Supplementary Figure 4). One variant, p.Glu50fs*3, introduces an early frameshift mutation, while the other, p.Tyr622*, represents a clinically relevant Alzheimer's disease-associated variant \supercite{Steinberg2015-mu}. Both variants likely produce severely truncated ABCA7 proteins or trigger nonsense-mediated decay; however, transcript rescue mechanisms such as exon skipping cannot be excluded \supercite{De_Roeck2017-hv}. We differentiated these iPSCs into induced neurons (iNs) via doxycycline-inducible NGN2 expression \supercite{Ho2016-kz} (Supplementary Figure 5a). Wild-type (WT) and ABCA7 LoF neurons both expressed neuronal markers, formed robust neuronal processes within 2–4 weeks (Supplementary Figure 5b,c), and exhibited electrophysiological activity (Extended Data Fig. 8a-e). Notably, ABCA7 LoF neurons demonstrated increased excitability, firing action potentials at lower thresholds than WT neurons (Extended Data Fig. 8f,g), consistent with neuronal hyperexcitability observed in Alzheimer's disease.

We next examined whether transcriptional signatures identified in ABCA7 LoF postmortem neurons were recapitulated in iNs using bulk RNA sequencing after four weeks of differentiation (Supplementary Table 10). Transcriptional perturbations between the p.Glu50fs*3 and p.Tyr622* variants strongly correlated (Pearson correlation = 0.84; Fig. 3b). Gene set enrichment analysis revealed 15 significantly perturbed pathways for each variant (FDR-adjusted p<0.05; WikiPathways; Supplementary Table 11). K/L partitioning of these pathways identified 9 transcriptional clusters perturbed in WT vs p.Tyr622* and 10 clusters in WT vs p.Glu50fs*3 (Fig. 3c, Extended Data Fig. 9a, Supplementary Table 12). These clusters showed substantial overlap between the two variants, with 8 of 9 clusters from p.Tyr622* significantly overlapping with 8 of 10 clusters from p.Glu50fs*3 (FDR-adjusted p<0.05; Fig. 1d, Extended Data Fig. 9b). Additionally, we observed significant concordance with transcriptional signatures from postmortem excitatory neurons, with 5 of 9 p.Tyr622*-associated clusters and 7 of 10 p.Glu50fs*3-associated clusters overlapping significantly with postmortem ones, predominantly with concordant directional changes (Fig. 3d, Extended Data Fig. 9c). For example, consistent with postmortem findings, p.Tyr622* iNs exhibited downregulated clusters associated with lipid metabolism (T.9 and T.13) and upregulated clusters related to cell cycle regulation and proteasomal activity (T.8 and T.14) compared to WT neurons (Fig. 3e). A mitochondrial cluster (T.10) showed the strongest overlap with postmortem neurons (cluster PM.1), consistently upregulated in both variant lines (Fig. 3d, Extended Data Fig. 9c). Together, these data support a causal relationship between ABCA7 LoF variants and multiple transcriptional disruptions in excitatory neurons, particularly affecting mitochondrial function, proteostasis, cell cycle regulation, and lipid metabolism.

\subsubsection{ABCA7 LoF impacts mitochondrial uncoupling}
To investigate mitochondrial alterations in ABCA7 LoF iNs, we analyzed the expression of 1,136 genes encoding mitochondrial proteins from the MitoCarta database (Supplementary Table 3). Upregulated genes in p.Tyr622* neurons included mitochondrial apoptosis pathway genes (e.g., \textit{CASP3}, \textit{BID}) and OXPHOS subunits, previously identified in clusters PM.1 and T.10 (Fig. 3f, Supplementary Table 13). Conversely, downregulated genes were enriched for β-oxidation (\textit{ACAD}, \textit{CPT}), mitochondrial metabolite transport (\textit{SLC25}), and oxidative stress detoxification (\textit{CAT}) (Fig. 3f, Supplementary Table 13). Similar gene expression profiles were observed in the p.Glu50fs*3 variant (Extended Data Fig. 9e).

We directly assessed mitochondrial function by measuring oxygen consumption rates (OCR) using the Seahorse assay (Extended Data 10a,b). During oxidative phosphorylation (OXPHOS), OCR-driven proton movement across the inner mitochondrial membrane maintains mitochondrial membrane potential (ΔѰm) and supports ATP production (Extended Data 10c). To account for differences in cell viability and mitochondrial abundance, we analyzed internally normalized OCR ratios. Spare respiratory capacity—representing the mitochondria's ability to increase respiration in response to demand—was comparable between WT and ABCA7 LoF neurons (Extended Data 10d,e). However, ABCA7 LoF neurons showed significantly reduced uncoupled mitochondrial respiration, defined as the proportion of basal oxygen consumption dedicated to maintaining membrane potential lost due to proton leakage rather than ATP synthesis \supercite{Divakaruni2022-bp}, compared to WT neurons (Fig. 3g, Extended Data 10f). The uncoupled mitochondrial OCR in WT neurons (approximately 20\%; Fig. 3) aligns with previous reports for neurons and other cell types \supercite{Divakaruni2011-uj,Jekabsons2004-fn,Jain2024-br}, indicating that ABCA7 LoF neurons exhibit abnormally low mitochondrial uncoupling. Consistent with this finding, expression of \textit{UCP2}—a mitochondrial uncoupling protein expressed in the brain \supercite{Kumar2022-bb}—was reduced in ABCA7 LoF neurons (Extended Data 10g).

Since reduced mitochondrial uncoupling typically results in elevated ΔѰm, we assessed ΔѰm using MitoHealth and TMRM dyes, both of which accumulate in mitochondria proportionally to the membrane potential. Both dyes showed higher fluorescence—indicative of elevated ΔѰm—in ABCA7 LoF neurons compared to WT neurons (Figure 3h,i). Signal specificity was confirmed by decreased fluorescence following FCCP-induced depolarization (Extended Data 10h). Because mitochondrial uncoupling regulates reactive oxygen species (ROS) generation, we next measured oxidative stress using CellROX dye. ABCA7 LoF neurons showed significantly increased fluorescent CellROX signal compared to WT neurons (Fig. 3j). Together, these data indicate that ABCA7 LoF variants decrease mitochondrial uncoupling, resulting in elevated membrane potential and increased oxidative stress.

\subsubsection{ABCA7 LoF alters phosphatidylcholine balance}
Since ABCA7 functions as a lipid transporter, we used LC-MS to examine lipid profiles in WT and ABCA7 LoF iNs (Supplementary Table 14). Comparing WT and p.Glu50fs*3 iNs revealed significant changes across several lipid classes, including neutral lipids, phospholipids, sphingolipids, and steroids (Extended Data Fig. 11a,b). Notably, triglycerides (TGs), particularly long-chain polyunsaturated species, were frequently elevated in p.Glu50fs*3 iNs (Extended Data Fig. 11b,c).
Consistent with ABCA7's known preference for phospholipids \supercite{Tomioka2017-sq,Picataggi2022-hf,Fang2025}, phosphatidylcholines (PCs)—key structural membrane components and potential ABCA7 substrates \supercite{LeThiMy2022-dp,Fang2025}—were prominently affected, with approximately 64\% of perturbed PC species increased in p.Glu50fs*3 iNs (Extended Data Fig. 11b). Analysis by fatty acid saturation showed significant enrichment of saturated PCs among upregulated species (hypergeometric p=0.026; Extended Data Fig. 11d). In contrast, several highly unsaturated polyunsaturated fatty acid (PUFA)-containing PCs showed decreased abundance (e.g., PC(44:7), PC(38:7); Extended Data Fig. 11e,f).

To verify if these lipid changes were conserved in p.Tyr622* iNs, lipidomics performed in positive ionization mode revealed similarly increased saturated PCs (hypergeometric p=0.044; Fig. 3k, Extended Data Fig. 11g,h). However, PUFA-containing PCs and long-chain triglycerides were not reliably detected in this analysis (Extended Data Fig. 11i,j).

\textit{De novo} PC synthesis occurs via the Kennedy pathway, followed by fatty acyl chain remodeling through the Lands cycle, mediated by LPCAT enzymes, with LPCAT3 specifically introducing PUFA chains \supercite{Boumann2003-ew,Wang2019-om,Zhao2008-pq}. \textit{LPCAT3} expression was reduced in both p.Tyr622* and p.Glu50fs*3 iNs compared to WT (Extended Data Fig. 11k,l), consistent with elevated saturated PC levels. Overall, these results indicate that ABCA7 LoF neurons accumulate neutral lipids, including long-chain polyunsaturated triglycerides and sterol lipids (zymosteryl), and show enriched saturated PC content.

\subsubsection{CDP-choline reverses ABCA7 LoF impacts}
Previous work showed that supplementation with exogenous choline, a soluble precursor for \textit{de novo} PC synthesis, normalized phospholipid saturation in yeast and improved APOE4-associated lipid phenotypes \supercite{Boumann2006-nz,Sienski2021-zt}. To test if CDP-choline similarly mitigates ABCA7 LoF-induced phenotypes in iNs, we treated p.Tyr622* iNs with CDP-choline for two weeks and performed targeted LC-MS analysis. CDP-choline treatment elevated extracellular CDP-choline from undetectable to detectable levels (Extended Data Fig. 12a, Supplementary Table 15). Furthermore, CDP and choline specifically accumulated in media conditioned by treated p.Tyr622* cells (Extended Data Fig. 12a), indicating extracellular hydrolysis. Intracellular choline increased significantly (Extended Data Fig. 12b), along with elevated expression of choline transporters (Extended Data Fig. 12c), confirming successful choline uptake by p.Tyr622* iNs.

We hypothesized that increased intracellular choline would enhance PC synthesis. Indeed, lipidomic analysis showed elevated levels of choline-containing phospholipids—particularly PCs, lysophosphatidylcholines (LPC), and sphingomyelins (SM)—alongside a reduction in a single TG species, with other neutral lipid species showing a similar downward trend (Fig. 4a, Supplementary Table 14). Correspondingly, \textit{PCYT1B}, the rate-limiting enzyme of Kennedy pathway-mediated PC synthesis, showed increased expression (Extended Data Fig. 12c). Additionally, LPCAT enzymes, including LPCAT3, exhibited higher expression following treatment (Extended Data Fig. 12d), consistent with increased saturated and unsaturated PC levels (Fig. 4a, Extended Data Fig. 12e). These findings suggest that CDP-choline enhances the synthesis and remodeling of choline-containing lipids in ABCA7 LoF iNs.

Next, we characterized changes induced by CDP-choline treatment using LC-MS-based metabolomics and bulk RNAseq. Although many metabolites altered by treatment could not be annotated, principal component analysis indicated that CDP-choline treatment reversed the separation between WT and p.Tyr622* iNs along the first principal component (PC1; Extended Data Fig. 12f). Transcriptomic analysis further demonstrated clear separation between treated and untreated samples (Extended Data Fig. 12g). Notably, the transcriptional signature of CDP-choline treatment negatively correlated with the p.Tyr622* signature (Fig. 4b), indicating partial restoration toward the WT state. K/L cluster analysis comparing untreated \textit{versus}treated p.Tyr622* samples revealed significant overlap in 7 of 9 clusters identified between p.Tyr622* and WT, with 5 clusters showing reversed directional changes after treatment (Fig. 4c-e).

Specifically, clusters associated with proteasomal and ribosomal functions (T+C.25, T+C.31), previously upregulated in p.Tyr622* iNs, were downregulated by CDP-choline treatment (Fig. 4d). Importantly, mitochondrial cluster T+C.26, strongly overlapping with mitochondrial cluster T.10 observed in postmortem data (PM.1), also reversed following treatment (Fig. 4e). Analysis of MitoCarta genes confirmed significant reversal in the expression of genes encoding mitochondrial proteins (Extended Data Fig. 12h), including reduced apoptosis-related genes (\textit{BID}, \textit{CASP3}; Fig. 3f), restored mitochondrial metabolic signatures (Supplementary Table 16), and elevated mitochondrial fusion regulators (\textit{MFN2}, \textit{OPA1}), which support mitochondrial biogenesis and function \supercite{Westermann2010-au}. Overall, CDP-choline treatment significantly reversed gene expression changes associated with ABCA7 LoF.

To assess whether CDP-choline treatment could restore mitochondrial uncoupling to WT levels, we repeated the Seahorse assay on p.Tyr622* iNs with and without treatment (Fig. 4i,j). CDP-choline treatment significantly increased uncoupled respiration in p.Tyr622* iNs to WT levels (Fig. 4g), without altering spare respiratory capacity (Extended Data Fig. 12k). Supporting this result, both TMRM and MitoHealth dyes showed lower fluorescence—indicative of decreased ΔѰm—in treated cells compared to untreated cells (Fig. 4h, Extended Data Fig. 12l). Additionally, CDP-choline significantly reduced oxidative stress, as indicated by decreased CellROX fluorescence (Fig. 4i).

\subsubsection{ CDP-choline reduces AD phenotypes}
Finally, we evaluated whether CDP-choline treatment could ameliorate key AD-associated phenotypes, since ABCA7 dysfunction has been linked to altered amyloid-β (Aβ) processing \supercite{Satoh2015-yu,Sakae2016-uy,Bamji-Mirza2018-xt,Chan2008-qu,De_Roeck2018-fw}. p.Tyr622* iNs secreted significantly higher Aβ40 and showed a trend toward increased Aβ42 secretion, although absolute levels remained relatively low (Extended Data Fig. 13a). To examine effects in a model with stronger pathology, we differentiated p.Tyr622* and WT lines into cortical organoids matured for approximately 6 months, a stage at which robust Aβ secretion was observed (approximately two- to four-fold higher than iNs; Extended Data Fig. 13b,c). Four-week treatment with 1 mM CDP-choline normalized Aβ40 and Aβ42 secretion in p.Tyr622* organoids to WT levels (Fig. 4j); this effect was concentration- and duration-dependent (Extended Data Fig. 13c). Furthermore, CDP-choline treatment at 100 µM significantly reduced neuronal hyperexcitability in dissociated cortical organoids, as assessed by electrophysiology (Fig. 4k).
