Loss-of-function (LoF) mutations in the lipid transporter ABCA7 are among the strongest genetic risk factors for late-onset AD. Here, we generated a transcriptional atlas of ABCA7 LoF effects across all major brain cell types in the human prefrontal cortex. Our dataset showed the highest levels of ABCA7 expression in excitatory neurons and strong evidence that ABCA7 LoF led to transcriptional perturbation in pathways related to lipid biosynthesis, mitochondrial respiration, and cellular stress, including up-regulation of DNA damage pathways, and changes to inflammatory and synaptic genes. Using iPSC-derived isogenic neuronal lines (iN) with and without ABCA7 LoF variants, we show that ABCA7 LoF leads to decreased mitochondrial decoupling, elevated mitochondrial membrane potential, and increased reactive oxygen species (ROS). Consistent with ABCA7’s role as a phospholipid transporter, ABCA7 LoF iNs exhibited significant imbalances in phosphatidylcholine composition, characterized by increased saturated PCs and decreased polyunsaturated (PUFA) PCs. A similar reduction in polyunsaturated phospholipids was recently observed in neuronal models of ALS/FTD, highlighting the broader significance of phospholipid saturation in neurodegenerative conditions \cite{Giblin2025-ri}. Treatment of ABCA7 LoF iN with CDP-choline increased phosphatidylcholine synthesis, upregulated expression of phosphatidylcholine remodeling enzymes, and rescued mitochondrial uncoupling, mitochondrial membrane potential, and the increase in ROS generation. In addition, CDP-choline supplementation also mitigated hyperexcitability and increased amyloid beta secretion in ABCA7 LoF iN. Together, our data indicate that the observed effects of ABCA7 LoF on neurons may be at least partially mediated by imbalances in phosphatidylcholine composition. 

\newcommand{\quoteI}{\textcolor{blue}{While the precise mechanism linking ABCA7 LoF to phosphatidylcholine imbalance remains unclear, disrupted ABCA7 floppase activity—responsible for phospholipid flipping across membrane leaflets—likely impacts membrane fluidity and curvature \cite{Takada2018-ce,Renne2018-fc}, important determinants of numerous cellular functions \cite{McMahon2015-gy,Yang2024-tz}. Changes in membrane composition may also broadly affect lipid metabolism by altering the activity of transcriptional regulators controlling lipid biosynthesis and remodeling genes (including LPCATs), which are responsive to shifts in membrane properties \cite{Ballweg2020-rv,Covino2018-hz}. Consistent with our observations and previous reports, CDP-choline supplementation supports \textit{de novo} synthesis of phosphatidylcholine species containing both saturated and polyunsaturated fatty acids \cite{Boumann2003-im}. Thus, CDP-choline may help restore phosphatidylcholine balance in ABCA7 LoF neurons by supporting the synthesis and remodeling of diverse phosphatidylcholine species. Given that phosphatidylcholine species are ubiquitous components of biological membranes—including abundant lipids within mitochondrial membranes \cite{Decker2024-ae}—imbalances in their fatty acyl chain composition could broadly impact cellular functions \cite{Wang2019-om,Van_der_Veen2017-ei}, including mitochondrial activity. Indeed, alterations in phospholipid saturation and composition have been linked to changes in mitochondrial dynamics, cristae morphology, bioenergetics, and membrane potential \cite{Decker2024-ae,Adachi2016-tg}. However, additional studies are needed to clarify the precise mechanisms by which phosphatidylcholine imbalances influence mitochondrial function and uncoupling dynamics.\label{quoteI-label}}} 
\quoteI

Mitochondrial dysfunction, including impaired mitochondrial uncoupling, is increasingly recognized as critical in aging, AD, and other neurodegenerative diseases. Although mitochondrial uncoupling was recently linked to frontotemporal dementia, its specific role in AD remains poorly investigated \cite{,Bano2023-qz,Zong2024-tn,Demine2019-qj,,Picca2023-gt}. Neurons maintain high mitochondrial oxidative phosphorylation (OXPHOS) to meet their significant energy demands \cite{Morant-Ferrando2023-va,Trigo2022-ym}. Mitochondrial uncoupling, actively regulated by mitochondrial proteins \cite{Park2023-fa,}, supports this energy demand by modulating mitochondrial membrane potential, reducing reactive oxygen species (ROS) \cite{Demine2019-qj,Shadel2015-kt}, and promoting mitochondrial biogenesis \cite{Korshunov1997-aj,Wisloff2005-ho,Andrews2005-yy,}. Impaired mitochondrial uncoupling, as observed in ABCA7 LoF neurons, can elevate oxidative stress, impair synaptic and calcium signaling, and contribute to neurodegeneration \cite{Korshunov1997-aj,Wisloff2005-ho,Andrews2005-yy,}. Increased oxidative stress also triggers DNA damage and inflammatory responses, as indicated by ABCA7 LoF transcriptomic signatures \cite{Robert2020-sc,Volanti2002-mc,Canty1999-oj,Schreck1992-zr}. Furthermore, neuronal dysfunction can affect surrounding glial cells, exacerbating pathological outcomes \cite{Byrns2024-id,Welch2022-ef}.

In line with our findings linking phosphatidylcholine imbalances to mitochondrial impairments in ABCA7 LoF neurons, a recent study in ABCA7 LoF neurospheroids independently revealed a link between phosphatidylglycerol deficiency and mitochondrial function \cite{Kawatani2023-vf}, further highlighting the importance of lipid-centric therapeutic interventions for ABCA7 LoF. Here, we offer a therapeutic strategy to reverse these dysfunctions - including ABCA7 LoF-induced AD pathology and neuronal hyperexcitability -through CDP-choline treatment, a readily available and safe dietary supplement \cite{Gavrilova2018-oi,Zeisel2009-xv,Blusztajn2017-nv}.  Recent work from our lab implicates phosphatidylcholine and fatty acyl saturation imbalances in APOE4 dysfunction \cite{Sienski2021-zt}, and in cognitive resilience to AD pathology \cite{Mathys2024-ex}, suggesting that phosphatidylcholine disruptions may be central to AD risk in large fractions of the population. Indeed, our work suggests that the common missense variant p.Ala1527Gly likely has convergent effects with ABCA7 LoF. Genetic interactions with other risk factors, including APOE4, may exacerbate otherwise subtle ABCA7 dysfunction, and contribute to risk in a significant subset of AD cases \cite{Wang2021-oa,Hemani2013-zr,Haig2011-vs,Zuk2012-uz}. As such, our study supports a growing body of literature, including recent studies on APOE4 \cite{Haney2024-fx,Victor2022-tl}, implicating lipid disruptions in the etiology of AD and pinpoints additional genotypes that may benefit from interventions on phosphatidylcholine metabolism.


% \textcolor{red}{How to mention the abeta and hyperactivity results? What about cholines role in acetylcholine? Previously had written something like:}

% \textcolor{red}{CDP-choline treatment also reduces toxic Aβ42 levels in ABCA7 LoF neurons, indicating that lipid metabolic defects may contribute to Alzheimer’s disease pathology in ABCA7 LoF neurons. This is consistent with previous studies linking membrane lipids and lipid modifications to APP processing \cite{Bhattacharyya2016-rs,Walter2013-qu}.}