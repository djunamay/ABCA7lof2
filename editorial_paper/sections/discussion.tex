\subsubsection{Discussion}
Here, we generated a transcriptional atlas to identify potential cell-type-specific impacts of ABCA7 LoF variants in the human prefrontal cortex. Excitatory neurons expressed the highest levels of ABCA7 and showed transcriptional alterations in pathways related to lipid biosynthesis, mitochondrial respiration, and cellular stress, including upregulation of DNA damage response genes, as well as changes in inflammatory and synaptic genes. Following experimental validation of predictions from this atlas, we observed that ABCA7 LoF variants impaired mitochondrial uncoupling, elevated mitochondrial membrane potential, and increased ROS levels in human induced neurons. Consistent with ABCA7's function as a phospholipid transporter, ABCA7 LoF altered PC composition in these neurons, characterized by increased saturated PCs and reduced highly polyunsaturated PCs. CDP-choline treatment increased PC synthesis, elevated expression of PC remodeling enzymes, and corrected mitochondrial uncoupling deficits, mitochondrial membrane potential, and oxidative stress. Furthermore, CDP-choline supplementation reduced neuronal hyperexcitability and amyloid-β secretion. 

Our findings indicate that ABCA7 LoF neurons accumulate saturated PC species, in line with recent reports of phospholipid saturation imbalance in human ALS/FTD neurons\supercite{Giblin2025-ri}. Treatment with CDP-choline, which boosts \textit{de novo} PC synthesis, effectively reduced downstream effects of ABCA7 LoF but did not fully normalize the lipid profile. Enhancing \textit{de novo} PC synthesis with CDP-choline may broaden the diversity of PC species by providing additional substrates for the Lands cycle. This cycle rapidly remodels the PC pool, generating diverse saturated and unsaturated PCs\supercite{Yaghmour2025-mo,ODonnell2022-ev}, potentially counteracting an excess of saturated species. Although further studies are needed to fully characterize these lipid alterations and their functional implications, and further investigate the specificity of CDP-choline treatment, our data suggest that disrupted PC metabolism may contribute to neuronal dysfunction associated with ABCA7 LoF variants.

PCs are abundant components of biological membranes, including mitochondrial membranes. Changes in their fatty acyl chain composition impact mitochondrial bioenergetics, dynamics, and membrane potential\supercite{Decker2024-ae}. Mitochondrial dysfunction, including impaired mitochondrial uncoupling, is increasingly linked to aging and neurodegeneration, although its specific role in Alzheimer's disease remains unclear\supercite{Crivelli2024-pf}. Neurons heavily depend on mitochondrial oxidative phosphorylation (OXPHOS) to meet their energy demands\supercite{Morant-Ferrando2023-va}, and regulated mitochondrial uncoupling supports neuronal health by controlling mitochondrial membrane potential, reducing reactive oxygen species, and promoting mitochondrial biogenesis\supercite{Korshunov1997-aj,Andrews2005-yy}. Elevated oxidative stress resulting from impaired uncoupling could contribute to neuronal DNA damage and inflammation observed in AD brains\supercite{Welch2022-bp}, both of which are transcriptionally evident in ABCA7 LoF carriers.

Further studies are needed to clarify how ABCA7 regulates PC composition. Previous work suggests ABCA7 transports PC and its Lands cycle derivatives, lysophosphatidylcholines (LPC)\supercite{Tomioka2017-sq}. We speculate that impaired ABCA7 floppase activity, which moves phospholipids between membrane leaflets, might initially cause PC accumulation at the inner leaflet. This buildup could trigger compensatory downregulation of PC synthesis. Supporting this idea, the Kennedy pathway’s rate-limiting enzyme, CTP:phosphocholine cytidylyltransferase, is activated upon binding to PC-deficient membranes\supercite{Cornell2016-zk}. Such conditions could shift the existing PC pool toward more saturated species, as previously shown in yeast\supercite{Boumann2006-nz}. Alternatively, impaired floppase activity involving PCs, LPCs, or other phospholipids transported by ABCA7, such as phosphatidylserines\supercite{Fang2025}, might indirectly alter PC composition by changing membrane fluidity and curvature\supercite{Takada2018-ce}. These membrane changes could then affect lipid metabolism by modulating enzymes such as LPCAT3, which helps maintain unsaturated phospholipid levels\supercite{Ballweg2020-rv,Ariyama2010-cc}.

Consistent with our findings linking PC imbalance to mitochondrial dysfunction in ABCA7 LoF neurons, a recent independent study reported mitochondrial impairment associated with phosphatidylglycerol deficiency in ABCA7-deficient neurospheroids\supercite{Kawatani2023-vf}, further emphasizing lipid metabolism as a therapeutic target. Here, we demonstrate that CDP-choline, a safe and widely available dietary supplement\supercite{Gavrilova2018-oi}reverses key aspects of ABCA7 LoF-induced neuronal dysfunction, including AD pathology and neuronal hyperexcitability. Recent studies from our laboratory similarly linked PC and fatty acid saturation imbalances to APOE4-associated neuronal dysfunction\supercite{Sienski2021-zt} and cognitive resilience to AD pathology\supercite{Mathys2024-ex}, highlighting broad relevance for PC disruptions in AD risk.

Supporting the therapeutic potential of targeting phospholipid metabolism in AD, dietary choline supplementation in APP/PS1 mouse models significantly reduced amyloid pathology\supercite{Velazquez2019-da}, aligning with our findings in cortical organoids. Additionally, recent studies in Drosophila ALS/FTD models demonstrated that \textit{in vivo} overexpression of fatty acid desaturases improved survival, likely through correction of phospholipid saturation imbalances\supercite{Giblin2025-ri}. Notably, a recent epidemiological study also linked higher dietary choline intake with reduced AD risk in humans\supercite{Karosas2025-px}, further highlighting the promise of targeting phosphatidylcholine pathways therapeutically in AD.

Our data additionally suggest that the common missense variant p.Ala1527Gly produces effects convergent with ABCA7 LoF. Genetic interactions between mild ABCA7 dysfunction and other AD risk factors, such as APOE4, could significantly amplify AD risk\supercite{Wang2021-oa}. Collectively, these findings align with a growing body of literature, including recent work on APOE4\supercite{Haney2024-fx,Victor2022-tl,Blanchard2022-cf}, highlighting lipid metabolic disruptions as central to AD pathogenesis and identifying additional genotypes that may benefit from phosphatidylcholine-targeted interventions.