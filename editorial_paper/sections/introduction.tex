\textbf{Main}\newline\newline
After APOE4, rare loss-of-function (LoF) mutations in ABCA7 caused by premature termination codons (PTCs) represent one of the strongest genetic risk factors for Alzheimer’s disease (AD), with an odds ratio of approximately 2 \supercite{Steinberg2015-mu,Holstege2022-vp}. Additionally, common single nucleotide polymorphisms in ABCA7 moderately increase AD risk \supercite{Holstege2022-vp,Kunkle2019-yo}, highlighting that ABCA7 dysfunction may contribute to risk in a significant proportion of AD cases. Despite their importance, the cellular mechanisms linking ABCA7 variants to increased AD risk remain poorly understood.

ABCA7 transports phospholipids across cellular membranes, maintaining membrane asymmetry and supporting lipid transport in the brain \supercite{Abe-Dohmae2004-wb,Wang2003-wh,Tomioka2017-nv,Picataggi2022-yp,Quazi2013-pe,Fang2025,Le2023-on}. Studies in rodent models and non-neural cell lines have implicated ABCA7 dysfunction in increased amyloid deposition, impaired clearance by astrocytes and microglia, and heightened glial inflammatory responses \supercite{Satoh2015-yu,Chan2008-qu,Kim2013-sv,Fu2016-qe,Aikawa2019-hv,Aikawa2021-vz}. Despite insights from these studies, there remains a notable lack of research directly investigating ABCA7 LoF effects in human neural tissues. The few existing human studies highlight changes in lipid metabolism as potential disease mechanisms \supercite{Kawatani2023-vf,Liu2021-zh,Duchateau2024-rf}.

Systematic profiling across multiple human neural cell types is needed to understand how ABCA7 LoF contributes to AD. Single-nucleus RNA sequencing (snRNA-seq) of human neural tissues has successfully identified cell type-specific transcriptional changes linked to AD risk variants, such as APOE and TREM2 \supercite{Brase2023-xk,Blanchard2022-cf,Sayed2021-qn}, providing insights into disease mechanisms and therapeutic opportunities. Here, we generated a cell type-specific transcriptomic atlas of ABCA7 LoF variants in the human prefrontal cortex (PFC) using snRNA-seq of postmortem brain tissue. We used these data to predict cell-type-specific impacts of ABCA7 variants and experimentally investigate potential therapeutic targets in human neurons.
