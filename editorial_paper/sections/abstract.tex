\textbf{Loss-of-function variants in the lipid transporter ABCA7 substantially increase Alzheimer’s disease risk \supercite{Steinberg2015-vj,Holstege2022-vp}, yet how they impact cellular states to drive disease remains unclear. Using single-nucleus RNA sequencing of human brain samples, we identified widespread gene expression changes across multiple neural cell types associated with rare ABCA7 loss-of-function variants. Excitatory neurons, which expressed the highest levels of ABCA7, showed disrupted lipid metabolism, mitochondrial function, DNA repair, and synaptic signaling pathways. Similar transcriptional disruptions occurred in neurons carrying the common Alzheimer's-associated variant ABCA7 p.Ala1527Gly\supercite{Kunkle2019-yo}, confirmed by molecular dynamics simulations to alter ABCA7 structure. Induced pluripotent stem cell-derived neurons with ABCA7 loss-of-function variants recapitulated these transcriptional changes, displaying impaired mitochondrial function, increased oxidative stress, and disrupted phosphatidylcholine metabolism. Supplementation with CDP-choline increased phosphatidylcholine synthesis, reversed these abnormalities, and normalized amyloid-β secretion and neuronal hyperexcitability—key Alzheimer's features exacerbated by ABCA7 dysfunction. Our results identify disrupted phosphatidylcholine metabolism as an important pathway underlying ABCA7-related Alzheimer’s risk and highlight a potential therapeutic approach.}\\

