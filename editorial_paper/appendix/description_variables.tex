\subsubsection*{Supplementary Notes}
\phantomsection
\addcontentsline{toc}{subsubsection}{Supplementary Notes}

\paragraph*{Supplementary Note 1.}
\phantomsection
\addcontentsline{toc}{paragraph}{Supplementary Note 1.}
Description of variables according to the Rush Alzheimer’s Disease Center Codebook.
\begin{enumerate}
    \item \textbf{study}: Indicates the cohort participants were recruited through: Religious Orders Study (ROS) or Memory and Aging Project (MAP)\supercite{Bennett2018-tn}.
    
    \item \textbf{pmi}: Postmortem interval, defined as the number of hours from the participant’s time of death until autopsy and tissue preservation.
    
    \item \textbf{age\_death}: Participant's age at the time of death.
    
    \item \textbf{msex}: Participant's self-reported sex, coded as “1” for male and “0” for female.
    
    \item \textbf{amyloid}: Mean amyloid-beta load, quantified as the percent cortical area occupied by amyloid-beta deposits via immunohistochemistry, averaged across 8 brain regions (at least 4 required): hippocampus, entorhinal cortex, midfrontal cortex, inferior temporal gyrus, angular gyrus, calcarine cortex, anterior cingulate cortex, and superior frontal cortex.
    
    \item \textbf{ceradsc}: CERAD neuropathological score providing a semiquantitative measure of neuritic plaque density, categorized into definite AD (1), probable AD (2), possible AD (3), or no AD (4), determined independently of age or clinical data.
    
    \item \textbf{nft}: Summary measure of neurofibrillary tangle burden, derived by microscopic assessment of silver-stained slides from five brain regions (midfrontal cortex, midtemporal cortex, inferior parietal cortex, entorhinal cortex, hippocampus), standardized regionally, and averaged into a single value.
    
    \item \textbf{braaksc}: Semi-quantitative Braak stage score for severity and distribution of neurofibrillary tangles (NFTs), assessed with Bielschowsky silver stain in the frontal, temporal, parietal, entorhinal cortex, and hippocampus: stages I-II indicate tangles primarily in entorhinal regions; stages III-IV signify limbic involvement such as hippocampus; and stages V-VI show moderate to severe neocortical involvement.
    
    \item \textbf{cogdx}: Final clinical consensus diagnosis regarding cognitive status at the time of death, derived by expert neurologists reviewing all available clinical data without knowledge of postmortem findings. Categories include: no cognitive impairment (1), mild cognitive impairment without other causes (2), mild cognitive impairment with another cause (3), Alzheimer's disease without another cause (probable AD) (4), Alzheimer's disease with another cause (possible AD) (5), and other primary dementia (6).
    
    \item \textbf{niareagansc}: The NIA-Reagan neuropathological scoring system classifies Alzheimer's disease likelihood into four levels (1: high, 2: intermediate, 3: low, 4: no AD) based on combined assessment of neurofibrillary tangles (Braak) and neuritic plaques (CERAD), evaluated postmortem without knowledge of clinical dementia status.
    
    \item \textbf{ad\_reagan}: Dichotomized version of the NIA-Reagan neuropathological score, categorizing participants into Alzheimer's-positive (1: high/intermediate) or Alzheimer's-negative (0: low/no AD).
    
    \item \textbf{apoe\_genotype}: APOE genotype identified by DNA extraction from peripheral blood mononuclear cells or brain tissue, using high-throughput sequencing to genotype codon 112 and codon 158 in exon 4 of the APOE gene on chromosome 19.
\end{enumerate}