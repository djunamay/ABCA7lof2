Loss-of-function (LoF) variants in the lipid transporter ABCA7 significantly increase Alzheimer's disease risk (odds ratio $\approx  2$), yet the underlying pathogenic mechanisms and specific neural cell types affected remain unclear. To investigate this, we generated single-nuclear RNA sequencing atlas of 36 human post-mortem prefrontal cortex samples, including 12 ABCA7 LoF carriers and 24 matched non-carriers. ABCA7 LoF was associated with transcriptional changes across all major neural cell types. Excitatory neurons, which expressed the highest levels of ABCA7, showed significant alterations in mitochondrial function, lipid metabolism, DNA damage responses, and synaptic signaling pathways. ABCA7 LoF-associated transcriptional changes in neurons were similarly perturbed in carriers of the common AD missense variant ABCA7 p.Ala1527Gly (n = 240 controls, 135 carriers), indicating that findings from our study may extend to large portions of the at-risk population. Human induced pluripotent stem cell (iPSC)-derived neurons carrying ABCA7 LoF variants closely recapitulated the transcriptional changes observed in human postmortem neurons, particularly those related to mitochondrial function. Biochemical experiments further demonstrated that ABCA7 LoF disrupts mitochondrial membrane potential, increases oxidative stress, and alters phospholipid homeostasis in neurons, notably elevating saturated phosphatidylcholine levels. Supplementation with CDP-choline to enhance de novo phosphatidylcholine synthesis effectively reversed these transcriptional changes, restored mitochondrial function, and reduced oxidative stress. Additionally, CDP-choline normalized amyloid-beta secretion and alleviated neuronal hyperexcitability in ABCA7 LoF neurons. This study provides a detailed transcriptomic profile of ABCA7 LoF-induced changes, highlighting phosphatidylcholine metabolism and mitochondrial dysfunction as key mechanisms in ABCA7-induced risk. Our findings suggest a promising therapeutic approach that may benefit a large proportion of individuals at increased risk for Alzheimer's disease.

