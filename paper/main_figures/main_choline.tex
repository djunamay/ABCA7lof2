\begin{figure}[ht]
%    \includegraphics[width=\textwidth]{./main_plots/fig_5_temp.png}
    \begin{subfigure}[t]{.2\textwidth}
        \begin{subfigure}[t]{\textwidth}
            \caption{}
            \includegraphics[width=\textwidth]{./main_plots/pca_rna_batch1.png}        
        \end{subfigure}  
            \begin{subfigure}[t]{\textwidth}
            \caption{}
            \includegraphics[width=\textwidth]{./main_plots/rna_correlation_plot.png}        
        \end{subfigure}  
    \end{subfigure}  
    \begin{subfigure}[t]{.2\textwidth}
        \begin{subfigure}[t]{\textwidth}
            \caption{}
            \includegraphics[width=\textwidth]{./main_plots/Y622_choline_kl_network.pdf}        
        \end{subfigure}  
            \begin{subfigure}[t]{\textwidth}
            \caption{}
            \includegraphics[width=\textwidth]{./main_plots/choline_y622_jaccard.png}        
        \end{subfigure}  
    \end{subfigure} 
    \begin{subfigure}[t]{.4\textwidth}
        \caption{}
        \includegraphics[width=\textwidth]{./main_plots/y622_choline_density.pdf}        
    \end{subfigure}  
    \begin{subfigure}[t]{.25\textwidth}
        \caption{}
        \includegraphics[width=\textwidth]{./main_plots/volcano_all_species_choline.png}        
    \end{subfigure}  
    \begin{subfigure}[t]{.15\textwidth}
        \caption{}
        \includegraphics[width=\textwidth]{./main_plots/choline_uncoupling.png}        
    \end{subfigure}  
    \begin{subfigure}[t]{.1\textwidth}
        \caption{}
        \includegraphics[width=\textwidth]{./main_plots/choline_mitohealth.png}        
    \end{subfigure}  
    \begin{subfigure}[t]{.2\textwidth}
        \caption{}
        \includegraphics[width=\textwidth]{./main_plots/tmrm_choline.png}        
    \end{subfigure}  
    \begin{subfigure}[t]{.3\textwidth}
        \caption{}
        \includegraphics[width=\textwidth]{./main_plots/tmrm_choline_images.png}        
    \end{subfigure}   
    \begin{subfigure}[t]{.3\textwidth}
        \caption{}
        \includegraphics[width=\textwidth]{./main_plots/choline_mito_degs.png}        
    \end{subfigure}   
    \begin{subfigure}[t]{.15\textwidth}
        \caption{}
        \includegraphics[width=\textwidth]{./main_plots/neurospheroid_example.png}        
    \end{subfigure} 
    \begin{subfigure}[t]{.35\textwidth}
        \caption{}
        \includegraphics[width=\textwidth]{./main_plots/abeta_elisa.png}        
    \end{subfigure}  
    \begin{subfigure}[t]{.15\textwidth}
        \caption{}
        \includegraphics[width=\textwidth]{./main_plots/temp.png}        
    \end{subfigure}  
    % \begin{subfigure}[t]{.25\textwidth}
    %     \caption{}
    %     \includegraphics[width=\textwidth]{./main_plots/rna_correlation_plot.png}        
    % \end{subfigure}  


    % \begin{subfigure}[t]{.25\textwidth}
    %     \caption{}
    %     \includegraphics[width=\textwidth]{./main_plots/choline_y622_jaccard.png}        
    % \end{subfigure}  

    % \par

    % \begin{subfigure}[t]{.2\textwidth}
    %     \caption{}
    %     \includegraphics[width=\textwidth]{coupling.png}        
    % \end{subfigure}  

    % \par

    % \begin{subfigure}[t]{.15\textwidth}
    %     \caption{}
    %     \includegraphics[width=\textwidth]{calcium_imaging.png}        
    % \end{subfigure}  
    \caption{
        \textbf{Supplementation with CDP-choline Reduces Neutral Lipid Accumulation, Restores Mitochondrial Function, and Reduces Amyloid Pathology in ABCA7 LoF Neurons.}\\[1ex]
        (A) Left: Representative images per condition as mean-intensity projections of the entire image (NeuN+) and within NeuN+ volumes considered for quantification (LipidSpot, PLIN2). Right: Quantification of neuronal LipidSpot dye fluorescence intensity. P-values computed by linear mixed-effects model on per-NeuN+ volume averages, including well-of-origin as a random effect. $N = 16$ (p.Tyr622* + H2O), $21$ wells (p.Tyr622* + CDP-choline) (5419 cells per condition) from three independent differentiation batches of 4-week-old iNs treated for 2 weeks (see Figure~\ref{fig:snRNA_cohort}2B). 
        (B) Relative mitochondrial uncoupling, quantified by Seahorse oxygen consumption assay for 4-week-old p.Tyr622* iNs treated with CDP-choline or vehicle control for 2 weeks. Relative uncoupling gives the proportion of basal oxygen consumption attributed to uncoupled proton leak. P-values computed by independent sample t-test. $N$ wells = 6 (p.Tyr622* + H2O), 8 (p.Tyr622* + CDP-choline). 
        (C) Left: Representative images per condition as mean-intensity projections of the entire image (NeuN+) and within NeuN+ volumes considered for quantification (MitoHealth, Hoechst). Right: Quantification of neuronal HCS MitoHealth dye fluorescence intensity as a measure of mitochondrial membrane potential. P-values computed by linear mixed-effects model on per-NeuN+ volume averages, including well-of-origin as a random effect. $N = 11$ (p.Tyr622* + H2O; datapoints from Figure~\ref{fig:main_mitochondrial}H), 12 wells (p.Tyr622* + CDP-choline) (3929 cells per condition) from three independent differentiation batches of 4-week-old iNs treated for 2 weeks (see Figure~\ref{fig:snRNA_cohort}2F). 
        (D) Left: Representative images per condition as mean-intensity projections of the entire image (NeuN+) and projections within NeuN+ volumes considered for quantification (Aβ42). Right: Quantification of neuronal Aβ42 fluorescence intensity. P-values computed by linear mixed-effects model on per-NeuN+ volume averages, including well-of-origin as a random effect. $N = 8$ (p.Tyr622* + H2O; datapointes from Figure~\ref{fig:quantification_Abeta42_iPSC_neurons}A) (1466 cells), 7 wells (p.Tyr622* + CDP-choline) (1102 cells) from 4-week-old iNs treated for 2 weeks. 
        (E) Left: Representative images per condition as mean-intensity projections of the entire image (NeuN+) and within NeuN+ volumes considered for quantification (β-secretase). Right: Quantification of neuronal β-secretase fluorescence intensity. P-values computed by linear mixed-effects model on per-NeuN+ volume averages, including well-of-origin as a random effect. $N = 4$ (p.Tyr622* + H2O) (3107 cells), 4 wells (p.Tyr622* + CDP-choline) (2829 cells) from 4-week-old iNs treated for 2 weeks. 
        (F) Proposed model of ABCA7 dysfunction in neurons: ABCA7 loss-of-function (LoF) induces a compositional shift away from phosphatidylcholines (PCs) (1,2), which may directly increase triglycerides (TGs) by increasing precursor availability (3, 4) or indirectly by impairing lipid droplet processing (5) and mitochondrial uncoupling (6). Impaired uncoupling can lead to multiple detrimental cellular effects, including a decreased ability to meet the cell’s energy demand, impaired regulation of carbon breakdown via oxidative phosphorylation (OXPHOS), and increased oxidative stress (7). Abbreviations: DG = diglyceride, TG = triglyceride, MG = monoglyceride, PC = phosphatidylcholine, ROS = reactive oxygen species. For (A-E),  + = treated with CDP-choline. - = treated with vehicle control. Boxes indicate per-condition dataset quartiles, and whiskers extend to the most extreme data points not considered outliers (i.e., within 1.5 times the interquartile range from the first or third quartile). For (A, C-E), individual data points indicate per-well averages of cell-level intensities. Representative images for the quantified channel were processed with condition-wide percentile-based background subtraction and thresholding. Representative images of cell soma underwent per-image percentile-based background subtraction and thresholding, reflecting the segmentation methodology. Schematic in (F) created with Biorender.com.
    }
    \label{fig:main_choline}
\end{figure}