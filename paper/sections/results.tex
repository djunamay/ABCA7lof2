\subsubsection{Single-nuclear transcriptomic profiling of human PFC from ABCA7 LoF-variant carriers} 
To investigate the cell type-specific impact of ABCA7 LoF variants in the human brain, we queried whole genome sequences of >1000 subjects from the Religious Order Study or the Rush Memory and Aging Project (collectively known as ROSMAP) for donors with Alzheimer’s disease diagnoses who are carriers of rare damaging variants in ABCA7 that result in a PTC. We identified 12 heterozygous carriers of ABCA7 LoF variants, including splice region variants (c.4416+2T>G and c.5570+5G>C), frameshift variants (p.Leu1403fs and p.Glu709fs), and nonsense ‘stop gained’ variants (p.Trp1245* and p.Trp1085*) (Figure~\ref{fig:main_atlas}A; Figure~\ref{fig:snRNA_cohort}A,B; Data~\ref{data:ptc_variants}). These variants have previously been associated with increased AD risk in genetic association studies (Table~\ref{tab:annotation_abca7})\cite{Steinberg2015-mu,Holstege2022-vp} and are presumed to induce risk via ABCA7 haploinsufficiency\cite{Duchateau2024-rf}. Analysis of published proteomic data\cite{Johnson2020-ip} confirmed that ABCA7 PTC-variant carriers indeed had lower ABCA7 protein levels in the human postmortem PFC compared to non-carriers (p=0.018; N = 180 Controls, 5 ABCA7 LoF; Figure~\ref{fig:main_atlas}B; Figure~\ref{fig:snRNA_cohort}C). 

We then performed snRNAseq on postmortem PFC samples from the 12 ABCA7 LoF variant carriers and 24 ABCA7 PTC non-carrier controls that were matched based on a number of potentially confounding variables, including AD pathology, age at death, post-mortem intervals, sex, APOE genotype, and cognition (Figure~\ref{fig:main_atlas}C;D; Figure~\ref{fig:snRNA_cohort}D,E; Figure~\ref{fig:snRNA_quality_annotation}A,B; Data~\ref{data:cohort_metadata}; Supplementary Text). We verified that none of the 36 selected subjects carried damaging variants in other AD risk genes (TREM2, SORL1, ATP8B4, ABCA1, and ADAM10)\cite{Holstege2022-vp} and we confirmed ABCA7 genotypes in a subset of ABCA7 LoF carriers and controls in the cohort by Sanger sequencing (Figure~\ref{fig:snRNA_cohort}A) and verified that all single-cell libraries matched their whole genome sequencing counterparts to ensure all samples were appropriately assigned to their respective genotypes (Figure~\ref{fig:snRNA_cohort}B).

For each individual, we obtained fresh frozen tissue from the BA10 region of the PFC and performed snRNAseq using the 10x Genomics Chromium platform, resulting in a total of 150,456 cells (102,710 cells after multiple rounds of quality control) (Figure~\ref{fig:main_atlas}E). After quality control (qc; dimensionality reduction, batch correction, and multiple rounds of clustering; Figure~\ref{fig:snRNA_quality_annotation}C-E; Methods) and cell type annotation with curated marker lists, we identified heterogeneous populations of inhibitory neurons (In, SYT1 & GAD1+), excitatory neurons (Ex, SYT1 & NRGN+), astrocytes (Ast, AQP4+), microglia (Mic, CSF1R+), oligodendrocytes (Oli, MBP & PLP1+), and oligodendrocyte precursor cells (OPCs, VCAN+) (Figure~\ref{fig:main_atlas}E; Figure~\ref{fig:snRNA_quality_annotation}F-I). The putative vascular cell cluster was small and did not meet qc cutoffs (Methods) and was therefore not considered for downstream analysis. After qc, individual-level gene expression profiles correlated well within cell types (mean correlation circa 0.95) and were well represented across individuals (Figure~\ref{fig:snRNA_quality_annotation}J-M).  

\subsubsection{Cell type-specific perturbations in the presence of ABCA7 LoF}
To explore ABCA7-LoF related gene expression changes across all major cell types, we filtered genes based on evidence for transcriptional perturbation in ABCA7 LoF (p<0.05, linear model, total \#genes passing this cutoff = 2,389) in at least one cell type (Ex, In, Ast, Mic, Oli, or OPC), while controlling for known and unknown covariates in the model and considering only genes with cell type-specific detection rates >10\% (see Methods; Data~\ref{data:degs}). Next, we projected perturbed genes from higher dimensional gene score space (where each dimension corresponds to cell type-specific ABCA7 LoF perturbation scores S, where S=sign(log(FC))*-log10(p-value))) into two dimensions for visualization and clustering (Figure~\ref{fig:snRNAseq_gene_scores}A; Methods). 
The 2D visual representation captured the transcriptional landscape of ABCA7 LoF gene changes across all major cell types (Fig1F; Figure~\ref{fig:snRNAseq_gene_scores}B,C). To summarize this landscape in terms of biological pathways, we assigned genes to clusters based on their projections in 2D space and quantified their enrichment for specific biological pathways (Figure~\ref{fig:main_atlas}G; Gene Ontology Biological Process database; Methods). This analysis indicated a number of biological pathways whose perturbation was broadly associated with ABCA7 LoF in postmortem human PFC, including pathways related to heat shock response, inflammation, cell dynamics and adhesion, synaptic function, DNA repair, and cellular metabolism (Figure~\ref{fig:main_atlas}G; Data~\ref{data:pathway_enrichments}). 

Decomposing the ABCA7 LoF transcriptional signature into cell type-specific patterns indicated which ABCA7 LoF-associated gene clusters were commonly or uniquely perturbed across major cell types in the PFC (Figure 1H; Figure~\ref{fig:snRNAseq_gene_scores}D; Data~\ref{data:degs}-5; Methods). For example, microglia exhibited a down-regulated heat shock response (cluster 4), also present to a lesser extent in neurons and OPCs (FDR-adjusted permutation p-value<0.01 & |score|>0.25; Figure~\ref{fig:main_atlas}H). Microglia and astrocytes displayed up-regulated inflammatory clusters, such as cluster 5 and cluster 10, respectively (Figure~\ref{fig:main_atlas}H). OPCs demonstrated perturbed cell motility and endocytic-associated clusters (clusters 1 and 0; Figure~\ref{fig:main_atlas}H). Both neuronal subtypes showed up-regulation of DNA repair-associated cluster 9 (e.g., NUP205, POLD3, VCP; Figure~\ref{fig:main_atlas}H) and down-regulation of clusters 6 and 14, characterized by cell adhesion and synaptic genes (e.g., SYT11, SHANK3, IFT56, GABRA3, NLGN1; Figure~\ref{fig:main_atlas}H). In addition, excitatory neurons exhibited uniquely perturbed clusters, including up-regulated cluster 8 (mitochondrial complex genes, e.g., COX7A2, antioxidant response e.g., PARK7) and down-regulated cluster 2 (lipid synthesis and transport genes, e.g., LDLR, APOL2, NR1H3, ACLY, FITM2; Figure~\ref{fig:main_atlas}H).

Together, these findings indicate that ABCA7 LoF variants may induce widespread, cell type-specific transcriptional changes in the human PFC. This single-cell atlas provides a rich resource for future studies aiming to elucidate the contributions of individual neural cell types to ABCA7 LoF-driven forms of AD risk. This resource will be made available for exploration via the UCSC Single Cell Browser and for further analysis via Synapse (accession ID: syn53461705).

\subsubsection{ABCA7 is expressed most highly in excitatory neurons}
Our snRNAseq data suggest that excitatory neurons expressed the highest levels of ABCA7, compared to other major cell types in the brain (Figure~\ref{fig:main_excitatory}A).  ABCA7 transcripts were detected (count>0) in ~ 30\% of excitatory neurons and ~ 15\% of inhibitory neurons, while the detection rate was considerably lower (<10\%) for microglia and astrocytes and an order of magnitude lower (<3\%) for oligodendrocytes and OPCs (Figure~\ref{fig:abca7_expression}A, B). We validated this expression pattern in an independent published dataset from\cite{Welch2022-ef} (Table~\ref{tab:external_datasets}), where bulk RNA sequencing of NeuN- (glial) and NeuN+ (neuronal) cell populations derived from six human postmortem temporal cortex samples showed significantly higher ABCA7 levels in the neuronal population versus the glial cell population (p=0.021, paired t-test; Figure~\ref{fig:abca7_expression}C). Several control genes, whose expression patterns in glial versus neuronal cells are well established (ABCA1, APOE, and NEUROD1), had expected expression patterns that matched those in the snRNAseq data (Figure~\ref{fig:abca7_expression}B,C). These results indicate that neurons, particularly excitatory neurons, are the primary ABCA7-expressing cell type in the aged human PFC. Given the relatively higher expression of ABCA7 in excitatory neurons and the evidence of transcriptional perturbations by ABCA7 LoF in this cell type, we focused our subsequent analysis specifically on excitatory neurons.

\subsubsection{ABCA7 LoF perturbations in excitatory neurons}
As an alternative approach to the unsupervised clustering of gene perturbation scores among all cell types, we next used prior knowledge of biological pathway structure to perform an in-depth characterization of perturbed biological processes specifically in ABCA7 LoF excitatory neurons. To this end, we first estimated statistical overrepresentation of biological gene sets (WikiPathways, N pathways = 472) among up and down-regulated genes in ABCA7 LoF excitatory neurons vs controls (by GSEA; Methods). We observed a total of 34 pathways with evidence for transcriptional perturbation at p<0.05 in excitatory neurons (Data~\ref{data:kl_clusters}). Enrichments of these pathways were driven by 268 unique genes (“leading edge” genes\cite{Subramanian2005-pu}; Data~\ref{data:kl_clusters}).   

To extract unique information from leading-edge genes and limit pathway redundancy, we next separated these genes and their associated pathway annotations into non-overlapping groups, formalized as a graph partitioning problem (Figure~\ref{fig:main_excitatory}B; Figure~\ref{fig:benchmarking_clustering}; Methods; Supplementary Text). Establishing gene-pathway groupings of approximately equal size revealed eight biologically interpretable “clusters” associated with ABCA7 LoF in excitatory neurons (Figure~\ref{fig:main_excitatory}B-D; Data~\ref{data:kl_clusters}). Predominantly, these gene clusters centered around two themes: (1) energy metabolism and homeostasis (C2, C7) and (2) DNA damage (C1, C4, C5), cell stress (C3, C6), and synaptic dysfunction (C0).

Clusters C2 and C7 were primarily defined by genes involved in cellular energetics, including genes related to lipid metabolism, mitochondrial function, and  OXPHOS (Figure~\ref{fig:main_excitatory}C,D). Cluster 2, characterized by transcriptional regulators of lipid homeostasis (e.g. NR1H3, ACLY, PPARD), exhibited evidence for down-regulation in ABCA7 LoF and featured pathways related to "SREBP signaling" and "lipid metabolism" (Figure~\ref{fig:main_excitatory}C,D; Data~\ref{data:kl_clusters}). Cluster 7 comprised multiple mitochondrial complex genes (e.g. COX7A2, NDUFV2) responsible for ATP generation from carbohydrate and lipid catabolism and showed up-regulation in ABCA7 LoF (Figure~\ref{fig:main_excitatory}C,D; Data~\ref{data:kl_clusters}).The remaining clusters C0, C1, and C3-C6 were characterized by DNA damage and proteasomal, inflammatory, and apoptotic mediators. Clusters C1, C3, C4, and C5 were up-regulated in ABCA7 LoF excitatory neurons and characterized by pathway terms such as "DNA damage response," "Parkin-Ubiquitin Proteasomal system,” "Signaling pathways in glioblastoma," and "Nucleotide metabolism," respectively (Figure~\ref{fig:main_excitatory}C,D; Data~\ref{data:kl_clusters}). They included up-regulated DNA damage/repair and proteasomal genes (e.g. RECQL, TLK2, BARD1, RBL2, MSH6, PSMD5, PSMD7) and were accompanied by increased expression of inflammatory mediators (e.g. C2, C1S, CLU, PIK3CD in C0, C4, C6) and down-regulated gene signatures related to synaptic function (C0), cytoskeletal function (C3), and apoptosis-related genes (C6) (Figure~\ref{fig:main_excitatory}C,D; Data~\ref{data:kl_clusters}). 

Together, these data suggest that ABCA7 LoF may disrupt energy metabolism in excitatory neurons and that these disruptions coincide with a state of increased cellular stress, characterized by genomic instability and neuronal dysfunction.

\subsubsection{ABCA7 LoF and common missense variants lead to overlapping neuronal perturbations}
ABCA7 LoF variants substantially increase AD risk (Odds Ratio = 2.03)\cite{Steinberg2015-mu} but are rare and therefore only contribute to a small portion of AD cases\cite{Duchateau2024-rf}. To evaluate whether ABCA7 LoF transcriptomic effects in neurons would generalize to more common, moderate-risk genetic variants in ABCA7, we examined the ROSMAP WGS cohort for carriers of the prevalent ABCA7 missense variant p.Ala1527Gly (rs3752246: Minor Allele Frequency ≈ 0.18; \% carriers >= 1 allele  ≈  30\%; Figure~\ref{fig:main_excitatory}E), which has risen to genome-wide significance for increased AD risk (Odds Ratio = 1.15 [1.11-1.18])\cite{Kunkle2019-yo,Holstege2022-vp,Naj2011-bs}, but whose molecular effects are unexplored. We identified 133 individuals carrying at least one copy of the p.Ala1527Gly risk variant and 227 non-carriers (Figure~\ref{fig:main_excitatory}F) for whom snRNAseq data of the post-mortem PFC were available\cite{Mathys2023-rs}. We ensured that none of these 360 individuals were part of our earlier ABCA7 LoF snRNAseq cohort as control samples and that none carried ABCA7 LoF variants. Using this cohort, we investigated whether excitatory neurons from p.Ala1527Gly carriers exhibited evidence of transcriptomic perturbations in the ABCA7 LoF-associated clusters C0-C7.

Remarkably, all clusters displayed directional trends in p.Ala1527Gly neurons consistent with the mean effect sizes observed in ABCA7 LoF neurons (Figure~\ref{fig:main_excitatory}C,G), while controlling for pathology, age, sex, and other covariates (Methods). Notably, 4 out of 8 clusters exhibited substantial evidence of perturbation in p.Ala1527Gly variant carriers, with perturbation directions aligning with predictions for ABCA7 LoF (Figure~\ref{fig:main_excitatory}G,H). Specifically, we observed an up-regulation in the DNA damage cluster C1 and the proteasomal cluster C3 in p.Ala1527Gly carriers compared to controls, suggesting a similar cell stress and genomic instability signature to ABCA7 LoF carriers (Figure~\ref{fig:main_excitatory}G,H), and a borderline significant up-regulation of the mitochondrial cluster C7, again consistent with ABCA7 LoF (Figure~\ref{fig:main_excitatory}G,H). Intriguingly, the most significantly perturbed cluster was the lipid cluster C2, which exhibited down-regulation (Figure~\ref{fig:main_excitatory}G,H) similar to our observations in ABCA7 LoF carriers. 

Because a missense variant can alter the dynamic structure of a protein, and mutations to glycine are known to introduce significantly greater local flexibility compared to alanine, we evaluated whether the convergent transcriptional signature could be attributed to changes in ABCA7 protein structure. Leveraging recent advances in protein structure simulation and the newly crystallized structures of ABCA7 in its closed (ATP-bound)  (Figure~\ref{fig:main_excitatory}I) and open (ATP-unbound) (Figure~\ref{fig:md_simulations}A) conformations\cite{Le_Thi_My2022-dp,Jumper2021-na} , we conducted molecular dynamics simulations of a 239 residue region with and without the p.Ala1527Gly substitutions over a 300ns simulation time (Figure~\ref{fig:main_excitatory}J; Figure~\ref{fig:md_simulations}A; Methods; Supplementary Text). These simulations revealed that the AD risk-associated G1527 mutation in the ATP-bound closed form induced local flexibility, evidencing large structural fluctuations over time, while the A1527 variant had a stabilizing effect on the ATP-bound closed conformation with only minor structural fluctuations (Figure~\ref{fig:main_excitatory}K-M). Both variants displayed stable conformational behavior in the unbound-open conformation (Figure~\ref{fig:md_simulations}A-F). Since the closed ATP-bound conformation is thought to mediate lipid extrusion to the plasma membrane and apolipoproteins\cite{Le_Thi_My2022-dp}, our data indicate that the p.Ala1527Gly substitution significantly impacts local secondary structure  stability and conformational dynamics of ABCA7, potentially affecting its lipid extrusion function.  

Together, our molecular dynamics simulations indicate that the p.Ala1527Gly substitution likely increases AD risk by perturbing ABCA7 structure, and therefore function. As such, our findings imply that rare and common ABCA7 LoF variants may mediate AD risk via similar, ABCA7-dependent mechanisms  and that findings from an in-depth study of ABCA7 LoF may extend to larger portions of the at-risk population. Moreover, the pronounced lipid signature associated with both LoF and the common missense variant in excitatory neurons, which aligns with ABCA7's functional role as a lipid transporter, suggests a potential upstream role of lipid disruptions in ABCA7-mediated risk.  

\subsubsection{Deriving human neurons with ABCA7 LoF variants}
To complement the correlative analyses in ABCA7 LoF human tissue, we next used CRISPR-Cas9 genome editing to generate two isogenic iPSC lines, each homozygous for ABCA7 LoF variants, from a parental line without ABCA7 variants (WT). The first LoF variant, ABCA7 p.Glu50fs*3, was generated by a single base-pair insertion in ABCA7 exon 3, resulting in a PTC early in the ABCA7 gene (Figure~\ref{fig:main_lipids}A; Figure~\ref{fig:differentiating_iPSC_neurons}A-C). This 5’-proximal PTC limits the possibility of producing a truncated protein with partial functionality, generating a near-complete knockout. The second LoF variant, ABCA7 p.Tyr622*, was generated by a single base-pair mutation in ABCA7 exon 15 (Figure~\ref{fig:main_lipids}A; Figure~\ref{fig:lipid_mitochondrial_perturbations}A-C). This PTC re-creates a variant previously observed in patients as associated with AD \cite{Steinberg2015-mu} and is meant to provide clinical context to ABCA7 dysfunction. 

We differentiated the isogenic iPSCs into neurons (iNs) through lentiviral delivery of a doxycycline-inducible NGN2 expression cassette as previously described\cite{Ho2016-kz} (Figure~\ref{fig:main_lipids}B; Figure~\ref{fig:differentiating_iPSC_neurons}D). At 2 and 4 weeks post-NGN2 induction, cells were immunoreactive for neuronal markers TUJ1 and MAP2 and showed robust neuronal processes by pan-axonal staining (Figure~\ref{fig:main_lipids}B; Figure~\ref{fig:differentiating_iPSC_neurons}E). Both WT and ABCA7 LoF lines had the ability to fire action potentials upon current injections (Figure~\ref{fig:main_lipids}C; Figure~\ref{fig:differentiating_iPSC_neurons}F-J). Though the ABCA7 genotype did not affect resting membrane potential (Figure~\ref{fig:differentiating_iPSC_neurons}H), ABCA7 LoF iNs fired more action potentials and with lesser magnitude of current injection than WT iNs (Figure~\ref{fig:differentiating_iPSC_neurons}I,J), indicating an ABCA7 LoF hyperexcitability phenotype. 

Previous studies suggest that ABCA7 dysfunction affects amyloid processing\cite{Satoh2015-yu,Sakae2016-uy,Bamji-Mirza2018-xt,Chan2008-qu,De_Roeck2018-zx}. To verify these findings in our ABCA7 LoF iNs, we measured Aβ42 and Aβ40 levels in media from iNs after four weeks of culture. Using enzyme-linked immunosorbent assay (ELISA), we found that ABCA7 LoF iNs exhibited a significant increase in both secreted Aβ42 and Aβ40 compared to wild-type cells (Figure~\ref{fig:differentiating_iPSC_neurons}K). Together, these data (1) confirm successful neuronal differentiation from iPSCs, and (2) show that ABCA7 LoF iNs recapitulate key AD-associated phenotypes of hyperexcitability and amyloid pathology.

\subsubsection{ABCA7 LoF iNs Recapitulate Excitatory Neuronal Transcriptional Signatures}
To determine whether the transcriptional changes associated with ABCA7 LoF in \emph{postmortem} human neurons recapitulated in ABCA7 LoF iNs, we performed bulk mRNA sequencing on ABCA7 WT, p.Glu50fs3, and p.Tyr622 iNs (N=2, N=5, and N=5, respectively) after four weeks of culture. A strong correlation in gene perturbation scores (-log10(p-value)*sign(log2(FC))) was observed between p.Glu50fs3 vs. WT and p.Tyr622 vs. WT comparisons (Pearson correlation coefficient = 0.84; Figure~\ref{fig:main_lipids}D), indicating that both ABCA7 LoF variants induced similar transcriptional perturbations.

As in our analysis of \emph{postmortem} neurons, we next performed gene set enrichment analysis (GSEA) on differentially expressed genes in p.Glu50fs3 vs. WT and p.Tyr622 vs. WT iNs, identifying 22 and 17 significantly perturbed pathways (FDR-adjusted p < 0.05), driven by 425 and 335 unique “leading edge” genes\cite{Subramanian2005-pu}. These genes clustered into 12 and 10 biologically interpretable groups, respectively, with substantial overlap between the two lines (Figure~\ref{fig:main_lipids}E; Figure~\ref{fig:bulk_RNAseq_supplement}A,B). The transcriptional signatures of ABCA7 LoF iNs closely mirrored those in \emph{postmortem} neurons, with most clusters exhibiting similar gene/pathway content and directional trends (Figure~\ref{fig:main_lipids}G). \textcolor{red}{Specifically, as in the postmortem data, we observed upregulation of mitochondrial OXPHOS-related clusters (C0, C1 in p.Tyr622*; C7 in postmortem brain), while a lipid biosynthesis cluster was downregulated and DNA damage clusters were upregulated. The meaningful overlap in gene and pathway content between iNs and postmortem human brain data supports a causal link between ABCA7 LoF and disruptions in mitochondrial function, lipid metabolism, and DNA integrity.}

\subsubsection{ABCA7 LoF impacts the neuronal lipidome}
\textcolor{red}{To investigate the effects of ABCA7 LoF variants on lipid homeostasis specifically in neurons, we performed untargeted LC-MS lipidomics on two independent iN differentiation batches containing the p.Glu50fs*3 ABCA7 mutation (N=3 p.Glu50fs*3 and N=3 WT for each batch), selecting the more severe mutation to maximize phenotypic effects. As lipidomic profiles across differentiation batches correlated well (Figure~\ref{fig:LCMS_supplement}A), we pooled samples from each batch to increase statistical power and reduce the impact of potential outliers. Overall, we observed that the lipid profiles of ABCA7 p.Glu50fs*3 iNs differed substantially from their WT counterparts and were linearly separable when projected on the second principal component (Figure~\ref{fig:main_lipids}H). We observed strong evidence (p<5e-3) for perturbation of sphingolipids (specifically, ceramides; (Cer)), phospholipids (including phosphatidylcholine (PC), cardiolipin (CL), phosphatidylserine (PS), and phosphatidylglycerol (PG)), and neutral lipids (specifically, triglycerides (TGs) and monoglycerides (MGs)), in ABCA7 p.Glu50fs*3 vs WT iNs (Figure~\ref{fig:main_lipids}I).} 

\textcolor{red}{Triglycerides were in absolute and relative terms the most frequently perturbed lipid species (N=31; 56\% of all detected TG species; hypergeometric enrichment p-value = 3.9e-15)(Data~\ref{data:ngn2_lipidome}; Figure~\ref{fig:main_excitatory}I). All differentially detected triglyceride species were up-regulated in ABCA7 LoF iNs (Figure~\ref{fig:main_lipids}I, J), indicating increased levels of neutral lipid accumulation in these cells. The up-regulated triglycerides tended to be polyunsaturated (Figure~\ref{fig:main_lipids}K). The conserved lipid cluster 2 across postmortem and bulk RNAseq data supported a down-regulation of lipid synthesis genes (e.g. SREBP), a decrease in TG hydrolysis genes (e.g. LPL), and lipid oxidation genes (e.g. CPT2) (Table~\ref{tab:cluster2_genes_y622}); Perhaps suggesting general response to lipid burden / storage and decreased metabolism of these lipids. PUFAs tend to be stored in TGs?.}

\textcolor{red}{While none of the remaining lipid classes were significantly enriched (hypergeometric p<0.05) among the differentially detected species (Figure~\ref{fig:main_lipids}I), we did note that phosphatidylcholine species emerged as the second most frequently altered lipid class in absolute terms, including a total of 21 perturbed species (p<5e-3; 16\% of all detected PC species)(Figure~\ref{fig:main_lipids}I, K). These changes were followed by a smaller number of perturbations to other lipid species, such as ceramides (N=5 changes) and cardiolipin (N=2 changes). Interestingly, the up-regulated PCs tended to be saturated and the down-regulated ones tended to be polyunsaturated; Perhaps suggesting suggesting diversion of polyunsaturated PCs to TGs and decreased export of the more saturated ones? Interestingly cluster 2 also included several desaturase genes which were down-regulated (e.g. SCD; Table~\ref{tab:cluster2_genes_y622}).}

\subsubsection{ABCA7 LoF alters the neuronal metabolome}
\textcolor{red}{To understand if these lipid changes were associated with global perturbations to the neuronal metabolome in ABCA7 LoF iNs, we performed untargeted metabolomic analysis on the aqueous fraction acquired during lipidomics preparations. Again, we observed consistency between batches and pooled the results for further analysis (Figure~\ref{fig:LCMS_supplement}B). Projection of metabolite levels onto the first principal component completely separated samples by genotype (Figure~\ref{fig:main_mitochondrial}A), suggesting that the metabolomes between WT and ABCA7 LoF iNs differ substantially.} 

\textcolor{red}{We observed 10 increased and 49 decreased metabolites in ABCA7 LoF vs WT iN (p-value < 0.05 & |log2FC|>1; Data~\ref{data:ngn2_metabolome}; Figure~\ref{fig:main_mitochondrial}B). Nine of these differentially abundant metabolites could be annotated with high confidence and were not detected in the background media (Methods). Annotated species with differential abundance included a number of carnitine species, glutamine-glutamate, inosine-inositol, and hypoxanthine, all of which were less abundant in ABCA7 LoF vs WT iNs (Figure~\ref{fig:main_mitochondrial}B). Together, these data implicate a number of perturbed metabolic processes in ABCA7 LoF including lipid catabolism, TCA cycle, and OXPHOS (via carnitine\cite{Virmani2022-uc}, glutamine-glutamate\cite{Yoo2020-lh,noauthor_2023-sp}), cellular redox state (via inosine-inositol\cite{Chatree2020-qn,Basile2022-dd}, hypoxanthine\cite{Furuhashi2020-oi,Lee2018-tk}), and excitatory synaptic signaling (via glutamine-glutamate\cite{Morland2022-dk,noauthor_2021-cn,noauthor_2022-jz} and carnitine\cite{Inazu2008-wg,noauthor_2016-gp,Janiri1991-sx}).} 

\textcolor{red}{We found a marked down-regulation of carnitine and its esterified forms, acetylcarnitine and propionylcarnitine, in p.Glu50fs3 vs. WT iNs (log2FC -1.35, -1.32, -3.55, respectively) (Figure~\ref{fig:main_mitochondrial}B, C). Carnitine and its derivatives facilitate the transfer of medium- and long-chain fatty acids into mitochondria for β-oxidation\cite{noauthor_2016-la,noauthor_2004-tm}. Their reduction in ABCA7 LoF iNs suggests impaired lipid β-oxidation, consistent with the down-regulation of several β-oxidation genes in these cells (Table~\ref{tab:cluster2_genes_y622}). Although lactate and glucose are the primary fuels for neuronal OXPHOS\cite{Dienel2018-dt,Trigo2022-ym,Yellen2018-kr}, up to 20\% of basal neuronal OXPHOS may rely on lipid β-oxidation\cite{Morant-Ferrando2023-va}. Our snRNAseq data from postmortem PFC show that excitatory neurons express key β-oxidation regulators (>10\% detectable levels), including HADHA, which catalyzes the final β-oxidation steps, and carnitine palmitoyltransferases (CPT1C, CPT1B), which mediate fatty acid transport into mitochondria (Figure~\ref{fig:LCMS_supplement}C). Together with transcriptional changes in mitochondrial-related genes, the observed carnitine depletion suggests altered mitochondrial bioenergetics in ABCA7 LoF neurons, potentially linking these defects to broader lipid metabolism disruptions.}

\subsubsection{ABCA7 LoF impairs mitochondrial uncoupling in neurons}
To directly assess mitochondrial function in ABCA7 LoF neurons, we measured the oxygen consumption rate (OCR) of WT and ABCA7 LoF iNs over time using the Seahorse metabolic flux assay (Agilent) (Figure~\ref{fig:oxygen_consumption_rates_iPSC_neurons}A,B). The mitochondrial membrane potential (ΔѰm) is maintained by the oxygen-driven movement of protons across the inner mitochondrial membrane during OXPHOS, which is then dissipated by ATP synthase and regulated proton leakage (Figure~\ref{fig:main_mitochondrial}D). Measuring OCR in the presence of mitochondrial inhibitors thus provides several functional readouts.

Because OCR can be influenced by cell viability and mitochondrial abundance\cite{Divakaruni2014-eq,Gu2021-ms}, we report internally normalized mitochondrial function metrics\cite{Divakaruni2022-rj} for WT, ABCA7 p.Glu50fs30, and ABCA7 p.Tyr622 iNs. We first assessed spare respiratory capacity by measuring the increase in oxygen consumption following pharmacological collapse of the proton gradient\cite{Divakaruni2022-rj} (Figure~\ref{fig:oxygen_consumption_rates_iPSC_neurons}C). We then quantified basal oxygen consumption attributed to ATP synthesis versus proton leak (i.e., uncoupled mitochondrial OCR)\cite{Divakaruni2014-eq} (Figure~\ref{fig:main_mitochondrial}E).

Although spare respiratory capacity was comparable between WT and ABCA7 LoF iNs (Figure~\ref{fig:oxygen_consumption_rates_iPSC_neurons}D), ABCA7 LoF iNs exhibited reduced uncoupled respiration (Figure~\ref{fig:main_mitochondrial}F), a finding replicated in two independent experiments (Figure~\ref{fig:oxygen_consumption_rates_iPSC_neurons}E,F). In WT iNs, uncoupled mitochondrial OCR (20\%) (Figure\ref{fig:main_mitochondrial}F) aligns with previous reports in neurons and other cell types\cite{Divakaruni2011-uj,Jekabsons2004-fn}, indicating that ABCA7 LoF iNs have abnormally low mitochondrial uncoupling. This was further supported by reduced expression of UCP2, the principal neuronal uncoupling protein \textcolor{red}{citation}, in ABCA7 LoF iNs (Figure~\ref{fig:oxygen_consumption_rates_iPSC_neurons}G) \textcolor{red}{specify significance}. Because decreased mitochondrial uncoupling often correlates with a hyperpolarized mitochondrial membrane potential (ΔѰm) \textcolor{red}{citation}, we first assessed ΔѰm in NeuN+ soma using MitoHealth dye, which accumul   ates in proportion to membrane potential. We then measured ΔѰm in both soma and processes using TMRM, followed by FCCP treatment to confirm signal specificity. As expected, ABCA7 LoF iNs displayed elevated ΔѰm compared to WT, as indicated by MitoHealth fluorescence intensity per NeuN+ surface (Figure~\ref{fig:main_mitochondrial}H) and TMRM staining (Figure~\ref{fig:main_mitochondrial}G).

To further investigate the mitochondrial state of ABCA7 LoF iNs, we analyzed 1136 mitochondrial genes from the MioCarta database, which contains genes with high-confidence annotation of mitochondrial localization, in our bulk RNAseq data. Several of the most significantly upregulated genes (e.g., CASP3, BID) were involved in the mitochondrial apoptosis pathway (Figure~\ref{fig:main_mitochondrial}E; Table~\ref{tab:y622_mito_genes}), \textcolor{red}{while supercomplex assembly genes (e.g., TIM) were also increased} (Figure~\ref{fig:main_mitochondrial}E; Table~\ref{tab:y622_mito_genes}). By contrast, downregulated genes were significantly enriched (padj<0.05) for metabolic functions, including beta-oxidation (ACAD and CPT genes), metabolite transfer into mitochondria (SLC25 genes), and oxidative stress detoxification (CAT) (Figure~\ref{fig:main_mitochondrial}E; Table~\ref{tab:y622_mito_genes}).

Together, these data demonstrate that ABCA7 LoF iNs exhibit a hyperpolarized mitochondrial membrane potential and reduced regulated uncoupling, alongside a transcriptomic profile consistent with mitochondrial stress and impaired lipid, \textcolor{red}{small metabolites}, and reactive oxygen metabolism.

\subsubsection{Treatment with CDP-choline reverses transcriptional signature}
\textcolor{red}{CDP-choline framing.}

\textcolor{red}{To test this, we treated the patient variant containing line ABCA7 p.Tyr622* iNs with CDP-choline, the rate-limiting precursor for phosphatidylcholine synthesis through the Kennedy pathway \cite{Zeisel2009-xv,Son2024-tu}, for 2 weeks and then performed bulk RNAseq. Comment here on the uptake of CDP-choline; what part in the pathway it is really targeting; reference the SLC genes and perhaps the PC/Choline related metabolism genes (Figure~\ref{fig:lipid_mitochondrial_effects_CDP_choline}A-C). Maybe this paragraph should just be on the rational of treating with Choline; whether it is being taken up and having it's expected effect on PC synthesis. Then have a separate section on the transcriptional changes.}

Transcriptionally, CDP-choline treatment induces significant changes, making treated samples distinct from untreated ones (Figure~\ref{fig:main_choline}A). Notably, treated samples cluster more closely with WT samples along PC dimension 1 (Figure~\ref{fig:main_choline}A). Strikingly, we observe a strong negative correlation between the transcriptional signature of CDP-choline treatment and that of p.Tyr622 (Figure~\ref{fig:main_choline}B), suggesting a partial reversal toward the wild-type state. \textcolor{red}{Cluster analysis reveals key clusters shifting after treatment, but now in the opposite direction—most notably those associated with mitochondrial function, DNA damage, and lipid metabolism (Figure~\ref{fig:main_choline}C-E). Quantitative overlap analysis confirms a highly significant inverse relationship between these treatment-induced clusters and those differentiating WT from p.Tyr622 (Figure~\ref{fig:main_choline}D). Together, these data suggest that CDP-choline treatment may reverse the transcriptional signature of ABCA7 LoF iNs, potentially restoring mitochondrial function and lipid metabolism.}

\subsubsection{Lipid and mitochondrial effects of CDP-choline treatment}
\textcolor{red}{We examined whether CDP-choline treatment reverses lipid alterations and whether these changes align with transcriptional signatures. Many key lipid-related genes showed rescue following treatment, including LPL, SC5D, LDLR, FABP3, PPARD, and SCD, as well as increased expression of CPT2 and SREBF1, along with several cholesterol biosynthesis genes and desaturases. These changes suggest a potential increase in fatty acid oxidation, fatty acid remodeling/desaturation, and lipid synthesis.} 

\textcolor{red}{To assess mitochondrial function, we measured oxygen consumption rates in ABCA7 p.Tyr622* iNs with and without CDP-choline treatment (Figure~\ref{fig:lipid_mitochondrial_effects_CDP_choline}D,E). Notably, CDP-choline restored uncoupled respiration in p.Tyr622* iNs to WT levels (Figure~\ref{fig:main_choline}G), while having no significant effect on SRC (Figure~\ref{fig:lipid_mitochondrial_effects_CDP_choline}F). In line with this increased uncoupling, CDP-choline supplementation reduced ΔѰm in p.Tyr622* iNs compared to vehicle-treated cells, as evidenced by decreased MitoHealth dye accumulation and TMRM staining (Figure~\ref{fig:main_choline}H,I). At the transcriptional level, CDP-choline treatment also reversed key mitochondrial stress-related gene expression patterns. Specifically, CASP3, BID, and CAT expression was rescued (Figure~\ref{fig:main_choline}J), while many Complex I genes were downregulated (Table~\ref{tab:choline_mito_genes}). Furthermore, CDP-choline treatment resulted in a significant inverse correlation across all MioCarta genes, culminating in an overall restoration of the mitochondrial transcriptional signature (Figure~\ref{fig:lipid_mitochondrial_effects_CDP_choline}G).}

\subsubsection{CDP-Choline Ameliorates AD-Associated Phenotypes in Neurospheroids}
To determine whether CDP-choline treatment could mitigate later-stage AD-associated phenotypes observed in ABCA7 LoF iNs—such as increased amyloid-beta secretion and neuronal hyperactivity—we used a neurospheroid system of increased maturity. Unlike iNs, neurospheroids better model these phenotypes due to their enhanced cellular complexity and extended culture period (Figure~\ref{fig:neurospheroid_figure}A; Figure~\ref{fig:main_choline}K).

Consistent with our iN findings, ABCA7 LoF neurospheroids secreted higher levels of Aβ40 and Aβ42 into the media (Figure~\ref{fig:main_choline}M; Figure~\ref{fig:neurospheroid_figure}B). Notably, treatment with 1 µM CDP-choline reduced this secretion to WT levels (Figure~\ref{fig:main_choline}L). However, this effect was not observed at lower CDP-choline concentrations or at time points where amyloid-beta secretion was not yet elevated (Figure~\ref{fig:main_choline}M; Figure~\ref{fig:neurospheroid_figure}C).

In addition to increased amyloid-beta secretion, ABCA7 LoF neurospheroids exhibited significant neuronal hyperactivity, as measured by electrophysiology (Figure~\ref{fig:neurospheroid_figure}D; Figure~\ref{fig:main_choline}N). CDP-choline treatment successfully rescued this hyperactivity phenotype (Figure~\ref{fig:main_choline}N). Further analysis of calcium signaling data supports these findings, indicating that CDP-choline restores neuronal activity to WT-like levels (Figure~\ref{fig:neurospheroid_figure}E).

