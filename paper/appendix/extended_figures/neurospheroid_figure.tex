\begin{figure}[ht]
    \begin{subfigure}[t]{\textwidth}
        \caption{}
        \includegraphics[width=\textwidth]{./extended_plots/neurospheroid_markers.png}        
    \end{subfigure}
    \begin{subfigure}[t]{0.2\textwidth}
        \caption{}
        \includegraphics[width=\textwidth]{./extended_plots/additional_abeta_timepoints.png}        
    \end{subfigure}
    \begin{subfigure}[t]{0.2\textwidth}
        \caption{}
        \includegraphics[width=\textwidth]{./extended_plots/other_choline_conc.png}        
    \end{subfigure}
    \begin{subfigure}[t]{0.2\textwidth}
        \caption{}
        \includegraphics[width=\textwidth]{./extended_plots/ephys_additional.png}        
    \end{subfigure}
    \begin{subfigure}[t]{0.2\textwidth}
        \caption{}
        \includegraphics[width=\textwidth]{./extended_plots/calcium_imaging.png}        
    \end{subfigure}
    \caption{
        \textbf{CDP-choline treatment in neurospheroids}\\[1ex]
        (A) Per cell type ABCA7 detection rate of major cell types in the post-mortem PFC as quantified by snRNA-seq. 
        (B) Normalized expression of indicated gene in glial cells (per-individual mean expression profiles across Oli, Opc, Ast, Mic) vs. neuronal cells (per-individual mean expression profiles across Ex and In) from post-mortem snRNA-seq data. 
        (C) Normalized expression of indicated genes in NeuN- vs. NeuN+ cells (N=6 individuals, from \cite{Welch2022-aa}; see Table 2). All p-values are computed by paired two-sided t-test. Boxes indicate per-condition dataset quartiles, and whiskers extend to the most extreme data points not considered outliers (i.e., within 1.5 times the interquartile range from the first or third quartile).
    }
    \label{fig:neurospheroid_figure}
\end{figure}
