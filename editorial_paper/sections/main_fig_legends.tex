\subsubsection{Main figure Legends}
\textbf{Fig. 1: Single-nuclear RNA-sequencing atlas of human post-mortem prefrontal cortex reveals cell-type-specific gene changes in ABCA7 LoF.}\newline  % ← the title
\textbf{a,} ABCA7 gene structure indicating variant locations studied here (average minor allele frequency <1\%). Exons are black rectangles; introns, black lines. Pie chart indicates frequency of ABCA7 PTC-variant carriers in ROSMAP cohort.
\textbf{b,} Overview of human snRNA-seq cohort (created with BioRender.com).
\textbf{c,} Metadata summary of snRNA-seq cohort ($N=36$ individuals).
\textbf{d,} ABCA7 protein abundance in postmortem prefrontal cortex from controls ($N=180$) vs. ABCA7 LoF carriers ($N=5$). Statistical comparison by Wilcoxon rank sum test. Boxes indicate quartiles; whiskers represent data within 1.5× interquartile range.
\textbf{e,} 2D UMAP projection of single-cell gene expression, colored by transcriptionally defined cell type.
\textbf{f,} 2D UMAP projection of ABCA7 LoF gene perturbation scores ($S = -\log_{10}(p)\times\text{sign}(\log_2(\text{FC}))$). Red: $S>1.3$, Blue: $S<-1.3$; point size reflects $|S|$. Up to top 10 genes labeled.
\textbf{g,} 2D UMAP projection colored by gene cluster assignment (Gaussian mixture model; see Methods). Top pathway enrichments per cluster shown (GO BP, hypergeometric enrichment, $p<0.01$).
\textbf{h,} Cell type-specific gene cluster scores ($SC=\text{mean}(S_i)$ for genes $i$ in cluster $c$). * indicates permutation FDR-adjusted $p<0.01$ and $|SC|>0.25$.

\textbf{Fig. 2: Transcriptional Perturbations in Excitatory Neurons in ABCA7 LoF and ABCA7 p.Ala1527Gly Variant Carriers.}\newline
\textbf{a,} (left) 2D UMAP projection of all cell types colored by log-transformed values of log-normalized ABCA7 expression ($\log(\text{Exp})$). (right) Kernighan-Lin (K/L) clustering of leading-edge genes from significantly perturbed pathways ($p<0.05$) in ABCA7 LoF excitatory neurons. Colors indicate distinct K/L clusters (0–7).
\textbf{b,} Gaussian kernel density plots of gene perturbation scores ($S=-\log_{10}(p)\times\text{sign}(\log_2(\text{FC}))$) per K/L cluster. Positive $S$ indicates upregulation in ABCA7 LoF. Solid lines show distribution means. Representative pathways with highest intra-cluster connectivity annotated per cluster.
\textbf{c,} Schematic of ABCA7 gene highlighting the p.Ala1527Gly codon change (purple arrow). Minor allele frequency (MAF) shown at right.
\textbf{d,} Overview of snRNA-seq cohort comparing ABCA7 p.Ala1527Gly carriers (homozygous/heterozygous) to non-carrier controls (MAF $\approx18\%$).
\textbf{e,} Perturbation (FGSEA scores) of ABCA7 LoF-associated gene clusters from (B) in excitatory neurons from p.Ala1527Gly carriers vs. controls. Top $p$-values ($p<0.1$) indicated. Positive scores represent upregulation in carriers.
\textbf{f,} Distribution of gene perturbation scores ($S$) for each K/L cluster comparing ABCA7 p.Ala1527Gly (no fill) vs. LoF variants (solid fill). Positive $S$ indicates upregulation.
\textbf{g,} Closed-conformation ABCA7 protein structure, highlighting domain (residues 1517–1756, yellow) used for molecular simulations. Lipid bilayer shown in orange. Expanded inset highlights Ala1527 (light grey) and Gly1527 (purple) residues.
\textbf{h,} Expanded inset from (G) with residues of interest indicated.
\textbf{i,} Root mean squared deviations (RMSD) of the closed-conformation ABCA7 domain (G) carrying Ala1527 (light grey) or Gly1527 (purple) during simulation, relative to reference closed conformation. Inset violin plot shows average $C_\alpha$ atom positional fluctuations.
\textbf{j,} Projection of $C_\alpha$ positional fluctuations onto the first two principal components during simulation for Ala1527 (top, light grey) and Gly1527 (bottom, purple).

\textbf{Fig. 3: ABCA7 LoF Impacts Regulation of Mitochondrial Uncoupling in Neurons.}\newline
\textbf{a,} Schematic of iPSC-derived isogenic neuronal lines harboring ABCA7 loss-of-function (LoF) variants. Gene structure shows exons (black rectangles) and introns (black lines). CRISPR-Cas9 introduced premature termination codons in exon 3 (p.Glu50fs3, blue) or exon 15 (p.Tyr622*, orange). Confocal images show MAP2 staining in iNs differentiated for 4 weeks (genotypes indicated).
\textbf{b,} Correlation of gene perturbation scores ($S = -\log_{10}(p)\times\text{sign}(\log_2(\text{FC}))$) by bulk mRNAseq comparing p.Glu50fs3 vs. WT and p.Tyr622* vs. WT iNs cultured for 4 weeks.
\textbf{c,} Kernighan-Lin (K/L) clustering of leading-edge genes from significantly perturbed pathways (Benjamini–Hochberg (BH) FDR-adjusted $p<0.05$) in p.Tyr622* vs. WT iNs. Colors represent distinct K/L clusters.
\textbf{d,} Heatmap of Jaccard index overlap between K/L gene clusters from p.Tyr622* neurons and clusters identified in human postmortem excitatory neurons. Red text denotes clusters with average score $S$ upregulated in ABCA7 LoF; blue text denotes clusters with average $S$ downregulated in ABCA7 LoF.
\textbf{e,} Gaussian kernel density plots of gene perturbation scores ($S$) within each cluster. Positive $S$ indicates upregulation in p.Tyr622*. Solid lines denote cluster means. Top enriched pathways with highest intra-cluster connectivity indicated.
\textbf{f,} Volcano plot of differential expression of genes with mitochondrial-localized protein products (MitoCarta) between p.Tyr622* and WT neurons.
\textbf{g,} Seahorse-measured mitochondrial uncoupled oxygen consumption rate (OCR) in WT and ABCA7 LoF and WT iNs cultured for 4 weeks. Each datapoint represents OCR from a single well. $N=18$ (WT), $17$ (p.Tyr622*), $13$ (p.Glu50fs3) wells, across two differentiation batches. Statistical comparison by independent-sample $t$-test.
\textbf{h,} Mitochondrial membrane potential quantified via HCS MitoHealth dye fluorescence intensity in ABCA7 LoF iNs cultured for 4 weeks. Each datapoint represents average intensity per well (NeuN+ volumes averaged). Statistical comparison via linear mixed-effects model, accounting for well-of-origin random effects. $N=8$ (WT), $11$ (p.Tyr622*), $9$ (p.Glu50fs3) wells; $\approx3000$ cells/condition, from three differentiation batches. Each NeuN/GFP image intensity was scaled relative to its maximum value, followed by gamma correction ($\gamma = 0.5$) for visualization.
\textbf{i,} Baseline mitochondrial membrane potential quantified by average TMRM fluorescence intensity per masked region (thresholded at 75th percentile) in ABCA7 LoF and WT iNs cultured for 4 weeks. Each datapoint represents average intensity per well. $N=4$ (WT), $5$ (p.Tyr622*) wells. Statistical comparison by independent-sample $t$-test.
\textbf{j,} Oxidative stress quantified by average CellROX fluorescence intensity per masked region (thresholded at 75th percentile) in p.Tyr622* and WT iNs cultured for 4 weeks. Each datapoint represents average intensity per well. $N=10$ wells per genotype. Statistical comparison by independent-sample $t$-test.
\textbf{k,} Volcano plot of differentially abundant lipid species between p.Tyr622* and WT iNs cultured for 4 weeks, colored by lipid class. Statistical comparisons by independent-sample $t$-tests followed by BH FDR adjustment. $N=10$ wells (WT) and $8$ wells (p.Tyr622*).

\textbf{Fig. 4: CDP-choline Treatment Rescues ABCA7 LoF-Induced Disruptions in Neurons.}\newline  % ← the title
\textbf{a,} Volcano plot of differentially abundant lipid species in p.Tyr622* iNs cultured for 4 weeks (treated with or without 100 $\mu$M CDP-choline during the final 2 weeks), colored by lipid class.  Statistical comparisons by independent-sample $t$-tests. $N=5$ wells per genotype.
\textbf{b,} Correlation of gene perturbation scores ($S = -\log_{10}(p)\times\text{sign}(\log_2(\text{FC}))$) comparing p.Tyr622* vs. WT and p.Tyr622* $\pm$ CDP-choline iNs.
\textbf{c,} Kernighan-Lin (K/L) clustering of leading-edge genes from significantly perturbed pathways (BH FDR-adjusted $p<0.05$) comparing p.Tyr622* $\pm$ CDP-choline iNs. Colors represent distinct K/L gene clusters, matched to p.Tyr622* vs. WT cluster colors based on Jaccard analysis in (D).
\textbf{d,} Heatmap of Jaccard index overlap between K/L clusters from p.Tyr622* vs. WT and p.Tyr622* $\pm$ CDP-choline iNs.
\textbf{e,} (left) Gaussian kernel density plots of gene perturbation scores ($S$, positive values indicate upregulation with CDP-choline treatment) for each cluster. Solid lines denote cluster means. (right) Representative pathways annotating the most genes per cluster.
\textbf{f,} Volcano plot of differential expression of genes with mitochondrial-localized protein products (MitoCarta) for p.Tyr622* $\pm$ CDP-choline iNs.
\textbf{g,} Mitochondrial uncoupling quantified by Seahorse assay (proportion of basal oxygen consumption due to proton leak) in p.Tyr622* $\pm$ CDP-choline iNs cultured for 4 weeks (treated with or without 100 $\mu$M CDP-choline during the final 2 weeks). Each datapoint represents OCR from a single well. Statistical comparisons via independent-sample $t$-tests. $N=6$ (vehicle), $8$ (CDP-choline) wells.
\textbf{h,} Average TMRM fluorescence intensity per mask (thresholded at 75th percentile) in p.Tyr622* $\pm$ CDP-choline iNs cultured for 4 weeks (treated with or without 100 $\mu$M CDP-choline during the final 2 weeks), under baseline and FCCP-treated conditions. $N=8$ wells in each condition.
\textbf{i,} Average CellROX fluorescence intensity per mask (thresholded at 75th percentile) in p.Tyr622* $\pm$ CDP-choline iNs cultured for 4 weeks (treated with or without 100 $\mu$M CDP-choline during the final 2 weeks). Each datapoint represents average intensity per well. $N=10$ wells in each condition.
\textbf{j,} Quantification of secreted A$\beta$ levels from media of cortical organoids derived from WT or p.Tyr622* iPSCs (cultured for 182 days), treated with or without 1 mM CDP-choline for 4 weeks. Each datapoint represents A$\beta$ levels measured for a single cortical organoid. $N=20$ (WT), $19$ (p.Tyr622*), and $14$ (p.Tyr622* + 1 mM CDP-choline) organoids.
\textbf{k,} Spontaneous action potentials recorded from dissociated cortical organoids derived from p.Tyr622* iPSCs (cultured for 150 days, followed by 2 weeks treatment post-dissociation), treated with or without 100 $\mu$M CDP-choline. Each datapoint represents an individual cell. $N=7$ (WT), $13$ (p.Tyr622*), and $9$ (p.Tyr622* + 100 $\mu$M CDP-choline) cells.
