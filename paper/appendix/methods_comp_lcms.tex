\subsubsection{LC-MS data analysis} 

\paragraph{Lipidomic data from postmortem tissue and iNs.}
Lipids were identified, and their signals integrated using the Lipidsearch © software (version 4.2.27, Mitsui Knowledge Industry, University of Tokyo). Integrations and peak quality were curated manually. For specific lipid species (see Figure 3A), statistical significance for differences in peak distributions between control and ABCA7 LoF were computed by two-sided unpaired t-test. For lipid classes (see Figure S9C), peaks were summed for individual lipid species of a given class, normalized to per-sample total lipid abundance, and compared by two-sided unpaired t-test.

\paragraph{Metabolomic data from iN.}
Data were analyzed using Compound Discoverer 3.2 (CD, Thermo Fisher Scientific). Identification was based on MS2/MS3 matching with a local mzVault library and corresponding retention time built with pure standards (level 1), or on mzCloud match (level 2). Each match was manually inspected. For specific metabolite comparisons, only metabolites with high-confidence annotation ('Level 1 ID', 'Level 2 ID', 'MasslistRT ID') that were not also detected in the background media were considered for analysis. Statistical significance for differences in normalized peak areas between control and ABCA7 LoF were computed by two-sided unpaired t-test.

