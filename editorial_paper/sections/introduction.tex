\subsubsection{Main}

After APOE4, rare loss-of-function (LoF) mutations in ABCA7 caused by premature termination codons (PTCs) are among the strongest genetic risk factors for Alzheimer’s disease (AD), with an odds ratio of approximately 2\supercite{Steinberg2015-mu,Holstege2022-vp}. Common single nucleotide polymorphisms in ABCA7 also moderately increase AD risk\supercite{Holstege2022-vp,Kunkle2019-yo}, suggesting ABCA7 dysfunction contributes substantially to disease risk in the broader population. Despite their significance, the precise cellular mechanisms through which ABCA7 LoF variants affect AD risk remain poorly defined.

ABCA7 functions by transporting phospholipids across cell membranes, thereby maintaining membrane asymmetry and facilitating lipid transport within the brain\supercite{Abe-Dohmae2004-wb,Wang2003-wh,Tomioka2017-nv}. Mouse model studies suggest that ABCA7 dysfunction promotes amyloid deposition, impairs amyloid clearance by astrocytes and microglia, and intensifies glial inflammatory responses\supercite{Satoh2015-yu,Kim2013-sv,Fu2016-qe,Aikawa2019-hv}. Additionally, recent research in human cell lines and tissues has identified disturbances in lipid metabolism as potential mechanisms linking ABCA7 dysfunction to AD risk\supercite{Kawatani2023-vf,Liu2021-zh,Duchateau2024-rf}. However, systematic analyses of ABCA7 LoF effects across various human brain cell types have not yet been performed. Investigations specifically addressing known ABCA7 PTC variants found in patients have been limited, with most existing studies focusing broadly on complete ABCA7 knockouts\supercite{Kawatani2023-vf,Tayran2024-qw,Kim2013-sv}.

Single-nucleus RNA sequencing (snRNA-seq) has effectively identified cell type-specific transcriptional changes linked to other AD-associated genes such as APOE and TREM2\supercite{Brase2023-xk,Blanchard2022-cf,Sayed2021-qn}, and provided insights into disease mechanisms and therapeutic targets. In this study, we generated a cell type-specific transcriptomic snRNA-seq atlas of ABCA7 LoF variants in the \textit{postmortem} human prefrontal cortex (PFC). Leveraging this resource, we identified the cell-type-specific correlates of ABCA7 LoF variants in the human brain - particularly in neurons - and experimentally investigated these predictions in human neurons harboring ABCA7 PTC variants.

