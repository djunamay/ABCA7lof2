\subsubsection{\underline{Bulk mRNA Sequencing of iNs}}

\paragraph{Bulk mRNA Sequencing Data Processing.}
Sequencing data were processed using the BMC/BCC pipeline version 1.8 (updated 06/06/2023). Further details of the pipeline are available at: \url{https://openwetware.org/wiki/BioMicroCenter:Software#BMC-BCC_Pipeline} by the MIT BioMicro Center. Fastq files were subsequently processed in-house using STAR and featureCounts. Human reference genome (GRCh38.p14, GENCODE release 47) and annotation files were downloaded from GENCODE and indexed using STAR (version recommended parameters). Sequencing reads were trimmed using Trim Galore with Nextera-specific settings and a stringency of 3, requiring a minimum overlap of 3 bases with adapter sequences for trimming. Trimmed reads were then mapped to the indexed human genome using STAR, with paired-end reads processed concurrently. Gene-level counts were generated using featureCounts with settings optimized for paired-end data: counting only read pairs with both ends aligned (-B), excluding pairs mapping to different chromosomes or strands (-C), and counting fragments rather than individual reads (-p). Counts were summarized at the exon level, and grouped by gene identifier.

\paragraph{Differential gene expression analysis.}
Differential gene expression analysis was performed using edgeR and limma-voom. Counts were filtered to retain protein-coding genes expressed at a minimum of 1 count-per-million (CPM) in at least one sample. Normalization factors were calculated, and linear modeling was conducted using limma's voom method, followed by empirical Bayes moderation (eBayes). Statistical comparisons were performed using contrasts tailored to experimental conditions (treatment and genotype) within batches, and results were summarized as log-fold changes with associated p-values.

\paragraph{Gene set enrichment analysis.}
Gene set enrichment analysis was conducted using Fast Gene Set Enrichment Analysis (FGSEA). Differentially expressed genes were ranked by a score calculated as the sign of log-fold change multiplied by the negative log-transformed p-value. Pre-defined gene sets were evaluated with FGSEA using 10,000 permutations, and significant pathways were identified based on adjusted p-values < 0.05. Leading-edge genes from these significant pathways were partitioned along with their associated pathways as described in "Gene-pathway clustering using Kernighan-Lin heuristic."

\paragraph{Gene-Pathway K/L Cluster Similarity Analysis.}
Jaccard indices were computed to assess similarity between gene-pathway clusters. Empirical p-values for the overlaps were obtained by comparing observed overlaps against 1000 random permutations, and p-values were adjusted using the Benjamini-Hochberg method to control the false discovery rate.