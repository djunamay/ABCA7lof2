\paragraph{Cortical spheroid generation}
Dorsal cortical spheroids were generated as previously described \cite{Sloan2018-ja}. In brief, iPSC were cultured until 80-90\% confluence, before dissociation and preparing a single cell suspension at 100 x 10^4 cells/mL in mTesr supplemented with 10 μM Rock inhibitor. Spheroids were then generated by distributing in 100uL / well in PrimeSurface® 96 Slit-well Plates (S-Bio, \#MS9096SZ). After 48 hours, differentiation was induced on days 0 - 5 via daily media change in Neural Induction Media comprised of DMEMf/12 (Life Technologies, cat. no. 11330-032), KnockOut serum replacement (Life Technologies, 10828-028), GlutaMAX, 2-Mercaptoethanol, Penicillin-Streptomycin, and SMAD inhibitors SB-431542 and dorsomorphin. From day 6 -12, media was switched to Neural Differentiation medium composed of Neurobasal A medium, B27 supplement, GlutaMAX, Penicillin–streptomycin, Human recombinant EGF (20 ng/ml), Human recombinant FGF2 (20 ng/ml). Daily media change was performed until day 16, followed by every other day until day 25. From day 25 onwards, EGF and FGF was removed and replaced with 20ng/mL Human recombinant brain-derived neurotrophic factor (BDNF; 450-02) and 20 ng/mL Human recombinant neurotrophin 3 (NT3; PeproTech, cat. no. 450-03). After day 45, media was changed two times per week. 

\subsubsection{Electrophysiological recordings on cortical organoids}

To produce 2D cultures from cortical spheroids, day 150 spheroids were washed in PBS, before transferring to 1.5 mL eppendorf tubes containing Accutase (Stem Cell Technologies, 07920) and placing in a 37C water bath for 40 minutes, with gentle agitation using a P1000 pipette every 5-10 minutes. Dissociated organoids were then plated on \#1 glass coverslips (Fisher Scientific, 50-194-4702) with PDL, laminin, and matrigel coating as previously described. 2D cultures were maintained with or without drug treatment for two weeks before electrophysiological recordings. Electrophysiological recordings were performed as described for the iNs.

\subsubsection{Cortical organoid Calcium Imaging}
Neurospheroids were transduced overnight with AAV1-GCaMP6s (Addgene, 100843) at a titre of 5 × 10¹⁰ viral genomes per spheroid, in spheroid maintenance media. 72 hours later, media was switched to BrainPhys supplemented with SM1 and N2. After two weeks, calcium activity was acquired at 2 Hz using a Zeiss LSM900 Confocal microscope, at 37°C and 5\% CO2. Data was analysed using FIJI….. 