\subsubsection{Molecular Dynamics Simulations Results} 
\paragraph{RMSD analysis of Ala1527 vs Gly1527 in open and closed states.}
To evaluate conformational stability, we conducted root mean square deviation (RMSD) analyses on ABCA7 under different states and mutations (Figure~\ref{fig:main_neurons}G,H; Figure~\ref{fig:md_simulations}A,B; Table~\ref{tab:abca7_structures}). RMSD values for the $C_\alpha$ atoms were calculated over the course of a 300 ns simulation period, comparing closed and open conformations, each harboring either the G1527 or A1527 mutation.

For the open ABCA7 conformation, both the G1527 and A1527 mutants exhibited relatively minor RMSD fluctuations (Figure~\ref{fig:md_simulations}A-D), with RMSD values for the A1527 mutant significantly lower compared to those of the G1527 mutant (Figure~\ref{fig:md_simulations}E). Overall, both mutants showed narrow RMSD distributions in the open state (Figure~\ref{fig:md_simulations}C, D), indicating generally stable conformational behavior. 

Differences in RMSD distributions between the two variants became more pronounced in the closed state: The RMSD profile of the closed conformation with the G1527 mutation exhibited substantial fluctuations throughout the simulation (Figure~\ref{fig:main_neurons}I), suggesting that the G1527 mutation significantly increases local structural flexibility. In contrast, the closed conformation harboring the A1527 mutation showed only minor RMSD fluctuations (Figure~\ref{fig:main_neurons}I), suggesting that the A1527 mutation confers greater structural stability and reduced flexibility in the closed ABCA7 conformation. Principal component analysis (PCA) further highlighted these differences visually; For the closed conformation, PCA projections of G1527 conformations over time were broad, indicating significant exploration of the conformational space (Figure~\ref{fig:main_neurons}J). Conversely, the PCA plot for the closed A1527 mutant displayed a tightly clustered distribution (Figure~\ref{fig:main_neurons}J), indicating limited conformational sampling over time and suggesting decreased conformational flexibility induced by the A1527 mutation.

\paragraph{Dihedral angle analysis of Ala1527 vs Gly1527 in open and closed states.}
\newcommand{\quoteM}{\textcolor{blue}{To further explore the local structural variations induced by a p.Ala1527Gly mutation, we analyzed backbone dihedral angles (phi/psi; $\phi/\psi$) for residues 1517-1537. In the open conformation, Gly1527 consistently occupied the $\alpha$-helical region of the Ramachandran plot throughout the simulation. In contrast, Ala1527 showed two distinct populations within the $\alpha$-helical region (Figure~\ref{fig:md_simulations_2}A), suggesting subtle local conformational differences, yet overall preservation of the $\alpha$-helical structure. These findings align closely with the RMSD analysis, and suggest similar conformational behaviors between the variants in the open state. \label{quoteM-label}}}

\quoteM

\newcommand{\quoteN}{\textcolor{blue}{However, significant structural differences emerged in the closed conformation: Ala1527 displayed two preferred conformations—one within the $\alpha$-helical region and another shifted toward the $\beta$-structure region—while Gly1527 explored a broader range of dihedral angles, indicative of greater structural flexibility (Figure~\ref{fig:md_simulations_2}A,B). This observation is in line with the RMSD analysis, indicating structural differences specifically in the closed state, with Gly1527 exhibiting significantly greater conformational flexibility compared to Ala1527.}}

\quoteN

\paragraph{Secondary structure analysis of Ala1527 vs Gly1527 in open and closed states.}
\newcommand{\quoteO}{\textcolor{blue}{To complement backbone angle analysis, we also evaluated secondary structure stability throughout the simulation. In the open state, secondary structure content was comparable between variants, maintaining similar $\alpha$-helical character (Figure~\ref{fig:md_simulations_2}C). Upon transitioning to the closed state, both variants experienced a substantial loss of $\alpha$-helical content across residues 1517-1537. This loss, however, was more pronounced in the Gly1527 variant compared to the Ala1527 variant, as residues 1520-1525 retained partial $\alpha$-helical structure more robustly in the Ala1527 variant compared to Gly1527 (Figure~\ref{fig:md_simulations_2}C).}}

\quoteO

Finally, structural alignment of the closed-state Gly1527 ABCA7 structure (PDB 8EOP) with the closed-state structures of ABCA1 (PDB 7TBW) and ABCA4 (PDB 7LKZ) revealed that residues corresponding to Gly1527 in ABCA7 (V1646 in ABCA1; I1671 in ABCA4) adopt stable $\alpha$-helical structures (Figure~\ref{fig:md_simulations_2}E,F). In contrast, the Gly1527 residue in ABCA7 exhibits significant flexibility and lacks defined $\alpha$-helical structure. Interestingly, our simulations indicate that the Ala1527 variant partially restores this local $\alpha$-helical conformation in ABCA7 (Figure~\ref{fig:md_simulations_2}C,D). These data suggest that the Gly1527 variant induces local structural changes that differentiate ABCA7 from its close homologs ABCA1 and ABCA4.
