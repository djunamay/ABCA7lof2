Over 50 million people worldwide have dementia, with a large fraction of cases caused by Alzheimer’s disease\cite{Alzheimers_Disease_International2020-xv}. Late-onset Alzheimer’s Disease (AD) affects individuals over the age of 65 and accounts for more than 95\% of all AD cases\cite{Alzheimers_Association2016-vq}. Though AD is a multifactorial disorder, twin studies suggest a strong genetic component (~70\% heritability)\cite{Karlsson2022-vv} contributing to AD disease risk and progression. Large scale genome-wide association studies implicate multiple genes in AD etiology\cite{Lambert2013-km,Marioni2019-os,Jansen2019-ww,Kunkle2019-yo,De_Rojas2021-gu,Wightman2021-km,Bellenguez2022-ao,Belloy2023-kj}. After APOE4, rare loss-of-function (LoF) mutations caused by premature termination codons (PTCs) in ATP-binding cassette transporter A7 (ABCA7), are among the strongest genetic factors for AD (odds ratio ~ 2)\cite{Steinberg2015-mu,De_Roeck2019-te,Reitz2013-eo,Bellenguez2022-ao,Holstege2022-vp,Lyssenko2021-gw}. In addition to LoF variants, several common single nucleotide polymorphisms in ABCA7 - depending on the population - moderately\cite{Steinberg2015-mu,De_Roeck2019-te,Reitz2013-eo,Bellenguez2022-ao,Le_Guennec2016-nl,Hollingworth2011-tr,Naj2011-bs} to strongly\cite{Reitz2013-eo} increase AD risk, suggesting that ABCA7 dysfunction may play a role in a significant proportion of AD cases. Despite its prevalence and potential impact, the mechanism by which ABCA7 dysfunction increases AD risk remains poorly characterized. 

ABCA7 is a member of the A subfamily of ABC transmembrane proteins\cite{Kim2008-zi} with high sequence homology to ABCA1, the primary lipid transporter responsible for cholesterol homeostasis and high-density lipoprotein genesis in the brain\cite{Koldamova2014-kd}. ABCA7 effluxes both cholesterol and phospholipids to APOA-I and APOE in in vitro studies\cite{Abe-Dohmae2004-wb,Wang2003-wh,Tomioka2017-nv,Picataggi2022-yp,Quazi2013-pe} and has been shown to be a critical regulator of lipid metabolism, immune cell functions, and amyloid processing\cite{Aikawa2018-ek,Tanaka2011-zo,Duchateau2023-ji,Kawatani2023-vf,Tayran2024-bo}. To date, study of ABCA7 LoF has been predominantly pursued in rodent knock-out models or in non-neural mammalian cell lines. These studies show that ABCA7 knock-out or missense variants  cause increased amyloid processing and deposition\cite{Satoh2015-yu,Sakae2016-uy,Chan2008-qu,Bamji-Mirza2018-xt}, reduced plaque clearance by astrocytes and microglia\cite{Kim2013-sv,Fu2016-qe}, and that ABCA7 LoF alters glial-mediated inflammatory responses\cite{Aikawa2019-hv,Aikawa2021-vz}.  While these studies shed light on potential mechanisms of ABCA7 risk in AD, studies investigating the effects of ABCA7 LoF in human cells and tissue are severely lacking, with only a small number published to date\cite{Kawatani2023-vf,Allen2017-vw,Liu2021-zh,Bamji-Mirza2018-xt}.  These human studies highlight a number of potential LoF-induced defects in human cells, including impacts on lipid metabolism and mitochondrial function\cite{Kawatani2023-vf}. However, comprehensive and unbiased profiling of multiple human neural cell types is needed to elucidate the mechanism by which ABCA7 LoF increases AD risk.

Recent work has used single-nuclear RNA sequencing of human neural tissue to reveal the cell type specific deficits associated with genetic variants in AD (including APOE4 and TREM2) and in other disorders\cite{Brase2023-xk,Blanchard2022-cf,Sayed2021-qn,Wamsley2024-zm,Kamath2022-if}, providing insight into disease mechanisms and possible therapeutic interventions. In this study, we generated a cell type-specific transcriptomic atlas of ABCA7 LoF effects in the human prefrontal cortex (PFC) by single-nuclear RNA sequencing (snRNAseq) of postmortem tissue from ABCA7 LoF variant carriers and matched control individuals. We found that ABCA7 LoF was linked to transcriptomic perturbations in all major brain cell types in the human PFC and that excitatory neurons, which expressed ABCA7 most highly, showed substantial transcriptomic perturbations  to genes related to lipid metabolism, mitochondrial respiration, DNA damage, and synaptic function. We complemented our transcriptomic analysis with biochemical experiments on postmortem human brain and neurons derived from isogenic induced pluripotent stem cells (iPSC).

Experiments in iPSC-derived neurons (iNs) revealed a metabolic shift away from phosphatidylcholine synthesis towards triglyceride accumulation in the presence of ABCA7 LoF, trends which were also observed in the postmortem human brain. This shift coincided with widespread metabolic changes related to carbon catabolism through mitochondrial oxidative phosphorylation (OXPHOS), indicating that ABCA7 LoF neurons may have decreased ability to metabolize lipids. Promoting phosphatidylcholine synthesis and reducing triglyceride accumulation by supplementing iNs with CDP-choline, a rate-limiting precursor for phosphatidylcholine synthesis, ameliorated this shift, suggesting that ABCA7 LoF-induced lipid disruptions mediate mitochondrial dysfunction in neurons. Supplementation with CDP-choline also rescued intracellular A𝛽42, the toxic form of amyloid-β\cite{Pauwels2012-ul,Fraser1991-tj,Pike1993-xs,Pettegrew2001-sm,Phillips2019-ev}, which was up-regulated in ABCA7 LoF neurons, suggesting a potential link between ABCA7 LoF-induced metabolic defects and AD pathology.

Together, our data suggest that neuronal ABCA7 plays an integral role in regulating intracellular lipid homeostasis critical for mitochondrial function. Our study reveals new insights into the mechanism through which damaging ABCA7 variants, including common ABCA7 missense variants, may increase AD risk by disrupting energy homeostasis in the cell. 

