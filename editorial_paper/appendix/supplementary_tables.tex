\subsubsection*{Supplementary Tables}
\phantomsection
\addcontentsline{toc}{subsubsection}{Supplementary Tables}

\captionsetup{justification=raggedright,singlelinecheck=false}
\renewcommand{\thetable}{S\arabic{table}}
\setcounter{table}{0}  

\renewcommand{\arraystretch}{1.2} 


\paragraph*{Supplementary Table 1.} Annotation of ABCA7 loss-of-function variants identified in this study. Columns include variant identifiers (rsID), coding sequence (HGVS.c) and protein (HGVS.p) notation (HGVS nomenclature), functional annotation, previously reported AD association, and number of carriers (N in cohort).
\phantomsection
\addcontentsline{toc}{paragraph}{Supplementary Table 1.}

\begin{longtable}{p{3.5cm} p{4.7cm} p{3cm} p{2.5cm} p{2.3cm} p{1cm}}
    \hline
    \textbf{rsID}          & \textbf{HGVS.c}           & \textbf{HGVS.p}    & \textbf{Annotation}                                                           & \textbf{AD association}                                                                   & \textbf{N in cohort} \\
    \hline
    \hline
    rs113809142            & c.4416+2T>G               & NA                 & splice donor variant \supercite{Allen2017-on}                                      & Steinberg et al (2015), Nature Genetics, Table 1 \supercite{Steinberg2015-vj}                   & 1 \\
    \hline
    rs200538373            & c.5570+5G>C               & NA                 & splice region variant \supercite{Allen2017-on,Steinberg2015-vj}                    & Steinberg et al (2015), Nature Genetics, Table 1 \supercite{Steinberg2015-vj}                   & 4 \\
    \hline
    rs538591288            & c.4208delT                & p.Leu1403fs        & frameshift variant \supercite{Allen2017-on}                                        & Steinberg et al (2015), Nature Genetics, Table 1 \supercite{Steinberg2015-vj}                   & 1 \\
    \hline
    rs547447016            & c.2126\_2132delAGCAGGG    & p.Glu709fs         & frameshift variant \supercite{Allen2017-on}                                        & Steinberg et al (2015), Nature Genetics, Table 1 \supercite{Steinberg2015-vj}                   & 4 \\
    \hline
    rs201060968            & c.3641G>A                & p.Trp1214*         & stop gained                                                                   & NA                                                                                        & 1 \\
    \hline
    19\_1053362\_G\_A       & c.3255G>A                & p.Trp1085*         & stop gained                                                                   & NA                                                                                        & 1 \\
    \hline
    \label{tab:annotation_abca7}
\end{longtable}

\paragraph*{Supplementary Table 2.} Rare ABCA7 loss-of-function variants identified in the ROSMAP cohort (all on chromosome 19). Columns include genomic position (POS), variant ID, reference (REF) and alternate (ALT) alleles, functional effect, coding sequence (HGVS\_C) and protein-level (HGVS\_P) annotations, and minor allele frequency (MAF). \\
\phantomsection
\addcontentsline{toc}{paragraph}{Supplementary Table 2.}
\begin{longtable}{%
    p{1.5cm}                   % POS
    p{2.8cm}                   % ID
    p{2cm}                     % REF
    p{1cm}                     % ALT
    p{1.5cm}    % Effect → wrap & ragged-right
    p{4.5cm}  % HGVS_C → wrap & ragged-right
    p{1.7cm}                     % HGVS_P
    p{1cm}                     % MAF
  }
\hline
\textbf{POS} & \textbf{ID} & \textbf{REF} & \textbf{ALT} & \textbf{Effect} & \textbf{HGVS\_C} & \textbf{HGVS\_P} & \textbf{MAF} \\
\hline
\hline
1047507 & rs547447016 & AGGAGCAG & A & frameshift variant                          & c.2126\_2132delAGCAGGG & p.Glu709fs  & 0.0025 \\
\hline
1053362 & no known rsID           & G        & A & stop gained                                 & c.3255G>A               & p.Trp1085*  & 0.0004  \\
\hline
1054255 & rs201060968 & G        & A & stop gained                                 & c.3641G>A               & p.Trp1214*  & 0.0004  \\
\hline
1055907 & rs538591288 & CT       & C & frameshift variant \& splice region variant & c.4208delT              & p.Leu1403fs & 0.0004  \\
\hline
1056244 & rs113809142 & T        & G & splice donor variant \& intron variant      & c.4416+2T>G             &    N/A         & 0.0008  \\
\hline
1061892 & rs200538373 & G        & C & splice region variant \& intron variant     & c.5570+5G>C             &    N/A         & 0.0021 \\
\hline
\label{tab:abca7_lof_simplified}
\end{longtable}


\paragraph*{Supplementary Table 3.} External datasets used. 
\phantomsection
\addcontentsline{toc}{paragraph}{Supplementary Table 3.}
\begin{longtable}{p{7cm} p{7cm} p{3cm}}
    \hline
    \textbf{Description} & \textbf{Access} & \textbf{Reference} \\ 
    \hline
    \hline
    postmortem human PFC proteomic data & \url{https://www.synapse.org/\#!Synapse:syn21449368} & \supercite{Johnson2020-ib} \\
    \hline
    Reference 1: Cell type specific marker genes for human brain & \url{https://osf.io/pqr9m/} & \supercite{Wang2018-im} \\
    \hline
    Reference 2: Cell type specific marker genes for human brain & \url{https://osf.io/pqr9m/} & \supercite{Franzen2019-hh} \\
    \hline
    Gene Ontology Biological Process 2023 & \url{https://maayanlab.cloud/Enrichr/\#libraries} & NA \\
    \hline
    NeuN+/- bulk RNA-sequencing from postmortem human brain & \url{https://www.synapse.org/\#!Synapse:syn33212353} & \supercite{Welch2022-ef} \\
    \hline
    WikiPathways 2019 Human & \url{https://maayanlab.cloud/Enrichr/\#libraries} & NA \\
    \hline
    snRNAseq from postmortem human PFC from p.Ala1527Gly variant-carriers and controls & \url{https://www.synapse.org/\#!Synapse:syn52293417} & \supercite{Mathys2023-rs} \\ 
    \hline
    Human MitoCarta3.0 & \url{https://www.broadinstitute.org/mitocarta/mitocarta30-inventory-mammalian-mitochondrial-proteins-and-pathways} & NA \\
    \hline
    Human prefrontal cortex layer markers based transcriptomics on dissected layers & \url{https://www.nature.com/articles/nn.4548#article-info} & \supercite{He2017-dq} \\
    \hline
    Human dorsolateral prefrontal cortex spatial transcriptomics markers & \url{https://www.nature.com/articles/s41593-020-00787-0#article-info} & \supercite{Maynard2021-mz} \\
    \hline
    \label{tab:external_datasets}
\end{longtable}

\paragraph*{Supplementary Table 11.} Enriched mitochondrial pathways (MitoCarta) comparing WT vs. p.Tyr622* mRNA identified using the Python package \texttt{gseapy}. Columns include pathway terms, enrichment scores ($-\log_{10}$ P-value), raw and FDR-corrected p-values, and associated mitochondrial genes. Only results with p $<$ 0.05 are shown.
\phantomsection
\addcontentsline{toc}{paragraph}{Supplementary Table 11.}
\begin{longtable}{p{5cm} p{2cm} p{1.5cm} p{1.5cm} p{7cm}}
    \hline
    \textbf{Term} & \textbf{score} & \textbf{P-value} & \textbf{FDR} & \textbf{Genes} \\
    \hline
    \hline
    Apoptosis & 1.900274 & 0.012581 & 0.415774 & BID; CASP3; CYCS; BCL2L1; BIK \\
    \hline
    OXPHOS & 1.894438 & 0.012752 & 0.415774 & NDUFA4; TMEM126A; ATP5MG; SDHAF2; NDUFA1; COX6A1; COX14; CYCS; COA4; NDUFS4; ATP5PB; ATP5MC3; NDUFAF2; MT-ND2 \\
    \hline
    Protein import and sorting & 1.755279 & 0.017568 & 0.415774 & TIMM8A; TIMM10; DNAJC19; SAMM50; TIMM23; TOMM22 \\
    \hline
    OXPHOS subunits & 1.552584 & 0.028017 & 0.477391 & NDUFA4; ATP5MG; NDUFA1; COX6A1; CYCS; NDUFS4; ATP5PB; ATP5MC3; MT-ND2 \\
    \hline
    Mitochondrial dynamics and surveillance & 1.473414 & 0.033619 & 0.477391 & BID; ATP5MG; SAMM50; CASP3; CYCS; BCL2L1; FUNDC1; BIK; FUNDC2 \\
    \hline
    Amino acid metabolism & -1.374004 & 0.042267 & 0.263353 & DLST; COMT; SLC25A44; MAOA; ABAT; SFXN3; GCAT; ALDH5A1; MCCC2; AADAT \\
    \hline
    Detoxification & -1.395286 & 0.040245 & 0.263353 & EPHX2; DHRS2; CYB5B; CAT; TXNRD2; MAOA; CYB5R3 \\
    \hline
    EF hand proteins & -1.441227 & 0.036205 & 0.263353 & RHOT2; SLC25A23; SLC25A25 \\
    \hline
    Lipid metabolism & -1.441792 & 0.036158 & 0.263353 & EPHX2; SLC25A1; ACADVL; GPAT2; CPT1C; ACADL; IDI1; ACP6; CROT; CYB5R3; CYP27A1; ACAD10 \\
    \hline
    Amidoxime reducing complex & -1.576863 & 0.026493 & 0.238440 & CYB5R3; CYB5B \\
    \hline
    Vitamin metabolism & -1.639877 & 0.022915 & 0.232016 & MMAB; DHRS4; PLPBP; PNPO; RFK; PC; SFXN3 \\
    \hline
    Gluconeogenesis & -1.844185 & 0.014316 & 0.165654 & PCK2; PC; SLC25A1 \\
    \hline
    Catechol metabolism & -1.857772 & 0.013875 & 0.165654 & COMT; MAOA \\
    \hline
    TCA-associated & -1.887037 & 0.012971 & 0.165654 & ACLY; PCK2; PC; SLC25A1 \\
    \hline
    ABC transporters & -2.026052 & 0.009418 & 0.165654 & ABCB6; ABCD2; ABCB8 \\
    \hline
    Xenobiotic metabolism & -2.103329 & 0.007883 & 0.165654 & EPHX2; DHRS2; CYB5B; MAOA; CYB5R3 \\
    \hline
    Vitamin B6 metabolism & -2.314663 & 0.004845 & 0.165654 & PNPO; PLPBP \\
    \hline
    Metabolism & -4.519649 & 0.000030 & 0.002448 & NMNAT3; DHRS4; CAT; GPAT2; DLAT; FECH; PNPO; SLC25A25; MCCC2; MMAB; COQ9; SLC25A1; PLPBP; ACLY; TXNRD2; ABAT; RFK; TK2; PC; IDI1; PCK2; EPHX2; DHRS2; ACADVL; IDH2; CPT1C; CROT; ALDH5A1; AADAT; ACAD10; TSTD1; CYB5B; DLST; SLC25A44; ABCB6; GATM; SLC25A23; MAOA; ACADL; SFXN3; ACP6; GCAT; COMT; CYP27A1; CYB5R3 \\
    \label{tab:y622_mito_genes}
\end{longtable}

\paragraph*{Supplementary Table 14.} Enriched mitochondrial pathways (MitoCarta) comparing p.Tyr622* + H\textsubscript{2}O vs. p.Tyr622* + CDP-choline mRNA identified using the Python package \texttt{gseapy}. Columns include pathway terms, enrichment scores ($-\log_{10}$ P-value), raw and FDR-corrected p-values, and associated mitochondrial genes. Only results with p $<$ 0.05 are shown.
\phantomsection
\addcontentsline{toc}{paragraph}{Supplementary Table 14.}
\begin{longtable}{p{5cm} p{2cm} p{1.5cm} p{1.5cm} p{7cm}}
    \hline
    \textbf{Term} & \textbf{score} & \textbf{P-value} & \textbf{FDR} & \textbf{Genes} \\
    \hline
    \hline
    EF hand proteins & 4.457699 & 0.000035 & 0.003451 & SLC25A12; SLC25A23; SLC25A25; EFHD1; MICU2; SLC25A13; MICU3; SLC25A24 \\
    \hline
    Small molecule transport & 3.256577 & 0.000554 & 0.027418 & ABCB10; SLC25A25; MPV17L; ABCD3; ABCD1; SLC25A12; SLC25A1; SLC25A13; MICU3; SFXN1; SLC25A29; SLC25A39; SLC25A43; MICU2; SFXN5; STARD7; SLC25A24; ABCD2; SLC25A15; SLC25A22; SLC25A23; VDAC1; MPV17 \\
    \hline
    Calcium homeostasis & 3.079591 & 0.000833 & 0.027474 & SLC25A12; SLC25A23; SLC25A25; EFHD1; VDAC1; LETM1; SLC25A13; MICU2; MICU3; SLC25A24 \\
    \hline
    Metabolism & 2.846777 & 0.001423 & 0.033450 & ABCB10; NT5DC2; ACSL6; ME3; CS; SLC25A12; PDSS2; HSD17B4; ACLY; TK2; ALDH3A2; IDI1; NADK2; ACADS; CPT2; STARD7; AADAT; ACAD10; AK3; NNT; SOD2; CHCHD7; SLC25A23; OGDH; NMNAT3; MT-CO1; GLS; GPAT2; SLC25A25; OAT; D2HGDH; SLC25A1; ABAT; RFK; SFXN1; ALDH1B1; OXCT1; TST; HIBCH; FHIT; FH; GLYCTK; LACTB; PNPO; SERAC1; ME2; GSR; AGPAT5; PCK2; SLC25A29; EPHX2; GLDC; SFXN5; PDK3; ALDH5A1; SPHK2; SLC25A24; TSTD1; MTFMT; ACP6; NT5M; ALDH9A1; GCAT; KMO; ECI1; CAT; DLAT; COQ2; PPM1K; PRXL2A; MCCC2; CISD3; SPTLC2; SLC25A13; SLC25A15; FASN; ALDH7A1; DLST; GATM; AK4; ACADSB \\
    \hline
    Signaling & 2.772263 & 0.001689 & 0.033450 & NLRX1; SLC25A12; PPTC7; SLC25A23; SLC25A25; EFHD1; VDAC1; LETM1; MACROD1; SLC25A13; MICU2; PDE2A; MICU3; SLC25A24; DELE1 \\
    \hline
    TCA-associated & 2.563737 & 0.002731 & 0.045055 & SLC25A1; ACLY; ME2; SFXN5; ME3; D2HGDH; PCK2 \\
    \hline
    Fusion & 2.466941 & 0.003412 & 0.048261 & OPA1; MIGA2; MFN2; MIGA1; MFN1 \\
    \hline
    Carbohydrate metabolism & 1.960909 & 0.010942 & 0.123756 & OXCT1; CS; SLC25A12; SLC25A1; FH; DLAT; ACLY; ME2; GLYCTK; DLST; OGDH; SLC25A13; ME3; SFXN5; PDK3; D2HGDH; PCK2 \\
    \hline
    Organelle contact sites & 1.856642 & 0.013911 & 0.123756 & FKBP8; SPIRE1; MIGA2; MFN2; MFN1; VDAC1 \\
    \hline
    ABC transporters & 1.852052 & 0.014059 & 0.123756 & ABCB10; ABCD1; ABCD2; ABCD3 \\
    \hline
    Phospholipid metabolism & 1.850381 & 0.014113 & 0.123756 & GPAT2; SERAC1; SPTLC2; LACTB; ACP6; STARD7; AGPAT5; SPHK2 \\
    \hline
    Amino acid metabolism & 1.823889 & 0.015001 & 0.123756 & GLS; OAT; PPM1K; MCCC2; SLC25A12; ABAT; SLC25A13; SFXN1; SLC25A29; GLDC; ALDH5A1; AADAT; HIBCH; ALDH9A1; SLC25A15; ALDH7A1; DLST; GCAT; ACADSB; KMO \\
    \hline
    Nucleotide metabolism & 1.357357 & 0.043918 & 0.334453 & AK3; NT5DC2; SLC25A23; SLC25A25; GATM; AK4; TK2; NT5M; SLC25A24; FHIT \\
    \hline
    OXPHOS assembly factors & -1.423365 & 0.037725 & 0.363536 & FOXRED1; TIMMDC1; COA3; COA6; COX7A2L; SDHAF3; FMC1; COX16; RAB5IF; NDUFAF6; SDHAF2; SURF1; TIMM21; BCS1L; COA7; NDUFAF1; COX14; LYRM2; TMEM70 \\
    \hline
    Protein import and sorting & -1.445375 & 0.035861 & 0.363536 & MTX2; TIMM8A; TIMM23; TIMM17B; TIMM10; UQCRC1; DNAJC19; PMPCA; TIMM8B; MTX1; TIMM21; TIMM22; TOMM5; GRPEL1 \\
    \hline
    CIII subunits & -1.603039 & 0.024944 & 0.293782 & UQCR10; CYC1; UQCRH; UQCRC1; UQCRFS1 \\
    \hline
    mtRNA granules & -1.605563 & 0.024799 & 0.293782 & MRPL47; ERAL1; MTPAP; ALKBH1; PTCD2; MRPS7; TFB1M; TRUB2; RMND1; TRMT10C \\
    \hline
    CII subunits & -1.645356 & 0.022628 & 0.293782 & SDHD; SDHC; SDHB \\
    \hline
    Complex II & -2.114075 & 0.007690 & 0.135856 & SDHAF2; SDHAF3; SDHC; SDHB; SDHD \\
    \hline
    OXPHOS subunits & -2.580372 & 0.002628 & 0.055714 & UQCR10; ATP5MG; SDHB; ATP5PB; UQCRH; NDUFA10; COX5B; COX7A2L; NDUFB3; NDUFA4; CYC1; SDHD; MT-ND4L; UQCRFS1; ATP5IF1; NDUFA1; NDUFV1; COX6A1; NDUFB8; ATP5F1A; SDHC; NDUFB6; NDUFA6; NDUFS4; COX7A2; UQCRC1; NDUFS2; ATP5F1C \\
    \hline
    Mitochondrial central dogma & -2.864986 & 0.001365 & 0.036163 & MRPL4; MRPL18; COA3; MTRES1; MRPL46; MTIF3; MRPS23; MRPL37; MRPL45; MRPL36; TRUB2; MRPS10; MRPL24; PTCD2; MRPL40; MRPS18C; UNG; TFB1M; METTL5; MRPS12; MRPL14; GATC; MRPL34; MRPL32; MRPL16; TSFM; MTPAP; MRPL47; APEX1; MRPS21; ALKBH1; TEFM; MRPL55; MRPL39; MRPS7; MRPS26; MRPL22; TIMM21; MRPL15; MTERF2; NGRN; MRPS18A; TRMT10C; MTERF1; ERAL1; MRM3; COX14; MRPS31; MRPL13; PTCD3; MRPS14; DAP3; MTERF3; MRPL27; MRPS28; RARS2; MRPS18B; RMND1; MRPS15 \\
    \hline
    OXPHOS & -3.420637 & 0.000380 & 0.013414 & UQCR10; FOXRED1; TIMMDC1; ATP5MG; COA3; SDHB; ATP5PB; UQCRH; NDUFA10; COX5B; COA6; COX7A2L; NDUFB3; NDUFA4; CYC1; SDHAF3; FMC1; SDHD; MT-ND4L; UQCRFS1; ATP5IF1; RAB5IF; COX16; NDUFAF6; SDHAF2; SURF1; NDUFV1; NDUFA1; COX6A1; TIMM21; NDUFB8; BCS1L; COA7; NDUFA12; NDUFAF1; ATP5F1A; COX14; SDHC; NDUFA6; NDUFB6; LYRM2; NDUFS4; COX7A2; UQCRC1; TMEM70; NDUFS2; ATP5F1C \\
    \hline
    Translation & -4.653989 & 0.000022 & 0.001176 & MRPL4; MRPL18; COA3; MTRES1; MRPL46; MTIF3; MRPS23; MRPL37; MRPL45; MRPL36; MRPS10; MRPL24; MRPL40; MRPS18C; TFB1M; MRPS12; MRPL14; GATC; MRPL34; MRPL32; MRPL16; TSFM; MRPL47; MRPS21; MRPL55; MRPL39; MRPS7; MRPS26; MRPL22; TIMM21; MRPL15; NGRN; MRPS18A; ERAL1; MRM3; COX14; MRPS31; MRPL13; PTCD3; DAP3; MRPS14; MTERF3; MRPL27; MRPS18B; RARS2; MRPS28; RMND1; MRPS15 \\
    \hline
    Mitochondrial ribosome & -6.408523 & 0.000000 & 0.000041 & MRPL4; MRPL18; MRPL46; MRPS23; MRPL37; MRPL45; MRPL36; MRPS10; MRPL24; MRPL40; MRPS18C; MRPS12; MRPL14; MRPL34; MRPL32; MRPL16; MRPL47; MRPS21; MRPL55; MRPL39; MRPS7; MRPS26; MRPL22; MRPL15; MRPS18A; MRPS31; MRPL13; PTCD3; DAP3; MRPS14; MRPL27; MRPS18B; MRPS28; MRPS15 \\
    \hline
    \label{tab:choline_mito_genes}
\end{longtable}

\paragraph*{Supplementary Table 15.} Experimentally determined 3D ABCA7 structures used in molecular dynamics simulations. Columns include system description (System), Protein Data Bank identifier (PDB ID), and conformational state (State).
\phantomsection
\addcontentsline{toc}{paragraph}{Supplementary Table 15.}
\begin{longtable}{p{6cm} p{5cm} p{6cm}}
    \hline
    \textbf{System}    & \textbf{PDB ID} & \textbf{State} \\
    \hline
    \hline
    CLOSE-G1527        & 8EOP           & HOLO         \\
    \hline
    CLOSE-A1527        & 8EOP           & HOLO         \\
    \hline
    OPEN-G1527         & 8EE6           & APO          \\
    \hline
    OPEN-A1527         & 8EE6           & APO          \\
    \hline
    \label{tab:abca7_structures}
\end{longtable}

\paragraph*{Supplementary Table 16.} Primers used for PCR and Sanger sequencing (SS). Columns indicate primer names (Primer) and their nucleotide sequences (Sequence).
\phantomsection
\addcontentsline{toc}{paragraph}{Supplementary Table 16.}
\begin{longtable}{p{7.5cm} p{10.5cm}}
    \hline
    \textbf{Primer} & \textbf{Sequence} \\
    \hline
    \hline
    rs547447016\_FOR              & 5’-ACGCTGGCCTGGATCTACTC-3’ \\
    \hline
    rs547447016\_REV              & 5’-TGCATGCGTGTGCCAAGAAG-3’ \\
    \hline
    chr19.1053362G>A\_rs201060968\_FOR   & 5’-CTGAAGCACCCCTTTGTCCAC-3’ \\
    \hline
    chr19.1053362G>A\_rs201060968\_REV   & 5’-GAAAGCGCTTGAGAAGCAGGG-3’ \\
    \hline
    chr19.1053362G>A\_REV\_SS      & 5’-GCTGCTCATAAACACGCTATTCATCCTTC-3’ \\
    \hline
    rs201060968\_FOR\_SS          & 5’-CATTGCTGGCCTAGACGTAA-3’ \\
    \hline
    ABCA7\_p.Glu50fs*3\_FOR       & 5’-GTGACGAAAGCGTTAAGCCC-3’ \\
    \hline
    ABCA7\_p.Glu50fs*3\_REV       & 5’-GCAGTGGCTTGTTTGGGAAG-3’ \\
    \hline
    ABCA7\_p.Tyr622*\_FOR         & 5’-CTGGTTCTGGTGCTCAAG-3’ \\
    \hline
    ABCA7\_p.Tyr622*\_REV         & 5’-CCTACGGCAGACGTCTTCAG-3’ \\
    \label{tab:pcr_primers}
\end{longtable}

\paragraph*{Supplementary Table 17.} Antibodies used. 
\phantomsection
\addcontentsline{toc}{paragraph}{Supplementary Table 17.}
\begin{longtable}{p{6cm} p{5cm} p{6cm}}
    \hline
    \textbf{Antibody name}                & \textbf{Company}      & \textbf{Catalog No.} \\
    \hline
    \hline
    NeuN                                  & Synaptic Systems      & 266004               \\
    \hline
    Tuj1                                  & BioLegend             & MMS-435P             \\
    \hline
    SM312        & Biolegend             & 837904               \\
    \hline
    MAP2                                  & Biolegend             & 822501               \\
    \hline
    \label{tab:antibodies_used}
\end{longtable}

    
    