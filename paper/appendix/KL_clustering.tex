\subsubsection{Choosing a partitioning heuristic for gene-pathway grouping}
\paragraph{Methods.}
The heatmap in Figure~\ref{fig:benchmarking_clustering}A highlights how frequently pathways within a pathway database, such as WikiPathways, share gene members. On average, every pathway shown in Figure~\ref{fig:benchmarking_clustering}A shares at least one gene with approximately 40\% of the other pathways, highlighting that there is redundancy in this matrix that could be summarized in simpler terms.

To summarize redundant gene-pathway information into a limited number of non-redundant gene-pathway groups, we reformulated the gene-pathway association problem as a bipartite graph $G$ constructed from all the genes in the Leading Edge subset (LE) and their associated pathways. LE was defined as the set of 268 genes driving the enrichment signal for pathways that passed a significance threshold of $p < 0.05$ (fGSEA) in Con vs. ABCA7 LoF excitatory neurons. $G$ was constructed from an $n \times m$ unweighted adjacency matrix, where $n$ represented the number of LE genes and $m$ the number of pathways associated with four or more LE genes, as specified in the WikiPathways database.

Graph partitioning involves segmenting the vertices of a graph into equal-sized partitions, optimizing for the minimal number of interconnecting edges (i.e., “total cut size”). We tested three prominent graph partitioning techniques, as outlined by Elsner (1997)\cite{Elsner1997-nt}, to approximate optimal partitioning. These methods include:

\begin{enumerate}
    \item \textbf{Recursive Spectral Bisection}: Implemented in Python using the numpy linear algebra package, this method was executed for $\log_2(N)$ iterations, yielding $N = 8$ partitions. A detailed description of the algorithm can be found in Elsner (1997)\cite{Elsner1997-nt}.
    \item \textbf{Multilevel Graph Partitioning}: Leveraging the METIS software package\cite{Karypis1997-it} in Python using the following parameters: `nparts=8`, `tpwgts=None`, `ubvec=None`, `recursive=False`.
    \item \textbf{Kernighan-Lin (K/L) Algorithm}: Based on its original paper\cite{Kernighan1970-rg}, this algorithm was implemented in Python and run with parameters set as $C=0$, `KL\_modified=True`, `random\_labels=True`, `unweighted=True`, and $K=50$.
\end{enumerate}

Additionally, the Spectral Clustering algorithm, a commonly used clustering method, was applied using the `SpectralClustering()` function from the `sklearn` Python package with default parameters, apart from `n\_clusters=8` and `assign\_labels='kmeans'`. We stipulated eight clusters for each algorithm, as qualitatively, this resolution seemed to strike a good balance to summarize main biological effects.

For benchmarking purposes, the three graph partitioning techniques and the spectral clustering algorithm were evaluated by segmenting graph $G$ into eight gene-pathway clusters using the respective algorithms. Spectral clustering was run over 1,000 initiations, while K/L and METIS were run over 50,000 iterations because their solutions were slightly more variable across runs. The deterministic bisection method was run only once. A randomized graph partitioning benchmark was also computed by permuting the eight cluster labels of approximately equivalent size for 1,000 initiations. Average losses were computed per algorithm on all initiations. The benchmarking process and source code are available at: GitHub Repository.

\paragraph{Results.}
Spectral clustering performed significantly better than all other algorithms based on the loss (Figure~\ref{fig:benchmarking_clustering}B). This was expected, as spectral clustering does not place a constraint on cluster size. Spectral clustering results were characterized by a single large cluster and many small clusters (Figure~\ref{fig:benchmarking_clustering}C), indicating that this clustering algorithm was highly susceptible to outliers and suggesting that graph partitioning, which imposes the constraint of equal partitioning, was a better approach to the problem of grouping genes and pathways into biologically informative groups. Indeed, all three graph partitioning algorithms divided the graph into more uniformly-sized groups (Figure~\ref{fig:benchmarking_clustering}C). Among the partitioning algorithms, K/L and METIS produced the most uniformly sized groups (Figure~\ref{fig:benchmarking_clustering}C) and also had significantly lower losses compared to the spectral bisection algorithm (Figure~\ref{fig:benchmarking_clustering}B). K/L and METIS solutions were very similar, with their respective best solutions (lowest loss) having an average Jaccard similarity index of 0.91 on the diagonal (Figure~\ref{fig:benchmarking_clustering}D,E). K/L and METIS solutions were also consistent across pairwise random initiations, both when comparing within K/L or METIS solutions (Rand Index=0.87 and 0.91, respectively) and when comparing all pairwise K/L and METIS solutions (Rand Index=0.88) (Figure~\ref{fig:benchmarking_clustering}F).

Overall, these results indicate the importance of non-redundant gene-pathway groupings to interpret biological effects. They also indicate that for some gene-pathway graphs, such as the one in this study, graph partitioning is a better approach than clustering.
