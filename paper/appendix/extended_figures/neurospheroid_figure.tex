\begin{figure}[H]
    \begin{subfigure}[t]{0.5\textwidth}
        \caption{}
        \includegraphics[width=\textwidth]{./main_plots/abeta_elisa_iN.png}        
    \end{subfigure}  
    \par
    \begin{subfigure}[t]{0.7\textwidth}
        \caption{}
        \includegraphics[width=\textwidth]{./extended_plots/neurospheroid_markers.png}        
    \end{subfigure}
    \par
    \begin{subfigure}[t]{0.7\textwidth}
        \caption{}
        \includegraphics[width=\textwidth]{./main_plots/abeta_elisa_3weeks.png}        
    \end{subfigure}
    % \begin{subfigure}[t]{0.4\textwidth}
    %     \caption{}
    %     \includegraphics[width=\textwidth]{./extended_plots/calcium_traces.png}        
    % \end{subfigure}
    \caption{
        \textbf{CDP-choline Treatment in Cortical Organoids}\\
    }
    \label{fig:neurospheroid_figure}
\end{figure}
\begin{itemize}
    \item[\textbf{(A)}] Amyloid-$\beta$ levels quantified by ELISA from media of 4-week-old iNs.
    \item[\textbf{(B)}] Representative images of cortical organoid slices from indicated genotypes.
    \item[\textbf{(C)}] Amyloid-$\beta$ levels quantified by ELISA from media of cortical organoids (176 days old), grouped by genotype and treated with 500~$\mu$M or 1~mM CDP-choline for 3 weeks. Samples correspond to organoids in Figure~\ref{fig:main_choline}K, analyzed one week prior to assays presented there.
\end{itemize}