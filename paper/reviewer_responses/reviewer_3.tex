\subsubsection{Referee \#3 (Remarks to the Author)}
The manuscript by Maydell et al to sheds light on how ABCA7 loss of functon(lof) may contribute to AD pathogenesis. ABCA7 lof variants are associated with an increased risk of Alzheimer's Disease (AD). The mechanisms by which these variants contribute to AD are not well understood. The current study aims to uncover the pathogenic mechanisms and affected neural cell types associated with ABCA7 LoF variants through single-nuclear RNA sequencing (snRNAseq) of human brain samples. The authors analyzed 36 post-mortem samples from the prefrontal cortex, including 12 ABCA7 LoF carriers and 24 matched non-carrier controls. They performed snRNAseq to examine gene expression changes in various neural cell types. ABCA7 LoF variants led to transcriptional changes in all major brain cell types, particularly excitatory neurons, affecting pathways related to lipid metabolism, mitochondrial function, cell cycle, and synaptic signaling. From lipidomic analysis ABCA7 LoF neurons showed intracellular triglyceride accumulation and decreased phosphatidylcholine levels. In metabolomic analysis. disrupted mitochondrial bioenergetics indicated impaired lipid breakdown by uncoupled respiration in ABCA7 LoF neurons. Treatment with CDP-choline, a precursor of phosphatidylcholine synthesis, reduced triglyceride accumulation, restored mitochondrial function, and decreased intracellular amyloid β-42 levels in ABCA7 LoF neurons. Also, the study generated a detailed transcriptomic atlas of ABCA7 LoF in the human brain, revealing cell type-specific gene expression changes and pathway perturbations. The therapeutic potential of CDP-choline in mitigating these effects highlights a possible intervention strategy. The comprehensive transcriptomic atlas derived from human patient/donor samples serves as a valuable resource for future studies exploring the role of lipid metabolism in AD pathology.
Overall, however, it is unclear whether the results constitute a fundamental advance in our mechanistic understanding of human ABCA7 dysfunction beyond correlating it with a myriad of lipid and fatty acid homeostatic/metabolic pathways. Many of these correlations are confirmatory, having been reported for ABCA7 lof mouse / cellular models. Considering the readouts for the experiments performed are indirect, it is unclear whether the phenotypic observations are directly stemming from ABCA7 dysfunction. In the absence biochemical validation at the protein level, the results do little to provide specifics of ABCA7 functionality in framework of AD progression and the manuscript would be more suited for a specialized audience. The description of residue 1527 could also be clarified considering:

In this manuscript, A1527G is listed as a missense variant whereas the canonical sequence of human ABCA7 (uniprot ID Q8IZY2) is G1527. G1527A has been reported to be a benign or mildly protective variant. The equivalent residue in mouse ABCA7 is indeed an A, but the authors need to clarify if they believe there is a mistake in the annotation for the human protein sequence. Their MD analysis, which compares the effect of an A or G at this position would still be valid if one considers an A in this position is ‘protective.’

ABCA7 was captured in multiple conformations by cryo-EM and its conformational transitions were shown to involve largely rigid body movements and significant contributions from bilayer lipids that are not considered here. The rationale for a G in this position, which is a flexible region between the ECD and TMD, leading to significant disruptions in the ability for of the transporter to adopt a closed conformation for lipid extrusion is tenuous considering it would allow for the same backbone torsion angles, with TMD closure dictated by binding of ATP. Additionally, the experimental structures and in vitro analysis by Le et al, EMBO J 2023 were done using transporters with the canonical G1527 matching Q8IZY2, showing that the protein produced displayed the expected conformational spectrum upon ATP binding.

\begin{enumerate}
    \item \textcolor{blue}{Our simulations were performed in a lipid membrane, but we only considered a single domain in the simulations. We do not compare open vs closed conformations, but the local effect of G1527, which does not trigger these overall conformational effects mentioned by the reviewer}
    \item \textcolor{blue}{Ramachandran plot of the residues A/G1527 over time to demonstrate that this reasoning is somewhat flawed. We are not saying ABCA7 G1527 cannot achieve the same conformation as ABCA7 A1527 (as indicated by CryoEM), but that ABCA7 G1527 likely explores multiple conformations, while A1527 stays in predominantly one conformation.}
    \item \textcolor{blue}{Secondary structure calculations over time to demonstrate that G1527 reduce the alpha helix content with respect to A1527}
    \item \textcolor{blue}{The cryoEM structure also indicate that the G1527 region is strongly flexible in the closed conformation.}
    \item \textcolor{blue}{G1527 in 8EOP (holo closed) has lost the alpha helix structure which is present in the 8EE6 (apo open). This is clear if we overlap ABCA1 and ABCA4 onto ABCA7, where in ABCA1 and ABCA4 the alpha helix is still present in the holo-closed conformations.}
\end{enumerate}

(Minor): The experimental structures were determined by single particle Cryo-EM, not crystallography as the authors state on lines 266-267.

\textcolor{blue}{We thank the reviewer for pointing out this oversight and have edited the text accordingly.}
