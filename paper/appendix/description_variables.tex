\paragraph{Description of variables according to the Rush Alzheimer’s Disease Center Codebook.}
\begin{enumerate}
    \item \textbf{age\_death}. Age at death
    \item \textbf{amyloid}. Overall amyloid level - Mean of 8 brain regions. Amyloid beta protein identified by molecularly-specific immunohistochemistry and quantified by image analysis. Value is percent area of cortex occupied by amyloid beta. Mean of amyloid beta score in 8 regions (4 or more regions are needed to calculate). The 8 regions are hippocampus, entorhinal cortex, midfrontal cortex, inferior temporal gyrus, angular gyrus, calcarine cortex, anterior cingulate cortex, superior frontal cortex.
    \item \textbf{braaksc}. Braak stage. Semi quantitative measure of neurofibrillary tangles. Braak Stage is a semi quantitative measure of severity of neurofibrillary tangle (NFT) pathology. Bielschowsky silver stain was used to visualize NFTs in the frontal, temporal, parietal, entorhinal cortex, and the hippocampus. Braak stages were based upon the distribution and severity of NFT pathology: Braak stages I and II indicate NFTs confined mainly to the entorhinal region of the brain; Braak stages III and IV indicate involvement of limbic regions such as the hippocampus; Braak stages V and VI indicate moderate to severe neocortical involvement.
    \item \textbf{ceradsc}. CERAD score. Semiquantitative measure of neuritic plaques.CERAD score is a semiquantitative measure of neuritic plaques. A neuropathologic diagnosis was made of no AD (value 4), possible AD (value 3), probable AD (value 2), or definite AD (value 1) based on semiquantitative estimates of neuritic plaque density as recommended by the Consortium to Establish a Registry for Alzheimer’s Disease (CERAD), modified to be implemented without adjustment for age and clinical diagnosis. A CERAD neuropathologic diagnosis of AD required moderate (probable AD) or frequent neuritic plaques (definite AD) in one or more neocortical regions. Diagnosis includes algorithm and neuropathologist’s opinion, blinded to age and all clinical data. Value 1: definite AD, Value 2: probable AD, Value 3: possible AD, Value 4: no AD.
    \item \textbf{cogdx}. Final consensus cognitive diagnosis. Clinical consensus diagnosis of cognitive status at time of death. At the time of death, all available clinical data were reviewed by a neurologist with expertise in dementia, and a summary diagnostic opinion was rendered regarding the most likely clinical diagnosis at the time of death. Summary diagnoses were made blinded to all postmortem data. Case conferences including one or more neurologists and a neuropsychologist were used for consensus on selected cases. Value 1: NCI, No cognitive impairment (No impaired domains), Value 2: MCI, Mild cognitive impairment (One impaired domain) and NO other cause of CI, Value 3: MCI, Mild cognitive impairment (One impaired domain) AND another cause of CI, Value 4: AD, Alzheimer’s disease and NO other cause of CI (NINCDS PROB AD), Value 5: AD, Alzheimer’s disease AND another cause of CI (NINCDS POSS AD), Value 6: Other dementia. Other primary cause of dementia
    \item \textbf{msex}. Sex. Self-reported sex, with “1” indicating male sex. 1 = Male 0 = Female
    \item \textbf{nft}. Neurofibrillary tangle burden. Neurofibrillary tangle summary based on 5 regions. Neurofibrillary tangle burden is determined by microscopic examination of silver-stained slides from 5 regions: midfrontal cortex, midtemporal cortex, inferior parietal cortex, entorhinal cortex, and hippocampus. The count of each region is scaled by dividing by the corresponding standard deviation. The 5 scaled regional measures are then averaged to obtain a summary measure for neurofibrillary tangle burden.
    \item \textbf{pmi}. postmortem interval. Time interval in hours from time of death to autopsy. postmortem interval (PMI) refers to the interval between death and tissue preservation in hours.
\end{enumerate}
