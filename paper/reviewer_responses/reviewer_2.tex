\subsubsection*{Referee \#2 (Remarks to the Author)}
LOF of ABCA7 increases risk of AD. the studies of its disease mechanisms are mostly done in animal models or in vitro cell models, which greatly limited their relavance to human disease. This study performed single-nucleus RNA sequencing (snRNA-seq) on rare ABCA7 loss-of-functon(LOF) carriers as well as an independent group of ABCA7 missense variants from the ROSMAP cohort. The study uncovered convergent neuronal pathways perturbed by ABCA7 variants, including those related to lipid metabolism and mitochondrial dysfunction. Further, robust and comprehensive omics analyses (lipidomics and metabolomics) were conducted to validate these observations using functional cellular assays in human induced pluripotent stem cell (iPSC)-derived neurons. Significantly, the study demonstrated that CDP-choline rescued LOF-induced dysregulation in mitochondrial function, lipid droplets, and amyloid biosynthesis. The mechanistic link between lipid disturbance, mitochondrial dysfunction, and amyloid pathology is novel. The robustness of the conclusions could be strengthened with additional controls.

\textcolor{blue}{We thank the reviewer for their encouraging comments and constructive suggestions.}

\textbf{Major comments:}

Figure 1D shows the combined demographics. It’s important to show the diagnosis, Cerad score and Braak scores, and gender distribution, stratified by the ABCA7 genotypes (control and LOF).

\textcolor{blue}{We thank the reviewer for pointing this out and have provided the stratified data in Figure\ref{fig:snRNA_cohort}D,E.}

Figure 1B, the numbers of samples for normal (n=180) and LOF (n=5) are not balanced. Increasing the number of LOF cases would strengthen the conclusion.

\textcolor{blue}{While we agree with the reviewer on this, given how rare the LoF cases are, these are the maximum number of ABCA7 LoF cases available in the ROSMAP proteomics dataset.}

Figure 5 needs to be strengthened. The statement of ABCA7 LoF iNs recapitulate AD amyloid pathology is weak in Figure. S9A,B and Figure 5D, ELISA is needed to support the effect of pGlu50fs on Abeta42. If BACE1 is reduced by the mutation (Fig. 5E), the levels of APP-CTF in the cell, and sAPPbeta in the medium should be assessed and quantified biochemically, in addition to IF, to confirm the reduction of enzyme activity.

\textcolor{blue}{We now show by ELISA that soluble Aβ40 and Aβ42 are significantly increased in media from ABCA7 LoF (p.Tyr622* and p.Glu50fs3) iNs relative to WT iNs (Figure~\ref{fig:differentiating_iPSC_neurons}H). Given the low levels of secreted Aβ at this early developmental stage, we also performed ELISA measurements in media from dorsal cortical organoids differentiated from WT, p.Tyr622, and p.Glu50fs*3 lines at six months maturity. Consistent with our earlier findings, these mature organoids exhibited elevated soluble Aβ40 and Aβ42 in both ABCA7 LoF lines compared to WT (Figure~\ref{fig:main_choline}L and Figure~\ref{fig:neurospheroid_figure}B). The previously included Aβ42 immunostaining data have been omitted to prevent confusion. We agree with the reviewer that the claim that BACE1 is reduced requires additional investigation. However, we believe this is beyond the scope of our study. Therefore, we have removed the BACE1 data from the manuscript.}

Figure 5A: a control line should be included to show the effect of pTyr622 on Lipid droplet. Also lipid staining should also be performed in pGlu50fs neurons, since the major lipidomics data were generated from pGlu50fs, not from pTyr622.

\textcolor{blue}{To address this point, we performed an additional LC-MS lipidomics experiment comparing p.Tyr622* iNs to WT iNs. Previously, LC-MS analysis of p.Glu50fs*3 iNs revealed increased long-chain triglycerides ($\geq32$ carbons; Figure~\ref{fig:main_lipids}C), consistent with triglyceride species recently identified as increased in iNs in the context of APOE4 \cite{Haney2024-bp}. However, these longer-chain triglycerides were not detected in the subsequent LC-MS experiment with p.Tyr622* iNs, where instead we observed fewer, shorter-chain triglycerides ($\leq32$ carbons; Figure~\ref{fig:main_lipids}C vs. Figure~\ref{fig:main_lipids}I). This discrepancy may reflect technical variability or inherent instability of certain triglyceride species. Given the consistent phosphatidylcholine changes across both ABCA7 LoF lines, we have now adjusted our interpretation to emphasize the robust increase in saturated phosphatidylcholine species, while discussing the triglyceride findings more cautiously [page~\pageref{quoteH-label}]:}

\begin{quote}
	\textcolor{blue}{"\quoteH"}
\end{quote}

\textbf{Other comments:}

Figure. S7E, F: a nuclei staining and a merged image should be provided for identification of signals in GLUT neurons.

\textcolor{blue}{Given limited postmortem tissue, we were unable to perform this experiment and have now removed Figure. S7E,F from the manuscript.}

It appears that only one clone is generated for human iPSC carrying ABCA7 variants (related to Fig. 3), it will be more convincing to have one more independent clone or line to replicate the findings for each variant.

\textcolor{blue}{We agree with the reviewer that including multiple clones per ABCA7 LoF variant would further strengthen our manuscript. However, generating and thoroughly characterizing additional clones was not feasible given the substantial time and resources required. Nevertheless, we believe our use of two independent ABCA7 LoF lines—each harboring distinct genetic variants and originating from separate clonal selection events—addresses this concern effectively. Additionally, we have now included new mRNA-sequencing data demonstrating strong concordance between these two independent ABCA7 LoF lines (Figure~\ref{fig:main_mitochondrial}B), and also with the human postmortem data, further reinforcing the validity and reproducibility of our findings.}

Figure 4D: The correlation is not convincing, especially due to the combination of two genotypes and the clustering of data points in the LOF line.

\textcolor{blue}{In line with another reviewer’s comments we have now removed this panel.}

