\subsubsection{\underline{LC-MS Experiments on iNs}} 

\paragraph{Biphasic Extraction.} 
iPSC-derived neurons were washed once with cold PBS (Fisher; Cat\#MT21040CM) and lifted off plate with a cell scraper in 1 mL cold PBS. Cells were centrifuged at 2000 xg for 5 min. PBS was removed, and cells were resuspended in 2 mL cold methanol for biphasic extraction. Chloroform (Sigma 1.02444) (4 mL; cold) was added to each vial, and mixed by vortexing for 1 min. Water (Sigma WX0001) (2 mL; cold) was added to each vial, and mixed by vortexing for 1 min. Vials were placed in 50 mL conical tubes and centrifuged for 10 min at 3000 rcf for phase separation. The lower, chloroform phase was collected (3 mL from each sample) and transferred to new vials. 
In instances where samples were prepared by the Harvard Center for Mass Spectrometry, cell pellets were provided in 500 µL of methanol, vortexed, and transferred to 8 mL glass vials. Each sample received an additional 1.5 mL of methanol and 4 mL of chloroform, followed by vortexing and incubation for 10 min in an ultrasound bath. Next, 2 mL of water was added, and samples were again vortexed. Phase separation was achieved by centrifugation at 800 rcf for 10 min at 4°C. The resulting upper aqueous phases were transferred into new glass vials designated for metabolomics analysis, while the lower chloroform phases were transferred separately for lipidomics analysis.
At least one blank sample (containing no cells) was prepared alongside each biphasic extraction experiment and processed identically through the LC-MS analysis pipeline.

\paragraph{Cell pellet sample preparation for LC-MS Lipidomics.} 
Subsequent sample preparation for lipidomics was performed by the Harvard Center for Mass Spectrometry. Samples were dried under nitrogen flow until approximately 1 mL remained, transferred into microcentrifuge tubes, and completely evaporated to dryness. Dried samples were resuspended in chloroform, with volumes scaled according to biomass (cell count) using a minimum of 60 µL, then split into two equal aliquots for positive and negative ionization mode analyses. For experiments using only positive ionization mode, samples were resuspended in a smaller biomass-scaled volume (minimum 20–25 µL) without splitting. Following resuspension, samples were centrifuged (maximum speed for 10 min or 18,000 rcf for 20 min at 4°C), and supernatants were transferred into microinserts for LC-MS analysis.

\paragraph{Cell pellet sample preparation for LC-MS Metabolomics.}
Subsequent sample preparation for metabolomics was performed by the Harvard Center for Mass Spectrometry. Samples were dried under nitrogen flow until approximately 1 mL remained, transferred into microcentrifuge tubes, and evaporated completely to dryness. Dried samples were resuspended in 50\% acetonitrile in water, using volumes scaled according to provided biomass (minimum ~20 µL). Following centrifugation at maximum speed for 10 min, a consistent volume (either 12 µL or 15 µL, depending on the batch) of supernatant from each sample was transferred into microinserts. The remaining supernatants from each batch were pooled separately to create batch-specific quality control (QC) samples.

\paragraph{Media preparation for LC-MS Metabolomics.}
Media samples (100 µL each) were transferred into microcentrifuge tubes containing 1 mL of methanol and incubated at -20°C for 2 hours. Following incubation, samples were centrifuged at 18,000 rcf for 20 min at -9°C, and supernatants were transferred into new tubes and evaporated to dryness under nitrogen flow. The dried samples were resuspended in 50 µL of 30\% acetonitrile in water containing 2 mM medronic acid, centrifuged again at 18,000 rcf for 20 min at 4°C, and the resulting supernatants were transferred into glass microinserts for LC-MS analysis.

\paragraph{LC-MS Lipidomics.}
LC-MS lipidomics analyses were conducted at the Harvard Center for Mass Spectrometry using a procedure adapted from \supercite{Miraldi2013-ng}. Samples were analyzed using an Orbitrap Exactive Plus mass spectrometer (Thermo Scientific) coupled to an Ultimate 3000 LC system (Thermo Scientific). Analyses were performed in both positive and negative ionization modes, in top 5 automatic data-dependent MS/MS mode. Chromatographic separation was performed on a Biobond C4 column (4.6 $\times$ 50 mm, 5 µm particle size; Dikma Technologies). The flow rate began at 100 µL min$^{-1}$ with 0\% mobile phase B (MB) for the initial 5 minutes, followed by an increase to 400 µL min$^{-1}$ over the next 50 minutes with a linear gradient of MB from 20\% to 100\%. The column was subsequently washed at 500 µL min$^{-1}$ for 8 minutes with 100\% MB, then re-equilibrated for 7 minutes at 500 µL min$^{-1}$ using 0\% MB. For positive ion mode, mobile phases consisted of buffer A (MA: 5 mM ammonium formate, 0.1\% formic acid, and 5\% methanol in water) and buffer B (MB: 5 mM ammonium formate, 0.1\% formic acid, 5\% water, and 35\% methanol in isopropanol). For negative ion mode, buffer A (MA) contained 0.03\% ammonium hydroxide and 5\% methanol in water, and buffer B (MB) contained 0.03\% ammonium hydroxide, 5\% water, and 35\% methanol in isopropanol. Lipids were identified, and their signals integrated using Lipidsearch software (version 4.2.27, Mitsui Knowledge Industry, University of Tokyo). Integration quality and peak selections were manually curated prior to data export.

\paragraph{LC-MS Metabolomics.}
LC-MS metabolomics analyses were performed at the Harvard Center for Mass Spectrometry using a Vanquish LC system coupled with an ID-X mass spectrometer (Thermo Fisher Scientific). Samples (5 µL injection) were analyzed on a ZIC-pHILIC peek-coated column (150 mm $\times$ 2.1 mm, 5 µm particle size; Sigma Aldrich) held at 40°C. Mobile phases comprised buffer A (20 mM ammonium carbonate and 0.1\% ammonium hydroxide in water) and buffer B (97\% acetonitrile in water). The gradient initiated at 93\% B, decreasing linearly to 40\% B over 19 minutes, further decreasing to 0\% B over the subsequent 9 minutes, held at 0\% B for 5 minutes, returned to 93\% B within 3 minutes, and finally re-equilibrated at 93\% B for 9 minutes. The flow rate was held constant at 0.15 mL min$^{-1}$, except for an initial 30-second ramp from 0.05 to 0.15 mL min$^{-1}$. Mass spectrometry data were acquired in polarity-switching mode at 120,000 resolution, with an AGC target of $1 \times 10^5$, covering an m/z range from 65 to 1000. MS1 acquisition employed polarity switching for all samples. MS2 and MS3 analyses were performed on pooled samples using the AcquireX DeepScan method, with five reinjections each in positive and negative ion modes separately. A mixture containing standards of targeted metabolites was prepared and analyzed immediately following the sample runs for targeted metabolite analysis.
