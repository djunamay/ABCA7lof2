\subsubsection{Molecular Dynamics Simulations }
The initial structure of ABCA7 was obtained from the Protein Data Bank (PDB) in unbound-open and bound-closed conformations, PDB IDs 8EE6 and 8EOP, respectively. These experimentally solved structures harbor the G1527 variant. The A1527 structure was generated by mutating the glycine residue to alanine using pymol software. 
 
The ABCA7 domain between residues 1517 and 1756 was embedded in a dipalmitoylphosphatidylcholine (DPPC) membrane using the CHARMM-GUI web server and oriented according to the Orientations of Proteins in Membranes (OPM) database. Four different simulations were performed using GROMACS 2022.3, as reported in Table~\ref{tab:abca7_structures}. The CHARMM36M force field was used for all simulations.  
 
The protein-membrane system was solvated in a cubic box with a minimum distance of 1.0 nm between the protein and the box edge, using the TIP3P water model. Energy minimization was performed using the steepest descent algorithm with a maximum force threshold of 1000 kJ/mol/nm to relieve any steric clashes or bad contacts. The system was equilibrated in six phases, each 125 ps long, to equilibrate volume (NVT) and pressure (NPT). The production run, 300 ns long, was performed in the NPT ensemble at 323 K using a v-rescale thermostat and 1 bar using the Parrinello-Rahman barostat. A 2 fs time step with h-bonds constraints was used with periodic boundary conditions applied in all directions. Long-range electrostatics were handled using the Particle Mesh Ewald (PME) method with a cutoff of 1.0 nm for non-bonded interactions. 
 
RMSD was calculated to monitor the conformational stability of a given structure over the course of the simulation by comparing the position of $C_\alpha$ at time $t$ under simulation to its reference position (in 8EOP or 8EE6). 
 
Principal Component Analysis (PCA) was conducted to identify the major conformational changes during the simulation. The analysis involved the following steps:
\begin{enumerate}
    \item A covariance matrix of the $C_\alpha$ atom positional fluctuations was constructed using the `gmx covar` tool. 
    \item The covariance matrix was diagonalized to obtain the eigenvalues and eigenvectors, representing the principal components (PCs). 
    \item The trajectory was projected onto the first two principal components (PC1 and PC2) using the `gmx anaeig` tool to visualize the dominant motions.  
    \item A kernel density estimate (KDE) plot implemented in seaborn python3 was used to visualize the first two eigenvectors for each simulation, corresponding to 45\%, 40\%, 40\% and 33\% of variance in 8EOP-G1527, 8EE6-A1527, 8EE6-G1527 and 8EOP-A1527 respectively. 
\end{enumerate}

Visualization of the trajectories was carried out using VMD software. 
