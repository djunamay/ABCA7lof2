\begin{figure}[H]
%    \includegraphics[width=\textwidth]{./main_plots/fig_5_temp.png}
    \begin{subfigure}[t]{.24\textwidth}
        \begin{subfigure}[t]{\textwidth}
            \caption{}
            \includegraphics[width=\textwidth]{./main_plots/all_lipids_choline_batch1.png}        
        \end{subfigure} 
        \begin{subfigure}[t]{\textwidth}
            \caption{}
            \vspace{-0.5cm}
            \centering
            \includegraphics[width=.8\textwidth]{./main_plots/rna_correlation_plot.png}        
        \end{subfigure}  
    \end{subfigure}  
    \hspace{.5cm}
    \begin{subfigure}[t]{.23\textwidth}
        \begin{subfigure}[t]{\textwidth}
            \caption{}
            \includegraphics[width=\textwidth]{./main_plots/Y622_choline_kl_network_network.pdf}        
        \end{subfigure}  
        \begin{subfigure}[t]{\textwidth}
            \caption{}
            \includegraphics[width=\textwidth]{./main_plots/jaccard_pT622_with_choline.png}        
        \end{subfigure} 
    \end{subfigure} 
    \hspace{.25cm}
    \begin{subfigure}[t]{.45\textwidth}
        \caption{}
        \includegraphics[width=\textwidth]{./main_plots/kl_densities_choline.png}        
    \end{subfigure}  
    \begin{subfigure}[t]{.3\textwidth}
        \caption{}
        \includegraphics[width=\textwidth]{./main_plots/choline_mito_degs.png}        
    \end{subfigure}  
    \hspace{.4cm} 
    \begin{subfigure}[t]{.2\textwidth}
        \caption{}
       % \vspace{.4cm}
        \includegraphics[width=\textwidth]{./main_plots/uncoupling_choline_quantification.png}        
    \end{subfigure}  
    \hspace{.4cm}  
    \begin{subfigure}[t]{.4\textwidth}
        \caption{}
        \vspace{-0.15cm}
        \includegraphics[width=\textwidth]{./main_plots/tmrm_choline.png}        
    \end{subfigure} 
   
    %%%% NEW ROW %%%%
    \begin{subfigure}[t]{.4\textwidth}
        \caption{}
        \includegraphics[width=\textwidth]{./main_plots/cellrox_images_choline.png}        
    \end{subfigure}
    \begin{subfigure}[t]{.39\textwidth}
        \caption{}
        \includegraphics[width=\textwidth]{./main_plots/abeta_elisa.png}        
    \end{subfigure}  
   % \hspace{.4cm}
    \begin{subfigure}[t]{.2\textwidth}
        \caption{}
        \includegraphics[width=\textwidth]{./main_plots/cortical_organoids_ephys.png}        
    \end{subfigure} 
    %\hspace{.4cm}
    \caption{
        \textbf{CDP-choline Treatment Rescues ABCA7 LoF-Induced Disruptions in Neurons.}\\
    }
    \label{fig:main_choline}
\end{figure}
\begin{itemize}
    \item[\textbf{(A)}] Volcano plot of differentially abundant lipid species in p.Tyr622* iNs cultured for 4 weeks (treated with or without 100 $\mu$M CDP-choline during the final 2 weeks), colored by lipid class.  Statistical comparisons by independent-sample $t$-tests.
    \item[\textbf{(B)}] Correlation of gene perturbation scores ($S = -\log_{10}(p)\times\text{sign}(\log_2(\text{FC}))$) comparing p.Tyr622* vs. WT and p.Tyr622* $\pm$ CDP-choline iNs.
    \item[\textbf{(C)}] Kernighan-Lin (K/L) clustering of leading-edge genes from significantly perturbed pathways (BH FDR-adjusted $p<0.05$) comparing p.Tyr622* $\pm$ CDP-choline iNs. Colors represent distinct K/L gene clusters, matched to p.Tyr622* vs. WT cluster colors based on Jaccard analysis in (D).
    \item[\textbf{(D)}] Heatmap of Jaccard index overlap between K/L clusters from p.Tyr622* vs. WT and p.Tyr622* $\pm$ CDP-choline iNs.
    \item[\textbf{(E)}] (left) Gaussian kernel density plots of gene perturbation scores ($S$, positive values indicate upregulation with CDP-choline treatment) for each cluster. Solid lines denote cluster means. (right) Representative pathways annotating the most genes per cluster.
    \item[\textbf{(F)}] Volcano plot of differential expression of genes with mitochondrial-localized protein products (MitoCarta) for p.Tyr622* $\pm$ CDP-choline iNs.
    \item[\textbf{(G)}] Mitochondrial uncoupling quantified by Seahorse assay (proportion of basal oxygen consumption due to proton leak) in p.Tyr622* $\pm$ CDP-choline iNs cultured for 4 weeks (treated with or without 100 $\mu$M CDP-choline during the final 2 weeks). Statistical comparisons via independent-sample $t$-tests. $N=6$ (vehicle), $8$ (CDP-choline) wells.
    \item[\textbf{(H)}] Average TMRM fluorescence intensity per mask (thresholded at 75th percentile) in p.Tyr622* $\pm$ CDP-choline iNs cultured for 4 weeks (treated with or without 100 $\mu$M CDP-choline during the final 2 weeks), under baseline and FCCP-treated conditions.
    \item[\textbf{(I)}] Average CellROX fluorescence intensity per mask (thresholded at 75th percentile) in p.Tyr622* $\pm$ CDP-choline iNs cultured for 4 weeks (treated with or without 100 $\mu$M CDP-choline during the final 2 weeks).
    \item[\textbf{(J)}] Quantification of secreted A$\beta$ levels from media of cortical organoids derived from WT or p.Tyr622* iNs (cultured for 182 days), treated with or without 1 mM CDP-choline for 4 weeks.
    \item[\textbf{(L)}] Spontaneous action potentials recorded from dissociated cortical organoids derived from p.Tyr622* iNs (cultured for 150 days, followed by 2 weeks treatment post-dissociation), treated with or without 1 mM CDP-choline.
\end{itemize}