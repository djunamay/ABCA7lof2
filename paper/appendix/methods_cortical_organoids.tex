\subsubsection{\underline{Experimental Methods on Dorsal Cortical Organoids}} 

\paragraph{Cortical organoid generation}
Dorsal cortical organoid were generated as previously described \cite{Sloan2018-ja}. In brief, iPSC were cultured until 80-90\% confluence, before dissociation and preparing a single cell suspension at $1 x 10^5$ cells/mL in mTesr supplemented with 10 μM Rock inhibitor. cortical organoids were then generated by distributing in 100uL / well in PrimeSurface® 96 Slit-well Plates (S-Bio, #MS9096SZ). After 48-72 hours, differentiation was induced on days 0 - 5 via daily media change in Neural Induction Media comprised of DMEMf/12 (Life Technologies, cat. no. 11330-032), KnockOut serum replacement (Life Technologies, 10828-028), GlutaMAX, 2-Mercaptoethanol, Penicillin-Streptomycin, and SMAD inhibitors SB-431542 and dorsomorphin. From days 6-16, media was switched to Neural Differentiation medium composed of Neurobasal A medium, B27 supplement, GlutaMAX, Penicillin–streptomycin, Human recombinant EGF (20 ng/ml), Human recombinant FGF2 (20 ng/ml). Daily media change was performed until day 16, followed by every other day until day 25. From day 25 onwards, EGF and FGF was removed and replaced with 20ng/mL Human recombinant brain-derived neurotrophic factor (BDNF; 450-02) and 20 ng/mL Human recombinant neurotrophin 3 (NT3; PeproTech, cat. no. 450-03). After day 45, media was changed two times per week. 

\paragraph{Aꞵ Enzyme linked immunosorbent assays on cortical organoids.}
Culture media was collected from cortical organoids at days 176-182, following 3-4 week treatments with 500 $\mu$M or 1 mM CDP-choline. Samples were kept on ice and immediately analyzed using ELISA to quantify levels of Aβ40 and Aβ42. ELISAs were performed according to the manufacturer’s protocols for Aβ40 (ThermoFisher Scientific, KHB3481) and Aβ42 (ThermoFisher Scientific, KHB3441). For the Aβ42 assay, 50 µL of undiluted media was used, whereas media samples were diluted (1:6.67 or 1:10) for the Aβ40 assay. The cortical organoid age (5-6 months) was selected based on empirical evidence showing the emergence of a robust amyloid phenotype at this developmental stage. CDP-choline concentrations were increased from 100 µM to 500 µM and 1 mM to compensate for reduced perfusion in cortical organoids compared to 2D iN cultures.

\paragraph{Electrophysiological recordings on cortical organoids}
To produce 2D cultures from cortical organoids, day 150 cortical organoids were washed in PBS, before transferring to 1.5 mL eppendorf tubes containing Accutase (Stem Cell Technologies, 07920) and placing in a 37C water bath for 40 minutes, with gentle agitation using a P1000 pipette every 5-10 minutes. Dissociated organoids were then plated on \#1 glass coverslips (Fisher Scientific, 50-194-4702) with PDL, laminin, and matrigel coating as previously described. 2D cultures were maintained with or without 100 $\mu$M CDP-choline treatment for two weeks before electrophysiological recordings. Electrophysiological recordings were performed as described for the iNs. This treatment duration and concentration were selected to align with conditions in 2D iNs experiments. The cortical organoid age was chosen based on empirical observations indicating the emergence of a robust amyloid phenotype at the developmental stage of 5-6 months.

Outliers in spontaneous action potentials were detected using the interquartile range (IQR) method. Specifically, data points falling below Q1 - 2 × IQR or above Q3 + 2 × IQR were identified as outliers. This method resulted in the removal of two data points exhibiting unusually high spontaneous action potentials: one in condition p.Tyr622* (value = 9.38) and one in condition p.Tyr622*+Choline (value = 6.15). Additionally, cells recording zero spontaneous potentials were excluded, as these were presumed to represent glial cells rather than neuronal cells.

\paragraph{Immunostaining of cortical organoids.}
20 $\mu$m cortical organoid sections were fixed in 4\% formaldehyde for 10 min at room temperature, and cortical organoids were then transferred to 30\% sucrose/PBS for dehydration at 4C for 2 to 3 days. cortical organoids were next embedded in optimal cutting temperature compound (OCT compound) and sliced as 20-mm sections using a Cryostat microtome (Leica). For immunostaining, sections were washed with PBS once, followed by permeabilized and blocked in PBS containing 0.2\% TritonX-100 and 10\% bovine serum albumin for 1 hour at room temperature, then incubated with primary antibody (MAP2 1:1000; NeuN 1:1000, diluted in blocking solution) at 4 °C overnight. Primary antibody was visualized using the appropriate secondary antibody conjugated to Alexa Fluor 488, or Alexa Fluor 594 (1:500, Thermo Fisher Scientific). Nuclei were visualized with Hoechst 33342 (1:1000, Thermo Fisher Scientific). All images were captured using a Zeiss LSM 900 confocal microscope and the ZEN software. 