\subsubsection{Single-nuclear transcriptomic profiling of human PFC from ABCA7 LoF-variant carriers} 
To investigate the cell type-specific impact of ABCA7 LoF variants in the human brain, we queried whole genome sequences of >1000 subjects from the Religious Order Study or the Rush Memory and Aging Project (collectively known as ROSMAP) for donors with Alzheimer’s disease diagnoses who are carriers of rare damaging variants in ABCA7 that result in a PTC. We identified 12 heterozygous carriers of ABCA7 LoF variants, including splice region variants (c.4416+2T>G and c.5570+5G>C), frameshift variants (p.Leu1403fs and p.Glu709fs), and nonsense ‘stop gained’ variants (p.Trp1245* and p.Trp1085*) (Figure~\ref{fig:main_atlas}A; Figure~\ref{fig:snRNA_cohort}A,B; Data~\ref{data:ptc_variants}). These variants have previously been associated with increased AD risk in genetic association studies (Table~\ref{tab:annotation_abca7})\cite{Steinberg2015-mu,Holstege2022-vp} and are presumed to induce risk via ABCA7 haploinsufficiency\cite{Duchateau2024-rf}. Analysis of published proteomic data\cite{Johnson2020-ip} confirmed that ABCA7 PTC-variant carriers indeed had lower ABCA7 protein levels in the human postmortem PFC compared to non-carriers (p=0.018; N = 180 Controls, 5 ABCA7 LoF; Figure~\ref{fig:main_atlas}B; Figure~\ref{fig:snRNA_cohort}C). 

We then performed snRNAseq on postmortem PFC samples from the 12 ABCA7 LoF variant carriers and 24 ABCA7 PTC non-carrier controls that were matched based on a number of potentially confounding variables, including AD pathology, age at death, post-mortem intervals, sex, APOE genotype, and cognition (Figure~\ref{fig:main_atlas}C;D; Figure~\ref{fig:snRNA_cohort}D,E; Figure~\ref{fig:snRNA_quality_annotation}A,B; Data~\ref{data:cohort_metadata}; Supplementary Text). We verified that none of the 36 selected subjects carried damaging variants in other AD risk genes (TREM2, SORL1, ATP8B4, ABCA1, and ADAM10)\cite{Holstege2022-vp} and we confirmed ABCA7 genotypes in a subset of ABCA7 LoF carriers and controls in the cohort by Sanger sequencing (Figure~\ref{fig:snRNA_cohort}A) and verified that all single-cell libraries matched their whole genome sequencing counterparts to ensure all samples were appropriately assigned to their respective genotypes (Figure~\ref{fig:snRNA_cohort}B).
For each individual, we obtained fresh frozen tissue from the BA10 region of the PFC and performed snRNAseq using the 10x Genomics Chromium platform, resulting in a total of 150,456 cells (102,710 cells after multiple rounds of quality control) (Figure~\ref{fig:main_atlas}E). After quality control (qc; dimensionality reduction, batch correction, and multiple rounds of clustering; Figure~\ref{fig:snRNA_quality_annotation}C-E; Methods) and cell type annotation with curated marker lists, we identified heterogeneous populations of inhibitory neurons (In, SYT1 & GAD1+), excitatory neurons (Ex, SYT1 & NRGN+), astrocytes (Ast, AQP4+), microglia (Mic, CSF1R+), oligodendrocytes (Oli, MBP & PLP1+), and oligodendrocyte precursor cells (OPCs, VCAN+) (Figure~\ref{fig:main_atlas}E; Figure~\ref{fig:snRNA_quality_annotation}F-I). The putative vascular cell cluster was small and did not meet qc cutoffs (Methods) and was therefore not considered for downstream analysis. After qc, individual-level gene expression profiles correlated well within cell types (mean correlation circa 0.95) and were well represented across individuals (Figure~\ref{fig:snRNA_quality_annotation}J-M).  

\subsubsection{Cell type-specific perturbations in the presence of ABCA7 LoF}
To explore ABCA7-LoF related gene expression changes across all major cell types, we filtered genes based on evidence for transcriptional perturbation in ABCA7 LoF (p<0.05, linear model, total \#genes passing this cutoff = 2,389) in at least one cell type (Ex, In, Ast, Mic, Oli, or OPC), while controlling for known and unknown covariates in the model and considering only genes with cell type-specific detection rates >10\% (see Methods; Data~\ref{data:degs}). Next, we projected perturbed genes from higher dimensional gene score space (where each dimension corresponds to cell type-specific ABCA7 LoF perturbation scores S, where S=sign(log(FC))*-log10(p-value))) into two dimensions for visualization and clustering (Figure~\ref{fig:snRNAseq_gene_scores}A; Methods). 
The 2D visual representation captured the transcriptional landscape of ABCA7 LoF gene changes across all major cell types (Fig1F; Figure~\ref{fig:snRNAseq_gene_scores}B,C). To summarize this landscape in terms of biological pathways, we assigned genes to clusters based on their projections in 2D space and quantified their enrichment for specific biological pathways (Figure~\ref{fig:main_atlas}G; Gene Ontology Biological Process database; Methods). This analysis indicated a number of biological pathways whose perturbation was broadly associated with ABCA7 LoF in postmortem human PFC, including pathways related to heat shock response, inflammation, cell dynamics and adhesion, synaptic function, DNA repair, and cellular metabolism (Figure~\ref{fig:main_atlas}G; Data~\ref{data:pathway_enrichments}). 

Decomposing the ABCA7 LoF transcriptional signature into cell type-specific patterns indicated which ABCA7 LoF-associated gene clusters were commonly or uniquely perturbed across major cell types in the PFC (Figure 1H; Figure~\ref{fig:snRNAseq_gene_scores}D; Data~\ref{data:degs}-5; Methods). For example, microglia exhibited a down-regulated heat shock response (cluster 4), also present to a lesser extent in neurons and OPCs (FDR-adjusted permutation p-value<0.01 & |score|>0.25; Figure~\ref{fig:main_atlas}H). Microglia and astrocytes displayed up-regulated inflammatory clusters, such as cluster 5 and cluster 10, respectively (Figure~\ref{fig:main_atlas}H). OPCs demonstrated perturbed cell motility and endocytic-associated clusters (clusters 1 and 0; Figure~\ref{fig:main_atlas}H). Both neuronal subtypes showed up-regulation of DNA repair-associated cluster 9 (e.g., NUP205, POLD3, VCP; Figure~\ref{fig:main_atlas}H) and down-regulation of clusters 6 and 14, characterized by cell adhesion and synaptic genes (e.g., SYT11, SHANK3, IFT56, GABRA3, NLGN1; Figure~\ref{fig:main_atlas}H). In addition, excitatory neurons exhibited uniquely perturbed clusters, including up-regulated cluster 8 (mitochondrial complex genes, e.g., COX7A2, antioxidant response e.g., PARK7) and down-regulated cluster 2 (lipid synthesis and transport genes, e.g., LDLR, APOL2, NR1H3, ACLY, FITM2; Figure~\ref{fig:main_atlas}H).

Together, these findings indicate that ABCA7 LoF variants may induce widespread, cell type-specific transcriptional changes in the human PFC. This single-cell atlas provides a rich resource for future studies aiming to elucidate the contributions of individual neural cell types to ABCA7 LoF-driven forms of AD risk. This resource will be made available for exploration via the UCSC Single Cell Browser and for further analysis via Synapse (accession ID: syn53461705).

\subsubsection{ABCA7 is expressed most highly in excitatory neurons}
Our snRNAseq data suggest that excitatory neurons expressed the highest levels of ABCA7, compared to other major cell types in the brain (Figure~\ref{fig:main_excitatory}A).  ABCA7 transcripts were detected (count>0) in ~ 30\% of excitatory neurons and ~ 15\% of inhibitory neurons, while the detection rate was considerably lower (<10\%) for microglia and astrocytes and an order of magnitude lower (<3\%) for oligodendrocytes and OPCs (Figure~\ref{fig:abca7_expression}A, B). We validated this expression pattern in an independent published dataset from\cite{Welch2022-ef} (Table~\ref{tab:external_datasets}), where bulk RNA sequencing of NeuN- (glial) and NeuN+ (neuronal) cell populations derived from six human postmortem temporal cortex samples showed significantly higher ABCA7 levels in the neuronal population versus the glial cell population (p=0.021, paired t-test; Figure~\ref{fig:abca7_expression}C). Several control genes, whose expression patterns in glial versus neuronal cells are well established (ABCA1, APOE, and NEUROD1), had expected expression patterns that matched those in the snRNAseq data (Figure~\ref{fig:abca7_expression}B,C). These results indicate that neurons, particularly excitatory neurons, are the primary ABCA7-expressing cell type in the aged human PFC. Given the relatively higher expression of ABCA7 in excitatory neurons and the evidence of transcriptional perturbations by ABCA7 LoF in this cell type, we focused our subsequent analysis specifically on excitatory neurons.

\subsubsection{ABCA7 LoF perturbations in excitatory neurons}
As an alternative approach to the unsupervised clustering of gene perturbation scores among all cell types, we next used prior knowledge of biological pathway structure to perform an in-depth characterization of perturbed biological processes specifically in ABCA7 LoF excitatory neurons. To this end, we first estimated statistical overrepresentation of biological gene sets (WikiPathways, N pathways = 472) among up and down-regulated genes in ABCA7 LoF excitatory neurons vs controls (by GSEA; Methods). We observed a total of 34 pathways with evidence for transcriptional perturbation at p<0.05 in excitatory neurons (Data~\ref{data:kl_clusters}). Enrichments of these pathways were driven by 268 unique genes (“leading edge” genes\cite{Subramanian2005-pu}; Data~\ref{data:kl_clusters}).   

To extract unique information from leading-edge genes and limit pathway redundancy, we next separated these genes and their associated pathway annotations into non-overlapping groups, formalized as a graph partitioning problem (Figure~\ref{fig:main_excitatory}B; Figure~\ref{fig:benchmarking_clustering}; Methods; Supplementary Text). Establishing gene-pathway groupings of approximately equal size revealed eight biologically interpretable “clusters” associated with ABCA7 LoF in excitatory neurons (Figure~\ref{fig:main_excitatory}B-D; Data~\ref{data:kl_clusters}). Predominantly, these gene clusters centered around two themes: (1) energy metabolism and homeostasis (C2, C7) and (2) DNA damage (C1, C4, C5), cell stress (C3, C6), and synaptic dysfunction (C0).

Clusters C2 and C7 were primarily defined by genes involved in cellular energetics, including genes related to lipid metabolism, mitochondrial function, and  OXPHOS (Figure~\ref{fig:main_excitatory}C,D). Cluster 2, characterized by transcriptional regulators of lipid homeostasis (e.g. NR1H3, ACLY, PPARD), exhibited evidence for down-regulation in ABCA7 LoF and featured pathways related to "SREBP signaling" and "lipid metabolism" (Figure~\ref{fig:main_excitatory}C,D; Data~\ref{data:kl_clusters}). Cluster 7 comprised multiple mitochondrial complex genes (e.g. COX7A2, NDUFV2) responsible for ATP generation from carbohydrate and lipid catabolism and showed up-regulation in ABCA7 LoF (Figure~\ref{fig:main_excitatory}C,D; Data~\ref{data:kl_clusters}).The remaining clusters C0, C1, and C3-C6 were characterized by DNA damage and proteasomal, inflammatory, and apoptotic mediators. Clusters C1, C3, C4, and C5 were up-regulated in ABCA7 LoF excitatory neurons and characterized by pathway terms such as "DNA damage response," "Parkin-Ubiquitin Proteasomal system,” "Signaling pathways in glioblastoma," and "Nucleotide metabolism," respectively (Figure~\ref{fig:main_excitatory}C,D; Data~\ref{data:kl_clusters}). They included up-regulated DNA damage/repair and proteasomal genes (e.g. RECQL, TLK2, BARD1, RBL2, MSH6, PSMD5, PSMD7) and were accompanied by increased expression of inflammatory mediators (e.g. C2, C1S, CLU, PIK3CD in C0, C4, C6) and down-regulated gene signatures related to synaptic function (C0), cytoskeletal function (C3), and apoptosis-related genes (C6) (Figure~\ref{fig:main_excitatory}C,D; Data~\ref{data:kl_clusters}). 

Together, these data suggest that ABCA7 LoF may disrupt energy metabolism in excitatory neurons and that these disruptions coincide with a state of increased cellular stress, characterized by genomic instability and neuronal dysfunction.

\subsubsection{ABCA7 LoF and common missense variants lead to overlapping neuronal perturbations}
ABCA7 LoF variants substantially increase AD risk (Odds Ratio = 2.03)\cite{Steinberg2015-mu} but are rare and therefore only contribute to a small portion of AD cases\cite{Duchateau2024-rf}. To evaluate whether ABCA7 LoF transcriptomic effects in neurons would generalize to more common, moderate-risk genetic variants in ABCA7, we examined the ROSMAP WGS cohort for carriers of the prevalent ABCA7 missense variant p.Ala1527Gly (rs3752246: Minor Allele Frequency ≈ 0.18; \% carriers >= 1 allele  ≈  30\%; Figure~\ref{fig:main_excitatory}E), which has risen to genome-wide significance for increased AD risk (Odds Ratio = 1.15 [1.11-1.18])\cite{Kunkle2019-yo,Holstege2022-vp,Naj2011-bs}, but whose molecular effects are unexplored. We identified 133 individuals carrying at least one copy of the p.Ala1527Gly risk variant and 227 non-carriers (Figure~\ref{fig:main_excitatory}F) for whom snRNAseq data of the post-mortem PFC were available\cite{Mathys2023-rs}. We ensured that none of these 360 individuals were part of our earlier ABCA7 LoF snRNAseq cohort as control samples and that none carried ABCA7 LoF variants. Using this cohort, we investigated whether excitatory neurons from p.Ala1527Gly carriers exhibited evidence of transcriptomic perturbations in the ABCA7 LoF-associated clusters C0-C7.

Remarkably, all clusters displayed directional trends in p.Ala1527Gly neurons consistent with the mean effect sizes observed in ABCA7 LoF neurons (Figure~\ref{fig:main_excitatory}C,G), while controlling for pathology, age, sex, and other covariates (Methods). Notably, 4 out of 8 clusters exhibited substantial evidence of perturbation in p.Ala1527Gly variant carriers, with perturbation directions aligning with predictions for ABCA7 LoF (Figure~\ref{fig:main_excitatory}G,H). Specifically, we observed an up-regulation in the DNA damage cluster C1 and the proteasomal cluster C3 in p.Ala1527Gly carriers compared to controls, suggesting a similar cell stress and genomic instability signature to ABCA7 LoF carriers (Figure~\ref{fig:main_excitatory}G,H), and a borderline significant up-regulation of the mitochondrial cluster C7, again consistent with ABCA7 LoF (Figure~\ref{fig:main_excitatory}G,H). Intriguingly, the most significantly perturbed cluster was the lipid cluster C2, which exhibited down-regulation (Figure~\ref{fig:main_excitatory}G,H) similar to our observations in ABCA7 LoF carriers. 

Because a missense variant can alter the dynamic structure of a protein, and mutations to glycine are known to introduce significantly greater local flexibility compared to alanine, we evaluated whether the convergent transcriptional signature could be attributed to changes in ABCA7 protein structure. Leveraging recent advances in protein structure simulation and the newly crystallized structures of ABCA7 in its closed (ATP-bound)  (Figure~\ref{fig:main_excitatory}I) and open (ATP-unbound) (Figure~\ref{fig:md_simulations}A) conformations\cite{Le_Thi_My2022-dp,Jumper2021-na} , we conducted molecular dynamics simulations of a 239 residue region with and without the p.Ala1527Gly substitutions over a 300ns simulation time (Figure~\ref{fig:main_excitatory}J; Figure~\ref{fig:md_simulations}A; Methods; Supplementary Text). These simulations revealed that the AD risk-associated G1527 mutation in the ATP-bound closed form induced local flexibility, evidencing large structural fluctuations over time, while the A1527 variant had a stabilizing effect on the ATP-bound closed conformation with only minor structural fluctuations (Figure~\ref{fig:main_excitatory}K-M). Both variants displayed stable conformational behavior in the unbound-open conformation (Figure~\ref{fig:md_simulations}A-F). Since the closed ATP-bound conformation is thought to mediate lipid extrusion to the plasma membrane and apolipoproteins\cite{Le_Thi_My2022-dp}, our data indicate that the p.Ala1527Gly substitution significantly impacts local secondary structure  stability and conformational dynamics of ABCA7, potentially affecting its lipid extrusion function.  

Together, our molecular dynamics simulations indicate that the p.Ala1527Gly substitution likely increases AD risk by perturbing ABCA7 structure, and therefore function. As such, our findings imply that rare and common ABCA7 LoF variants may mediate AD risk via similar, ABCA7-dependent mechanisms  and that findings from an in-depth study of ABCA7 LoF may extend to larger portions of the at-risk population. Moreover, the pronounced lipid signature associated with both LoF and the common missense variant in excitatory neurons, which aligns with ABCA7's functional role as a lipid transporter, suggests a potential upstream role of lipid disruptions in ABCA7-mediated risk.  

\subsubsection{Lipid disruptions in ABCA7 LoF PFC}
Our findings revealed a notable downregulation of various genes and pathways involved in fatty acid biosynthesis in ABCA7 LoF neurons from postmortem human PFC (Figure~\ref{fig:main_atlas}G,H; Figure~\ref{fig:main_excitatory}B-D; Figure~\ref{fig:lipid_mitochondrial_perturbations}A,B), while pathways related to lipid breakdown, particularly OXPHOS involving the tricarboxylic acid (TCA) cycle via lipid oxidation, were upregulated (Figure~\ref{fig:lipid_mitochondrial_perturbations}C,D). We validated a subset of these gene expression changes by RNAscope on postmortem human brain samples, including down-regulation of ACLY, a converter of mitochondria-derived citrate into acetyl-CoA for fatty acid synthesis, and up-regulation of SCP2, a lipid transfer protein involved in lipid β-oxidation (Figure~\ref{fig:lipid_mitochondrial_perturbations}E,F). These findings suggest that ABCA7 LoF may impact the balance between lipid synthesis and catabolism through mitochondrial carbon flux.

Since ABCA7 is known to function as a lipid transporter with an affinity for phospholipids, including phosphatidylcholines\cite{Tomioka2017-nv,Picataggi2022-yp}, we next performed unbiased mass spectrometry-based lipidomics on a subset of post-mortem human brains from the snRNAseq cohort with and without ABCA7 LoF variants (N=8 individuals per group, matched on multiple covariates; Data~\ref{data:pfc_lipidome}; Methods) to explore the effects of ABCA7 LoF on lipid composition. We noted a modest decrease of multiple phospholipid species, including a significant decrease in cardiolipin and phosphatidylcholine (p<0.05) (Figure~\ref{fig:main_lipids}A) and a trend towards increased  abundance of triglycerides (Figure~\ref{fig:main_lipids}A). Notably, phospholipids (decreased) and triglycerides (increased) share a common precursor, diacylglycerol (Figure~\ref{fig:main_lipids}A). 

\subsubsection{Deriving human neurons with ABCA7 LoF variants}
To complement the correlative analyses in ABCA7 LoF human tissue, we next used CRISPR-Cas9 genome editing to generate two isogenic iPSC lines, each homozygous for ABCA7 LoF variants, from a parental line without ABCA7 variants (WT). The first LoF variant, ABCA7 p.Glu50fs*3, was generated by a single base-pair insertion in ABCA7 exon 3, resulting in a PTC early in the ABCA7 gene (Figure~\ref{fig:main_lipids}B; Figure~\ref{fig:differentiating_iPSC_neurons}A,C). This 5’-proximal PTC limits the possibility of producing a truncated protein with partial functionality, generating a near-complete knockout. The second LoF variant, ABCA7 p.Tyr622*, was generated by a single base-pair mutation in ABCA7 exon 15 (Figure~\ref{fig:main_lipids}B; Figure~\ref{fig:lipid_mitochondrial_perturbations}B,C). This PTC re-creates a variant previously observed in patients as associated with AD \cite{Steinberg2015-mu} and is meant to provide clinical context to ABCA7 dysfunction. 

We differentiated the isogenic iPSCs into neurons (iNs) through lentiviral delivery of a doxycycline-inducible NGN2 expression cassette as previously described\cite{Ho2016-kz} (Figure~\ref{fig:main_lipids}C; Figure~\ref{fig:differentiating_iPSC_neurons}D). At 2 and 4 weeks post-NGN2 induction, cells were immunoreactive for neuronal markers TUJ1 and MAP2 and showed robust neuronal processes by pan-axonal staining (Figure~\ref{fig:main_lipids}D; Figure~\ref{fig:differentiating_iPSC_neurons}E). Both WT and ABCA7 LoF lines had the ability to fire action potentials upon current injections (Figure~\ref{fig:main_lipids}E,F; Figure~\ref{fig:differentiating_iPSC_neurons}F-J). Though the ABCA7 genotype did not affect resting membrane potential (Figure~\ref{fig:differentiating_iPSC_neurons}H), ABCA7 LoF iNs fired more action potentials and with lesser magnitude of current injection than WT iNs (Figure~\ref{fig:differentiating_iPSC_neurons}I,J), indicating an ABCA7 LoF hyperexcitability phenotype. 

Previous studies suggest that ABCA7 dysfunction affects amyloid processing\cite{Satoh2015-yu,Sakae2016-uy,Bamji-Mirza2018-xt,Chan2008-qu,De_Roeck2018-zx}. To verify that this was the case in our ABCA7 LoF lines, we examined Aβ42 levels - considered the most toxic species\cite{Pauwels2012-ul,Fraser1991-tj,Pike1993-xs,Pettegrew2001-sm,Phillips2019-ev} - and observed  increased intracellular Aβ42 by immunohistochemistry in the p.Glu50fs*3 iNs (p = 0.016), and a similar trend of borderline significance in p.Tyr622* iNs (p = 0.069) when compared to control iNs (Figure~\ref{fig:quantification_Abeta42_iPSC_neurons}A,B). Together, these data (1) confirm successful neuronal differentiation from iPSCs, and (2) show that ABCA7 LoF iNs recapitulate key AD-associated phenotypes of hyperexcitability and amyloid pathology.

\subsubsection{ABCA7 LoF impacts the neuronal lipidome}
Lipidomics on postmortem brain tissue with and without ABCA7 LoF had suggested decreases in phospholipids and increases in triglycerides (Figure~\ref{fig:main_lipids}A). To investigate the effects of ABCA7 LoF variants on lipid homeostasis specifically in neurons, we performed untargeted LC-MS lipidomics on two independent iN differentiation batches containing the p.Glu50fs*3 ABCA7 mutation (N=3 p.Glu50fs*3 and N=3 WT for each batch), selecting the more severe mutation to maximize phenotypic effects. As lipidomic profiles across differentiation batches correlated well (Figure~\ref{fig:snRNA_cohort}0A,B), we pooled samples from each batch to increase statistical power and reduce the impact of potential outliers. Overall, we observed that the lipid profiles of ABCA7 p.Glu50fs*3 iNs differed substantially from their WT counterparts and were linearly separable when projected on the second principal component (Figure~\ref{fig:main_lipids}G). We observed strong evidence (p<5e-3) for perturbation of sphingolipids (specifically, ceramides; (Cer), phospholipids (including phosphatidylcholine (PC), cardiolipin (CL), phosphatidylserine (PS), and phosphatidylglycerol (PG)), and neutral lipids (specifically, triglycerides (TGs) and monoglycerides (MGs)), in ABCA7 p.Glu50fs*3 vs WT iNs (Figure~\ref{fig:main_lipids}H,I). 
Triglycerides were in absolute and relative terms the most frequently perturbed lipid species (N=31; 56\% of all detected TG species; hypergeometric enrichment p-value = 3.9e-15)(Data~\ref{data:ngn2_lipidome}; Figure~\ref{fig:main_excitatory}I). All differentially detected triglyceride species were up-regulated in ABCA7 LoF iNs (Figure~\ref{fig:main_lipids}I, J), indicating increased levels of neutral lipid accumulation in these cells. While none of the remaining lipid classes were significantly enriched (hypergeometric p<0.05) among the differentially detected species (Figure~\ref{fig:main_lipids}I), we did note that phosphatidylcholine species emerged as the second most frequently altered lipid class in absolute terms, including a total of 21 perturbed species (p<5e-3; 16\% of all detected PC species)(Figure~\ref{fig:main_lipids}I, K). These changes were followed by a smaller number of perturbations to other lipid species, such as ceramides (N=5 changes) and cardiolipin (N=2 changes). Together, these data indicate a striking triglyceride accumulation accompanied by bi-directional changes for different phosphatidylcholine species, as well as a number of changes to other phospho- and sphingolipid species. 

\subsubsection{ABCA7 LoF induces a compositional shift from phospatidylcholines towards triglycerides}
Phosphatidylcholines, which represent a major constituent of cellular membranes and the phospholipid envelope of lipid droplets (LDs), and triglycerides, which store fatty acids inside of lipid droplets, share diglycerides (DGs) as a common precursor\cite{Tomioka2017-nv,Picataggi2022-yp}. Importantly, diglycerides are not only converted into but can also be derived from phospholipids or triglycerides and then broken down into first monoglycerides and finally fatty acids for energy (Figure~\ref{fig:main_lipids}L). Breakdown of triglycerides is believed to be impacted by the phosphatidylcholine to triglyceride (PC:TG) ratio, which affects lipid droplet size and lipase access to their triglyceride core\cite{Krahmer2011-xy,Guo2008-xt,Fei2011-bl,Schott2019-zq}. Therefore, the relative abundance of phosphatidylcholines and triglycerides serves as an important indicator of metabolic state in the cell. We observed a dramatically altered PC:TG ratio (Figure~\ref{fig:main_lipids}M) in the ABCA7 LoF iN that was accompanied by a ~80\% increase (log2FC = 0.86) in the fraction of triglycerides relative to total lipid content (Figure~\ref{fig:main_lipids}N) and a circa 81\% decrease (log2FC = -2.3 in the fraction of monoglycerides (Figure~\ref{fig:main_lipids}O). These changes suggest an ABCA7 LoF-induced metabolic shift away from triglyceride breakdown into monoglycerides and towards triglyceride accumulation. Furthermore, the steep increase in the relative triglyceride abundance together with a more moderate change in the relative diglyceride abundance (log2FC = 0.14; +10\%) (Figure~\ref{fig:main_lipids}P) and a decrease in the relative phosphatidylcholine abundance (log2FC = -0.65; -36\%) (Figure~\ref{fig:main_lipids}Q) suggest that diglycerides are less often converted into phosphatidylcholine and more often into triglycerides in the presence of ABCA LoF (Figure~\ref{fig:main_lipids}R). Notably, we observed a similar trend related to relatively increased triglyceride and decreased phosphatidylcholine abundance in the postmortem human PFC of ABCA7 LoF carriers (Figure~\ref{fig:quantification_Abeta42_iPSC_neurons}C). 

\subsubsection{ABCA7 LoF alters the neuronal metabolome}
To understand if these lipid changes were associated with global perturbations to the neuronal metabolome in ABCA7 LoF iNs, we performed untargeted metabolomic analysis on the aqueous fraction acquired during lipidomics preparations. Again, we observed consistency between batches and pooled the results for further analysis (Figure~\ref{fig:snRNA_cohort}0D). Projection of metabolite levels onto the first principal component completely separated samples by genotype (Figure~\ref{fig:main_mitochondrial}A), suggesting that the metabolomes between WT and ABCA7 LoF iNs differ substantially. 
We observed 10 increased and 49 decreased metabolites in ABCA7 LoF vs WT iN (p-value < 0.05 & |log2FC|>1; Data~\ref{data:ngn2_metabolome}; Figure~\ref{fig:main_mitochondrial}B). Nine of these differentially abundant metabolites could be annotated with high confidence and were not detected in the background media (Methods). Annotated species with differential abundance included a number of carnitine species, glutamine-glutamate, inosine-inositol, and hypoxanthine, all of which were less abundant in ABCA7 LoF vs WT iNs (Figure~\ref{fig:main_mitochondrial}B). Together, these data implicate a number of perturbed metabolic processes in ABCA7 LoF including lipid catabolism, TCA cycle, and OXPHOS (via carnitine\cite{Virmani2022-uc}, glutamine-glutamate\cite{Yoo2020-lh,noauthor_2023-sp}), cellular redox state (via inosine-inositol\cite{Chatree2020-qn,Basile2022-dd}, hypoxanthine\cite{Furuhashi2020-oi,Lee2018-tk}), and excitatory synaptic signaling (via glutamine-glutamate\cite{Morland2022-dk,noauthor_2021-cn,noauthor_2022-jz} and carnitine\cite{Inazu2008-wg,noauthor_2016-gp,Janiri1991-sx}). 
ABCA7 LoF decreases carnitine abundance

We were particularly intrigued by the down-regulation of carnitine and its esterified forms, acetylcarnitine and propionylcarnitine, which were consistently and strongly reduced in p.Glufs50*3 vs WT iNs  (log2FC -1.35, -1.32, -3.55,, respectively) (Figure~\ref{fig:main_mitochondrial}B, C). Carnitine and its derivatives mediate the rate-limiting transfer of medium and long chain free fatty acids across the mitochondrial membrane for β-oxidation \cite{noauthor_2016-la,noauthor_2004-tm}. Reduced amounts of carnitines in the ABCA7 LoF iN therefore suggest a decrease in neuronal β-oxidation of lipids relative to WT iN. While the primary substrates for neuronal OXPHOS are lactate and glucose\cite{Dienel2018-dt,Trigo2022-ym,Yellen2018-kr}, a recent study indicates that up to 20\% of basal neuronal OXPHOS may be fueled by lipid β-oxidation\cite{Morant-Ferrando2023-va} Indeed, our snRNAseq data from postmortem human PFC show that excitatory neurons express several regulators of β-oxidation at detectable levels (>10\%), including HADHA, which catalyzes the ultimate steps in β-oxidation, and carnitine palmitoyltransferases (CPT1C, CPT1B), which mediate the carnitine-dependent transport of fatty acids into mitochondria for oxidation (Figure~\ref{fig:snRNA_cohort}0E). We also observed a significant positive correlation between carnitine abundance and monoglycerides, an intermediate in the hydrolysis of triglycerides into free fatty acids, in both WT and ABCA7 LoF iNs (Figure~\ref{fig:main_mitochondrial}D). Together with the transcriptional changes to mitochondrial-related genes (Figure~\ref{fig:main_excitatory}C; Figure~\ref{fig:lipid_mitochondrial_perturbations}C,D), the observed decrease in carnitines  points to altered mitochondrial bioenergetics in ABCA7 LoF neurons and suggests a possible connection to the lipid disruptions detected in these cells.

\subsubsection{ABCA7 LoF impairs mitochondrial uncoupling in neurons}
To directly assess mitochondrial function in ABCA7 LoF neurons, we next measured the oxygen consumption rate (OCR) of WT and ABCA7 LoF iNs over time using the live-cell Seahorse metabolic flux assay (Agilent). The mitochondrial membrane potential (ΔѰm) is maintained via the oxygen-dependent movement of protons across the inner mitochondrial membrane during OXPHOS and dissipated through the membrane-bound ATP synthase and by regulated proton leakage along the gradient (Figure~\ref{fig:main_mitochondrial}E). The OCR, in the presence of various mitochondrial inhibitors, therefore allows several readouts of mitochondrial function\cite{Divakaruni2014-eq}.

Since measures of oxygen consumption are sensitive to differences in viable cell number and mitochondrial abundance\cite{Divakaruni2014-eq,Gu2021-ms}, parameters which may be affected by ABCA7 LoF mutations, we report internally-normalized measures of mitochondrial function derived from the OCR curves\cite{Divakaruni2022-rj} of WT, ABCA7 p.Glu50fs*30, and ABCA7 p.Tyr622* iNs (Figure~\ref{fig:snRNA_cohort}1A,B).  First, we  report the measured increase in oxygen consumption upon pharmacologically collapsing the proton gradient, which indicates the cellular spare respiratory capacity \cite{Divakaruni2022-rj}(Figure~\ref{fig:snRNA_cohort}1C).  Secondly, we report the proportion of basal oxygen consumption attributed to ATP synthesis vs proton leak across the membrane  (i.e. “relative uncoupling”; colored panels in Figure~\ref{fig:main_mitochondrial}F) \cite{Divakaruni2014-eq}. 

While we did not observe a significant difference in spare respiratory capacity between WT and ABCA7 LoF iNs (Figure~\ref{fig:snRNA_cohort}1D), we did observe a substantial decrease in relative uncoupling in ABCA7 LoF vs WT iNs (Figure~\ref{fig:main_mitochondrial}G; Figure~\ref{fig:snRNA_cohort}1E). Levels of relative uncoupling observed in our WT iNs (circa 20\%) (Figure~\ref{fig:main_mitochondrial}G) are in line with previous reports under basal conditions in neurons and other cell types\cite{Divakaruni2011-uj,Jekabsons2004-fn}, indicating that ABCA7 LoF neurons have abnormally low levels of uncoupling. Impaired mitochondrial uncoupling is generally associated with a hyperpolarized mitochondrial membrane potential, ΔѰm\cite{Demine2019-qj,Zorov2021-dq}. To verify that this was also the case in ABCA7 LoF iNs, we incubated iNs with a fluorescent dye that accumulates in mitochondria proportionally to ΔѰm. Indeed, we observed an increase in ΔѰm assessed by mean fluorescence intensity per NeuN+ surface in ABCA7 LoF vs WT iNs (Figure~\ref{fig:main_mitochondrial}H; Figure~\ref{fig:snRNA_cohort}1F). 

Cells actively control their ΔѰm through expression of mitochondrial uncoupling proteins\cite{noauthor_2021-tx,noauthor_2021-im}, which limit the formation of damaging reactive oxygen species induced by high ΔѰm\cite{Suski2012-zo} (Figure~\ref{fig:main_mitochondrial}E). Free fatty acids, in particular, are known to promote mitochondrial uncoupling\cite{Hoang2015-vm,noauthor_1999-dt,Ardalan2022-bk}, perhaps allowing cells to safely metabolize lipids through mitochondrial β-oxidation. Together, these data suggest that ABCA7 LoF neurons may have decreased ability to dissipate their ΔѰm and maintain basal levels of carbon oxidation via regulated uncoupling. 

\subsubsection{Treatment with CDP-choline rescues lipid and mitochondrial defects}
Our data indicate a relative metabolic shift away from phosphatidylcholines towards triglycerides in ABCA7 LoF neurons, accompanied by changes in mitochondrial metabolites and function. Decreasing phosphatidylcholine synthesis can indirectly increase triglyceride levels, given their common precursor\cite{Farese2023-bb}.  In addition, deficient phosphatidylcholine at the mitochondrial membrane may directly impair mitochondrial lipid breakdown, perhaps further compounding the triglyceride accumulation\cite{Schuler2016-tr,Szymkowicz2019-be,Mejia2015-tw,Prola2021-uz}. Therefore, enhancing phosphatidylcholine synthesis in ABCA7 LoF neurons may rescue triglyceride accumulation and mitochondrial function in these neurons.

To test this, we treated ABCA7 p.Tyr622* iNs with CDP-choline, the rate-limiting precursor for phosphatidylcholine synthesis through the Kennedy pathway \cite{Zeisel2009-xv,Son2024-tu}, for 2 weeks and then evaluated neutral lipid accumulation and mitochondrial bioenergetics. Because triglycerides are stored primarily in the form of neutral lipid droplets\cite{noauthor_2024-sd}, we measured neutral lipid accumulation by staining with a lipid-droplet specific dye, LipidSpot. Fluorescence intensity of  LipidSpot and an antibody against perilipin 2 (PLIN2), a protein embedded in lipid-droplet envelopes\cite{Olzmann2019-qv}, correlated well, indicating that the LipidSpot stain labels bonafide lipid droplets in iNs (Figure~\ref{fig:main_choline}A; Figure~\ref{fig:snRNA_cohort}2A). Treatment of ABCA7 p.Tyr622* iNs with CDP-choline reduced the neutral lipid burden per NeuN+ surface, as measured by LipidSpot fluorescence intensity, compared to vehicle treatment (Figure~\ref{fig:main_choline}A; Figure~\ref{fig:snRNA_cohort}2B). 

By measuring the oxygen consumption rate in ABCA7 p.Tyr622* iNs with and without CDP-choline, we found that CDP-choline increased relative uncoupling in p.Tyr622* iNs back to WT levels (Figure~\ref{fig:main_choline}B; Figure~\ref{fig:snRNA_cohort}2C,D), with no significant effect on SRC (Figure~\ref{fig:snRNA_cohort}2E). In line with this increased uncoupling, CDP-choline supplementation decreased ΔѰm in p.Tyr622* iNs vs vehicle treated cells, as shown by reduced MitoHealth dye accumulation (Figure~\ref{fig:main_choline}C). These data indicate that supplementation with CDP-choline rescued both triglyceride accumulation and re-established control of the mitochondrial membrane potential. 

\subsubsection{Treatment with CDP-choline decreases amyloid-β production}
Changes in membrane phospholipid content influence membrane properties including curvature and fluidity\cite{Van_der_Veen2017-pj}. Because changes to membrane fluidity are important regulators of amyloid precursor protein (APP) processing\cite{Bharadwaj2018-yu,Yang2014-vl}, we wondered whether phosphatidylcholine defects in ABCA7 LoF iNs were also responsible for increased levels of Aβ42 processing described earlier (Figure~\ref{fig:quantification_Abeta42_iPSC_neurons}A,B) and in previous literature\cite{Satoh2015-yu,Sakae2016-uy,Chan2008-qu,Bamji-Mirza2018-xt}. To test this, we measured intracellular Aβ42 levels in p.Tyr622* iNs treated with CDP-choline. Indeed we found a significant reduction in Aβ42 fluorescence per NeuN+ volume after treatment with CDP-choline (Figure~\ref{fig:main_choline}D). Other studies suggest that the activity of secretases, which cleave APP to produce amyloid, is also modulated by membrane fluidity or by phospholipid binding\cite{Walter2013-qu,Takasugi2011-iq,Kalvodova2005-kb}. CDP-choline treatment also decreased fluorescence intensity of the β-amyloid secretase enzyme BACE (Figure~\ref{fig:main_choline}E), indicating reduced amyloidogenic processing in treated p.Tyr622* iNs. Together, our data suggest that treatment with CDP-choline rescues increased amyloid processing in ABCA7-LoF iNs, linking the aforementioned ABCA7 LoF-induced metabolic defects in neurons (Figure~\ref{fig:main_choline}F) to AD pathology.  

