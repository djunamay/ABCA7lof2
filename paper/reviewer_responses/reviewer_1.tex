\subsubsection*{Referee \#1 (Remarks to the Author)}
In this manuscript, von Maydell et al. analyzed the single nucleus transcriptomes of AD patients with or without ABCA7 LoF mutations and found that ABCA7 LoF affects pathways including lipid metabolism, mitochondrial function, and others. Through cross validation with other datasets and using iPSC-derived neurons, the authors showed that ABCA7 LoF led to lipid accumulation and a decrease in mitochondrial uncoupling in neurons. Modulating ABCA7 LoF through CDP-choline treatment may reverse these effects.
While the study provides novel evidence of the important roles that phosphatidylcholine disruption plays in AD pathogenesis, some key information is missing. Detailed comments are listed below.

\textcolor{blue}{We thank the reviewer for their encouraging comments and constructive suggestions.}

Fig. 1F, what are the genes overlapping in the different cell types? are these changes relevant for the risk of ABCA7 LOF on AD? The thresholds of the volcano plots should be indicated.

\textcolor{blue}{We have now added a supplementary heat map showing overlapping genes across cell types (Figure~\ref{fig:snRNAseq_gene_scores_2}A). For visualization purposes, we included only genes significantly perturbed (p<0.05) in ABCA7 loss-of-function (LoF) brains in at least two cell types. Figure~\ref{fig:snRNAseq_gene_scores_2}B provides functional annotations for these genes, grouping them into key categories. These categories highlight pathways consistently identified in our study, such as lipid metabolism, mitochondrial function, and DNA damage response, suggesting these pathways may contribute to Alzheimer's disease risk linked to ABCA7 LoF in other cell types, in addition to neurons.}

\textcolor{blue}{We appreciate the reviewer's comment. However, these plots represent projections of differentially expressed gene scores across major cell types for simplified visualization and comparison in 2D space, rather than standard volcano plots. To clearly distinguish them from volcano plots, we have now added axis labels indicating the UMAP dimensions in Figure~\ref{fig:main_atlas}F}

Fig. 1G, it is not clear how the clusters were defined and genes under each pathway/cluster were selected. PCA the and thresholds of volcano plots should be provided.

\textcolor{blue}{We have now carefully revised our description of this analysis in the text to clarify our approach [page~\pageref{quoteF-label}]:}
\begin{quote}
	\textcolor{blue}{["\dots]} \quoteF\\\\
	\quoteG\textcolor{blue}{"}
\end{quote}

Fig. 3E-F, the data of the 3 iNs should be provided.

\textcolor{blue}{We have now included the data for all three lines in Figure~\ref{fig:iN_phenotypes}A,B}.

Fig. S9. The increase of Aβ42 is not evident, Amyloid accumulation should be confirmed with another method like immunoblotting or ELISA. Images with more cells within the same ROI should be provided, for example like in Fig.4H. Same applies to Fig 5A, 5D and 5E.

\textcolor{blue}{We have now confirmed increased soluble Aβ40 and Aβ42 levels by ELISA in media from ABCA7 LoF (p.Tyr622* and p.Glu50fs*3) iNs compared to WT iNs and present these data in Figure~\ref{fig:differentiating_iPSC_neurons}H. Since secreted Aβ levels were low at this early developmental stage, we also differentiated WT, p.Tyr622*, and p.Glu50fs*3 lines into dorsal cortical organoids and repeated ELISA measurements on the media at a more mature timepoint (six months). These experiments similarly confirmed elevated soluble Aβ40 and Aβ42 in media from p.Tyr622* and p.Glu50fs*3 cortical organoids compared to WT cortical organoids. We now present these data in Figure~\ref{fig:main_choline}L and Figure~\ref{fig:neurospheroid_figure}B. We have removed the immunostaining data for Aβ42 to avoid confusion.}

Is the number of up- and down-regulated PCs highlighted in Fig. 3K the sameas in Fig.3I.

\textcolor{blue}{We thank the reviewer for highlighting this inconsistency. We have now revised the relevant panels, consistently applying the criteria of adjusted p-value < 0.05 and absolute log2 fold change > 1. These updated data are presented in Figure~\ref{fig:main_lipids}A,B,D,E}

Line 365-380, the lipidomics analysis is unclear, oversimplified, and seemingly biased which I am afraid may affect accurate data interpretation. Here are my comments: 1) in Fig. 3I, the PC lipid species display 13 up- and 8 down-regluated species, but in Fig. 3Q, the relative PC abundance in total is even decreased, the two results are contradictory. 2) which PC lipid species were selected to calculate the ratio of PC:TG. TG species are uniformly increased, which can predominantly drive the decrease in this ratio, likely not specifically related to PC. 3) For MG and DG, when nearly no lipids in each class are significantly changed, interpreting the relative abundance of the entire lipid class is difficult and highly speculative. Based on the lipdidomics results, TG seems to accumulate and the complex changes in PC and other lipids are suggestive. But the conclusion that diglycerides are less often converted into phosphatidylcholine and more often into triglycerides in the presence of ABCA LoF is highly speculative. It is important here to measure lipid flux using istope tracers to confirm the conclusion.

\textcolor{blue}{We agree with the reviewer that our initial analysis was overly complex, and that emphasizing lipid ratios could lead to misinterpretation. To address this, we have substantially revised our lipidomics analysis comparing p.Glu50fs3 vs. WT iNs and added a new lipidomics experiment using the second ABCA7 LoF line (p.Tyr622 vs. WT iNs). We removed relative abundance analyses, instead highlighting specific changes at the individual lipid species level. These revisions allowed for a more detailed evaluation of phosphatidylcholine changes, which we discuss below.}

\textcolor{blue}{\underline{Regarding comments 2 and 3):} We have removed the relative abundance panels (Fig. 3 L–R) and simply provide the unbiased visualizations of individual lipid species grouped by major lipid categories in Figure~\ref{fig:main_lipids}A,B,G,H. These updated panels highlight changes at the individual lipid-species level rather than general lipid classes or ratios. For lipid classes that were frequently perturbed—specifically triglycerides and phosphatidylcholines—we provide additional analyses categorizing species based on fatty acid saturation and chain length in Figure~\ref{fig:main_lipids}C–E,H–J, given the important influence these features have on lipid function \cite{Wang2019-om}}.	

\textcolor{blue}{\underline{Regarding comment 1):} We agree with the reviewer that phosphatidylcholine changes are complex, with many species showing increased abundance. Instead of aggregating phosphatidylcholine species into a single ratio, we now categorize individual species by their fatty acid saturation profiles (Figure~\ref{fig:main_lipids}D,E). This analysis revealed significant enrichment of saturated and mono-unsaturated fatty acids among phosphatidylcholine species up-regulated in p.Glu50fs3 vs. WT iNs, while down-regulated species predominantly contained poly-unsaturated fatty acids. We confirmed a similar increase in saturated phosphatidylcholine species in p.Tyr622* vs. WT iNs (Figure~\ref{fig:main_lipids}G,H and Figure~\ref{fig:main_mitochondrial}K). Additionally, mRNA sequencing of p.Tyr622*, p.Glu50fs3, and WT iNs indicated decreased expression of key phosphatidylcholine remodeling enzymes, including LPCAT3, which incorporates poly-unsaturated fatty acids into phosphatidylcholine \cite{Zhao2008-pq} (Figure~\ref{fig:main_lipids}K,L). Together, these findings reveal complex alterations in phosphatidylcholine abundance and saturation in ABCA7 LoF neurons. We have updated the subsection "ABCA7 LoF induces phosphatidylcholine imbalance in neurons" to reflect these new data [page~\pageref{quoteA-label}]:}
\begin{quote}
	\textcolor{blue}{"}\quoteA\\\\
	\quoteB"\textcolor{blue}{"}
\end{quote}

Line 380: it is Fig. S10C (instead of Fig. S9C). Furthermore, the observed trends in TG and PC levels in human samples is minimal and not significant to suggest increased TG and decreased PC levels in ABCA7 LoF carriers. These data should be removed.
 
\textcolor{blue}{We have now removed these data from the paper.}

The increase of TG in ABCA7 LoF should be confirmed by lipid staining.

\textcolor{blue}{Our LCMS data for p.Glu50fs*3 revealed increased levels of long-chain triglyceride species with a total carbon count of $\geq$32 (Figure~\ref{fig:main_lipids}C), consistent with triglyceride species recently observed to accumulate in iNs in the context of APOE4 \cite{Haney2024-bp}. However, these longer-chain triglycerides were not detected in a subsequent LCMS experiment on p.Tyr622 iNs, where we instead observed fewer and shorter-chain triglycerides ($\leq$32 carbons; Figure~\ref{fig:main_lipids}C vs. Figure~\ref{fig:main_lipids}I). This discrepancy may reflect technical variability or instability of certain triglyceride species. Therefore, we have adjusted our interpretation to emphasize the robust and consistent increase in saturated phosphatidylcholine species detected in both ABCA7 LoF lines, while presenting triglyceride findings more cautiously [page~\pageref{quoteH-label}]:}
\begin{quote}
	\textcolor{blue}{"\quoteH"}
\end{quote}

Fig. 4A: In the PCA analysis, PC1 and PC2 explain only 0.18\% and 0.15\% of the variations, which are extremely low. Was it a mistake? If not, this indicates that the differences between the two groups are not actually captured by PC1 and PC2. The authors should review their analysis and consider testing other dimension reduction methods.

\textcolor{blue}{We thank the reviewer for pointing out this error. PC1 and PC2 explain 18\% and 15\% of the variance, respectively, not 0.18\% and 0.15\% as indicated in the figure. We have now corrected this panel.}

Fig. 4B, what were the upregulated metabolites?

\textcolor{blue}{Due to inherent limitations of LCMS analyses, typically only a small fraction of detected metabolites can be confidently annotated. In this specific analysis, none of the up-regulated metabolites could be annotated with high confidence. While these data are intriguing, we acknowledge that they have caused confusion among reviewers. Thus, we now limit our discussion of these findings to supplementary materials (Figure~\ref{fig:oxygen_consumption_rates_iPSC_neurons}). Although our data indicate notable differences in overall metabolite composition between ABCA7 LoF and WT iNs, further studies are required to fully characterize and interpret these metabolic changes.}

Fig. 4D, what is the significance of the correlation. If the dots in each group in the correlation plots represent technical replicates, the aggregated data (mean or median) should be used here.

\textcolor{blue}{Another reviewer expressed similar concerns regarding the interpretation and clarity of these plots, and asked us to remove these plots.}

Fig. 4E, how OCR data were normalized should be explained.

\textcolor{blue}{We have now updated the text to more clearly explain this point[page~\pageref{quoteC-label}]:}
\begin{quote}
	\textcolor{blue}{"\quoteC"}
\end{quote}

Line 457-459, CDP-choline cannot be directly taken up by cells. Although cells may uptake its hydrolyzed products, it remains important to confirm the cellular increase of CDP-choline.

\textcolor{blue}{We thank the reviewer for raising this important point. To determine whether CDP-choline treatment increases cellular choline levels, we performed targeted LCMS analyses of both media and cell extracts with and without CDP-choline treatment. Our experiments revealed that: (1) CDP-choline treatment raised CDP-choline levels in the media from non-detectable to detectable levels; (2) CDP and choline specifically accumulated in media conditioned by treated p.Tyr622* cells but not in treated media without cells, indicating—as suggested by the reviewer—that iNs hydrolyze CDP-choline into CDP and choline for separate uptake; and (3) targeted LCMS analyses showed a significant increase in cellular choline after treatment. CDP itself was not detected in any sample, while CDP-choline was detected at low levels, increasing from 1 of 8 (13\%) untreated samples to 4 of 8 (50\%) after treatment in p.Tyr622* cells.}

\textcolor{blue}{In agreement with this, mRNA sequencing showed increased expression of PCYT1B, encoding the rate-limiting enzyme for converting choline back into CDP-choline \cite{Lykidis1998-rj}. Furthermore, LCMS experiments on p.Tyr622* cells treated with CDP-choline showed significant (p<0.05, |logFC|>1) increases in several phosphatidylcholine species, as well as increases in other choline-containing lipids, including lysophosphatidylcholine—a direct phosphatidylcholine metabolite indicating acyl-chain remodeling \cite{Wang2019-om}—and sphingomyelin. The only lipid species that decreased significantly after CDP-choline treatment was a triglyceride species. Taken together, these data suggest that CDP-choline treatment leads to cellular uptake of choline, conversion into CDP-choline, and enhanced synthesis of multiple choline-containing lipids, including phosphatidylcholine.}

\textcolor{blue}{We have now updated the text with a detailed discussion of these results[page~\pageref{quoteD-label}]:}
\begin{quote}
	\textcolor{blue}{"\quoteD"}
\end{quote}

Line 453-455, from the data provided, it is unclear how PC and mitochondrial function are linked in ABCA7 LoF. Few questions remain to be addressed,
\begin{itemize}
	\item whether the change of PC affects mitochondrial membrane structure in ABCA7 LOF?

	\item whether CDP-choline increases cellular PC levels and whether this increase affects mitochondrial membrane structure in ABCA7 LoF?
	
	\item Whether CDP-choline specifically decreases the TG species that accumulate in ABCA7 LoF or affects different groups of TGs also remains to be determined.
\end{itemize}

\textcolor{blue}{\underline{Regarding item \# 1:} We have tried to address the structural question using three methods but have run into some technical issues. 1) TEM images, 2) Expansion microscopy, 3) mitotracker. We can show the mitotracker data here but say that additional difficulty is due to the mitochondrial membrane potential phenotype.}

\textcolor{blue}{\underline{Regarding item \#2:} As noted previously, we have now performed mRNA analysis on p.Tyr622* iNs following CDP-choline treatment, which revealed increased expression of PCYT1B (Figure~\ref{fig:choline_treatment}C), the rate-limiting enzyme for converting choline into CDP-choline \cite{Lykidis1998-rj}, after treatment. Consistent with this, LCMS analysis of CDP-choline-treated p.Tyr622* cells demonstrated significant (p<0.05, |logFC|>1) increases in multiple phosphatidylcholine species, as well as lysophosphatidylcholine—a phosphatidylcholine remodeling intermediate of the Land's cycle \cite{Wang2019-om}. We also observed increased levels of both saturated and unsaturated phosphatidylcholine species (Figure~\ref{fig:main_choline}A), along with restoration of previously decreased LPCAT3 expression after treatment (Figure~\ref{fig:choline_treatment}D). These findings indicate that CDP-choline likely rescues mitochondrial phenotypes by enhancing phosphatidylcholine synthesis and remodeling. We have moved previous discussions regarding the relationship between phosphatidylcholine and mitochondrial function (previously Line 453-455) to the discussion on page~\pageref{quoteI-label}, which now reads as follows:}

\begin{quote}
	\textcolor{blue}{"\quoteI"}
\end{quote}

\textcolor{blue}{\underline{Regarding item \#3:} Although we observed a modest decrease in triglyceride species following CDP-choline treatment (Figure~\ref{fig:main_choline}A), we were unable to detect the long-chain triglycerides with evidence for accumulation in p.Glu50fs*3 iNs (Figure~\ref{fig:main_lipids}C) in our LCMS experiment on p.Tyr622* iNs (Figure~\ref{fig:main_lipids}I). Therefore, we are cautious about drawing definitive conclusions regarding triglyceride accumulation in response to CDP-choline treatment, and we have updated the manuscript text accordingly to reflect this nuanced interpretation.}

Are the dyes used in Fig. 4H and Fig.5C the same? If different, both should be examined in Fig. 4H and 5C.

\textcolor{blue}{These are the same dyes.}

Line 479-487: The clearance of amyloid-β by CDP-choline is interesting yet perplexing. In the secretase model, it is unclear whether the clearance is related to the mitochondrial membrane or the membranes of other organelles via TAG or PC. More importantly, it is unclear whether the mechanisms involving TG and PC are actually causal for ABCA7 LOF-associated AD phenotypes or risk, and whether CDP-choline has therapeutic potential for AD. A mouse model of ABCA7 LOF-induced AD is necessary to address these questions.

\textcolor{blue}{\underline{Regarding the secretase data:} We agree with the reviewer that these findings are intriguing. However, investigating the precise mechanism through which CDP-choline reduces Aβ levels—now confirmed by ELISA in media from iNs and aged cortical organoids—is beyond the scope of the current study. Therefore, we have removed the secretase data from the manuscript.}

\textcolor{blue}{\underline{Regarding the causality of lipid dysregulation in ABCA7 LoF-associated AD:} We believe that our updated analyses provide strong support for an upstream role of phosphatidylcholine imbalance in ABCA7 LoF-associated risk. Specifically: (1) CDP-choline treatment significantly reversed ABCA7 LoF-related transcriptional, metabolic, and mitochondrial phenotypes, restoring them toward WT levels (updated Figure~\ref{fig:main_choline}); (2) targeted and untargeted metabolomics and lipidomics experiments indicate that CDP-choline treatment enhances synthesis and remodeling of phosphatidylcholine and its derivatives (Figure~\ref{fig:main_choline}A; Figure~\ref{fig:choline_treatment}A-D); and (3) phosphatidylcholine species were consistently perturbed across both p.Glu50fs3 and p.Tyr622* induced neurons (Figure~\ref{fig:main_choline}A; Figure~\ref{fig:main_lipids}A,B). Together, these findings highlight phosphatidylcholine imbalance in ABCA7 LoF neurons and demonstrate that intervening in phosphatidylcholine metabolism ameliorates ABCA7 LoF-associated defects. Given ABCA7’s established role as a phospholipid transporter, including phosphatidylcholine transport, our data strongly support phosphatidylcholine imbalance as an upstream mechanism driving dysfunction in ABCA7 LoF neurons.}

\textcolor{blue}{\underline{Regarding the therapeutic potential of CDP-choline:} We demonstrate that CDP-choline treatment effectively reduces several key Alzheimer's disease -related pathologies, including mitochondrial dysfunction, reactive oxygen species accumulation, amyloid pathology, and neuronal hyperexcitability. Importantly, our data suggest these benefits may generalize to carriers of the common ABCA7 variant (p.Ala1527Gly) (Figure~\ref{fig:main_neurons}), suggesting that CDP-choline—a readily available dietary supplement—may represent a promising therapeutic strategy to reduce AD risk in broader populations. This conclusion is further supported by recent findings from our lab showing CDP-choline's effectiveness in reversing APOE4-associated cellular phenotypes \cite{Sienski2021-zt} and linking choline metabolism to cognitive resilience in AD \cite{Mathys2024-ex}. Choline supplementation is currently under clinical investigation as an AD treatment strategy \cite{Cummings2024-cu}. While testing CDP-choline efficacy in a mouse model would be valuable, we consider this beyond the scope of the present study.}

% \textcolor{blue}{The relationship between mitochondrial dysfunction and increased reactive oxygen species (ROS)—as observed in our ABCA7 LoF neurons—is well established, with ROS accumulation being a hallmark of aging and Alzheimer's disease (AD) \cite{Welch2022-bp, Nissanka2018-yq}. Previous research suggests that alterations in membrane lipid composition, including changes in phospholipid saturation, may influence amyloid production and aggregation \cite{Ntarakas2019-lz, Nissanka2018-yq, Yang2014-fo}. Additionally, extensive evidence supports a close connection between amyloid pathology and neuronal hyperexcitability \cite{Cai2023-fw}, a phenotype we also observe in ABCA7 LoF iNs and cortical organoids. While our findings demonstrate that targeting phospholipid metabolism with CDP-choline treatment reduces all three of these key AD-related pathologies, fully elucidating the mechanistic relationship is beyond the scope of the current study,, supporting its therapeutic potential for ABCA7 LoF-associated AD. ..}

Line 161-164, line 215-217, line 256-260, across many datasets, DNA damage/repair pathways were detected; this was not explored further. Yet DNA damage/repair is closely related to lipid dysregulation and protein aggregation, it is therefore interesting to explore whether DNA damage/repair was involved in the phenotypic changes upon ABCA7 LoF.

\textcolor{blue}{We thank the reviewer for this insightful point. Given that DNA damage pathways consistently emerged in our datasets, we agree that exploring this connection further is valuable. DNA damage frequently occurs downstream of increased reactive oxygen species (ROS) \cite{Welch2022-bp}, and decreased mitochondrial uncoupling—as observed in our ABCA7 LoF lines—is well-known to cause oxidative damage in cells \cite{Jain2024-br, Crivelli2024-pf}. To directly test this, we measured ROS using CellROX staining, revealing significantly increased ROS levels in ABCA7 LoF neurons, which were effectively reversed by CDP-choline treatment (Figure~\ref{fig:main_mitochondrial}J and Figure~\ref{fig:main_choline}J). We have updated the Results and Discussion sections accordingly, highlighting this clearer connection between mitochondrial dysfunction, ROS accumulation, and potential genotoxic effects in ABCA7 LoF neurons, as originally indicated by our transcriptomic analyses.}

	
