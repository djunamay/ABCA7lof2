\begin{figure}[H]
    \begin{subfigure}[t]{0.6\textwidth}
        \caption{}
        \hspace{2cm}
        \includegraphics[width=\textwidth]{./extended_plots/iN_induction_cartoon.png}        
    \end{subfigure}
    \par
    \begin{subfigure}[t]{0.8\textwidth}
        \caption{}
        \hspace{2cm}
        \includegraphics[width=\textwidth]{./extended_plots/iN_markers.png}        
    \end{subfigure}
    \caption{
        \textbf{iN Markers.}\\
    }
    \label{fig:iN_markers}
\end{figure}
\caption{
    \textbf{Differentiating and Profiling iPSC-Derived Neurons Harboring ABCA7 PTC Variants.}\\
}
\label{fig:differentiating_iPSC_neurons}
\end{figure}
\begin{itemize}
\item[\textbf{(A)}] Sanger sequencing chromatogram confirming single nucleotide insertion in ABCA7 exon 3 to introduce a premature termination codon into the isogenic iPSC line ABCA7 p.Glu50fs*3 using CRISPR-Cas9 gene editing. 
\item[\textbf{(B)}] Sanger sequencing chromatogram confirming patient single nucleotide polymorphism in ABCA7 exon 15 to introduce a premature termination codon into the isogenic iPSC line ABCA7 p.Tyr622* using CRISPR-Cas9 gene editing. 
\item[\textbf{(C)}] Normal karyotypes were observed for control, ABCA7 p.Glu50fs*3, and ABCA7 p.Tyr622* isogenic iPSC lines. 
\item[\textbf{(D)}] iPSCs were plated at low density for NGN2 viral transduction. Expression of NGN2 was driven by doxycycline (DOX) induction with puromycin (PURO) selection, then re-plated to match neuronal densities. Neurons were maintained for 4 weeks (DIV 28) before experimentation (Created with BioRender.com). 
\item[\textbf{(E)}] Neuronal marker gene expression in 2 and 4-week matured iNs. 
\end{itemize}