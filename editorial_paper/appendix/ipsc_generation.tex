\subsubsection{\underline{Culture and generation of human isogenic iPSCs}} 

A control parental line was derived from a 75-year-old female (AG09173) with an APOE3/3 genotype by the Picower Institute for Learning and Memory iPSC Facility as first described \supercite{Lin2018-ma}. Two ABCA7 LoF isogenic lines were derived from parental AG09173. ABCA7 p.Glu50fs*3, generated by Synthego (www.synthego.com), contains a premature termination codon in exon 3 (Figure~\ref{fig:snRNA_cohort}1A), which to our knowledge has not been discovered in patients, but is functionally analogous to patient loss-of-function mutations.

ABCA7 p.Tyr622* contains a patient-derived mutation (Y622*) \supercite{De_Roeck2019-bs} and was generated in house by CRISPR-Cas9 genome editing. The CRISPR/Cas9-ABCA7-Y622* sgRNA plasmid was prepared followed by the published protocol \supercite{Ran2013-sf}. In brief, a sgRNA sequence within 10 nucleotides from the target site was designed using the CRISPR/Cas9 Design Tool (http://crispr.mit.edu). The oligomer pairs (forward: 5’-CACCGCCCCTACAGCCACCCGGGCG-3’ and reverse: 5’-AAACCGCCCGGGTGGCTGTAGGGGC-3’) were annealed and cloned into pSpCas9-2A-GFP (PX458) plasmid (Addgene \#48138). Plasmid DNA was submitted for Sanger sequencing to confirm correct ABCA7 sgRNA sequence (Figure~\ref{fig:snRNA_cohort}1B).

AG09173 iPSCs were dissociated with Accutase (Thermo Fisher Scientific) supplemented with 10 μM ROCK inhibitor (Tocris) for electroporation using Amaxa and Human Stem Cell Nucleofector Kit I (Lonza). ~5x106 cells were resuspended in 100 μl of reaction buffer supplemented with 7.5 μg of CRISPR/Cas9-ABCA7 sgRNA plasmid and 15 μg of single-strand oligodeoxynucleotide (ssODN) template \seqsplit{(5'-GGTGCGCGCCCCCAGGCCAATCCAGGAGCTGCACCCTAAGCTCCCGTTGCCTCTCACAGCTGGGAGACATCCTCCCCTAGAGCCACCCGGGCGTCGTCTTCCTGTTCTTGGCAGCCTTCGCGGTGGCCACGGTGACCCAGAGCTTCCTGCTCAGCGCCTTCTTCTCCCGCGCCAACCTGG-3')}. This reaction mixture was nucleofected with program A-23, resuspended with media supplemented with 10 μM ROCK inhibitor and seeded on MEF plates. Two days after electroporation, cells were dissociated and filtered through Falcon polystyrene test tubes (Corning \#352235), transferred to Falcon polypropylene test tubes (Corning \#352063) and sorted by BD FACS Aria IIU in FACS Facility at the Whitehead Institute and seeded as single cells in media supplemented with 1X Penicillin-Streptomycin (P/S, Gemini Bio-products) and 10 μM ROCK inhibitor. After sufficient colony growth, each colony was transferred in part to a 12-well plate while the remainder was collected and used to extract genomic DNA (Qiagen DNeasy Blood & Tissue Kit, Cat. No. 69504) and screen for the Y622* mutation by sanger sequencing.

All lines used were confirmed to have normal karyotypes before use and periodically reviewed (Cell Line Genetics) (Figure~\ref{fig:snRNA_cohort}0C). All human iPSCs were maintained at 37°C and 5\% CO2, in feeder-free conditions in mTeSR-1 medium (Cat \#85850; STEMCELL Technologies) on Matrigel-coated plates (Cat \# 354277; Corning; hESC-Qualified Matrix). iPSCs were passaged at 60–80\% confluence using ReLeSR (Cat\# 05872; STEMCELL Technologies) and reseeded between 1:6 and 1:24 (depending on desired density) onto Matrigel-coated plates.