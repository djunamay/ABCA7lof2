\subsubsection{Molecular Dynamics Simulations Results} 
The root mean square deviation (RMSD) analysis was performed to assess the conformational stability of ABCA7 in different states and mutations (Figure~\ref{fig:main_excitatory}I; Figure~\ref{fig:md_simulations}A). The RMSD values of the $C_\alpha$ atoms were calculated over the 300 ns simulation period for both the closed and open conformations, with G1527 and A1527 mutations. 

The RMSD plot for ABCA7 in the closed conformation with the G1527 mutation (Figure~\ref{fig:main_excitatory}J, K) exhibited large fluctuations throughout the simulation (Figure~\ref{fig:main_excitatory}L). The distribution of RMSD values was broad, indicating significant conformational variability (Figure~\ref{fig:main_excitatory}L; Figure~\ref{fig:md_simulations}C). The high RMSD values suggest that the G1527 mutation in the closed state induces larger flexibility, leading to substantial deviations from the initial structure. 

In contrast, the RMSD analysis for the closed conformation with the A1527 mutation (Figure~\ref{fig:main_excitatory}J,K) showed minor fluctuations (Fig.2L). The RMSD values were consistently lower than those observed for the G1527 mutation, and the distribution of RMSD values followed a normal distribution (Figure~\ref{fig:main_excitatory}L; Figure~\ref{fig:md_simulations}C). This indicates that the A1527 mutation confers lower local flexibility and greater local stability to the closed conformation of ABCA7. 

For the open conformation of ABCA7, both G1527 and A1527 mutations (Figure~\ref{fig:md_simulations}A,C) exhibited minor RMSD fluctuations (Figure~\ref{fig:md_simulations}D). The RMSD values for these mutations were significantly lower compared to the closed conformation with the G1527 mutation (Figure~\ref{fig:md_simulations}E). The RMSD distributions for both mutations in the open conformation were narrow (Figure~\ref{fig:md_simulations}C; Figure~\ref{fig:md_simulations}D), suggesting local stable conformational behavior. 

Principal component analysis (PCA) was performed to further investigate the conformational dynamics of ABCA7 in the different states and mutations. The PCA of the closed conformation with the G1527 mutation showed a widespread along the first principal component (PC1), reflecting the large conformational space sampled during the simulation (Figure~\ref{fig:main_excitatory}M). This is consistent with the broad RMSD distribution, indicating high flexibility and multiple conformational states. 

In contrast, the closed conformation with the A1527 mutation showed a much more confined spread along PC1 and PC2 (Figure~\ref{fig:main_excitatory}M). This limited spread corresponds to the minor RMSD fluctuations observed, suggesting a locally more stable and less flexible structure. 
For the open conformation, PCA indicated two distinct clusters for both G1527 and A1527 mutations (Figure~\ref{fig:md_simulations}D). These clusters represent two closely related conformational states, indicating that both mutations stabilize the open conformation into two main conformational states with minimal fluctuations. 
