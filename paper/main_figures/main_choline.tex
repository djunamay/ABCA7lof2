\begin{figure}[H]
%    \includegraphics[width=\textwidth]{./main_plots/fig_5_temp.png}
    \begin{subfigure}[t]{.24\textwidth}
        \begin{subfigure}[t]{\textwidth}
            \caption{}
            \includegraphics[width=\textwidth]{./main_plots/all_lipids_choline_batch1.png}        
        \end{subfigure} 
        \begin{subfigure}[t]{\textwidth}
            \caption{}
            \vspace{-0.5cm}
            \centering
            \includegraphics[width=.8\textwidth]{./main_plots/rna_correlation_plot.png}        
        \end{subfigure}  
    \end{subfigure}  
    \hspace{.5cm}
    \begin{subfigure}[t]{.23\textwidth}
        \begin{subfigure}[t]{\textwidth}
            \caption{}
            \includegraphics[width=\textwidth]{./main_plots/Y622_choline_kl_network_network.pdf}        
        \end{subfigure}  
        \begin{subfigure}[t]{\textwidth}
            \caption{}
            \includegraphics[width=\textwidth]{./main_plots/jaccard_pT622_with_choline.png}        
        \end{subfigure} 
    \end{subfigure} 
    \hspace{.25cm}
    \begin{subfigure}[t]{.45\textwidth}
        \caption{}
        \includegraphics[width=\textwidth]{./main_plots/kl_densities_choline.png}        
    \end{subfigure}  
    \vspace{.05cm}
    \begin{subfigure}[t]{.24\textwidth}
        \caption{}
        \includegraphics[width=\textwidth]{./main_plots/choline_mito_degs.png}        
    \end{subfigure}  
    \hspace{.4cm} 
    \begin{subfigure}[t]{.17\textwidth}
        \caption{}
        \vspace{.3cm}
        \includegraphics[width=\textwidth]{./main_plots/pca_plot_y622_choline_metab.png}        
    \end{subfigure} 
    \hspace{.4cm}
    \begin{subfigure}[t]{.17\textwidth}
        \caption{}
        \vspace{.4cm}
        \includegraphics[width=\textwidth]{./main_plots/uncoupling_choline_quantification.png}        
    \end{subfigure}  
    \hspace{.4cm}  
    \begin{subfigure}[t]{.35\textwidth}
        \caption{}
        \vspace{-0.15cm}
        \includegraphics[width=\textwidth]{./main_plots/tmrm_choline.png}        
    \end{subfigure} 
   
    %%%% NEW ROW %%%%
    \begin{subfigure}[t]{.35\textwidth}
        \caption{}
        \includegraphics[width=\textwidth]{./main_plots/cellrox_images_choline.png}        
    \end{subfigure}
    \hspace{.4cm}
    \begin{subfigure}[t]{.2\textwidth}
        \caption{}
        \includegraphics[width=\textwidth]{./main_plots/cortical_organoids_ephys.png}        
    \end{subfigure} 
    \hspace{.4cm}
    \begin{subfigure}[t]{.4\textwidth}
        \caption{}
        \includegraphics[width=\textwidth]{./main_plots/abeta_elisa.png}        
    \end{subfigure}  
    \caption{
        \textbf{CDP-choline Treatment Rescues ABCA7 LoF-Induced Disruptions in Neurons.}\\
    }
    \label{fig:main_choline}
\end{figure}
\begin{itemize}
    \item[\textbf{(A)}] Volcano plot showing differentially abundant lipid species between p.Tyr622* iNs with and without CDP-choline treatment, colored by lipid class. 
    \item[\textbf{(B)}] PCA plot of per-cell gene expression in p.Tyr622* iNs treated with CDP-choline or vehicle control. 
    \item[\textbf{(C)}] Kernighan-Lin (K/L) clustering of leading-edge genes from pathways perturbed in p.Tyr622* vs. WT iNs (p-adjusted < 0.05). Colors represent distinct K/L gene clusters, with clusters assigned p.Tyr622* vs. WT labels based on Jaccard analysis in 
    \item[\textbf{(D)}] Heatmap showing Jaccard index-based overlap between K/L clusters from p.Tyr622* vs. WT iNs and p.Tyr622* vs. CDP-choline iNs. 
    \item[\textbf{(E)}] Left: Gaussian kernel density estimate plots of gene scores (S) for genes in each cluster (S > 0 indicates upregulation in ABCA7 LoF). Solid lines show distribution means. Right: Representative pathways annotating the most genes within each cluster. 
    \item[\textbf{(F)}] Volcano plot of differentially expressed genes in p.Tyr622* iNs after CDP-choline treatment, highlighting those with mitochondrial protein localization. 
    \item[\textbf{(G)}] LC-MS metabolite profile projection onto the first two principal components. 
    \item[\textbf{(H)}] Relative mitochondrial uncoupling quantified by Seahorse oxygen consumption assay in 4-week-old p.Tyr622* iNs treated with CDP-choline or vehicle for 2 weeks. Uncoupling represents the proportion of basal oxygen consumption attributed to proton leak. P-values from independent t-tests, wells = 6 (p.Tyr622* + H2O), 8 (p.Tyr622* + CDP-choline). 
    \item[\textbf{(I)}] Average TMRM fluorescence intensity in p.Tyr622* iNs with or without CDP-choline, per mask (binarized TMRM signal based on the 75th percentile threshold) under baseline and FCCP treatment. 
    \item[\textbf{(J)}] Average CellROX fluorescence intensity in p.Tyr622* iNs with or without CDP-choline, per mask (binarized CellROX signal based on the 75th percentile threshold). 
    \item[\textbf{(K)}] Quantification of secreted Aβ levels in p.Tyr622* iNs with or without CDP-choline in cortical organoids. 
    \item[\textbf{(L)}] Spontaneous action potential measurement in p.Tyr622* iNs with or without CDP-choline in dissociated cortical organoids.
\end{itemize}